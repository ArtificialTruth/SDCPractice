%% Generated by Sphinx.
\def\sphinxdocclass{report}
\documentclass[letterpaper,10pt,english]{sphinxmanual}
\ifdefined\pdfpxdimen
   \let\sphinxpxdimen\pdfpxdimen\else\newdimen\sphinxpxdimen
\fi \sphinxpxdimen=.75bp\relax

\PassOptionsToPackage{warn}{textcomp}
\usepackage[utf8]{inputenc}
\ifdefined\DeclareUnicodeCharacter
 \ifdefined\DeclareUnicodeCharacterAsOptional
  \DeclareUnicodeCharacter{"00A0}{\nobreakspace}
  \DeclareUnicodeCharacter{"2500}{\sphinxunichar{2500}}
  \DeclareUnicodeCharacter{"2502}{\sphinxunichar{2502}}
  \DeclareUnicodeCharacter{"2514}{\sphinxunichar{2514}}
  \DeclareUnicodeCharacter{"251C}{\sphinxunichar{251C}}
  \DeclareUnicodeCharacter{"2572}{\textbackslash}
 \else
  \DeclareUnicodeCharacter{00A0}{\nobreakspace}
  \DeclareUnicodeCharacter{2500}{\sphinxunichar{2500}}
  \DeclareUnicodeCharacter{2502}{\sphinxunichar{2502}}
  \DeclareUnicodeCharacter{2514}{\sphinxunichar{2514}}
  \DeclareUnicodeCharacter{251C}{\sphinxunichar{251C}}
  \DeclareUnicodeCharacter{2572}{\textbackslash}
 \fi
\fi
\usepackage{cmap}
\usepackage[T1]{fontenc}
\usepackage{amsmath,amssymb,amstext}
\usepackage{babel}
\usepackage{times}
\usepackage[Bjarne]{fncychap}
\usepackage[,numfigreset=1,mathnumfig]{sphinx}

\usepackage{geometry}

% Include hyperref last.
\usepackage{hyperref}
% Fix anchor placement for figures with captions.
\usepackage{hypcap}% it must be loaded after hyperref.
% Set up styles of URL: it should be placed after hyperref.
\urlstyle{same}
\addto\captionsenglish{\renewcommand{\contentsname}{Practice guide}}

\addto\captionsenglish{\renewcommand{\figurename}{Fig.}}
\addto\captionsenglish{\renewcommand{\tablename}{Table}}
\addto\captionsenglish{\renewcommand{\literalblockname}{Listing}}

\addto\captionsenglish{\renewcommand{\literalblockcontinuedname}{continued from previous page}}
\addto\captionsenglish{\renewcommand{\literalblockcontinuesname}{continues on next page}}

\addto\extrasenglish{\def\pageautorefname{page}}

\setcounter{tocdepth}{1}



\title{Readthedocs Documentation}
\date{Jun 04, 2018}
\release{}
\author{Matthew Welch}
\newcommand{\sphinxlogo}{\vbox{}}
\renewcommand{\releasename}{}
\makeindex

\begin{document}

\maketitle
\sphinxtableofcontents
\phantomsection\label{\detokenize{index::doc}}


\sphinxstylestrong{Thijs Benschop}

\sphinxstylestrong{Cathrine Machingauta}

\sphinxstylestrong{Matthew Welch}

Affiliations: Thijs Benschop, Consultant The World Bank, Cathrine
Machingauta, The World Bank, Matthew Welch, The World Bank

Correspondence: Matthew Welch, \sphinxhref{mailto:mwelch@worldbank.org}{mwelch@worldbank.org}

Acknowledgments: The authors thank Olivier Dupriez (The World Bank) for
his technical comments and inputs throughout the process.

The production of this guide was made possible through a World Bank
Knowledge for Change II Grant: KCP II - A microdata dissemination
challenge: Balancing data protection and data utility. Grant number: TF
015043, Project Number P094376. As well as from United Kingdom - DFID
funding to the World Bank Multi-Donor Trust Fund - International
Household Survey and Accelerated Data Program \textendash{} TF071804/TF011722.


\chapter{Introduction}
\label{\detokenize{intro:statistical-disclosure-control-for-microdata-a-practice-guide}}\label{\detokenize{intro:introduction}}\label{\detokenize{intro::doc}}
National statistics agencies are mandated to collect
microdata %
\begin{footnote}[1]\sphinxAtStartFootnote
Microdata are unit-level data obtained from sample surveys, censuses
and administrative systems. They provide information about
characteristics of individual people or entities such as households,
business enterprises, facilities, farms or even geographical areas
such as villages or towns. They allow in-depth understanding of
socio-economic issues by studying relationships and interactions
among phenomena. Microdata are thus key to designing projects and
formulating policies, targeting interventions and monitoring and
measuring the impact and results of projects, interventions and
policies.
%
\end{footnote} from surveys and censuses to inform and
measure policy effectiveness. In almost all countries, statistics acts
and privacy laws govern these activities. These laws require that
agencies protect the identity of respondents, but may also require that
agencies disseminate the results and, in appropriate cases, the
microdata. Data producers who are not part of national statistics
agencies are also often subject to restrictions, through privacy laws or
strict codes of conduct and ethics that require a similar commitment to
privacy protection. This has to be balanced against the increasing
requirement from funders that data produced using donor funds be made
publically available.

This tension between complying with confidentiality requirements while
at the same time requiring that microdata be released means that a
demand exists for practical solutions for applying Statistical
Disclosure Control (SDC), also known as microdata anonymization. The
provision of adequate solutions and technical support has the potential
to “unlock” a large number of datasets.

The \sphinxhref{http://ihsn.org}{International Household Survey Network} (IHSN)
and the World Bank have contributed to successful programs that have
generated tools, resources and guidelines for the curation, preservation
and dissemination of microdata and resulted in the documentation of
thousands of surveys by countries and agencies across the world. While
these programs have ensured substantial improvements in the preservation
of data and dissemination of good quality metadata, many agencies are
still reluctant to allow access to the microdata. The reasons are
technical, legal, ethical and political, and sometimes involve a fear of
being criticized for not providing perfect data. When combined with the
tools and guidelines already developed by the IHSN/World Bank for the
curation, preservation and dissemination of microdata, tools and
guidelines for the anonymization of microdata should further reduce or
remove some of these obstacles.

Working with the IHSN, PARIS21 (OECD), Statistics Austria and the Vienna
University of Technology, the World Bank has contributed to the
development of an open source software package for SDC, called
\sphinxstyleemphasis{sdcMicro}. The package was developed for use with the open source \sphinxstyleemphasis{R}
statistical software, available from the Comprehensive R Archive Network
(CRAN) at \sphinxurl{http://cran.us.r-project.org}. The package includes numerous
methods for the assessment and reduction of disclosure risk in
microdata.

Ensuring that a free open source solution is available to agencies was
an important step forward, but not a sufficient one. There is still
limited consolidated and reported knowledge on the impact of disclosure
risk reduction methods on data utility. This limited access to knowledge
combined with a lack of experience in using the tools and methods makes
it difficult for many agencies to implement optimal solutions, i.e.,
solutions that meet their obligations towards both privacy protection
and the release of data useful for policy monitoring and evaluation.
This practice guide attempts to fill this critical gap by:
\begin{enumerate}
\item {} 
consolidating knowledge gained at the World Bank through
experiments conducted during a large-scale evaluation of
anonymization techniques

\item {} 
translating the experience and key results into practical
guidelines

\end{enumerate}

It should be stressed that SDC is only one part of the data release
process, and its application must be considered within the complete data
release framework. The level and methods of SDC depend on the laws of
the country, the sensitivity of the data and the access policy (i.e.,
who will gain access) considered for release. Agencies that are
currently releasing data are already using many of the methods described
in this guide and applying appropriate access polices to their data
before release. The primary objective of this guide is to provide a
primer to those new to the process who are looking for guidance on both
theory and practical implementation. This guide is not intended to
prescribe or advocate for changes in methods that specific data
producers are already using and which they have designed to fit and
comply with their existing data release policies.

The guide seeks to provide practical steps to those agencies that want
to unlock access to their data in a safe way and ensure that the data
remain fit for purpose.


\section{Building a knowledge base}
\label{\detokenize{intro:building-a-knowledge-base}}
The release of data is important, as it allows researchers and
policymakers to replicate officially published results, generate new
insights into issues, avoid duplication of surveys and provide greater
returns to the investment in the survey process.

Both the production of reports, with aggregate tables of indicators and
statistics, and the release of microdata result in privacy challenges to
the producer. In the past, for many agencies, the only requirement was
to release a report and some key indicators. The recent movement around
Open Data, Open Government and transparency means that agencies are
under greater pressure to release their microdata to allow broader use
of data collected through public and donor funds. This guide focuses on
the methods and processes for the release of microdata.

Releasing data in a safe way is required to protect the integrity of the
statistical system, by ensuring agencies honor their commitment to
respondents to protect their identity. Agencies do not widely share, in
substantial detail, their knowledge and experience using SDC and the
processes for creating safe data with other agencies. This makes it
difficult for agencies new to the process to implement solutions. To
fill this experience and knowledge gap, we evaluated the use of a broad
suite of SDC methods on a range of survey microdata covering important
development topics related to health, labor, education, poverty and
inequality. The data we used were all previously treated to make them
safe for release. Given that their producers had already treated these
data, it was not possible, nor was it our goal, to pass any judgment on
the safety of these data, many of which are in the public domain The
focus was rather on measuring the effects that various methods would
have on the risk-utility trade-off for microdata produced to measure
common development indicators. We used the experience from this
large-scale experimentation to inform our discussion of the processes
and methods in this guide.

\sphinxstylestrong{Important:} At no point was any attempt made to re-identify, through
matching or any other method, any respondents in the surveys we used in
building our knowledge base. All risk assessments were based on
frequencies and probabilities.


\section{Using this guide}
\label{\detokenize{intro:using-this-guide}}
The methods discussed in this guide originate from a large body of
literature on SDC. The processes underlying many of the methods are the
subject of extensive academic research and many, if not all, of them are
used extensively by agencies experienced in preparing microdata for
release.

Where possible, for each method and topic, we provide elaborate
examples, references to the original or seminal work describing the
methods and algorithms in detail and recommended readings. This, when
combined with the discussion of the method and practical considerations
in this guide, should allow the reader to understand the methods and
their strengths and weaknesses. It should also provide enough detail for
readers to use an existing software solution to implement the methods or
program the methods in statistical software of their choice.

For the examples in this guide, we use the open source and free package
for SDC called \sphinxstyleemphasis{sdcMicro} as well as the statistical software \sphinxstyleemphasis{R}.
\sphinxstyleemphasis{sdcMicro} is an add-on package to the statistical software \sphinxstyleemphasis{R}. The
package was developed and is maintained by Matthias Templ, Alexander
Kowarik and Bernhard Meindl. {\color{red}\bfseries{}{}`}See and the GitHub of the developers. The
GitHub repository can also be used to submit bugs found in the
package. %
\begin{footnote}[2]\sphinxAtStartFootnote
\sphinxurl{http://cran.r-project.org/web/packages/sdcMicro/index.html}\sphinxurl{https://github.com/alexkowa/sdcMicro}
%
\end{footnote} The statistical software \sphinxstyleemphasis{R} and the \sphinxstyleemphasis{sdcMicro}
package, as well as any other packages needed for the SDC process, are
freely available from the Comprehensive R Archive Network (CRAN) mirrors
(\sphinxurl{http://cran.r-project.org/}). The software is available for Linux,
Windows and Macintosh operating systems. We chose to use \sphinxstyleemphasis{R} and
\sphinxstyleemphasis{sdcMicro} because it is freely available, accommodates all main data
formats and is easy to adapt by the user. The World Bank, through the
IHSN, has also provided funding towards the development of the
\sphinxstyleemphasis{sdcMicro} package to ensure it meets the requirements of the agencies
we support.

This guide does not provide a review of all other available packages for
implementing the SDC process. Our concern is more with providing
practical insight into the application of the methods. We would,
however, like to highlight one particular other software package that is
commonly used by agencies: \(\mu\)-ARGUS %
\begin{footnote}[3]\sphinxAtStartFootnote
\(\mu\)-ARGUS is available at: \sphinxurl{http://neon.vb.cbs.nl/casc/mu.htm}. The
software was recently ported to open source.
%
\end{footnote}. \(\mu\)-ARGUS is
developed by Statistics Netherlands. \sphinxstyleemphasis{sdcMicro} and \(\mu\)-ARGUS are both
widely used in statistics offices in the European Union and implement
many of the same methods.

The user needs some knowledge of \sphinxstyleemphasis{R} to use \sphinxstyleemphasis{sdcMicro}. It is beyond the
scope of this guide to teach the use of \sphinxstyleemphasis{R}, but we do provide
throughout the guide code examples on how to implement the necessary
routines in \sphinxstyleemphasis{R}. %
\begin{footnote}[4]\sphinxAtStartFootnote
There are many free resources for learning \sphinxstyleemphasis{R} available on the web.
One place to start would be the CRAN \sphinxstyleemphasis{R} Project page:
\sphinxurl{http://cran.r-project.org/other-docs.html}
%
\end{footnote} We also present a number of case
studies that include the code for the anonymization of a number of demo
datasets using \sphinxstyleemphasis{R}. Through these case studies, we demonstrate a number
of approaches to the anonymization process in \sphinxstyleemphasis{R}. {\color{red}\bfseries{}{[}\#foot17{]}\_}


\section{Outline of this guide}
\label{\detokenize{intro:outline-of-this-guide}}
This guide is divided into the following main sections:
\begin{enumerate}
\item {} 
Section 2 is a primer on SDC.

\item {} 
Section 3 gives an introduction to different release types for
microdata.

\item {} 
Sections 4 to 6 cover SDC methods, risk and utility measurement.
Here the goal is to provide knowledge that allows the reader to
independently apply and execute the SDC process. This section is
enriched with real examples as well as code snippets from the
\sphinxstyleemphasis{sdcMicro} package. The interested reader can also find more
information in the references and recommended readings at the end
of each section.

\item {} 
Section 7 gives an overview of issues encountered when carrying
out anonymization with the \sphinxstyleemphasis{sdcMicro} package in \sphinxstyleemphasis{R}, which exceed
basic \sphinxstyleemphasis{R} knowledge. This section also includes tips and solutions
to some of the common issues and problems that might be
encountered when applying SDC methods in \sphinxstyleemphasis{R} with \sphinxstyleemphasis{sdcMicro.}

\item {} 
Section 8 provides a step-by-step guide to disclosure control,
which draws upon the knowledge presented in the previous sections.

\item {} 
Section 9 presents a number of detailed case studies that
demonstrate the use of the methods, their implementation in
\sphinxstyleemphasis{sdcMicro} and the process that should be followed to reach the
optimal risk-utility solution.

\end{enumerate}


\chapter{Glossary and list of acronyms}
\label{\detokenize{glossary_acr::doc}}\label{\detokenize{glossary_acr:glossary-and-list-of-acronyms}}

\section{List of Acronyms}
\label{\detokenize{glossary_acr:list-of-acronyms}}

\begin{savenotes}\sphinxatlongtablestart\begin{longtable}{|l|l|}
\hline

\endfirsthead

\multicolumn{2}{c}%
{\makebox[0pt]{\sphinxtablecontinued{\tablename\ \thetable{} -- continued from previous page}}}\\
\hline

\endhead

\hline
\multicolumn{2}{r}{\makebox[0pt][r]{\sphinxtablecontinued{Continued on next page}}}\\
\endfoot

\endlastfoot

AFR
&
Sub Saharan Africa
\\
\hline
COICOP
&
Classification of Individual
Consumption by Purpose
\\
\hline
CRAN
&
Comprehensive R Archive Network
\\
\hline
CTBIL
&
Contingency Table-Based
Information Loss
\\
\hline
DHS
&
Demographic and Health Surveys
\\
\hline
DIS
&
Data Intrusion Simulation
\\
\hline
EAP
&
East Asia and the Pacific
\\
\hline
ECA
&
Europe and Central Asia
\\
\hline
EU
&
European Union
\\
\hline
GIS
&
Geographical Information System
\\
\hline
GPS
&
Global Positioning System
\\
\hline
GUI
&
Graphical User Interface
\\
\hline
HIV/AIDS
&
Human Immunodeficiency
Virus/Acquired Immune Deficiency
Syndrome
\\
\hline
I2D2
&
International Income Distribution
Database
\\
\hline
IHSN
&
International Household Survey
Network
\\
\hline
LAC
&
Latin America and the Caribbean
\\
\hline
LSMS
&
Living Standards Measurement
Survey
\\
\hline
MDAV
&
Maximum Distance Average Vector
\\
\hline
MDG
&
Millennium Development Goal
\\
\hline
MENA
&
Middle East and North America
\\
\hline
MICS
&
Multiple Indicator Cluster Survey
\\
\hline
MME
&
Mean Monthly Expenditures
\\
\hline
MMI
&
Mean Monthly Income
\\
\hline
MSU
&
Minimal Sample Uniques
\\
\hline
NSI
&
National Statistical Institute
\\
\hline
NSO
&
National Statistical Office
\\
\hline
OECD
&
Organization for Economic
Cooperation and Development
\\
\hline
PARIS21
&
Partnership in Statistics for
Development in the 21$^{\text{st}}$
century
\\
\hline
PRAM
&
Post Randomization Method
\\
\hline
PC
&
Principal Component
\\
\hline
PUF
&
Public Use File
\\
\hline
SA
&
South Asia
\\
\hline
SDC
&
Statistical Disclosure Control
\\
\hline
SSE
&
Sum of Squared Errors
\\
\hline
SHIP
&
Survey-based Harmonized
Indicators Program
\\
\hline
SUDA
&
Special Uniques Detection
Algorithm
\\
\hline
SUF
&
Scientific Use File
\\
\hline
UNICEF
&
United Nations Children’s Fund
\\
\hline
\end{longtable}\sphinxatlongtableend\end{savenotes}


\section{Glossary}
\label{\detokenize{glossary_acr:glossary}}

\begin{savenotes}\sphinxatlongtablestart\begin{longtable}{|l|l|}
\hline

\endfirsthead

\multicolumn{2}{c}%
{\makebox[0pt]{\sphinxtablecontinued{\tablename\ \thetable{} -- continued from previous page}}}\\
\hline

\endhead

\hline
\multicolumn{2}{r}{\makebox[0pt][r]{\sphinxtablecontinued{Continued on next page}}}\\
\endfoot

\endlastfoot

Administrative data
&
Data collected for administrative
purposes by government agencies.
Typically, administrative data
require specific SDC methods.
\\
\hline
Anonymization
&
Use of techniques that convert
confidential data into anonymized
data/ removal or masking of
identifying information from
datasets.
\\
\hline
Attribute disclosure
&
Attribute disclosure occurs if an
intruder is able to determine new
characteristics of an individual
or organization based on the
information available in the
released data.
\\
\hline
Categorical variable
&
A variable that takes values over
a finite set, e.g., gender. Also
called factor in \sphinxstyleemphasis{R}.
\\
\hline
Confidentiality
&
Data confidentiality is a
property of data, usually
resulting from legislative
measures, which prevents it from
unauthorized
disclosure.$^{\text{2}}$
\\
\hline
Confidential data
&
Data that will allow
identification of an individual
or organization, either directly
or indirectly. \sphinxhref{l}{Australian Bureau
of Statistics,  {[}7{]}\_}
\\
\hline
Continuous variable
&
A variable with which numerical
and arithmetic operations can be
performed, e.g., income.
\\
\hline
Data protection
&
Data protection refers to the set
of privacy-motivated laws,
policies and procedures that aim
to minimize intrusion into
respondents’ privacy caused by
the collection, storage and
dissemination of personal data.
\sphinxhref{l}{OECD,  {[}8{]}\_}
\\
\hline
Deterministic methods
&
Anonymization methods that follow
a certain algorithm and produce
the same results if applied
repeatedly to the same data with
the same set of parameters.
\\
\hline
Direct identifier
&
A variable that reveals directly
and unambiguously the identity of
a respondent, e.g., names, social
identity numbers.
\\
\hline
Disclosure
&
Disclosure occurs when a person
or an organization recognizes or
learns something that they did
not already know about another
person or organization through
released data. %
\begin{footnote}[1]\sphinxAtStartFootnote
\sphinxurl{http://www.nss.gov.au/nss/home.nsf/pages/Confidentiality+-+Glossary}
%
\end{footnote}
See also Identity disclosure,
Attribute disclosure and
Inferential disclosure.
\\
\hline
Disclosure risk
&
A disclosure risk occurs if an
unacceptably narrow estimation of
a respondent’s confidential
information is possible or if
exact disclosure is possible with
a high level of
confidence. %
\begin{footnote}[2]\sphinxAtStartFootnote
\sphinxurl{http://stats.oecd.org/glossary}
%
\end{footnote}
Disclosure risk also refers to
the probability that successful
disclosure could occur.
\\
\hline
End user
&
The user of the released
microdata file after
anonymization. Who is the end
user depends on the release type.
\\
\hline
Factor variable
&
Factor variables are one way to
classify categorical variables in
\sphinxstyleemphasis{R}.
\\
\hline
Hierarchical structure
&
Data is made up of collections of
records that are interconnected
through links, e.g., individuals
belonging to groups/households or
employees belonging to companies.
\\
\hline
Identifier
&
An identifier is a variable/
information that can be used to
establish identity of an
individual or organization.
Identifiers can lead to direct or
indirect identification.
\\
\hline
Identity disclosure
&
Identity disclosure occurs if an
intruder associates a known
individual or organization with a
released data record.
\\
\hline
Indirect identification
&
Indirect identification occurs
when the identity of an
individual or organization is
disclosed, not using direct
\sphinxhref{http://www.nss.gov.au/nss/home.nsf/pages/Confidentiality+-+Glossary\#4}{identifiers},
but through a combination of
unique characteristics in key
variables. %
\begin{footnote}[9]\sphinxAtStartFootnote
Australian Bureau of Statistics,
\sphinxurl{http://www.nss.gov.au/nss/home.nsf/pages/Confidentiality+-+Glossary}
%
\end{footnote}
\\
\hline
Inferential disclosure
&
Inferential disclosure occurs if
an intruder is able to determine
the value of some characteristic
of an individual or organization
more accurately with the released
data than otherwise would have
been possible.
\\
\hline
Information loss
&
Information loss refers to the
reduction of the information
content in the released data
relative to the information
content in the raw data.
Information loss is often
measured with respect to common
analytical measures, such as
regressions and indicators. See
also Utility.
\\
\hline
Interval
&
A set of numbers between two
designated endpoints that may or
may not be included. Brackets
(e.g., {[}0, 1{]}) denote a closed
interval, which includes the
endpoints 0 and 1. Parentheses
(e.g., (0, 1) denote an open
interval, which does not include
the endpoints.
\\
\hline
Intruder
&
A user who misuses released data
by trying to disclose information
about an individual or
organization, using a set of
characteristics known to the
user.
\\
\hline
\sphinxstyleemphasis{k}-anonymity
&
The risk measure
\(k\)-anonymity is based on
the principle that the number of
individuals in a sample sharing
the same combination of values
(key) of categorical key
variables should be higher than a
specified
threshold\(\text{\ k}\).
\\
\hline
Key
&
A combination or pattern of key
variables/quasi-identifiers.
\\
\hline
Key variables
&
A set of variables that, in
combination, can be linked to
external information to
re-identify respondents in the
released dataset. Key variables
are also called
“quasi-identifiers” or “implicit
identifiers”.
\\
\hline
Microaggregation
&
Anonymization method that is
based on replacing values for a
certain variable with a common
value for a group of records. The
grouping of records is based on a
proximity measure of variables of
interest. The groups of records
are also used to calculate the
replacement value.
\\
\hline
Microdata
&
A set of records containing
information on individual
respondents or on
economic entities. Such records
may contain responses to a survey
questionnaire or administrative
forms.
\\
\hline
Noise addition
&
Anonymization method based on
adding or multiplying a
stochastic or randomized number
to the original values to protect
data from exact matching with
external files. Noise addition is
typically applied to continuous
variables.
\\
\hline
Non-perturbative methods
&
Anonymization methods that reduce
the detail in the data or
suppress certain values (masking)
without distorting the data
structure.
\\
\hline
Observation
&
A set of data derived from an
object/unit of experiment, e.g.,
an individual (in
individual-level data), a
household (in household-level
data) or a company (in company
data). Observations are also
called “records”.
\\
\hline
Original data
&
The data before SDC/anonymization
methods were applied. Also called
“raw data” or “untreated data”.
\\
\hline
Outlier
&
An unusual value that is
correctly reported but is not
typical of the rest of the
population. Outliers can also be
observations with an unusual
combination of values for
variables, such as 20-year-old
widow. On their own age, 20 and
widow are not unusual values, but
their combination may
be. %
\begin{footnote}[10]\sphinxAtStartFootnote
Australian Bureau of Statistics,
\sphinxurl{http://www.nss.gov.au/nss/home.nsf/pages/Confidentiality+-+Glossary}
%
\end{footnote}
\\
\hline
Perturbative methods
&
Anonymization methods that alter
values slightly to limit
disclosure risk by creating
uncertainty around the true
values, while retaining as much
content and structure as
possible, e.g. microaggregation
and noise addition.
\\
\hline
Population unique
&
The only record in the population
with a particular set of
characteristics, such that the
individual or organization can be
distinguished from other units in
the population based on that set
of characteristics.
\\
\hline
Post Randomization Method (PRAM)
&
Anonymization method for
microdata in which the scores of
a categorical variable are
altered according to certain
probabilities. It is thus
intentional misclassification
with known misclassification
probabilities. OECD, %
\begin{footnote}[11]\sphinxAtStartFootnote
\sphinxurl{http://stats.oecd.org/glossary}
%
\end{footnote}
\\
\hline
Probabilistic methods
&
Anonymization methods that depend
on a probability mechanism or a
random number-generating
mechanism. Every time a
probabilistic method is used, a
different outcome is generated.
\\
\hline
Privacy
&
Privacy is a concept that applies
to data subjects while
confidentiality applies to
data. The concept is defined as
follows: “It is the status
accorded to data which has been
agreed upon between the person or
organization furnishing the data
and the organization receiving it
and which describes the degree of
protection which will be
provided.” %
\begin{footnote}[5]\sphinxAtStartFootnote
\sphinxurl{http://stats.oecd.org/glossary}
%
\end{footnote}
\\
\hline
Public Use File (PUF)
&
Type of release of microdata
file, which is freely available
to any user, for example on the
internet.
\\
\hline
Quasi-identifiers
&
A set of variables that, in
combination, can be linked to
external information to
re-identify respondents in the
released dataset.
Quasi-identifiers are also called
“key variables” or “implicit
identifiers”.
\\
\hline
Raw data
&
The data before SDC/anonymization
methods were applied. Also called
“original data” or “untreated
data”.
\\
\hline
Recoding
&
Anonymization method for
microdata in which groups of
existing categories/values are
replaced with new values, e.g.
the values ‘protestant’, and
‘catholic’ are replaced with
‘Christian’. Recoding reduces the
detail in the data. Recoding of
continuous variables leads to a
transformation from continuous to
categorical, e.g. creating income
bands.
\\
\hline
Record
&
A set of data derived from an
object/unit of experiment, e.g.,
an individual (in
individual-level data), a
household (in household-level
data) or a company (in company
data). Records are also called
“observations”.
\\
\hline
Regression
&
A statistical process of
measuring the relation between
the mean value of one variable
and corresponding values of other
variables.
\\
\hline
Re-identification risk
&
See Disclosure risk
\\
\hline
Release
&
Dissemination \textendash{} the release to
users of information obtained
through a statistical activity.
\sphinxfootnotemark[5]
\\
\hline
Respondents
&
Individuals or units of
observation whose
information/responses to a survey
make up the data file.
\\
\hline
Sample unique
&
The only record in the sample
with a particular set of
characteristics, such that the
individual or organization can be
distinguished from other units in
the sample based on that set of
characteristics.
\\
\hline
Scientific Use File (SUF)
&
Type of release of microdata
file, which is only available to
selected researchers under
contract. Also known as “licensed
file”, “microdata under contract”
or “research file”.
\\
\hline
\sphinxstyleemphasis{sdcMicro}
&
An \sphinxstyleemphasis{R} based package authored by
Templ, M., Kowarik, A. and
Meindl, B. with tools for the
anonymization of microdata, i.e.
for the creation of public- and
scientific-use files.
\\
\hline
\sphinxstyleemphasis{sdcMicroGUI}
&
A GUI for the \sphinxstyleemphasis{R} based
\sphinxstyleemphasis{sdcMicro} package, which allows
users to use the \sphinxstyleemphasis{sdcMicro} tools
without \sphinxstyleemphasis{R} knowledge.
\\
\hline
Sensitive variables
&
Sensitive or confidential
variables are those whose values
must not be discovered for any
respondent in the dataset. The
determination of sensitive
variables is often subject to
legal and ethical concerns.
\\
\hline
Statistical Disclosure Control
(SDC)
&
Statistical Disclosure Control
techniques can be defined as the
set of methods to reduce the risk
of disclosing information on
individuals, businesses or other
organizations. Such methods are
only related to the dissemination
step and are usually based on
restricting the amount of or
modifying the data
released. %
\begin{footnote}[12]\sphinxAtStartFootnote
OECD, \sphinxurl{http://stats.oecd.org/glossary}
%
\end{footnote}
\\
\hline
Suppression
&
Data suppression involves not
releasing information that is
considered unsafe because it
fails confidentiality rules being
applied. Sometimes this is done
is by replacing values signifying
individual attributes with
missing values. In the context of
this guide, usually to achieve a
desired level of \sphinxstyleemphasis{k}- anonymity.
\\
\hline
Threshold
&
An established level, value,
margin or point at which values
that fall above or below it will
deem the data safe or unsafe. If
unsafe, further action will need
to be taken to reduce the risk of
identification.
\\
\hline
Utility
&
Data utility describes the value
of data as an analytical
resource, comprising analytical
completeness and analytical
validity.
\\
\hline
Untreated data
&
The data before SDC/anonymization
methods were applied. Also called
“raw data” or “original data”.
\\
\hline
Variable
&
Any characteristic, number or
quantity that can be measured or
counted for each unit of
observation.
\\
\hline
\end{longtable}\sphinxatlongtableend\end{savenotes}


\chapter{Statistical Disclosure Control (SDC): An Introduction}
\label{\detokenize{SDC_intro:statistical-disclosure-control-sdc-an-introduction}}\label{\detokenize{SDC_intro::doc}}

\section{Need for SDC}
\label{\detokenize{SDC_intro:need-for-sdc}}
A large part of the data collected by statistical agencies cannot be
published directly due to privacy and confidentiality concerns. These
concerns are both of legal and ethical nature. SDC seeks to treat and
alter the data so that the data can be published or released without
revealing the confidential information it contains, while, at the same
time, limit information loss due to the anonymization of the data. In
this guide, we discuss only disclosure control for
microdata. %
\begin{footnote}[1]\sphinxAtStartFootnote
There is another strand of literature on the anonymization of tabular
data, see e.g., Hundepool et al. (2012).
%
\end{footnote} Microdata are datasets that provide
information on a set of variables for each individual respondent.
Respondents can be natural persons, but also legal entities such as
companies.

The aim of anonymizing microdata is to transform the datasets to achieve
an “acceptable level” of disclosure risk. The level of acceptability of
disclosure risk and the need for anonymization are usually at the
discretion of the data producer and guided by legislation. These are
formulated in the dissemination policies and programs of the data
providers and based on considerations including “{[}…{]} the costs and
expertise involved; questions of data quality, potential misuse and
misunderstanding of data by users; legal and ethical matters; and
maintaining the trust and support of respondents” (Dupriez and Boyko,
2010). There is a moral, ethical and legal obligation for the data
producers to ensure that data provided by the respondents are used only
for statistical purposes.

In some cases, the dissemination of microdata is a legal obligation,
but, in most cases, the legislation will formulate restrictions. Thus, a
country’s legislative framework will shape its microdata dissemination
policy. It is crucial for data producers to “ensure there is a sound
legal and ethical base (as well as the technical and methodological
tools) for protecting confidentiality. This legal and ethical base
requires a balanced assessment between the public good of
confidentiality protection on the one hand, and the public benefits of
research on the other. A decision on whether or not to provide access
might depend on the merits of specific research proposals and the
credibility of the researcher, and there should be some allowance for
this in the legal arrangements.” (Dupriez and Boyko, 2010).

“Data access arrangements should respect the legal rights and legitimate
interests of all stakeholders in the public research enterprise. Access
to, and use of, certain research data will necessarily be limited by
various types of legal requirements, which may include restrictions for
reasons of:
\begin{itemize}
\item {} 
National security: data pertaining to intelligence, military
activities, or political decision making may be classified and
therefore subject to restricted access.

\item {} 
Privacy and confidentiality: data on human subjects and other
personal data are subject to restricted access under national laws
and policies to protect confidentiality and privacy. However,
anonymization or confidentiality procedures that ensure a
satisfactory level of confidentiality should be considered by
custodians of such data to preserve as much data utility as possible
for researchers.

\item {} 
Trade secrets and intellectual property rights: data on, or from,
businesses or other parties that contain confidential information may
not be accessible for research. (…)” (Dupriez and Boyko, 2010).

Box 1, extracted from Dupriez and Boyko (2010), provides several
examples of statistical legislation on microdata release.

\end{itemize}

Box 1: Examples of statistical legislation on microdata release

Besides the legal and ethical concerns and codes of conducts of agencies
producing statistics, SDC is important because it guarantees data
quality and response rates in future surveys. If respondents feel that
data producers are not protecting their privacy, they might not be
willing to participate in future surveys. “{[}…{]} one incident,
particularly if it receives strong media attention, could have a
significant impact on respondent cooperation and therefore on the
quality of official statistics” (ibid.). At the same time, if data users
are unable to gain enough utility from the data due to excessive or
inappropriate SDC protection, or are unable to access the data, then the
large investment in producing the data will be lost.


\section{The risk-utility trade-off in the SDC process}
\label{\detokenize{SDC_intro:the-risk-utility-trade-off-in-the-sdc-process}}
SDC is characterized by the trade-off between risk of disclosure and
utility of the data for end users. The risk\textendash{}utility scale extends
between two extremes; (i) no data is released (zero risk of disclosure)
and thus users gain no utility from the data, to (ii) data is released
without any treatment, and thus with maximum risk of disclosure, but
also maximum utility to the user (i.e., no information loss). The goal
of a well-implemented SDC process is to find the optimal point where
utility for end users is maximized at an acceptable level of risk.
Figure 2.1 illustrates this trade-off. The triangle corresponds to the
raw data. The raw data have no information loss, but generally have a
disclosure risk higher than the acceptable level. The other extreme is
the square, which corresponds to no data release. In that case there is
no disclosure risk, but also no utility from the data for the users. The
points in-between correspond to different choices of SDC methods and/or
parameters for these methods applied to different variables. The SDC
process looks for the SDC methods and the parameters for those methods
and applies these in a way that reduces the risk sufficiently, while
minimizing the information loss.

\noindent\sphinxincludegraphics[width=6.03524in,height=4.3072in]{{image1}.png}

Figure 2.1: Risk-utility trade-off

SDC cannot achieve total risk elimination, but can reduce the risk to an
acceptable level. Any application of SDC methods will suppress or alter
values in the data and as such decrease the utility (i.e., result in
information loss) when compared to the original data. A common thread
that will be emphasized throughout this guide will be that the process
of SDC should prioritize the goal of protecting respondents, while at
the same time keeping the data users in mind to limit information loss.
In general, the lower the disclosure risk, the higher the information
loss and the lower the data utility for end-users.

In practice, choosing SDC methods is partially trial and error: after
applying methods, disclosure risk and data utility are re-measured and
compared to the results of other choices of methods and parameters. If
the result is satisfactory, the data can be released. We will see that
often the first attempt will not be the optimal one. The risk may not be
sufficiently reduced or the information loss may be too high and the
process has to be repeated with different methods or parameters until a
satisfactory solution is found. Disclosure risk, data utility and
information loss in the SDC context and how to measure them are
discussed in subsequent chapters of this guide.

Again, it must be stressed that the level of SDC and methods applied
depend to a large extent on the entire data release framework. For
example, a key consideration is to whom and under what conditions the
data are to be released (see also Chapter 3). If data are to be released
as public use data, then the level of SDC applied will necessarily need
to be higher than in the cases where data are released under license
conditions to trusted users after careful vetting. With careful
preparation, data may be released under both public and licensed
versions. We discuss how this might be achieved later in the guide.


\section{Release Types}
\label{\detokenize{SDC_intro:release-types}}
This section discusses data release. Rather than rewriting work that has
already been conducted through the World Bank and its partners at the
IHSN, this section extracts from an excellent guide published by Dupriez
and Boyko (2010).

The trade-off between risk and utility in the anonymization process
depends greatly on who the users are %
\begin{footnote}[2]\sphinxAtStartFootnote
See Section 5 in Dupriez and Boyko (2010) as to who the users of
microdata are and to whom microdata should be made available.
%
\end{footnote} and under
what conditions a microdata file is released. Generally, three types of
data release methods are practiced and apply to different target groups.
\begin{itemize}
\item {} 
\sphinxstylestrong{Public Use File (PUF)}: the data “are available to anyone agreeing
to respect a core set of easy-to-meet conditions. Such conditions
relate to what cannot be done with the data (e.g. the data cannot be
sold), upon gaining access to the data. In some cases PUFs are
disseminated with no conditions; often being made available on-line,
{[}e.g. on the website of the statistical agency{]}. These data are made
easily accessible because the risk of identifying individual
respondents is considered minimal. Minimising the risk of disclosure
involves eliminating all content that can identify respondents
directly—for instance, names, addresses and telephone numbers. In
addition this requires purging relevant indirect identifiers from the
microdata file. These vary across survey designs, but
commonly-suppressed indirect identifiers include geographical
information below the sub-national level at which the sample is
representative. Occasionally, certain records may be suppressed also
from PUFs, as might variables characterised by extremely skewed
distribution or outliers. However, in lieu of deleting entire records
or variables from microdata files, alternative SDC methods can
minimise the risk of disclosure while maximizing information content.
Such methods include top-and-bottom coding, local suppression or
using data perturbation techniques {[}(see Chapter 5 for an overview of
anonymization methods){]}. PUFs are typically generated from census
data files using a sub-set {[}or sample{]} of records rather than the
entire file and {[}from sample surveys, such as{]} household surveys.”
(Dupriez and Boyko, 2010).

\item {} 
\sphinxstylestrong{Scientific Use File (SUF)} (also known as a licensed file,
microdata under contract or research file): the “dissemination is
restricted to users who have received authorization to access them
after submitting a documented application and signing an agreement
governing the data’s use. While typically licensed files are also
anonymised to ensure the risk of identifying individuals is minimised
when used in isolation, they may still {[}potentially{]} contain
identifiable data if linked with other data files. Direct identifiers
such as respondents’ names must be removed from a licensed dataset.
The data files may, however, still contain indirect variables that
could identify respondents by matching them to other data files such
as voter lists, land registers or school records. When disseminating
licensed files, the recommendation is to establish and sign an
agreement between the data producer and external bona fide users \textendash{}
trustworthy users with legitimate need to access the data. Such an
agreement should govern access and use of such microdata
files %
\begin{footnote}[3]\sphinxAtStartFootnote
Appendix B provides an example of a blanket agreement.
%
\end{footnote}. Sometimes, licensing agreements are only
entered into with users affiliated to an appropriate sponsoring
institution. i.e., research centers, universities or development
partners. It is further recommended that, before entering into a data
access and use agreement, the data producer asks potential users to
complete an application form to demonstrate the need to use a
licensed file (instead of the PUF version, if available) for a stated
statistical or research purpose” (Dupriez and Boyko, 2010). This also
allows the data producer to learn which characteristics of the data
are important for the users, which is valuable information for
optimizing future anonymization processes.

\item {} 
\sphinxstylestrong{Microdata available in a controlled research data center} (also
known as data enclave): “Some files may be offered to users under
strict conditions in a data enclave. This is a facility {[}(often on
the premises of the data provider){]} equipped with computers not
linked to the internet or an external network and from which no
information can be downloaded via USB ports, CD-DVD or other drives.
Data enclaves contain data that are particularly sensitive or allow
direct or easy identification of respondents. Examples include
complete population census datasets, enterprise surveys and certain
health related datasets containing highly-confidential information.
Users interested in accessing a data enclave will not necessarily
have access to the full dataset \textendash{} only to the particular data subset
they require. They will be asked to complete an application form
demonstrating a legitimate need to access these data to fulfill a
stated statistical or research purpose {[}…{]} The outputs generated must
be scrutinised by way of a full disclosure review before release.
Operating a data enclave may be expensive \textendash{} it requires special
premises and computer equipment. It also demands staff with the
skills and time to review outputs before their removal from the data
enclave in order to ensure there is no risk of disclosure. Such staff
must be familiar with data analysis and be able to review the request
process and manage file servers. Because of the substantial operating
costs and technical skills required, some statistical agencies or
other official data producers opt to collaborate with academic
institutions or research centres to establish and manage data
enclaves.”

\end{itemize}

There are other data access possibilities besides these, such as
teaching files, files for other specific purposes, remote execution or
remote access. Obviously, the required level of protection depends on
the type of release; a PUF file must be protected to a much larger
extent than a SUF file, which in turn has to be protected more than a
file which is only available in an on-site facility. Section 8.3 gives
more guidance on the choice of the release type and its implications for
the anonymization process. The same microdata set can be released in
different ways for different users, e.g., as SUF and teaching file.
Section 8.3 discusses the particular issues of multiple releases of one
dataset.

The first step for any agency that wants to release data would be
formulation of clear data dissemination policies for the release of
microdata. We will see later that deciding on the level of anonymization
needed will depend partly on knowing under what conditions the data will
be released. Access policies and conditions provide the framework for
the whole release process.

The following sections further specify the conditions under which
microdata should be provided under different release types.


\section{Conditions for PUFs}
\label{\detokenize{SDC_intro:conditions-for-pufs}}
“Generally, data regarded as public are open to anyone with access to an
{[}National Statistical Office{]} (NSO) website. It is, however, normally
good practice to include statements defining suitable uses for and
precautions to be adopted in using the data. While these may not be
legally binding, they serve to sensitise the user. Prohibitions such as
attempts to link the data to other sources can be part of the ‘use
statement’ to which the user must agree, on-line, before the data can be
downloaded. {[}…{]} Dissemination of microdata files necessarily involves
the application of rules or principles. {[}Box 2{]} below {[}taken from
Dupriez and Boyko (2010){]} shows basic principles normally applying to
PUFs.” (Dupriez and Boyko, 2010).

Box 2: Conditions for accessing and using PUFs


\section{Conditions for SUFs}
\label{\detokenize{SDC_intro:conditions-for-sufs}}
“For {[}SUFs{]}, terms and conditions must include the basic common
principles plus some additional ones applying to the researcher’s
organisation. There are two options: firstly, data are provided to a
researcher or a team for a specific purpose; secondly, data are provided
to an organization under a blanket agreement for internal use, e.g., to
an international body or research agency. In both cases, the
researcher’s organisation must be identified, as must suitable
representatives to sign the licence” (Dupriez and Boyko, 2010).

\sphinxstyleemphasis{Access to a researcher or research team for a specific purpose}

“If data are provided for an individual research project, the research
team must be identified. This is covered by requiring interested users
to complete a formal request to access the data (a model of such a
request form is provided in Appendix 1 {[}in Dupriez and Boyko (2010){]}).
The conditions to obtain the data (see example in Box 3) will specify
that the files will not be shared outside the organisation and that data
will be stored securely. To the possible extent, the intended use of the
data \textendash{} including a list of expected outputs and the organisation’s
dissemination policy \textendash{} must be identified. Access to licensed datasets
is only granted when there is a legally-registered sponsoring agency,
e.g., government ministry, university, research centre or national or
international organization” (Dupriez and Boyko, 2010).

Box 3: Conditions for accessing and using SUFs

\sphinxstyleemphasis{Blanket agreement to an organization}

“In the case of a blanket agreement, where it is agreed the data can be
used widely but securely within the receiving organisation, the licence
should ensure compliance, with a named individual formally assuming
responsibility for this. Each additional user must be made aware of the
terms and conditions that apply to data files: this can be achieved by
having to sign an affidavit. Where such an agreement exists, with
security in place, it is not necessary for users to destroy the data
after use” (Dupriez and Boyko, 2010). Appendix B provides an example of
the formulation of such an agreement.


\section{Conditions for microdata available in a controlled research data center}
\label{\detokenize{SDC_intro:conditions-for-microdata-available-in-a-controlled-research-data-center}}
Access to microdata in research data centers is “used for particularly
sensitive data or for more detailed data for which sufficient
anonymisation to release them outside the NSO premises is not possible.
These can be referred to also as data laboratories or research data
centres. A {[}research data centre{]} may be located at the NSO headquarters
or in major centres such as universities close to the research
community. They are used to give researchers access to complete data
files but without the risk of releasing confidential data. In a typical
{[}research data centre{]}, NSO staff supervise access and use of the data;
the computers must not be able to communicate outside the {[}research data
centre{]}; and the results obtained by the researchers must be screened
for confidentiality by an NSO analyst before taken outside. A model of a
data enclave access policy is provided in Appendix 2 {[}in Dupriez and
Boyko (2010){]}, and a model of a data enclave access request form is in
Appendix 3 {[}in Dupriez and Boyko (2010){]}” (Dupriez and Boyko, 2010).

Research data centers “have the advantage of providing access to
detailed microdata but the disadvantage of requiring researchers to work
at a different location. And they are expensive to set up and operate.
It is, however, quite likely that many countries have used on-site
researchers as a way of providing access to microdata. These researchers
are sworn in under the statistics’ acts in the same way as regular NSO
employees. This approach tends to favour researchers who live near NSO
headquarters.” (Dupriez and Boyko, 2010)

\begin{sphinxadmonition}{note}{Recommended Reading Material on Release Types}

Dupriez, O., \& Boyko, E. (2010). \sphinxstyleemphasis{Dissemination of Microdata Files;
Principles, Procedures and Practices.} International Household Survey
Network (IHSN).
\end{sphinxadmonition}


\chapter{Measuring Risk}
\label{\detokenize{measure_risk::doc}}\label{\detokenize{measure_risk:measuring-risk}}

\section{Types of disclosure}
\label{\detokenize{measure_risk:types-of-disclosure}}
Measuring disclosure risk is an important part of the SDC process: risk
measures are used to judge whether a data file is safe enough for
release. Before measuring disclosure risk, we have to define what type
of disclosure is relevant for the data at hand. The literature commonly
defines three types of disclosure; we take these directly from \sphinxhref{bibliography.html\#Lamb93}{Lamb93}
(see also \sphinxhref{bibliography.html\#HDFG12}{HDFG12}).
\begin{itemize}
\item {} 
\sphinxstylestrong{Identity disclosure}, which occurs if the intruder associates a
known individual with a released data record. For example, the
intruder links a released data record with external information, or
identifies a respondent with extreme data values. In this case, an
intruder can exploit a small subset of variables to make the linkage,
and once the linkage is successful, the intruder has access to all
other information in the released data related to the specific
respondent.

\item {} 
\sphinxstylestrong{Attribute disclosure}, which occurs if the intruder is able to
determine some new characteristics of an individual based on the
information available in the released data. Attribute disclosure
occurs if a respondent is correctly re-identified and the dataset
contains variables containing information that was previously unknown
to the intruder. Attribute disclosure can also occur without identity
disclosure. For example, if a hospital publishes data showing that
all female patients aged 56 to 60 have cancer, an intruder then knows
the medical condition of any female patient aged 56 to 60 in the
dataset without having to identify the specific individual.

\item {} 
\sphinxstylestrong{Inferential disclosure}, which occurs if the intruder is able to
determine the value of some characteristic of an individual more
accurately with the released data than would otherwise have been
possible. For example, with a highly predictive regression model, an
intruder may be able to infer a respondent’s sensitive income
information using attributes recorded in the data, leading to
inferential disclosure.

\end{itemize}

SDC methods for microdata are intended to prevent identity and attribute
disclosure. Inferential disclosure is generally not addressed in SDC in
the microdata setting, since microdata is distributed precisely so that
researchers can make statistical inference and understand relationships
between variables. In that sense, inference cannot be likened to
disclosure. Also, inferences are designed to predict aggregate, not
individual, behavior, and are therefore usually poor predictors of
individual data values.


\section{Classification of variables}
\label{\detokenize{measure_risk:classification-of-variables}}
For the purpose of the SDC process, we use the classifications of
variables described in the following paragraphs (see \hyperref[\detokenize{measure_risk:fig24}]{Fig.\@ \ref{\detokenize{measure_risk:fig24}}}
for an overview). The initial classification of variables into identifying and
non-identifying variables depends on the way the variables can be used
by intruders for re-identification (\sphinxhref{bibliography.html\#HDFG12}{HDFG12}; \sphinxhref{bibliography.html\#TeMK14}{TeMK14}):
\begin{itemize}
\item {} 
\sphinxstylestrong{Identifying variables:} these contain information that can lead to
the identification of respondents and can be further categorized as:
\begin{itemize}
\item {} 
\sphinxstylestrong{Direct identifiers} reveal directly and unambiguously the
identity of the respondent. Examples are names, passport numbers,
social identity numbers and addresses. Direct identifiers should
be removed from the dataset prior to release. Removal of direct
identifiers is a straightforward process and always the first step
in producing a safe microdata set for release. Removal of direct
identifiers, however, is often not sufficient.

\item {} 
\sphinxstylestrong{Quasi-identifiers} (\sphinxstylestrong{or key variables}) contain information
that, when combined with other quasi-identifiers in the dataset,
can lead to re-identification of respondents. This is especially
the case when they can be used to match the information with other
external information or data. Examples of quasi-identifiers are
race, birth date, sex and ZIP/postal codes, which might be easily
combined or linked to publically available external information
and make identification possible. The combinations of values of
several quasi-identifiers are called keys (see also {\hyperref[\detokenize{measure_risk:levels-of-risk}]{\sphinxcrossref{Levels of Risk}}}).
The values of quasi-identifiers themselves often do not lead to
identification (e.g. male/female), but a combination of several
values of quasi-identifier can render a record unique (e.g. male,
14 years, married) and hence identifiable. It is not generally
advisable to simply remove quasi-identifiers from the data to
solve the problem. In many cases, they will be important variables
for any sensible analysis. In practice, any variable in the
dataset could potentially be used as a quasi-identifier. SDC
addresses this by identifying variables as quasi-identifiers and
anonymizing them while still maintaining the information in the
dataset for release.

\end{itemize}

\item {} 
\sphinxstylestrong{Non-identifying} variables are variables that cannot be used for
re-identification of respondents. This could be because these
variables are not contained in any other data files or other external
sources and are not observable to an intruder. Non-identifying
variables are nevertheless important in the SDC process, since they
may contain confidential/sensitive information, which may prove
damaging should disclosure occur as a result of identity disclosure
based on identifying variables.

These classifications of variables depend partially on the
availability of external datasets that might contain information
that, when combined with the current data, could lead to disclosure.
The identification and classification of variables as
quasi-identifiers depends, amongst others, on the availability of
information in external datasets. An important step in the SDC
process is to define a list of possible disclosure scenarios based on
how the quasi-identifiers might be combined with each other and
information in external datasets and then treating the data to
prevent disclosure. We discuss disclosure scenarios in more detail in
{\hyperref[\detokenize{measure_risk:disclosure-scenarios}]{\sphinxcrossref{Disclosure scenarios}}}.

\end{itemize}

For the SDC process, it is also useful to further classify the
quasi-identifiers into \sphinxstylestrong{categorical}, \sphinxstylestrong{continuous} and
\sphinxstylestrong{semi-continuous} variables. This classification is important for
determining the appropriate SDC methods for that variable, as well as
the validity of risk measures.
\begin{itemize}
\item {} 
\sphinxstylestrong{Categorical} variables take values over a finite set, and any
arithmetic operations using them are generally not meaningful or not
allowed. Examples of categorical variables are gender, region and
education level.

\item {} 
\sphinxstylestrong{Continuous} variables can take on an infinite number of values in
a given set. Examples are income, body height and size of land plot.
Continuous variables can be transformed into categorical variables by
constructing intervals (such as income bands). %
\begin{footnote}[1]\sphinxAtStartFootnote
Recoding a continuous variable is sometimes useful in cases where the
data contains only a few continuous variables. We will see in Section
3 that many methods used for risk calculation depend on whether the
variables are categorical. We will also see that it is easier for the
measurement of risk if the data contains only categorical or only
continuous variables.
%
\end{footnote}

\item {} 
\sphinxstylestrong{Semi-continuous} variables are continuous variables that take on
values that are limited to a finite set. An example is age measured
in years, which could take on values in the set \{0, 1, …, 100\}. The
finite nature of the values for these variables means that they can
be treated as categorical variables for the purpose of
SDC. %
\begin{footnote}[2]\sphinxAtStartFootnote
This is discussed in greater detail in the following sections. In
cases where the number of possible values is large, recoding the
variable, or parts of the set it takes values on, to obtain fewer
distinct values is recommended.
%
\end{footnote}

\end{itemize}

Apart from these classifications of variables, the SDC process further
classifies variables according to their sensitivity or confidentiality.
Both quasi-identifiers and non-identifying variables can be classified
as \sphinxstylestrong{sensitive} (or confidential) or \sphinxstylestrong{non-sensitive} (or
non-confidential). This distinction is not important for direct
identifiers, since direct identifiers are removed from the released
data.
\begin{itemize}
\item {} 
\sphinxstylestrong{Sensitive} variables contain confidential information that should
not be disclosed without suitable treatment using SDC methods to
reduce disclosure risk. Examples are income, religion, political
affiliation and variables concerning health. Whether a variable is
sensitive depends on the context and country: a certain variable can
be considered sensitive in one country and non-sensitive in another.

\item {} 
\sphinxstylestrong{Non-sensitive} variables contain non-confidential information on
the respondent, such as place of residence or rural/urban residence.
The classification of a variable as non-sensitive, however, does not
mean that it does not need to be considered in the SDC process.
Non-sensitive variables may still serve as quasi-identifiers when
combined with other variables or other external data.

\end{itemize}

\begin{figure}[htbp]
\centering
\capstart

\noindent\sphinxincludegraphics{{image24}.png}
\caption{Classification of variables}\label{\detokenize{measure_risk:fig24}}\label{\detokenize{measure_risk:id15}}\end{figure}


\section{Disclosure scenarios}
\label{\detokenize{measure_risk:disclosure-scenarios}}
Evaluation of disclosure risk is carried out with reference to the
available data sources in the environment where the dataset is to be
released. In this setting, disclosure risk is the possibility of
correctly re-identifying an individual in the released microdata file by
matching their data to an external file based on a set of
quasi-identifiers. The risk assessment is done by identifying so-called
disclosure or intrusion scenarios. A disclosure scenario describes the
information potentially available to the intruder (e.g., census data,
electoral rolls, population registers or data collected by private
firms) to identify respondents and the ways such information can be
combined with the microdata set to be released and used for
re-identification of records in the dataset. Typically, these external
datasets include direct identifiers. In that case, the re-identification
of records in the released dataset leads to identity and, possibly,
attribute disclosure. The main outcome of the evaluation of disclosure
scenarios is the identification of a set of quasi-identifiers (i.e., key
variables) that need to be treated during the SDC process (see Elliot et
al., 2010).

An example of a disclosure scenario could be the spontaneous recognition
of a respondent by a researcher. For instance, while going through the
data, the researcher recognizes a person with an unusual combination of
the variables age and marital status. Of course, this can only happen if
the person is well-known or is known to the researcher. Another example
of a disclosure scenario for a publicly available file would be if
variables in the data could be linked to a publically available
electoral register. An intruder might try matching the entire dataset
with individuals in the register. However, this might be difficult and
take specialized expertise, or software, and other conditions have to be
fulfilled. Examples are that the point in time the datasets were
collected should approximately match and the content of the variables
should be (nearly) identical. If these conditions are not fulfilled,
exact matching is much less likely.

\begin{sphinxadmonition}{note}{Note:}
Not all external data is
necessarily in the public domain. Also privately owned datasets or
datasets which are not released should be taken into consideration for
determining the suitable disclosure scenario.
\end{sphinxadmonition}

\begin{sphinxadmonition}{note}{Info-box - Disclosure scenarios and different release types}

A dataset can have more than one disclosure scenario. Disclosure scenarios
also differ depending on the data access type that the data will be released
under; for example, Public Use Files (PUF) or Scientific Use Files (SUF, also
known as licensed) or in a data enclave. The required level of protection,
the potential avenues of disclosure as well as the availability of other external
data sources differ according to the access type under which the data will be
released. For example, the user of a Scientific Use File (SUF) might be
contractually restricted by an agreement as to what they are allowed to do
with the data, whereas a Public Use File (PUF) might be freely available on
the internet under a much looser set of conditions. PUFs will in general require
more protection than SUFs and SUFs will require more protection than those files
only released in an data enclave. Disclosure scenarios should be developed with
all of this in mind.
\end{sphinxadmonition}

The evaluation of disclosure risk is based on the quasi-identifiers,
which are identified in the analysis of disclosure risk scenarios. The
disclosure risk directly depends on the inclusion or exclusion of
variables in the set of quasi-identifiers chosen. This step in the SDC
process (making the choice of quasi-identifiers) should therefore be
approached with great thought and care. We will see later, as we discuss
the steps in the SDC process in more detail, that the first step for any
agency is to undertake an exercise in which an inventory is compiled of
all datasets available in the country. Both datasets released by the
national statistical office and from other sources are considered and
their availability to intruders as well as the variables included in
these datasets is analyzed. It is this information that will serve as a
key metric when deciding which variables to choose as potential
identifiers, as well as dictate the level of SDC and methods needed.


\section{Levels of risk}
\label{\detokenize{measure_risk:levels-of-risk}}
With microdata from surveys and censuses, we often have to be concerned
about disclosure at the individual or unit level, i.e., identifying
individual respondents. Individual respondents are generally natural
persons, but can also be units, such as companies, schools, health
facilities, etc. Microdata files often have a hierarchical structure
where individual units belong to groups, e.g., people belong to
households. The most common hierarchical structure in microdata is the
household structure in household survey data. Therefore, in this guide,
we sometimes call disclosure risk for data with a hierarchical structure
“household risk”. The concepts, however, apply equally to establishment
data and other data with hierarchical structures, such as school data
with pupils and teachers or company data with employees.

We will see that this hierarchical structure is important to take into
consideration when measuring disclosure risk. For hierarchical data,
information collected at the higher hierarchical level (e.g., household
level) would be the same for all individuals in the group belonging to
that higher hierarchical level (e.g., household). %
\begin{footnote}[3]\sphinxAtStartFootnote
Besides variables collected at the higher hierarchical level, also
variables collected at the lower level but with no (or little)
variation within the groups formed by the hierarchical structure
should be treated as higher level variables. An example could be
mother tongue, where most households are monolingual, but the
variable is collected at the individual level.
%
\end{footnote}
Some typical examples of variables that would have the same values for
all members of the same higher hierarchical unit are, in the case of
households, those relating to housing and household income. These
variables differ from survey to survey and from country to
country. %
\begin{footnote}[4]\sphinxAtStartFootnote
Religion, for example, can be shared by all household members in
some countries, whereas in other countries this variable is measured
at the individual level and mixed-religion households exist.
%
\end{footnote} This hierarchical structure creates a
further level of disclosure risk for two reasons:
\begin{enumerate}
\item {} 
if one individual in the household is re-identified, the household structure allows for
re-identification of the other household members in the same household,

\item {} 
values of variables for other household members that are common for
all household members can be used for re-identification of another
individual of the same household. This is discussed in more detail in
{\hyperref[\detokenize{measure_risk:household-risk}]{\sphinxcrossref{Household Risk}}}.

\end{enumerate}

In the following pages, we first discuss risk measures used to evaluate
disclosure risk in the absence of a hierarchical structure. This
includes risk measures that seek to aggregate the individual risk for
all individuals in the microdata file; the objective is to quantify a
global disclosure risk measure for the file. We then discuss how risk
measures change when taking the hierarchical structure of the data into
account.

We will also discuss how risk measures differ for categorical and
continuous key variables. For categorical variables, we will use the
concept of uniqueness of combinations of values of quasi-identifiers
(so-called “keys”) used to identify individuals at risk. The concept of
uniqueness, however, is not useful for continuous variables, since it is
likely that all or many individuals will have unique values for that
variable, by definition of a continuous variable. Risk measures for
categorical variables are generally a priori measures, i.e., they can be
evaluated before applying anonymization methods since they are based on
the principle of uniqueness. Risk measures for continuous variables are
a posteriori measures; they are based on comparing the microdata before
and after anonymization and are, for example, based on the proximity of
observations between the original and the treated (anonymized) datasets.

Files that are limited to only categorical or only continuous key
variables are easiest for risk measurement. We will see in later
sections that, in cases where both types of variables are present,
recoding of continuous variables into categories is one approach to use
to simplify the SDC process, but we will also see that from a utility
perspective this may not be desirable. An example might be the use of
income quintiles instead of the actual income variables. We will see
that measuring the risk of disclosure based on the categorical and
continuous variables separately is generally not a valid approach.

The risk measures discussed in the next section are based on several
assumptions. In general, these measures rely on quite restrictive
assumptions and will often lead to conservative risk estimates. These
conservative risk measures may overstate the risk as they assume a
worst-case scenario. Two assumptions should, however, be fulfilled for
the risk measures to be valid and meaningful; the microdata should be a
sample of a larger population (no census) and the sampling weights
should be available. \sphinxhref{anon\_methods.html\#Specialcase:censusdata}{Special case: census data}
briefly discusses how to deal with
census data.


\section{Individual risk}
\label{\detokenize{measure_risk:individual-risk}}

\subsection{Categorical key variables and frequency counts}
\label{\detokenize{measure_risk:categorical-key-variables-and-frequency-counts}}
The main focus of risk measurement for categorical quasi-identifiers is
on identity disclosure. Measuring disclosure risk is based on the
evaluation of the probability of correct re-identification of
individuals in the released data. We use measures based on the actual
microdata to be released. In general, the rarer a combination of values
of the quasi-identifiers (i.e., key) of an observation in the sample,
the higher the risk of identity disclosure. An intruder that tries to
match an individual who has a relatively rare key within the sample data
with an external dataset in which the same key exists will have a higher
probability of finding a correct match than when a larger number of
individuals share the same key. This can be illustrated with the
following example that is illustrated in \hyperref[\detokenize{measure_risk:tab41}]{Table \ref{\detokenize{measure_risk:tab41}}}.

\hyperref[\detokenize{measure_risk:tab41}]{Table \ref{\detokenize{measure_risk:tab41}}} shows values for 10 respondents for the quasi-identifiers
“residence”, “gender”, “education level” and “labor status”. In the
data, we find seven unique combinations of values of quasi-identifiers
(i.e., patterns or keys) of the four quasi-identifiers. Examples of keys
are \{‘urban’, ‘female’, ‘secondary incomplete’, ‘employed’\} and
\{‘urban’, ‘female’, ‘primary incomplete’, ‘non-LF’\}. Let \(f_{k}\)
be the sample frequency of the \sphinxstyleemphasis{k}$^{\text{th}}$ key, i.e., the number of
individuals in the sample with values of the quasi-identifiers that
coincide with the \sphinxstyleemphasis{k}$^{\text{th}}$ key. This would be 2 for the key
\{urban, female, secondary incomplete, employed\}, since this key is
shared by individuals 1 and 2 and 1 for the key (‘urban’, ‘female’,
‘primary incomplete’, ‘non-LF’), which is unique to individual 3. By
definition, \(f_{k}\) is the same for each record sharing a
particular key.

The fewer the individuals with whom an individual shares his or her
combination of quasi-identifiers, the more likely the individual is to
be correctly matched in another dataset that contains these
quasi-identifiers. Even when direct identifiers are removed from the
dataset, that individual has a higher disclosure risk than others,
assuming that their sample weights are the same. \hyperref[\detokenize{measure_risk:tab41}]{Table \ref{\detokenize{measure_risk:tab41}}} reports the
sample frequencies \(f_{k}\) of the keys for all individuals.
Individuals with the same keys have the same sample frequency. If
\(f_{k} = 1\), this individual has a unique combination of values of
quasi-identifiers and is called “sample unique”. The dataset in \hyperref[\detokenize{measure_risk:tab41}]{Table \ref{\detokenize{measure_risk:tab41}}}
contains four sample uniques. Risk measures are based on this sample
frequency.


\begin{savenotes}\sphinxattablestart
\centering
\sphinxcapstartof{table}
\sphinxcaption{Example dataset showing sample frequencies,
                     population frequencies and individual disclosure risk}\label{\detokenize{measure_risk:tab41}}\label{\detokenize{measure_risk:id16}}
\sphinxaftercaption
\begin{tabulary}{\linewidth}[t]{|T|T|T|T|T|T|T|T|T|}
\hline
\sphinxstyletheadfamily 
No
&\sphinxstyletheadfamily 
Residence
&\sphinxstyletheadfamily 
Gender
&\sphinxstyletheadfamily 
Education level
&\sphinxstyletheadfamily 
Labor status
&\sphinxstyletheadfamily 
Weight
&\sphinxstyletheadfamily 
f\_k
&\sphinxstyletheadfamily 
F\_k
&\sphinxstyletheadfamily 
risk
\\
\hline
1
&
Urban
&
Female
&
Secondary incomplete
&
Employed
&
180
&
2
&
360
&
0.0054
\\
\hline
2
&
Urban
&
Female
&
Secondary incomplete
&
Employed
&
180
&
2
&
360
&
0.0054
\\
\hline
3
&
Urban
&
Female
&
Primary incomplete
&
Non-LF
&
215
&
1
&
215
&
0.0251
\\
\hline
4
&
Urban
&
Male
&
Secondary complete
&
Employed
&
76
&
2
&
152
&
0.0126
\\
\hline
5
&
Rural
&
Female
&
Secondary complete
&
Unemployed
&
186
&
1
&
186
&
0.0282
\\
\hline
6
&
Urban
&
Male
&
Secondary complete
&
Employed
&
76
&
2
&
152
&
0.0126
\\
\hline
7
&
Urban
&
Female
&
Primary complete
&
Non-LF
&
180
&
1
&
180
&
0.0290
\\
\hline
8
&
Urban
&
Male
&
Post-secondary
&
Unemployed
&
215
&
1
&
215
&
0.0251
\\
\hline
9
&
Urban
&
Female
&
Secondary incomplete
&
Non-LF
&
186
&
2
&
262
&
0.0074
\\
\hline
10
&
Urban
&
Female
&
Secondary incomplete
&
Non-LF
&
76
&
2
&
262
&
0.0074
\\
\hline
\end{tabulary}
\par
\sphinxattableend\end{savenotes}

In \hyperref[\detokenize{measure_risk:code41}]{Listing \ref{\detokenize{measure_risk:code41}}}, we show how to use the \sphinxstyleemphasis{sdcMicro} package to create a
list of sample frequencies \(f_{k}\) for each record in a dataset.
This is done by using the \sphinxstyleemphasis{sdcMicro} function freq(). A value of 2 for
an observation means that in the sample, there is one more individual
with exactly the same combination of values for the selected key
variables. In \hyperref[\detokenize{measure_risk:code41}]{Listing \ref{\detokenize{measure_risk:code41}}}, the function freq() is applied to
“sdcInitial”, which is an \sphinxstyleemphasis{sdcMicro} object. Footnote %
\begin{footnote}[5]\sphinxAtStartFootnote
The code examples in this guide are based on \sphinxstyleemphasis{sdcMicro} objects. An
\sphinxstyleemphasis{sdcMicro} object contains, amongst others, the data and identifies
all the specified key variables. The code below creates a data.frame
with the data from \hyperref[\detokenize{measure_risk:tab41}]{Table \ref{\detokenize{measure_risk:tab41}}} and the \sphinxstyleemphasis{sdcMicro} objects “sdcInitial”
used in most examples in this section.

\sphinxSetupCodeBlockInFootnote
\fvset{hllines={, ,}}%
\begin{sphinxVerbatim}[commandchars=\\\{\}]
\PYG{k+kn}{library}\PYG{p}{(}sdcMicro\PYG{p}{)}

\PYG{c+c1}{\PYGZsh{} Set up dataset}

data \PYG{o}{\PYGZlt{}\PYGZhy{}} \PYG{k+kp}{as.data.frame}\PYG{p}{(}\PYG{k+kp}{cbind}\PYG{p}{(}\PYG{k+kp}{as.factor}\PYG{p}{(}\PYG{k+kt}{c}\PYG{p}{(}\PYG{l+s}{\PYGZsq{}}\PYG{l+s}{Urban\PYGZsq{}}\PYG{p}{,}
\PYG{l+s}{\PYGZsq{}}\PYG{l+s}{Urban\PYGZsq{}}\PYG{p}{,} \PYG{l+s}{\PYGZsq{}}\PYG{l+s}{Urban\PYGZsq{}}\PYG{p}{,} \PYG{l+s}{\PYGZsq{}}\PYG{l+s}{Urban\PYGZsq{}}\PYG{p}{,} \PYG{l+s}{\PYGZsq{}}\PYG{l+s}{Rural\PYGZsq{}}\PYG{p}{,} \PYG{l+s}{\PYGZsq{}}\PYG{l+s}{Urban\PYGZsq{}}\PYG{p}{,} \PYG{l+s}{\PYGZsq{}}\PYG{l+s}{Urban\PYGZsq{}}\PYG{p}{,} \PYG{l+s}{\PYGZsq{}}\PYG{l+s}{Urban\PYGZsq{}}\PYG{p}{,}
\PYG{l+s}{\PYGZsq{}}\PYG{l+s}{Urban\PYGZsq{}}\PYG{p}{,} \PYG{l+s}{\PYGZsq{}}\PYG{l+s}{Urban\PYGZsq{}}\PYG{p}{)}\PYG{p}{)}\PYG{p}{,}
\PYG{k+kp}{as.factor}\PYG{p}{(}\PYG{k+kt}{c}\PYG{p}{(}\PYG{l+s}{\PYGZsq{}}\PYG{l+s}{Female\PYGZsq{}}\PYG{p}{,} \PYG{l+s}{\PYGZsq{}}\PYG{l+s}{Female\PYGZsq{}}\PYG{p}{,} \PYG{l+s}{\PYGZsq{}}\PYG{l+s}{Female\PYGZsq{}}\PYG{p}{,} \PYG{l+s}{\PYGZsq{}}\PYG{l+s}{Male\PYGZsq{}}\PYG{p}{,}
\PYG{l+s}{\PYGZsq{}}\PYG{l+s}{Female\PYGZsq{}}\PYG{p}{,} \PYG{l+s}{\PYGZsq{}}\PYG{l+s}{Male\PYGZsq{}}\PYG{p}{,} \PYG{l+s}{\PYGZsq{}}\PYG{l+s}{Female\PYGZsq{}}\PYG{p}{,} \PYG{l+s}{\PYGZsq{}}\PYG{l+s}{Male\PYGZsq{}}\PYG{p}{,} \PYG{l+s}{\PYGZsq{}}\PYG{l+s}{Female\PYGZsq{}}\PYG{p}{,} \PYG{l+s}{\PYGZsq{}}\PYG{l+s}{Female\PYGZsq{}}\PYG{p}{)}\PYG{p}{)}\PYG{p}{,}
\PYG{k+kp}{as.factor}\PYG{p}{(}\PYG{k+kt}{c}\PYG{p}{(}\PYG{l+s}{\PYGZsq{}}\PYG{l+s}{Sec in\PYGZsq{}}\PYG{p}{,} \PYG{l+s}{\PYGZsq{}}\PYG{l+s}{Sec in\PYGZsq{}}\PYG{p}{,} \PYG{l+s}{\PYGZsq{}}\PYG{l+s}{Prim in\PYGZsq{}}\PYG{p}{,} \PYG{l+s}{\PYGZsq{}}\PYG{l+s}{Sec com\PYGZsq{}}\PYG{p}{,} \PYG{l+s}{\PYGZsq{}}\PYG{l+s}{Sec com\PYGZsq{}}\PYG{p}{,} \PYG{l+s}{\PYGZsq{}}\PYG{l+s}{Sec com\PYGZsq{}}\PYG{p}{,} \PYG{l+s}{\PYGZsq{}}\PYG{l+s}{Prim com\PYGZsq{}}\PYG{p}{,} \PYG{l+s}{\PYGZsq{}}\PYG{l+s}{Post\PYGZhy{}sec\PYGZsq{}}\PYG{p}{,} \PYG{l+s}{\PYGZsq{}}\PYG{l+s}{Sec in\PYGZsq{}}\PYG{p}{,} \PYG{l+s}{\PYGZsq{}}\PYG{l+s}{Sec in\PYGZsq{}}\PYG{p}{)}\PYG{p}{)}\PYG{p}{,}
\PYG{k+kp}{as.factor}\PYG{p}{(}\PYG{k+kt}{c}\PYG{p}{(}\PYG{l+s}{\PYGZsq{}}\PYG{l+s}{Emp\PYGZsq{}}\PYG{p}{,} \PYG{l+s}{\PYGZsq{}}\PYG{l+s}{Emp\PYGZsq{}}\PYG{p}{,} \PYG{l+s}{\PYGZsq{}}\PYG{l+s}{Non\PYGZhy{}LF\PYGZsq{}}\PYG{p}{,} \PYG{l+s}{\PYGZsq{}}\PYG{l+s}{Emp\PYGZsq{}}\PYG{p}{,} \PYG{l+s}{\PYGZsq{}}\PYG{l+s}{Unemp\PYGZsq{}}\PYG{p}{,} \PYG{l+s}{\PYGZsq{}}\PYG{l+s}{Emp\PYGZsq{}}\PYG{p}{,} \PYG{l+s}{\PYGZsq{}}\PYG{l+s}{Non\PYGZhy{}LF\PYGZsq{}}\PYG{p}{,} \PYG{l+s}{\PYGZsq{}}\PYG{l+s}{Unemp\PYGZsq{}}\PYG{p}{,} \PYG{l+s}{\PYGZsq{}}\PYG{l+s}{Non\PYGZhy{}LF\PYGZsq{}}\PYG{p}{,}\PYG{l+s}{\PYGZsq{}}\PYG{l+s}{Non\PYGZhy{}LF\PYGZsq{}}\PYG{p}{)}\PYG{p}{)}\PYG{p}{,}
\PYG{k+kp}{as.factor}\PYG{p}{(}\PYG{k+kt}{c}\PYG{p}{(}\PYG{l+s}{\PYGZsq{}}\PYG{l+s}{yes\PYGZsq{}}\PYG{p}{,} \PYG{l+s}{\PYGZsq{}}\PYG{l+s}{yes\PYGZsq{}}\PYG{p}{,} \PYG{l+s}{\PYGZsq{}}\PYG{l+s}{yes\PYGZsq{}}\PYG{p}{,} \PYG{l+s}{\PYGZsq{}}\PYG{l+s}{yes\PYGZsq{}}\PYG{p}{,} \PYG{l+s}{\PYGZsq{}}\PYG{l+s}{yes\PYGZsq{}}\PYG{p}{,} \PYG{l+s}{\PYGZsq{}}\PYG{l+s}{no\PYGZsq{}}\PYG{p}{,} \PYG{l+s}{\PYGZsq{}}\PYG{l+s}{no\PYGZsq{}}\PYG{p}{,} \PYG{l+s}{\PYGZsq{}}\PYG{l+s}{yes\PYGZsq{}}\PYG{p}{,} \PYG{l+s}{\PYGZsq{}}\PYG{l+s}{no\PYGZsq{}}\PYG{p}{,} \PYG{l+s}{\PYGZsq{}}\PYG{l+s}{yes\PYGZsq{}}\PYG{p}{)}\PYG{p}{)}\PYG{p}{,}
\PYG{k+kt}{c}\PYG{p}{(}\PYG{l+m}{180}\PYG{p}{,} \PYG{l+m}{180}\PYG{p}{,} \PYG{l+m}{215}\PYG{p}{,} \PYG{l+m}{76}\PYG{p}{,} \PYG{l+m}{186}\PYG{p}{,} \PYG{l+m}{76}\PYG{p}{,} \PYG{l+m}{180}\PYG{p}{,} \PYG{l+m}{215}\PYG{p}{,} \PYG{l+m}{186}\PYG{p}{,} \PYG{l+m}{76}\PYG{p}{)}\PYG{p}{)}\PYG{p}{)}

\PYG{c+c1}{\PYGZsh{} Specify variable names}

\PYG{k+kp}{names}\PYG{p}{(}data\PYG{p}{)} \PYG{o}{\PYGZlt{}\PYGZhy{}} \PYG{k+kt}{c}\PYG{p}{(}\PYG{l+s}{\PYGZsq{}}\PYG{l+s}{Residence\PYGZsq{}}\PYG{p}{,} \PYG{l+s}{\PYGZsq{}}\PYG{l+s}{Gender\PYGZsq{}}\PYG{p}{,} \PYG{l+s}{\PYGZsq{}}\PYG{l+s}{Educ\PYGZsq{}}\PYG{p}{,} \PYG{l+s}{\PYGZsq{}}\PYG{l+s}{Lstat\PYGZsq{}}\PYG{p}{,} \PYG{l+s}{\PYGZsq{}}\PYG{l+s}{Health\PYGZsq{}}\PYG{p}{,} \PYG{l+s}{\PYGZsq{}}\PYG{l+s}{Weights\PYGZsq{}}\PYG{p}{)}

\PYG{c+c1}{\PYGZsh{} Set up sdcMicro object with specified quasi\PYGZhy{}identifiers and weight variable}

sdcInitial \PYG{o}{\PYGZlt{}\PYGZhy{}} createSdcObj\PYG{p}{(}dat \PYG{o}{=} data\PYG{p}{,}
keyVars \PYG{o}{=} \PYG{k+kt}{c}\PYG{p}{(}\PYG{l+s}{\PYGZsq{}}\PYG{l+s}{Residence\PYGZsq{}}\PYG{p}{,} \PYG{l+s}{\PYGZsq{}}\PYG{l+s}{Gender\PYGZsq{}}\PYG{p}{,} \PYG{l+s}{\PYGZsq{}}\PYG{l+s}{Educ\PYGZsq{}}\PYG{p}{,} \PYG{l+s}{\PYGZsq{}}\PYG{l+s}{Lstat\PYGZsq{}}\PYG{p}{)}\PYG{p}{,} weightVar \PYG{o}{=} \PYG{l+s}{\PYGZsq{}}\PYG{l+s}{Weights\PYGZsq{}}\PYG{p}{)}
\end{sphinxVerbatim}
%
\end{footnote}
shows how to initialize the \sphinxstyleemphasis{sdcMicro} object for
this example. For a complete discussion of \sphinxstyleemphasis{sdcMicro} objects as well as
instructions on how to create \sphinxstyleemphasis{sdcMicro} objects, we refer to
\begin{quote}

\sphinxhref{sdcMicro.html\#Objectsofclass*sdcMicroObj*}{Objects of class Objects of class *sdcMicroObj*}.
\sphinxstyleemphasis{sdcMicro} objects are used when doing SDC with \sphinxstyleemphasis{sdcMicro}. The
\end{quote}

function freq() displays the sample frequency for the keys constructed
on a defined set of quasi-identifiers. \hyperref[\detokenize{measure_risk:code41}]{Listing \ref{\detokenize{measure_risk:code41}}} corresponds to the
data in \hyperref[\detokenize{measure_risk:tab41}]{Table \ref{\detokenize{measure_risk:tab41}}}.

\def\sphinxLiteralBlockLabel{\label{\detokenize{measure_risk:code41}}}
\sphinxSetupCaptionForVerbatim{Calculating \(\mathbf{f}_{\mathbf{k}}\) using \sphinxstyleemphasis{sdcMicro}}
\fvset{hllines={, ,}}%
\begin{sphinxVerbatim}[commandchars=\\\{\},numbers=left,firstnumber=1,stepnumber=1]
 \PYG{c+c1}{\PYGZsh{} Frequency of the particular combination of key variables (keys) for each record in the sample}
 freq\PYG{p}{(}sdcInitial\PYG{p}{,} type \PYG{o}{=} \PYG{l+s}{\PYGZsq{}}\PYG{l+s}{fk\PYGZsq{}}\PYG{p}{)}
 \PYG{l+m}{2} \PYG{l+m}{2} \PYG{l+m}{1} \PYG{l+m}{2} \PYG{l+m}{1} \PYG{l+m}{2} \PYG{l+m}{1} \PYG{l+m}{1} \PYG{l+m}{2} \PYG{l+m}{2}
\end{sphinxVerbatim}

For sample data, it is more interesting to look at \(F_{k}\), the
population frequency of a combination of quasi-identifiers (key)
\(k\), which is the number of individuals in the population with the
key that corresponds to key \(k\). The population frequency
is unknown if the microdata is a sample and not a census. Under certain
assumptions, the expected value of the population frequencies can be
computed using the sample design weight \(w_{i}\) (in a simple
sample, this is the inverse of the inclusion probability) for each
individual \(i\)
\begin{equation*}
\begin{split}F_{k} = \sum_{i|key\ of\ individual\ i\ corresponds\ to\ key\ k}^{}w_{i}\end{split}
\end{equation*}
\(F_{k}\) is the sum of the sample weights of all records with the
same key \sphinxstyleemphasis{k}. Hence, like \(f\) is, \(F_{k}\) is the same for
each record with key \sphinxstyleemphasis{k}. The risk of correct re-identification is the
probability that the key is matched to the correct individual in the
population. Since every individual in the sample with key \(k\)
corresponds to \(F_{k}\) individuals in the population, the
probability of correct re-identification is \(1/F_{k}.\ \)This is
the probability of re-identification in the worst-case scenario and can
be interpreted as disclosure risk. Individuals with the same key have
the same frequencies, i.e., the frequency of the key.

If \(F_{k} = 1\), the key \(k\) is both a sample and a
population unique and the disclosure risk would be 1. Population uniques
are an important factor to consider when evaluating risk, and deserve
special attention. \hyperref[\detokenize{measure_risk:tab41}]{Table \ref{\detokenize{measure_risk:tab41}}} also shows \(F_{k}\) for the example
dataset. This is further discussed in the case studies in Chapter 9.

Besides \(f_{k}\), the sample frequency of key
\(k\) (i.e., the number of individuals in the sample with
the combination of quasi-identifiers corresponding to the combination
specified in key \(k\)) and \(F_{k}\), the estimated population
frequency of key \(k\), can be displayed in \sphinxstyleemphasis{sdcMicro}. \hyperref[\detokenize{measure_risk:code42}]{Listing \ref{\detokenize{measure_risk:code42}}}
illustrates how to return lists of length \sphinxstyleemphasis{n} of frequencies for all
individuals. The frequencies are displayed for each individual and not
for each key.

\def\sphinxLiteralBlockLabel{\label{\detokenize{measure_risk:code42}}}
\sphinxSetupCaptionForVerbatim{Calculating the sample and population frequencies using \sphinxstyleemphasis{sdcMicro}}
\fvset{hllines={, ,}}%
\begin{sphinxVerbatim}[commandchars=\\\{\},numbers=left,firstnumber=1,stepnumber=1]
 \PYG{c+c1}{\PYGZsh{} Sample frequency of individual’s key}
 freq\PYG{p}{(}sdcInitial\PYG{p}{,} type \PYG{o}{=} \PYG{l+s}{\PYGZsq{}}\PYG{l+s}{fk\PYGZsq{}}\PYG{p}{)}
 \PYG{l+m}{2} \PYG{l+m}{2} \PYG{l+m}{1} \PYG{l+m}{2} \PYG{l+m}{1} \PYG{l+m}{2} \PYG{l+m}{1} \PYG{l+m}{1} \PYG{l+m}{2} \PYG{l+m}{2}

 \PYG{c+c1}{\PYGZsh{} Population frequency of individual’s key}
 freq\PYG{p}{(}sdcInitial\PYG{p}{,} type \PYG{o}{=} \PYG{l+s}{\PYGZsq{}}\PYG{l+s}{Fk\PYGZsq{}}\PYG{p}{)}
 \PYG{l+m}{360} \PYG{l+m}{360} \PYG{l+m}{215} \PYG{l+m}{152} \PYG{l+m}{186} \PYG{l+m}{152} \PYG{l+m}{180} \PYG{l+m}{215} \PYG{l+m}{262} \PYG{l+m}{262}
\end{sphinxVerbatim}

In practice, this approach leads to conservative risk estimates, as it
does not adequately take the sampling methods into account. In this
case, the estimates of re-identification risk may be estimated too high.
If this overestimated risk is used, the data may be overprotected (i.e.,
information loss will be higher than was necessary) when applying SDC
measures. Instead, a Bayesian approach to risk measurement is
recommended, where the posterior distribution of \(F_{k}\) is used
(see e.g., Hundepool et al., 2012) to estimate an individual risk
measure \(r_{k}\) for each key \(k\).

The risk measure \(r_{k}\) is, as \(f_{k}\) and \(F_{k}\),
the same for all individuals sharing the same pattern of values of key
variables and is referred to as individual risk. The values
\(r_{k}\) can also be interpreted as the probability of disclosure
for the individuals or as the probability for a successful match with
individuals chosen at random from an external data file with the same
values of the key variables. This risk measure is based on certain
assumptions{}` %
\begin{footnote}[6]\sphinxAtStartFootnote
The assumptions for this risk measure are strict and the risk is
estimated in many cases higher than the actual risk. Among other
assumptions, it is assumed that all individuals in the sample are
also included in the external file used by the intruder to match
against. If this is not the case, the risk is much lower; if the
individual in the released file is not contained in the external
file, the probability of a correct match is zero. Other assumptions
are that the files contain no errors and that both sets of data were
collected simultaneously, i.e. they contain the same information.
These assumptions will often not hold generally, but are necessary
for computation of a measure. An example of a violation of the last
assumptions is could occur if datasets are collected at different
points in time and records have changed. This could happen when
people move or change jobs and makes correct matching impossible. The
assumptions assume a worst-case scenario.
%
\end{footnote}, which are strict and may lead to a
relatively conservative risk measure. In \sphinxstyleemphasis{sdcMicro}, the risk measure
\(r_{k}\) is automatically computed when creating an \sphinxstyleemphasis{sdcMicro}
object and saved in the “risk” slot %
\begin{footnote}[7]\sphinxAtStartFootnote
See Section 7.5 for more information on slots and the \sphinxstyleemphasis{sdcMicro}
object structure.
%
\end{footnote}. \hyperref[\detokenize{measure_risk:code43}]{Listing \ref{\detokenize{measure_risk:code43}}}
shows how to retrieve the risk measures using \sphinxstyleemphasis{sdcMicro} for our
example. The risk measures are also presented in \hyperref[\detokenize{measure_risk:tab41}]{Table \ref{\detokenize{measure_risk:tab41}}}.

\def\sphinxLiteralBlockLabel{\label{\detokenize{measure_risk:code43}}}
\sphinxSetupCaptionForVerbatim{The individual risk slot in the \sphinxstyleemphasis{sdcMicro} object}
\fvset{hllines={, ,}}%
\begin{sphinxVerbatim}[commandchars=\\\{\},numbers=left,firstnumber=1,stepnumber=1]
     sdcInitial\PYG{o}{@}risk\PYG{o}{\PYGZdl{}}individual risk fk Fk

     \PYG{p}{[}\PYG{l+m}{1}\PYG{p}{,}\PYG{p}{]} \PYG{l+m}{0.005424520} \PYG{l+m}{2} \PYG{l+m}{360}
     \PYG{p}{[}\PYG{l+m}{2}\PYG{p}{,}\PYG{p}{]} \PYG{l+m}{0.005424520} \PYG{l+m}{2} \PYG{l+m}{360}
     \PYG{p}{[}\PYG{l+m}{3}\PYG{p}{,}\PYG{p}{]} \PYG{l+m}{0.025096439} \PYG{l+m}{1} \PYG{l+m}{215}
     \PYG{p}{[}\PYG{l+m}{4}\PYG{p}{,}\PYG{p}{]} \PYG{l+m}{0.012563425} \PYG{l+m}{2} \PYG{l+m}{152}
     \PYG{p}{[}\PYG{l+m}{5}\PYG{p}{,}\PYG{p}{]} \PYG{l+m}{0.028247279} \PYG{l+m}{1} \PYG{l+m}{186}
     \PYG{p}{[}\PYG{l+m}{6}\PYG{p}{,}\PYG{p}{]} \PYG{l+m}{0.012563425} \PYG{l+m}{2} \PYG{l+m}{152}
     \PYG{p}{[}\PYG{l+m}{7}\PYG{p}{,}\PYG{p}{]} \PYG{l+m}{0.029010932} \PYG{l+m}{1} \PYG{l+m}{180}
     \PYG{p}{[}\PYG{l+m}{8}\PYG{p}{,}\PYG{p}{]} \PYG{l+m}{0.025096439} \PYG{l+m}{1} \PYG{l+m}{215}
     \PYG{p}{[}\PYG{l+m}{9}\PYG{p}{,}\PYG{p}{]} \PYG{l+m}{0.007403834} \PYG{l+m}{2} \PYG{l+m}{262}
     \PYG{p}{[}\PYG{l+m}{10}\PYG{p}{,}\PYG{p}{]} \PYG{l+m}{0.007403834} \PYG{l+m}{2} \PYG{l+m}{262}
\end{sphinxVerbatim}

The main factors influencing the individual risk are the sample
frequencies \(f_{k}\) and the sampling design weights \(w_{i}\).
If an individual is at relatively high risk of disclosure, in our
example this would be individuals 3, 5, 7 and 8 in \hyperref[\detokenize{measure_risk:tab41}]{Table \ref{\detokenize{measure_risk:tab41}}} and
\hyperref[\detokenize{measure_risk:code43}]{Listing \ref{\detokenize{measure_risk:code43}}}, the probability that a potential intruder correctly matches these
individuals with an external data file is high \sphinxstylestrong{relative to the other
individuals in the released data.} In our example, the reason for the
high risk is the fact that these individuals are sample uniques
(\(f_{k} = 1\)). This risk is the worst-case scenario risk and does
not imply that the person will be re-identified with certainty with this
probability. For instance, if an individual included in the microdata is
not included in the external data file, the probability for a correct
match is zero. Nevertheless, the risk measure computed based on the
frequencies will be positive.


\subsection{\protect\(k\protect\)-anonymity}
\label{\detokenize{measure_risk:anonymity}}
The risk measure \(k\)-anonymity is based on the principle that, in a safe
dataset, the number of individuals sharing the same combination of
values (keys) of categorical quasi-identifiers should be higher than a
specified threshold \(k\). \(k\)-anonymity is a risk
measure based on the microdata to be released, since it only takes the
sample into account. An individual violates \(k\)-anonymity if the
sample frequency count \(f_{k}\) for the key \(k\) is smaller
than the specified threshold \(k\). For example, if an
individual has the same combination of quasi-identifiers as two other
individuals in the sample, these individuals satisfy 3-anonymity but
violate 4-anonymity. In the dataset in \hyperref[\detokenize{measure_risk:tab41}]{Table \ref{\detokenize{measure_risk:tab41}}}, six individuals
satisfy 2-anonymity and four violate 2-anonymity. The individuals that
violate 2-anonymity are sample uniques. The risk measure is the number
of observations that violates k-anonymity for a certain value of \sphinxstyleemphasis{k},
which is
\begin{equation*}
\begin{split}\sum_{i}^{}{I(f_{k} < k)},\end{split}
\end{equation*}
where \(I\) is the indicator function and \(i\) refers to the
\(i\)$^{\text{th}}$ record. This is simply a count of the number of
individuals with a sample frequency of their key lower than \(k\).
The count is higher for larger \(k\), since if a record satisfies
\(k\)-anonimity, it also satisfies (\(k + 1\))-anonimity. The
risk measure \(k\)-anonymity does not consider the sample weights,
but it is important to consider the sample weights when determining the
required level of \(k\)-anonymity. If the sample weights are large,
one individual in the dataset represents more individuals in the target
population, the probability of a correct match is smaller, and hence the
required threshold can be lower. Large sample weights go together with
smaller datasets. In a smaller dataset, the probability to find another
record with the same key is smaller than in a larger dataset. This
probability is related to the number of records in the population with a
particular key through the sample weights.

In \sphinxstyleemphasis{sdcMicro} we can display the number of observations violating a
given \(k\)-anonymity threshold. In \hyperref[\detokenize{measure_risk:code44}]{Listing \ref{\detokenize{measure_risk:code44}}}, we use \sphinxstyleemphasis{sdcMicro}
to calculate the number of violators for the thresholds \(k = 2\)
and \(k = 3\). Both the absolute number of violators and the
relative number as percentage of the number of individuals in the sample
are given. In the example, four observations violate 2-anonimity and all
10 observations violate 3-anonymity.

\def\sphinxLiteralBlockLabel{\label{\detokenize{measure_risk:code44}}}
\sphinxSetupCaptionForVerbatim{Using the print() function to display observations violating k-anonymity}
\fvset{hllines={, ,}}%
\begin{sphinxVerbatim}[commandchars=\\\{\},numbers=left,firstnumber=1,stepnumber=1]
 \PYG{k+kp}{print}\PYG{p}{(}sdcInitial\PYG{p}{,} \PYG{l+s}{\PYGZsq{}}\PYG{l+s}{kAnon\PYGZsq{}}\PYG{p}{)}

 Number of observations violating
 \PYG{o}{\PYGZhy{}}  \PYG{l+m}{2}\PYG{o}{\PYGZhy{}}anonymity\PYG{o}{:} \PYG{l+m}{4}
 \PYG{o}{\PYGZhy{}}  \PYG{l+m}{3}\PYG{o}{\PYGZhy{}}anonymity\PYG{o}{:} \PYG{l+m}{10}
 \PYG{o}{\PYGZhy{}}\PYG{o}{\PYGZhy{}}\PYG{o}{\PYGZhy{}}\PYG{o}{\PYGZhy{}}\PYG{o}{\PYGZhy{}}\PYG{o}{\PYGZhy{}}\PYG{o}{\PYGZhy{}}\PYG{o}{\PYGZhy{}}\PYG{o}{\PYGZhy{}}\PYG{o}{\PYGZhy{}}\PYG{o}{\PYGZhy{}}\PYG{o}{\PYGZhy{}}\PYG{o}{\PYGZhy{}}\PYG{o}{\PYGZhy{}}\PYG{o}{\PYGZhy{}}\PYG{o}{\PYGZhy{}}\PYG{o}{\PYGZhy{}}\PYG{o}{\PYGZhy{}}\PYG{o}{\PYGZhy{}}\PYG{o}{\PYGZhy{}}\PYG{o}{\PYGZhy{}}\PYG{o}{\PYGZhy{}}\PYG{o}{\PYGZhy{}}\PYG{o}{\PYGZhy{}}\PYG{o}{\PYGZhy{}}\PYG{o}{\PYGZhy{}}
 Percentage of observations violating
 \PYG{o}{\PYGZhy{}}  \PYG{l+m}{2}\PYG{o}{\PYGZhy{}}anonymity\PYG{o}{:} \PYG{l+m}{40} \PYG{o}{\PYGZpc{}}
\PYG{o}{ \PYGZhy{}  3\PYGZhy{}anonymity: 100 \PYGZpc{}}
\end{sphinxVerbatim}

For other levels of \(k\)-anonymity, it is possible to compute the
number of violating individuals by using the sample frequency counts in
the \sphinxstyleemphasis{sdcMicro} object. The number of violators is the number of
individuals with sample frequency counts smaller than the specified
threshold \(k\). In \hyperref[\detokenize{measure_risk:code45}]{Listing \ref{\detokenize{measure_risk:code45}}}, we show an example of how to
calculate any threshold for \(k\) using the already-stored risk
measures available after setting up an \sphinxstyleemphasis{sdcMicro} object in \sphinxstyleemphasis{R}.
\(k\) can be replaced with any required threshold. The choice of the
required threshold that all individuals in the microdata file should
satisfy depends on many factors and is discussed further in Section 4.3
on local suppression. In many institutions, typically required
thresholds for \(k\)-anonymity are 3 and 5.

\def\sphinxLiteralBlockLabel{\label{\detokenize{measure_risk:code45}}}
\sphinxSetupCaptionForVerbatim{Computing k-anonymity violations for other values of k}
\fvset{hllines={, ,}}%
\begin{sphinxVerbatim}[commandchars=\\\{\},numbers=left,firstnumber=1,stepnumber=1]
 \PYG{k+kp}{sum}\PYG{p}{(}sdcInitial\PYG{o}{@}risk\PYG{o}{\PYGZdl{}}individual\PYG{p}{[}\PYG{p}{,}\PYG{l+m}{2}\PYG{p}{]} \PYG{o}{\PYGZlt{}} k\PYG{p}{)}
\end{sphinxVerbatim}

It is important to note that missing values (‘NA’s in
\sphinxstyleemphasis{R} %
\begin{footnote}[8]\sphinxAtStartFootnote
In \sphinxstyleemphasis{sdcMicro} it is important to use the standard missing value code
NA instead of other codes, such as 9999 or strings. In Chapter 6, we
further discuss how to set other missing value codes to NA in \sphinxstyleemphasis{R}.
This is necessary to ensure that the methods in \sphinxstyleemphasis{sdcMicro} function
properly. When missing values have codes other than NA, the missing
value codes are interpreted as a distinct factor level in the case of
categorical variables.
%
\end{footnote}) are treated as if they were any other value.
Two individuals with keys \{‘Male’, NA, ‘Employed’\} and \{‘Male’,
‘Secondary complete’, ‘Employed’\} share the same key, and similarly,
\{‘Male’, NA, ‘Employed’\} and \{‘Male’, ‘Secondary incomplete’,
‘Employed’\} also share the same key. Therefore, the missing value in the
first key is first interpreted as ‘Secondary complete’, and then as
‘Secondary incomplete’. This is illustrated in \hyperref[\detokenize{measure_risk:tab42}]{Table \ref{\detokenize{measure_risk:tab42}}}.

\begin{sphinxadmonition}{note}{Note:}
The sample frequency of the third record is 3, since it is regarded to share
its key both with the first and second record.
\end{sphinxadmonition}

This principle is used when applying local suppression to achieve a certain level of
\(k\)-anonymity (see Section 5.2.2) and is based on the fact that
the value NA could replace any value.


\begin{savenotes}\sphinxattablestart
\centering
\sphinxcapstartof{table}
\sphinxcaption{Example dataset to illustrate the effect of missing values on k-anonymity}\label{\detokenize{measure_risk:tab42}}\label{\detokenize{measure_risk:id17}}
\sphinxaftercaption
\begin{tabulary}{\linewidth}[t]{|T|T|T|T|T|}
\hline
\sphinxstyletheadfamily 
No
&\sphinxstyletheadfamily 
Gender
&\sphinxstyletheadfamily 
Education level
&\sphinxstyletheadfamily 
Labor status
&\sphinxstyletheadfamily 
f\_k
\\
\hline
1
&
Male
&
Secondary complete
&
Employed
&
2
\\
\hline
2
&
Male
&
Secondary incomplete
&
Employed
&
2
\\
\hline
3
&
Male
&
NA
&
Employed
&
3
\\
\hline
\end{tabulary}
\par
\sphinxattableend\end{savenotes}

If a dataset satisfies \(k\)-anonymity, an intruder will always find
at least \(k\) individuals with the same combination of
quasi-identifiers. \(k\)-anonymity is often a necessary requirement
for anonymization for a dataset before release, but is not necessarily a
sufficient requirement. The \(k\)-anonymity measure is only based on
frequency counts and does not take (differences in) sample weights into
account. Often \(k\)-anonymity is achieved by first applying
recoding and subsequently applying local suppression, and in some cases
by microaggregation, before using other risk measures and disclosure
methods to further reduce disclosure risk. These methods are discussed
in Chapter 5.


\subsection{\protect\(l\protect\)-diversity}
\label{\detokenize{measure_risk:diversity}}
\(k\)-anonymity has been criticized for not being restrictive
enough. Sensitive information might be disclosed even if the data
satisfies \(k\)-anonymity. This might occur in cases where the data
contains sensitive (non-identifying) categorical variables that have the
same value for all individuals that share the same key. Examples of such
sensitive variables are those containing information on an individual’s
health status. \hyperref[\detokenize{measure_risk:tab43}]{Table \ref{\detokenize{measure_risk:tab43}}} illustrates this problem by using the same data
as previously used, but adding a sensitive variable, ”health”. The first
two individuals satisfy 2-anonymity for the key variables “residence”,
“gender”, “education level” and “labor status”. This means that an
intruder will find at least two individuals when matching to the
released microdata set based on those four quasi-identifiers.
Nevertheless, if the intruder knows that someone belongs to the sample
and has the key \{‘Urban’, ‘Female’, ‘Secondary incomplete’ and
‘Employed’\}, with certainty the health status is disclosed (‘yes’),
because both observations with this key have the same value. This
information is thus disclosed without the necessity to match exactly to
the individual. This is not the case for the individuals with the key
\{‘Urban’, ‘Male’, ‘Secondary complete’, ‘Employed’\}. Individuals 4 and 6
have different values (‘yes’ and ‘no’) for “health”, and thus the
intruder would not gain information about the health status from this
dataset by matching an individual to one of these individuals.


\begin{savenotes}\sphinxattablestart
\centering
\sphinxcapstartof{table}
\sphinxcaption{l-diversity illustration}\label{\detokenize{measure_risk:tab43}}\label{\detokenize{measure_risk:id18}}
\sphinxaftercaption
\begin{tabulary}{\linewidth}[t]{|T|T|T|T|T|T|T|T|T|}
\hline
\sphinxstyletheadfamily 
No
&\sphinxstyletheadfamily 
Residence
&\sphinxstyletheadfamily 
Gender
&\sphinxstyletheadfamily 
Education level
&\sphinxstyletheadfamily 
Labor status
&\sphinxstyletheadfamily 
Health
&\sphinxstyletheadfamily 
f\_k
&\sphinxstyletheadfamily 
F\_k
&\sphinxstyletheadfamily 
l-diversity
\\
\hline
1
&
Urban
&
Female
&
Secondary incomplete
&
Employed
&
yes
&
2
&
360
&
1
\\
\hline
2
&
Urban
&
Female
&
Secondary incomplete
&
Employed
&
yes
&
2
&
360
&
1
\\
\hline
3
&
Urban
&
Female
&
Primary incomplete
&
Non-LF
&
yes
&
1
&
215
&
1
\\
\hline
4
&
Urban
&
Male
&
Secondary complete
&
Employed
&
yes
&
2
&
152
&
2
\\
\hline
5
&
Rural
&
Female
&
Secondary complete
&
Unemployed
&
yes
&
1
&
186
&
1
\\
\hline
6
&
Urban
&
Male
&
Secondary complete
&
Employed
&
no
&
2
&
152
&
2
\\
\hline
7
&
Urban
&
Female
&
Primary complete
&
Non-LF
&
no
&
1
&
180
&
1
\\
\hline
8
&
Urban
&
Male
&
Post-secondary
&
Unemployed
&
yes
&
1
&
215
&
1
\\
\hline
9
&
Urban
&
Female
&
Secondary incomplete
&
Non-LF
&
no
&
2
&
262
&
2
\\
\hline
10
&
Urban
&
Female
&
Secondary incomplete
&
Non-LF
&
yes
&
2
&
262
&
2
\\
\hline
\end{tabulary}
\par
\sphinxattableend\end{savenotes}

The concept of (distinct) \(l\)-diversity addresses this shortcoming
of \(k\)-anonymity (see Machanavajjhala et al., 2007). A dataset
satisfies \(l\)-diversity if for every key \sphinxstyleemphasis{k} there are at least
\sphinxstyleemphasis{l} different values for each of the sensitive variables. In the
example, the first two individuals satisfy only 1-diversity, individuals
4 and 6 satisfy 2-diversity. The required level of \(l\)-diversity
depends on the number of possible values the sensitive variable can
take. If the sensitive variable is a binary variable, the highest level
if \(l\)-diversity that can be achieved is 2. A sample unique will
always only satisfy 1-diversity.

To compute \(l\)-diversity for sensitive variables in \sphinxstyleemphasis{sdcMicro},
the function ldiversity() can be used. This is illustrated in \hyperref[\detokenize{measure_risk:code46}]{Listing \ref{\detokenize{measure_risk:code46}}}.
As arguments, we specify the names of the sensitive
variables %
\begin{footnote}[9]\sphinxAtStartFootnote
Alternatively, the sensitive variables can be specified when
creating the \sphinxstyleemphasis{sdcMicro} object using the function createSdcObj() in
the \sphinxstyleemphasis{sensibleVar} argument. This is further explained in Section 7.5.
In that case, the argument \sphinxstyleemphasis{ldiv\_index} does not have to be specified
in the ldiversity() function. and the variables in the \sphinxstyleemphasis{sensibleVar}
argument will automatically be used to compute \(l\)-diversity.
%
\end{footnote} in the file as well as a constant for
recursive \(l\)-diversity, %
\begin{footnote}[10]\sphinxAtStartFootnote
Besides distinct \(l\)-diversity, there are other
\(l\)-diversity methods: entropy and recursive. Distinct
\(l\)-diversity is most commonly used.
%
\end{footnote} and the code for
missing values in the data. The output is saved in the “risk” slot of
the \sphinxstyleemphasis{sdcMicro} object. The result shows the minimum, maximum, mean and
quantiles of the \(l\)-diversity scores for all individuals in the
sample. The output in \hyperref[\detokenize{measure_risk:code46}]{Listing \ref{\detokenize{measure_risk:code46}}} reproduces the results based on the
data in \hyperref[\detokenize{measure_risk:tab43}]{Table \ref{\detokenize{measure_risk:tab43}}}.

\def\sphinxLiteralBlockLabel{\label{\detokenize{measure_risk:code46}}}
\sphinxSetupCaptionForVerbatim{\(l\)-diversity function in \sphinxstyleemphasis{sdcMicro}}
\fvset{hllines={, ,}}%
\begin{sphinxVerbatim}[commandchars=\\\{\},numbers=left,firstnumber=1,stepnumber=1]
 \PYG{c+c1}{\PYGZsh{} Computing l\PYGZhy{}diversity}

 sdcInitial \PYG{o}{\PYGZlt{}\PYGZhy{}} ldiversity\PYG{p}{(}obj \PYG{o}{=} sdcInitial\PYG{p}{,} ldiv\PYGZus{}index \PYG{o}{=} \PYG{k+kt}{c}\PYG{p}{(}\PYG{l+s}{\PYGZdq{}}\PYG{l+s}{Health\PYGZdq{}}\PYG{p}{)}\PYG{p}{,} l\PYGZus{}recurs\PYGZus{}c \PYG{o}{=} \PYG{l+m}{2}\PYG{p}{,} missing \PYG{o}{=} \PYG{k+kc}{NA}\PYG{p}{)}
 \PYG{c+c1}{\PYGZsh{} Output for l\PYGZhy{}diversity}
 sdcInitial\PYG{o}{@}risk\PYG{o}{\PYGZdl{}}ldiversity

 \PYG{o}{\PYGZhy{}}\PYG{o}{\PYGZhy{}}\PYG{o}{\PYGZhy{}}\PYG{o}{\PYGZhy{}}\PYG{o}{\PYGZhy{}}\PYG{o}{\PYGZhy{}}\PYG{o}{\PYGZhy{}}\PYG{o}{\PYGZhy{}}\PYG{o}{\PYGZhy{}}\PYG{o}{\PYGZhy{}}\PYG{o}{\PYGZhy{}}\PYG{o}{\PYGZhy{}}\PYG{o}{\PYGZhy{}}\PYG{o}{\PYGZhy{}}\PYG{o}{\PYGZhy{}}\PYG{o}{\PYGZhy{}}\PYG{o}{\PYGZhy{}}\PYG{o}{\PYGZhy{}}\PYG{o}{\PYGZhy{}}\PYG{o}{\PYGZhy{}}\PYG{o}{\PYGZhy{}}\PYG{o}{\PYGZhy{}}\PYG{o}{\PYGZhy{}}\PYG{o}{\PYGZhy{}}\PYG{o}{\PYGZhy{}}\PYG{o}{\PYGZhy{}}
 L\PYG{o}{\PYGZhy{}}Diversity Measures
 \PYG{o}{\PYGZhy{}}\PYG{o}{\PYGZhy{}}\PYG{o}{\PYGZhy{}}\PYG{o}{\PYGZhy{}}\PYG{o}{\PYGZhy{}}\PYG{o}{\PYGZhy{}}\PYG{o}{\PYGZhy{}}\PYG{o}{\PYGZhy{}}\PYG{o}{\PYGZhy{}}\PYG{o}{\PYGZhy{}}\PYG{o}{\PYGZhy{}}\PYG{o}{\PYGZhy{}}\PYG{o}{\PYGZhy{}}\PYG{o}{\PYGZhy{}}\PYG{o}{\PYGZhy{}}\PYG{o}{\PYGZhy{}}\PYG{o}{\PYGZhy{}}\PYG{o}{\PYGZhy{}}\PYG{o}{\PYGZhy{}}\PYG{o}{\PYGZhy{}}\PYG{o}{\PYGZhy{}}\PYG{o}{\PYGZhy{}}\PYG{o}{\PYGZhy{}}\PYG{o}{\PYGZhy{}}\PYG{o}{\PYGZhy{}}\PYG{o}{\PYGZhy{}}
 Min.  \PYG{l+m}{1}st Qu.  Median    Mean   \PYG{l+m}{3}rd Qu.    Max.
 \PYG{l+m}{1.0}   \PYG{l+m}{1.0}      \PYG{l+m}{1.0}       \PYG{l+m}{1.4}    \PYG{l+m}{2.0}        \PYG{l+m}{2.0}

 \PYG{c+c1}{\PYGZsh{} l\PYGZhy{}diversity score for each record}
 sdcInitial\PYG{o}{@}risk\PYG{o}{\PYGZdl{}}ldiversity\PYG{p}{[}\PYG{p}{,}\PYG{l+s}{\PYGZsq{}}\PYG{l+s}{Health\PYGZus{}Distinct\PYGZus{}Ldiversity\PYGZsq{}}\PYG{p}{]}

 \PYG{p}{[}\PYG{l+m}{1}\PYG{p}{]} \PYG{l+m}{1} \PYG{l+m}{1} \PYG{l+m}{1} \PYG{l+m}{2} \PYG{l+m}{1} \PYG{l+m}{2} \PYG{l+m}{1} \PYG{l+m}{1} \PYG{l+m}{2} \PYG{l+m}{2}
\end{sphinxVerbatim}

\(l\)-diversity is useful if the data contains categorical sensitive
variables that are not quasi-identifiers themselves. It is not possible
to select quasi-identifiers to calculate the \(l\)-diversity.
\(l\)-diversity has to be calculated for each sensitive variable
separately.


\section{Special Uniques Detection Algorithm (SUDA)}
\label{\detokenize{measure_risk:special-uniques-detection-algorithm-suda}}
The previously discussed risk measures depend on identifying key
variables for which there may be information available from other
sources or other datasets, and which, when combined with the current
data, may lead to re-identification. In practice, however, it might not
always be possible to conduct an inventory of all available datasets and
their variables and thus assess all known external linkages and risks.

To overcome this, an alternative heuristic measure based on special
uniques has been developed to determine the riskiness of a record, which
leads to a SUDA metric or score (see Elliot et al., 2002). These
measures are based on the search for special uniques. To find these
special uniques, algorithms, called SUDA (Special Uniqueness Detection
Algorithm), have been developed. SUDA algorithms are based on the
concept of special uniqueness, which is introduced in the next
subsection. Since this is a heuristic approach, its performance is only
tested in actual datasets, which is done in Elliot et al. (2002) for UK
census data. These tests have shown that the performance of the
algorithm leads to good risk estimates for these test datasets.


\subsection{Sample unique vs. special unique}
\label{\detokenize{measure_risk:sample-unique-vs-special-unique}}
The previous measures of risk focused on the uniqueness of the key of a
record in the dataset. \hyperref[\detokenize{measure_risk:tab44}]{Table \ref{\detokenize{measure_risk:tab44}}} reproduces the data from \hyperref[\detokenize{measure_risk:tab41}]{Table \ref{\detokenize{measure_risk:tab41}}}. The
sample dataset has 10 records and four pre-determined quasi-identifiers
\{“Residence”, “Gender”, “Education level” and “Labor status”\}. Given the
four quasi-identifiers, we have seven distinct patterns in those key
variables, or keys (e.g., \{‘Urban’, ‘Female’, ‘Secondary incomplete’,
‘Employed’\}). The sample frequency counts of the first and second
records equal 2, because the two records share the same pattern (i.e.,
\{‘Urban’, ‘Female’, ‘Secondary incomplete’, ‘Employed’\}). Record 3 is a
sample unique because it is the only individual in the sample who is a
female living in an urban area who is employed without completing
primary school. Similarly, records 5, 7 and 8 are sample uniques,
because they possess distinct patterns with respect to the four key
variables.


\begin{savenotes}\sphinxattablestart
\centering
\sphinxcapstartof{table}
\sphinxcaption{Sample uniques and special uniques}\label{\detokenize{measure_risk:tab44}}\label{\detokenize{measure_risk:id19}}
\sphinxaftercaption
\begin{tabulary}{\linewidth}[t]{|T|T|T|T|T|T|T|T|T|}
\hline
\sphinxstyletheadfamily 
No
&\sphinxstyletheadfamily 
Residence
&\sphinxstyletheadfamily 
Gender
&\sphinxstyletheadfamily 
Education level
&\sphinxstyletheadfamily 
Labor status
&\sphinxstyletheadfamily 
Weight
&\sphinxstyletheadfamily 
f\_k
&\sphinxstyletheadfamily 
F\_k
&\sphinxstyletheadfamily 
risk
\\
\hline
1
&
Urban
&
Female
&
Secondary incomplete
&
Employed
&
180
&
2
&
360
&
0.0054
\\
\hline
2
&
Urban
&
Female
&
Secondary incomplete
&
Employed
&
180
&
2
&
360
&
0.0054
\\
\hline
3
&
Urban
&
Female
&
Primary incomplete
&
Non-LF
&
215
&
1
&
215
&
0.0251
\\
\hline
4
&
Urban
&
Male
&
Secondary complete
&
Employed
&
76
&
2
&
152
&
0.0126
\\
\hline
5
&
Rural
&
Female
&
Secondary complete
&
Unemployed
&
186
&
1
&
186
&
0.0282
\\
\hline
6
&
Urban
&
Male
&
Secondary complete
&
Employed
&
76
&
2
&
152
&
0.0126
\\
\hline
7
&
Urban
&
Female
&
Primary complete
&
Non-LF
&
180
&
1
&
180
&
0.0290
\\
\hline
8
&
Urban
&
Male
&
Post-secondary
&
Unemployed
&
215
&
1
&
215
&
0.0251
\\
\hline
9
&
Urban
&
Female
&
Secondary incomplete
&
Non-LF
&
186
&
2
&
262
&
0.0074
\\
\hline
10
&
Urban
&
Female
&
Secondary incomplete
&
Non-LF
&
76
&
2
&
262
&
0.0074
\\
\hline
\end{tabulary}
\par
\sphinxattableend\end{savenotes}

In addition to the records 3, 5, 7 and 8 in \hyperref[\detokenize{measure_risk:tab44}]{Table \ref{\detokenize{measure_risk:tab44}}} being sample
uniques with respect to the key variable set \{“Residence”, “Gender”,
“Education level”, “Labor status”\}, we can find unique patterns in these
records without even having to consider the complete set of key
variables. For instance, a unique pattern can be found in record 5 when
we look only at the variables “Education level” and “Labor status”
(\{‘Secondary complete’, ‘Unemployed’\}). While the values \{‘Secondary
complete’\} and \{‘Unemployed’\} are not unique in the sample, the
combination of them, \{‘Secondary complete’, ‘Unemployed’\} makes record 5
unique. This variable subset is referred to as the Minimal Sample Unique
(MSU) as any smaller subset of this set of variables is not unique (in
this case \{‘Secondary complete’\} and \{‘Unemployed’\}). It is an MSU of
size 2.{[}\#foot33{]}\_ This holds as well
for three other combinations in record 5, i.e., \{‘Female’, ‘Unemployed’\}
and \{‘Female’, ‘Secondary Complete’\}, which are also MSUs of size 2 and
\{‘Rural’\} of size 1. In total, record 5 has four
MSUs %
\begin{footnote}[11]\sphinxAtStartFootnote
There are more combinations of quasi-identifiers that make record 5
unique (e.g., \{‘Rural’, ‘Female’\} and \{‘Female’, ‘Secondary
Complete’, ‘Unemployed’\}. These combinations, however, are not
considered MSUs because they do not fulfill the \sphinxstylestrong{minimal} subset
requirement. They contain subsets that are MSUs.
%
\end{footnote}. To determine if a set is an MSU of size
\(k\), we check whether it fulfills the minimal requirement. It
suffices to check whether all subsets of size \(k\)-1 of the MSU are
unique. If any of these subsets are also unique in the sample, the set
found may be a sample unique, but violates the minimal requirement and
is hence not an MSU. The unique subset of size \(k\)-1 could be a
MSU. In our example, to determine if the MSU \{‘Secondary complete’,
‘Unemployed’\} is a MSU, we checked as to whether its subsets \{‘Secondary
complete’\} and \{‘Unemployed’\} were not unique in the sample. By
definition, only sample uniques can be special uniques.

The SUDA algorithm identifies all the MSUs in the sample, which in turn
are used to assign a SUDA score to each record. This score indicates how
“risky” a record is. The potential risk of the records is determined
based on two observations:
\begin{itemize}
\item {} 
The smaller the size of the MSU within a record (i.e., the fewer
variables are needed to reach uniqueness), the greater the risk of
the record

\item {} 
The larger the number of MSUs possessed by a record, the greater the
risk of the record

\end{itemize}

A record is defined as a special unique if it is a sample unique both on
the complete set of quasi-identifiers (e.g., in the data in \hyperref[\detokenize{measure_risk:tab44}]{Table \ref{\detokenize{measure_risk:tab44}}},
the variables “Residence”, ”Gender”, “Education level” and “Labor
status”) and simultaneously has at least one MSU (Elliot et al., 1998).
Special uniques can be classified according to the number and size of
subsets that are MSUs. Research has shown that special uniques are more
likely to be population uniques than random uniques (Elliot et al.,
2002) and are thus relevant for risk assessment.


\subsection{Calculating SUDA scores}
\label{\detokenize{measure_risk:calculating-suda-scores}}
The SUDA algorithm is used to search for MSUs in the data among the
sample uniques to determine which sample uniques are also special
uniques i.e., have subsets that are also unique (see Elliot et al.,
2005). First the SUDA algorithm is used to identify the MSUs for each
sample unique. To simplify the search and because smaller subsets are
more important for disclosure risk, the search is limited to a maximum
subset size. Subsequently, a score is assigned to each individual, which
ranks the individuals according to their level of risk.

For each MSU of size \(k\) contained in a given record, a score is
computed by \(\prod_{i = k}^{M}{(ATT - i)}\), where \(M\) is the
user-specified maximum size of MSUs %
\begin{footnote}[12]\sphinxAtStartFootnote
OECD, \sphinxurl{http://stats.oecd.org/glossary}
%
\end{footnote}, and
\(\text{ATT}\) is the total number of attributes or variables in the
dataset. By definition, the smaller the size \(k\) of the MSU, the
larger the score for the MSU, which reflects greater risk (see Elliot et
al., 2005). The final SUDA score for each record is computed by adding
the scores for each MSU in the record. In this way, records with more
MSUs are assigned a higher SUDA score, which also reflects the higher
risk. The SUDA score ranks the individuals according to their level of
risk. The higher the SUDA score, the riskier the sample unique.
\begin{quote}

Calculating SUDA scores \textendash{} a simplified example
\end{quote}

To illustrate how SUDA scores are calculated, we compute the SUDA scores
for the sample uniques in the data in \hyperref[\detokenize{measure_risk:tab45}]{Table \ref{\detokenize{measure_risk:tab45}}}, which replicates the
data from \hyperref[\detokenize{measure_risk:tab45}]{Table \ref{\detokenize{measure_risk:tab45}}}. Record 5 contains four MSUs: \{Rural\} of size 1, and
\{‘Secondary Complete’, ‘Unemployed’\}, \{‘Female’, ‘Unemployed’\} and
\{Female, Secondary Complete\} of size 2. Suppose the maximum size of MSUs
we search for in the data, \(M\), is set at 3. Knowing that,
\(\text{ATT}\), the number of selected key variables in the dataset,
is 4; the score assigned to \{Rural\} is computed by
\(\prod_{i = 1}^{3}{(4 - i)} = 3*2*1 = 6\); and the score assigned
to \{Secondary complete, Unemployed\}, \{Female, Unemployed\} and \{Female,
Secondary Complete\} is
\(\prod_{i = 2}^{3}\left( 4 - i \right) = 2*1 = 2\). The SUDA score
for the fifth record in \hyperref[\detokenize{measure_risk:tab45}]{Table \ref{\detokenize{measure_risk:tab45}}} is then \(6 + 2 + 2 + 2 = 12\),
which is the sum of these four scores per MSU. The SUDA scores for the
other sample uniques are computed accordingly %
\begin{footnote}[13]\sphinxAtStartFootnote
The third record has one MSU, \{‘Primary incomplete’\}; the seventh
record has one MSU, \{‘Primary complete’\}; and the eighth record has
three MSUs, \{‘Urban, Unemployed’\}, \{‘Male, Unemployed’\} and
\{‘Post-secondary’\}.
%
\end{footnote}. The
values that are in the MSUs in the sample uniques are shaded in \hyperref[\detokenize{measure_risk:tab45}]{Table \ref{\detokenize{measure_risk:tab45}}}.
Records that are not sample uniques (\(f_{k} > 1\)) cannot be
special uniques and are assigned the score 0.


\begin{savenotes}\sphinxattablestart
\centering
\sphinxcapstartof{table}
\sphinxcaption{Illustrating the calculation of SUDA and DIS-SUDA scores}\label{\detokenize{measure_risk:tab45}}\label{\detokenize{measure_risk:id20}}
\sphinxaftercaption
\begin{tabulary}{\linewidth}[t]{|T|T|T|T|T|T|T|T|T|}
\hline
\sphinxstyletheadfamily 
No
&\sphinxstyletheadfamily 
Residence
&\sphinxstyletheadfamily 
Gender
&\sphinxstyletheadfamily 
Education level
&\sphinxstyletheadfamily 
Labor status
&\sphinxstyletheadfamily 
Weight
&\sphinxstyletheadfamily 
f\_k
&\sphinxstyletheadfamily 
SUDA score
&\sphinxstyletheadfamily 
DIS-SUDA
\\
\hline
1
&
Urban
&
Female
&
Secondary incomplete
&
Employed
&
180
&
2
&
0
&
0.0000
\\
\hline
2
&
Urban
&
Female
&
Secondary incomplete
&
Employed
&
180
&
2
&
0
&
0.0000
\\
\hline
3
&
Urban
&
Female
&
Primary incomplete
&
Non-LF
&
215
&
1
&
6
&
0.0051
\\
\hline
4
&
Urban
&
Male
&
Secondary complete
&
Employed
&
76
&
2
&
0
&
0.0000
\\
\hline
5
&
Rural
&
Female
&
Secondary complete
&
Unemployed
&
186
&
1
&
12
&
0.0107
\\
\hline
6
&
Urban
&
Male
&
Secondary complete
&
Employed
&
76
&
2
&
0
&
0.0000
\\
\hline
7
&
Urban
&
Female
&
Primary complete
&
Non-LF
&
180
&
1
&
6
&
0.0051
\\
\hline
8
&
Urban
&
Male
&
Post-secondary
&
Unemployed
&
215
&
1
&
10
&
0.0088
\\
\hline
9
&
Urban
&
Female
&
Secondary incomplete
&
Non-LF
&
186
&
2
&
0
&
0.0000
\\
\hline
10
&
Urban
&
Female
&
Secondary incomplete
&
Non-LF
&
76
&
2
&
0
&
0.0000
\\
\hline
\end{tabulary}
\par
\sphinxattableend\end{savenotes}

To estimate record-level disclosure risks, SUDA scores can be used in
combination with the Data Intrusion Simulation (DIS) metric (Elliot and
Manning, 2003), a method for assessing disclosure risks for the entire
dataset (i.e., file-level disclosure risks). Roughly speaking, the
DIS-SUDA method distributes the file-level risk measure generated by the
DIS metric between records according to the SUDA scores of each record.
This way, SUDA scores are calibrated against a consistent measure to
produce the DIS-SUDA scores, which provide the record-level disclosure
risk. These scores are used to compute the conditional probability that
a unique match found by an intruder between the sample unique in the
released microdata and an external data source is also a correct match,
and hence a successful disclosure. The DIS-SUDA measure can be computed
in \sphinxstyleemphasis{sdcMicro}. Since the DIS score is a probability, its values are in
the interval \(\lbrack 0,\ 1\rbrack\). A full description of the
DIS-SUDA method is provided by Elliot and Manning (2003).

Note that unlike the risk methods discussed earlier, the DIS-SUDA score
does not fully account for the sampling weights. Risk measures based on
the previous methods (i.e., negative binomial models) will in general
have lower risks for those records with greater sampling weight, given
the same sample frequency count, than those measured using DIS-SUDA.
Therefore, instead of replacing the risk measures introduced in the
previous section, the SUDA scores and DIS-SUDA approach should be used
as a complementary method. As mentioned earlier, DIS-SUDA is
particularly useful in situations where taking an inventory of all
already available datasets and their variables is difficult.


\subsection{Application of SUDA, DIS-SUDA using \sphinxstyleemphasis{sdcMicro}}
\label{\detokenize{measure_risk:application-of-suda-dis-suda-using-sdcmicro}}
Both SUDA and DIS-SUDA scores can be computed using \sphinxstyleemphasis{sdcMicro} (Templ et
al., 2014). Given that the search for MSUs with the SUDA algorithm can
be computationally demanding, \sphinxstyleemphasis{sdcMicro} uses an improved SUDA2
algorithm, which more effectively locates the boundaries of the search
space for MSUs (Manning et al., 2008).

SUDA scores can be calculated using the suda2() function in \sphinxstyleemphasis{sdcMicro}.
It is important to specify the missing argument in suda2(). This should
match the code for missing values in your dataset. In \sphinxstyleemphasis{R} this is most
likely the \sphinxstyleemphasis{R} standard missing value, NA. We mention this because \sphinxstylestrong{the
default missing value code in the sdcMicro suda2() function is -999 and
will most likely need to be changed to ‘NA’ when using most R
datasets.} The scores are saved in the risk slot of the \sphinxstyleemphasis{sdcMicro}
object. The syntax in \hyperref[\detokenize{measure_risk:code47}]{Listing \ref{\detokenize{measure_risk:code47}}} shows how to retrieve the output.

\def\sphinxLiteralBlockLabel{\label{\detokenize{measure_risk:code47}}}
\sphinxSetupCaptionForVerbatim{Evaluating SUDA scores}
\fvset{hllines={, ,}}%
\begin{sphinxVerbatim}[commandchars=\\\{\},numbers=left,firstnumber=1,stepnumber=1]
 \PYG{c+c1}{\PYGZsh{} Evaluating SUDA scores for the specified variables}
 sdcInitial \PYG{o}{\PYGZlt{}\PYGZhy{}} suda2\PYG{p}{(}obj \PYG{o}{=} sdcInitial\PYG{p}{,} missing \PYG{o}{=} \PYG{k+kc}{NA}\PYG{p}{)}

 \PYG{c+c1}{\PYGZsh{} The results are saved in the risk slot of the sdcMicro object}
 \PYG{c+c1}{\PYGZsh{} SUDA scores}
 sdcInitial\PYG{o}{@}risk\PYG{o}{\PYGZdl{}}suda2\PYG{o}{\PYGZdl{}}score

 \PYG{p}{[}\PYG{l+m}{1}\PYG{p}{]} \PYG{l+m}{0.00} \PYG{l+m}{0.00} \PYG{l+m}{1.75} \PYG{l+m}{0.00} \PYG{l+m}{3.25} \PYG{l+m}{0.00} \PYG{l+m}{1.75} \PYG{l+m}{2.75} \PYG{l+m}{0.00} \PYG{l+m}{0.00}

 \PYG{c+c1}{\PYGZsh{} DIS\PYGZhy{}SUDA scores}
 sdcInitial\PYG{o}{@}risk\PYG{o}{\PYGZdl{}}suda2\PYG{o}{\PYGZdl{}}disScore

 \PYG{p}{[}\PYG{l+m}{1}\PYG{p}{]} \PYG{l+m}{0.000000000} \PYG{l+m}{0.000000000} \PYG{l+m}{0.005120313} \PYG{l+m}{0.000000000} \PYG{l+m}{0.010702061}
 \PYG{p}{[}\PYG{l+m}{6}\PYG{p}{]} \PYG{l+m}{0.000000000} \PYG{l+m}{0.005120313} \PYG{l+m}{0.008775093} \PYG{l+m}{0.000000000} \PYG{l+m}{0.000000000}

 \PYG{c+c1}{\PYGZsh{} Summary of DIS\PYGZhy{}SUDA scores}
 sdcInitial\PYG{o}{@}risk\PYG{o}{\PYGZdl{}}suda2

 Dis suda scores \PYG{k+kp}{table}\PYG{o}{:}
 \PYG{o}{\PYGZhy{}} \PYG{o}{\PYGZhy{}} \PYG{o}{\PYGZhy{}} \PYG{o}{\PYGZhy{}} \PYG{o}{\PYGZhy{}} \PYG{o}{\PYGZhy{}} \PYG{o}{\PYGZhy{}} \PYG{o}{\PYGZhy{}} \PYG{o}{\PYGZhy{}} \PYG{o}{\PYGZhy{}} \PYG{o}{\PYGZhy{}}
 thresholds number
 \PYG{l+m}{1}        \PYG{o}{\PYGZgt{}} \PYG{l+m}{0}      \PYG{l+m}{6}
 \PYG{l+m}{2}      \PYG{o}{\PYGZgt{}} \PYG{l+m}{0.1}      \PYG{l+m}{4}
 \PYG{l+m}{3}      \PYG{o}{\PYGZgt{}} \PYG{l+m}{0.2}      \PYG{l+m}{0}
 \PYG{l+m}{4}      \PYG{o}{\PYGZgt{}} \PYG{l+m}{0.3}      \PYG{l+m}{0}
 \PYG{l+m}{5}      \PYG{o}{\PYGZgt{}} \PYG{l+m}{0.4}      \PYG{l+m}{0}
 \PYG{l+m}{6}      \PYG{o}{\PYGZgt{}} \PYG{l+m}{0.5}      \PYG{l+m}{0}
 \PYG{l+m}{7}      \PYG{o}{\PYGZgt{}} \PYG{l+m}{0.6}      \PYG{l+m}{0}
 \PYG{l+m}{8}      \PYG{o}{\PYGZgt{}} \PYG{l+m}{0.7}      \PYG{l+m}{0}
 \PYG{o}{\PYGZhy{}} \PYG{o}{\PYGZhy{}} \PYG{o}{\PYGZhy{}} \PYG{o}{\PYGZhy{}} \PYG{o}{\PYGZhy{}} \PYG{o}{\PYGZhy{}} \PYG{o}{\PYGZhy{}} \PYG{o}{\PYGZhy{}} \PYG{o}{\PYGZhy{}} \PYG{o}{\PYGZhy{}} \PYG{o}{\PYGZhy{}}
\end{sphinxVerbatim}

To compare DIS scores before and after applying SDC methods, it may be
useful to use histograms or density plots of these scores. \hyperref[\detokenize{measure_risk:code48}]{Listing \ref{\detokenize{measure_risk:code48}}}
shows how to generate histograms of the SUDA scores summarized in
\hyperref[\detokenize{measure_risk:code47}]{Listing \ref{\detokenize{measure_risk:code47}}}. The histogram is shown in \hyperref[\detokenize{measure_risk:fig2}]{Fig.\@ \ref{\detokenize{measure_risk:fig2}}}. All outputs relate to
the data used in the example. In our case, we have not applied any SDC
method to the data yet and thus have only the plots for the initial
values. Typically, after applying SDC methods, one would recalculate the
SUDA scores and compare them to the original values. One way to quickly
see the differences would be to rerun these visualizations and compare
them to the base for risk changes.

\def\sphinxLiteralBlockLabel{\label{\detokenize{measure_risk:code48}}}
\sphinxSetupCaptionForVerbatim{Histogram and density plots of DIS-SUDA scores}
\fvset{hllines={, ,}}%
\begin{sphinxVerbatim}[commandchars=\\\{\},numbers=left,firstnumber=1,stepnumber=1]
 \PYG{c+c1}{\PYGZsh{} Plot a histogram of disScore}
 hist\PYG{p}{(}sdcInitial\PYG{o}{@}risk\PYG{o}{\PYGZdl{}}suda2\PYG{o}{\PYGZdl{}}disScore\PYG{p}{,} main \PYG{o}{=} \PYG{l+s}{\PYGZsq{}}\PYG{l+s}{Histogram of DIS\PYGZhy{}SUDA scores\PYGZsq{}}\PYG{p}{)}

 \PYG{c+c1}{\PYGZsh{} Density plot}
 density \PYG{o}{\PYGZlt{}\PYGZhy{}} density\PYG{p}{(}sdcInitial\PYG{o}{@}risk\PYG{o}{\PYGZdl{}}suda2\PYG{o}{\PYGZdl{}}disScore\PYG{p}{)}
 plot\PYG{p}{(}density\PYG{p}{,} main \PYG{o}{=} \PYG{l+s}{\PYGZsq{}}\PYG{l+s}{Density plot of DIS\PYGZhy{}SUDA scores\PYGZsq{}}\PYG{p}{)}
\end{sphinxVerbatim}

\begin{figure}[htbp]
\centering
\capstart

\noindent\sphinxincludegraphics{{image2}.png}
\caption{Visualizations of DIS-SUDA scores}\label{\detokenize{measure_risk:fig2}}\label{\detokenize{measure_risk:id21}}\end{figure}


\section{Risk measures for continuous variables}
\label{\detokenize{measure_risk:risk-measures-for-continuous-variables}}
The principle of rareness or uniqueness of combinations of
quasi-identifiers (keys) is not useful for continuous variables, because
it is likely that all or many individuals will have unique keys.
Therefore, other approaches are exploited for measuring the disclosure
risk of continuous variables. These methods are based on uniqueness of
the values in the neighborhood of the original values. The uniqueness is
defined in different ways: in absolute terms (interval measure) or
relative terms (record linkage). Most measures are a posteriori
measures: they are evaluated after anonymization of the raw data,
compare the treated data with the raw data and evaluate for each
individual the distance between the values in the raw and the treated
data. This means that these methods are not useful for identifying
individuals at risk within the raw data, but rather show the
distance/difference between the dataset before and after anonymization
and can therefore be interpreted as evaluation of the anonymization
method. For that reason, they resemble the information loss measures
discussed in Chapter 6. Finally, risk measures for continuous
quasi-identifiers are also based on outlier detection. Outliers play an
important role in the re-identification of these records.


\subsection{Record linkage}
\label{\detokenize{measure_risk:record-linkage}}
Record linkage is an a posteriori method that evaluates the number of
correct linkages when linking the perturbed values with the original
values. The linking algorithm is based on the distance between the
original and the perturbed values (i.e., distance-based record linkage).
The perturbed values are matched with the closest individual. It is
important to note that this method does not give information on the
initial risk, but is rather a measure to evaluate the perturbation
algorithm (i.e., it is designed to indicate the level of uncertainty
introduced into the variable by counting the number of records that
could be correctly matched).

Record linkage algorithms differ with respect to which distance measure
is used. When a variable has very different scaling than other
continuous variables in the dataset, rescaling the variables before
using record linkage is recommended. Very different scales may lead to
undesired results when measuring the multivariate distance between
records based on several continuous variables. Since these methods are
based on both the raw data and treated data, examples of their
applications require the introduction of SDC methods and are therefore
postponed to the case studies in Chapter 9.

Besides distance-based record linkage, another method for linking is
probabilistic record linkage (see Domingo-Ferrer and Torra, 2003). The
literature shows, however, that results from distance-based record
linkage are better than the results from probabilistic record linkage.
Individuals in the treated data that are linked to the correct
individuals in the raw data are considered at risk of disclosure.


\subsection{Interval measure}
\label{\detokenize{measure_risk:interval-measure}}
Successful application of an SDC method should result in perturbed
values that are considered not too close to their initial values; if the
value is relatively close, re-identification may be relatively easy. In
the application of interval measures, intervals are created around each
perturbed value and then a determination is made as to whether the
original value of that perturbed observation is contained in this
interval. Values that are within the interval around the initial value
after perturbation are considered too close to the initial value and
hence unsafe and need more perturbation. Values that are outside of the
intervals are considered safe. The size of the intervals is based on the
standard deviation of the observations and a scaling parameter. This
method is implemented in the function dRisk() in \sphinxstyleemphasis{sdcMicro}. \hyperref[\detokenize{measure_risk:code49}]{Listing \ref{\detokenize{measure_risk:code49}}}
shows how to print or display the risk value computed by \sphinxstyleemphasis{sdcMicro} by
comparing the income variables before and after anonymization. “sdcObj”
is an \sphinxstyleemphasis{sdcMicro} object and “compExp“ is a vector containing the names
of the income variables. The size of the intervals is \(k\) times
the standard deviation, where \(k\) is a parameter in the function
dRisk(). The larger \(k\), the larger the intervals are, and hence
the larger the number of observations within the interval constructed
around their original values and the higher the risk measure. The result
1 indicates that all (100 percent) the observations are outside the
interval of 0.1 times the standard deviation around the original values.

\def\sphinxLiteralBlockLabel{\label{\detokenize{measure_risk:code49}}}
\sphinxSetupCaptionForVerbatim{Example with the function dRisk()}
\fvset{hllines={, ,}}%
\begin{sphinxVerbatim}[commandchars=\\\{\},numbers=left,firstnumber=1,stepnumber=1]
 dRisk\PYG{p}{(}obj \PYG{o}{=} sdcObj\PYG{o}{@}origData\PYG{p}{[}\PYG{p}{,}compExp\PYG{p}{]}\PYG{p}{,} xm \PYG{o}{=} sdcObj\PYG{o}{@}manipNumVars\PYG{p}{[}\PYG{p}{,}compExp\PYG{p}{]}\PYG{p}{,} k \PYG{o}{=} \PYG{l+m}{0.1}\PYG{p}{)}
 \PYG{p}{[}\PYG{l+m}{1}\PYG{p}{]} \PYG{l+m}{1}
\end{sphinxVerbatim}

For most values, this is a satisfactory approach. It is not a sufficient
measure for outliers, however. After perturbation, outliers will stay
outliers and are easily re-identifiable, even if they are sufficiently
far from their initial values. Therefore, outliers should be treated
with caution.


\subsection{Outlier detection}
\label{\detokenize{measure_risk:outlier-detection}}
Outliers are important for measuring re-identification risk in
continuous microdata. Continuous data are often skewed, especially
right-skewed. This means that there are a few outliers with very high
values relative to the other observations of the same variable. Examples
are income in household data, where only few individuals/households may
have very high incomes, or turnover data for firms that are much larger
than other firms in the sample are. In cases like these, even if these
values are perturbed, it may still be easy to identify these outliers,
since they will stay the largest values even after perturbation. (The
perturbation will have created uncertainty as to the exact value, but
because the value started out so much further away from other
observations, it may still be easy to link to the high-income individual
or very large firm.). Examples would be the only doctor in a
geographical area with a high income or one single large firm in one
industry type. Therefore, identifying outliers in continuous data is an
important step when identifying individuals at high risk. In practice,
identifying the values of a continuous variable that are larger than a
predetermined \(p\%\)-percentile might help identify outliers, and
thus units at greater risk of identification. The value of \(p\)
depends on the skewness of the data.

We can calculate the \(p\%\)-percentile of a continuous variable in
\sphinxstyleemphasis{R} and show the individuals who have income larger than this
percentile. \hyperref[\detokenize{measure_risk:code410}]{Listing \ref{\detokenize{measure_risk:code410}}} provides an illustration for the 90$^{\text{th}}$
percentile.

\def\sphinxLiteralBlockLabel{\label{\detokenize{measure_risk:code410}}}
\sphinxSetupCaptionForVerbatim{Computing 90 \% percentile of variable income}
\fvset{hllines={, ,}}%
\begin{sphinxVerbatim}[commandchars=\\\{\},numbers=left,firstnumber=1,stepnumber=1]
 \PYG{c+c1}{\PYGZsh{} Compute the 90 \PYGZpc{} percentile for the variable income*}
 perc90 \PYG{o}{\PYGZlt{}\PYGZhy{}} quantile\PYG{p}{(}\PYG{k+kp}{file}\PYG{p}{[}\PYG{p}{,}\PYG{l+s}{\PYGZsq{}}\PYG{l+s}{income\PYGZsq{}}\PYG{p}{]}\PYG{p}{,} \PYG{l+m}{0.90}\PYG{p}{,} na.rm \PYG{o}{=} \PYG{k+kc}{TRUE}\PYG{p}{)}

 \PYG{c+c1}{\PYGZsh{} Show the ID of observations with values for income larger than the 90 \PYGZpc{} percentile}
 \PYG{k+kp}{file}\PYG{p}{[}\PYG{p}{(}\PYG{k+kp}{file}\PYG{p}{[}\PYG{p}{,} \PYG{l+s}{\PYGZsq{}}\PYG{l+s}{income\PYGZsq{}}\PYG{p}{]} \PYG{o}{\PYGZgt{}=} perc90\PYG{p}{)}\PYG{p}{,} \PYG{l+s}{\PYGZsq{}}\PYG{l+s}{ID\PYGZsq{}}\PYG{p}{]}
\end{sphinxVerbatim}

A second approach for outlier detection is a posteriori measure
comparing the treated and raw data. An interval is constructed around
the perturbed values as described in the previous section. If the
original values fall into the interval around the perturbed values, the
perturbed values are considered unsafe since they are too close to the
original values. There are different ways to construct such intervals,
such as rank-based intervals and standard deviation-based intervals.
Templ and Meindl (2008) propose a robust alternative for these
intervals. They construct the intervals based on the squared Robust
Mahalanobis Distance (RMD) of the individual values. The intervals are
scaled by the RMD such that outliers obtain larger intervals and hence
need to have a larger perturbation in order to be considered safe than
values that are not outliers. This method is implemented in \sphinxstyleemphasis{sdcMicro}
in the function dRiskRMD(), which is an extension of the dRisk()
function. This method is illustrated in the case studies in Chapter 9.


\section{Global risk}
\label{\detokenize{measure_risk:global-risk}}
To construct one aggregate risk measure at the global level for the
complete dataset, we can aggregate the measures for risk at the
individual level in several ways. Global risk measures should be used
with caution: behind an acceptable global risk can hide some very
high-risk records that are compensated by many low risk records.


\subsection{Mean of individual risk measures}
\label{\detokenize{measure_risk:mean-of-individual-risk-measures}}
A straightforward way of aggregating the individual risk measures is
taking the mean of all individuals in the sample, which is equal to
summing over all keys in the sample if multiplied by the sample
frequencies of these keys and dividing by the sample size n:
\begin{equation*}
\begin{split}R_{1} = \frac{1}{n}\sum_{i}^{}r_{k} = \frac{1}{n}\sum_{k}^{}{f_{k}r}_{k}\end{split}
\end{equation*}
\(r_{k}\) is the individual risk of key \(k\) that the
\(i\)$^{\text{th}}$ individual shares (see Section 4.5.1). This measure
is reported as global risk in \sphinxstyleemphasis{sdcMicro}, is stored in the “risk” slot
and can be displayed as shown in \hyperref[\detokenize{measure_risk:code411}]{Listing \ref{\detokenize{measure_risk:code411}}}. It indicates that the
average re-identification probability is 0.01582 or 0.1582 \%.

\def\sphinxLiteralBlockLabel{\label{\detokenize{measure_risk:code411}}}
\sphinxSetupCaptionForVerbatim{Computation of the global risk measure}
\fvset{hllines={, ,}}%
\begin{sphinxVerbatim}[commandchars=\\\{\},numbers=left,firstnumber=1,stepnumber=1]
 \PYG{c+c1}{\PYGZsh{} Global risk (average re\PYGZhy{}identification probability)}
 sdcInitial\PYG{o}{@}risk\PYG{o}{\PYGZdl{}}global\PYG{o}{\PYGZdl{}}risk

 \PYG{p}{[}\PYG{l+m}{1}\PYG{p}{]} \PYG{l+m}{0.01582}
\end{sphinxVerbatim}

The global risk in the example data in \hyperref[\detokenize{measure_risk:tab41}]{Table \ref{\detokenize{measure_risk:tab41}}} is 0.01582, which is
the expected proportion of all individuals in the sample that could be
re-identified by an intruder. Another way of expressing the global risk
is the number of expected re-identifications, \(n*R_{1}\), which is
in the example 10 * 0.01582. The expected number of re-identifications
is also saved in the \sphinxstyleemphasis{sdcMicro} object. \hyperref[\detokenize{measure_risk:code412}]{Listing \ref{\detokenize{measure_risk:code412}}} shows how to
display this.

\begin{sphinxadmonition}{note}{Note:}
This global risk measure should be used with
caution. The average risk can be relatively low, but a few individuals
could have a very high probability of re-identification.
\end{sphinxadmonition}

An easy way to check for this is to look at the distribution of the individual risk
values or the number of individuals with risk values above a certain
threshold, as shown in the next section.

\def\sphinxLiteralBlockLabel{\label{\detokenize{measure_risk:code412}}}
\sphinxSetupCaptionForVerbatim{Computation of expected number of re-identifications}
\fvset{hllines={, ,}}%
\begin{sphinxVerbatim}[commandchars=\\\{\},numbers=left,firstnumber=1,stepnumber=1]
 \PYG{c+c1}{\PYGZsh{} Global risk (expected number of reidentifications)}
 sdcInitial\PYG{o}{@}risk\PYG{o}{\PYGZdl{}}global\PYG{o}{\PYGZdl{}}risk\PYGZus{}ER

     \PYG{p}{[}\PYG{l+m}{1}\PYG{p}{]} \PYG{l+m}{0.1582}
\end{sphinxVerbatim}


\subsection{Count of individuals with risks larger than a certain threshold}
\label{\detokenize{measure_risk:count-of-individuals-with-risks-larger-than-a-certain-threshold}}
All individuals belonging to the same key have the same individual risk,
\(r_{k}\). Another way of expressing the total risk in the sample is
the total number of observations that exceed a certain threshold of
individual risk. Setting the threshold can be absolute (e.g., all those
individuals who have a disclosure risk higher than 0.05 or 5\%) or
relative (e.g., all those individuals with risks higher than the upper
quartile of individual risk). \hyperref[\detokenize{measure_risk:code413}]{Listing \ref{\detokenize{measure_risk:code413}}} shows how, using \sphinxstyleemphasis{R}, one
would count the number of observations with an individual
re-identification risk higher than 5\%. In the example, no individual has
a higher disclosure risk than 0.05.

\def\sphinxLiteralBlockLabel{\label{\detokenize{measure_risk:code413}}}
\sphinxSetupCaptionForVerbatim{Number of individuals with individual risk higher than the threshold 0.05}
\fvset{hllines={, ,}}%
\begin{sphinxVerbatim}[commandchars=\\\{\},numbers=left,firstnumber=1,stepnumber=1]
 \PYG{k+kp}{sum}\PYG{p}{(}sdcInitial\PYG{o}{@}risk\PYG{o}{\PYGZdl{}}individual\PYG{p}{[}\PYG{p}{,}\PYG{l+m}{1}\PYG{p}{]} \PYG{o}{\PYGZgt{}} \PYG{l+m}{0.05}\PYG{p}{)}

     \PYG{p}{[}\PYG{l+m}{1}\PYG{p}{]} \PYG{l+m}{0}
\end{sphinxVerbatim}

These calculations can then be used to treat data for individuals whose
risk values are above a predetermined threshold. We will see later that
there are methods in \sphinxstyleemphasis{sdcMicro}, such as localSupp(), that can be used
to suppress values of certain key variables for those individuals with
risk above a specified threshold. This is explained further in Section
5.2.2.


\section{Household risk}
\label{\detokenize{measure_risk:household-risk}}
In many social surveys, the data have a hierarchical structure where an
individual belongs to a higher-level entity (see Section 4.4). Typical
examples are households in social surveys or pupils in schools.
Re-identification of one household member can lead to re-identification
of the other household members, too. It is therefore easy to see that if
we take the household structure into account, the re-identification risk
is the risk that at least one of the household members is re-identified.

\(r^{h} = P(A_{1} \cup A_{2} \cup\) …
\(\cup A_{J}) = 1 - \prod_{j = 1}^{J}{1 - P(A_{j})}\),

where \(A_{j}\) is the event that the \(j\)$^{\text{th}}$ member of
the household is re-identified and \(P\left( A_{j} \right) = r_{k}\)
is the individual disclosure risk of the \(jt\)$^{\text{h}}$ member.
For example, if a household member has three members with individual
disclosure risks based on their respective keys 0.02, 0.03 and 0.03,
respectively, the household risk is
\begin{equation*}
\begin{split}1 - (1 - 0.02)(1 - 0.03)(1 - 0.03)) = 0.078\end{split}
\end{equation*}
The hierarchical or household risk cannot be lower than the individual
risk, and the household risk is always the same for all household
members. The household risk should be used in cases where the data
contain a hierarchical structure, i.e., where a household structure is
present in the data. Using \sphinxstyleemphasis{sdcMicro}, if a household identifier is
specified (in the argument \sphinxstyleemphasis{hhId} in the function createSdcObj()) while
creating an \sphinxstyleemphasis{sdcMicro} object, the household risk will automatically be
computed. \hyperref[\detokenize{measure_risk:code414}]{Listing \ref{\detokenize{measure_risk:code414}}} shows how to display these risk measures.

\def\sphinxLiteralBlockLabel{\label{\detokenize{measure_risk:code414}}}
\sphinxSetupCaptionForVerbatim{Computation of household risk and expected number of re-identifications}
\fvset{hllines={, ,}}%
\begin{sphinxVerbatim}[commandchars=\\\{\},numbers=left,firstnumber=1,stepnumber=1]
 \PYG{c+c1}{\PYGZsh{} Household risk}
 sdcInitial\PYG{o}{@}risk\PYG{o}{\PYGZdl{}}global\PYG{o}{\PYGZdl{}}hier\PYGZus{}risk

 \PYG{c+c1}{\PYGZsh{} Household risk (expected number of reidentifications}
 sdcInitial\PYG{o}{@}risk\PYG{o}{\PYGZdl{}}global\PYG{o}{\PYGZdl{}}hier\PYGZus{}risk\PYGZus{}ER
\end{sphinxVerbatim}

\begin{sphinxadmonition}{note}{Note:}
The size of a household is an important identifier itself,
especially for large households. Suppression of the actual size variable
(e.g., number of household members), however, does not suffice to remove
this information from the dataset, as a simple count of the household
members for a particular household will allow reconstructing this
variable as long as a household ID is in the data, which allows
assigning individuals to households. We flag this for the reader’s
attention as it is important. Further discussion on approaches to the
SDC process that take into account the household structure where it
exists can be found in Section** \sphinxstylestrong{5.4.}
\end{sphinxadmonition}

\begin{sphinxadmonition}{note}{Recommended Reading Material on Risk Measurement}

Elliot, Mark J, Anna Manning, Ken Mayes, John Gurd, and Michael Bane.
2005. “SUDA: A Program for Detecting Special Uniques.” \sphinxstyleemphasis{Joint
UNECE/Eurostat Work Session on Statistical Data Confidentiality}.
Geneva.

Hundepool, Anco, Josep Domingo-Ferrer, Luisa Franconi, Sarah Giessing,
Eric Schulte Nordholt, Keith Spicer, and Peter Paul de Wolf. 2012.
\sphinxstyleemphasis{Statistical Disclosure Control.} Chichester: John Wiley \& Sons Ltd.
doi:10.1002/9781118348239.

Lambert, Diane. 1993.”Measures of Disclosure Risk and Harm.” \sphinxstyleemphasis{Journal of
Official Statistics} 9(2) : 313-331.

Machanavajjhala, Ashwin, Daniel Kifer, Johannes Gehrke, and
Muthuramakrishnan Venkitasubramaniam. 2007. “L-diversity: Privacy Beyond
K-anonymity.” \sphinxstyleemphasis{ACM Trans. Knowl. Discov. Data} 1 (Article 3)
(1556-4681). doi:10.1145/1217299.1217302.
http://www.truststc.org/pubs/465/L\%20Diversity\%20Privacy.pdf. Accessed
October 5, 2015.

Templ, Matthias, Bernhard Meindl, Alexander Kowarik, and Shuang Chen.
2014. “Introduction to Statistical Disclosure Control (SDC).”
\sphinxurl{http://www.ihsn.org/home/sites/default/files/resources/ihsn-working-paper-007-Oct27.pdf}\sphinxstyleemphasis{.}
August 1. Accessed November 13, 2014.
\end{sphinxadmonition}


\chapter{Anonymization Methods}
\label{\detokenize{anon_methods:anonymization-methods}}\label{\detokenize{anon_methods::doc}}
This chapter describes the SDC methods most commonly used. All methods
are implementable in \sphinxstyleemphasis{R} by using the \sphinxstyleemphasis{sdcMicro} package. We discuss for
every method for what type of data the method is suitable, both in terms
of data characteristics and type of data. Furthermore, options such as
specific parameters for each method are discussed as well as their
impacts. %
\begin{footnote}[1]\sphinxAtStartFootnote
We also show code examples in \sphinxstyleemphasis{R,} which are drawn from findings we
gathered by applying these methods to a large collection of datasets.
%
\end{footnote} These findings are meant as guidance but
should be used with caution, since every dataset has different
characteristics and our findings may not always address your particular
dataset. The last three sections of this chapter are on the
anonymization of variables and datasets with particular characteristics
that deserve special attention. Section 5.4 deals with for anonymizing
geographical data, such as GPS coordinates, Section 5.5 discusses the
anonymization of data with a hierarchical structure (household
structure) and Section 5.6 describes the peculiarities of dealing with
and releasing census microdata.

To determine which anonymization methods are suitable for specific
variables and/or datasets, we begin by presenting some classifications
of SDC methods.


\section{Classification of SDC methods}
\label{\detokenize{anon_methods:classification-of-sdc-methods}}
SDC methods can be classified as \sphinxstylestrong{non-perturbative} and
\sphinxstylestrong{perturbative} (see Hundepool et al., 2012).
\begin{itemize}
\item {} 
\sphinxstylestrong{Non-perturbative methods} reduce the detail in the data by
generalization or suppression of certain values (i.e., masking)
without distorting the data structure.

\item {} 
\sphinxstylestrong{Perturbative methods} do not suppress values in the dataset but
perturb (i.e., alter) values to limit disclosure risk by creating
uncertainty around the true values.

Both non-perturbative and perturbative methods can be used for
categorical and continuous variables.

\end{itemize}

We also distinguish between \sphinxstylestrong{probabilistic} and \sphinxstylestrong{deterministic} SDC
methods.
\begin{itemize}
\item {} 
\sphinxstylestrong{Probabilistic methods} depend on a probability mechanism or a
random number-generating mechanism. Every time a probabilistic method
is used, a different outcome is generated. For these methods it is
often recommended that a seed be set for the random number generator
if you want to produce replicable results.

\item {} 
\sphinxstylestrong{Deterministic methods} follow a certain algorithm and produce the
same results if applied repeatedly to the same data with the same set
of parameters.

\end{itemize}

SDC methods for microdata intend to prevent identity and attribute
disclosure. Different SDC methods are used for each type of disclosure
control. Methods such as recoding and suppression are applied to
quasi-identifiers to prevent identity disclosure, whereas top coding a
quasi-identifier (e.g., income) or perturbing a sensitive variable
prevent attribute disclosure.

As this practice guide is written around the use of the \sphinxstyleemphasis{sdcMicro}
package, we discuss only SDC methods that are implemented in the
\sphinxstyleemphasis{sdcMicro} package or can be easily implemented in \sphinxstyleemphasis{R}. These are the
most commonly applied methods from the literature and used in most
agencies experienced in using these methods. Table 5.1 gives an overview
of the SDC methods discussed in this chapter, their classification,
types of data to which they are applicable and their function names in
the \sphinxstyleemphasis{sdcMicro} package.

Table 5.1: SDC methods and corresponding functions in \sphinxstyleemphasis{sdcMicro}


\begin{savenotes}\sphinxattablestart
\centering
\begin{tabulary}{\linewidth}[t]{|T|T|T|T|}
\hline
\sphinxstyletheadfamily 
Method
&\sphinxstyletheadfamily 
Classification of SDC method
&\sphinxstyletheadfamily 
Data Type
&\sphinxstyletheadfamily 
Function in sdcMicro
\\
\hline
Global recoding
&
non-perturbative, determinitic
&
continuous and categorical
&
\sphinxhref{http://www.rdocumentation.org/packages/sdcMicro/functions/globalrecode/}{globalRecode} ,
\sphinxhref{http://www.rdocumentation.org/packages/sdcMicro/functions/groupVars-methods/}{groupVars}
\\
\hline
Top and bottom coding
&
non-perturbative, determinitic
&
continuous and categorical
&
\sphinxhref{http://www.rdocumentation.org/packages/sdcMicro/functions/topBotCoding/}{topBotCoding}
\\
\hline
Local
suppression
&
non-perturbative, determinitic
&
categorical
&
\sphinxhref{http://www.rdocumentation.org/packages/sdcMicro/functions/localSuppression/}{localSuppression}, localSupp
\\
\hline
PRAM
&
perturbative,
probabilistic
&
categorical
&
\sphinxhref{http://www.rdocumentation.org/packages/sdcMicro/functions/pram/}{pram}
\\
\hline
Micro aggregation
&
perturbative,
probabilistic
&
continuous
&
\sphinxhref{http://www.rdocumentation.org/packages/sdcMicro/functions/microaggregation/}{microaggregation}
\\
\hline
Noise addition
&
perturbative,
probabilistic
&
continuous
&
\sphinxhref{http://www.rdocumentation.org/packages/sdcMicro/functions/addNoise/}{addNoise}
\\
\hline
Shuffling
&
perturbative,
probabilistic
&
continuous
&
\sphinxhref{http://www.rdocumentation.org/packages/sdcMicro/functions/shuffle/}{shuffle}
\\
\hline
Rank swapping
&
perturbative,
probabilistic
&
continuous
&
\sphinxhref{http://www.rdocumentation.org/packages/sdcMicro/functions/rankSwap/}{rankSwap}
\\
\hline
\end{tabulary}
\par
\sphinxattableend\end{savenotes}


\section{Non-perturbative methods}
\label{\detokenize{anon_methods:non-perturbative-methods}}

\subsection{Recoding}
\label{\detokenize{anon_methods:recoding}}
Recoding is a deterministic method used to decrease the number of
distinct categories or values for a variable. This is done by combining
or grouping categories for categorical variables or constructing
intervals for continuous variables. Recoding is applied to all
observations of a certain variable and not only to those at risk of
disclosure. There are two general types of recoding: global recoding and
top and bottom coding.


\subsubsection{Global recoding}
\label{\detokenize{anon_methods:global-recoding}}
Global recoding combines several categories of a categorical variable or
constructs intervals for continuous variables. This reduces the number
of categories available in the data and potentially the disclosure risk,
especially for categories with few observations, but also, importantly,
it reduces the level of detail of information available to the analyst.
To illustrate recoding, we use the following example. Assume that we
have five regions in our dataset. Some regions are very small and when
combined with other key variables in the dataset, produce high
re-identification risk for some individuals in those regions. One way to
reduce risk would be to combine some of the regions by recoding them. We
could, for example, make three groups out of the five, call them
‘North’, ‘Central’ and ‘South’ and re-label the values accordingly. This
reduces the number of categories in the variable region from five to
three. \sphinxstylestrong{NOTE: Any grouping should be some logical grouping and not a
random joining of categories.} Examples would be grouping districts
into provinces, municipalities into districts, or clean water categories
together. Grouping all small regions without geographical proximity
together is not necessarily the best option from the utility
perspective. Table 5.2 illustrates this with a very simplified example
dataset. Before recoding, three individuals have distinct keys, whereas
after recoding (grouping ‘Region 1’ and ‘Region 2’ into ‘North’, ‘Region
3’ into ‘Central’ and ‘Region 4’ and ‘Region 5’ into ‘South’), the
number of distinct keys is reduced to four and the frequency of every
key is at least two, based on the three selected quasi-identifiers. The
frequency counts of the keys \(f_{k}\) are shown in the last column
of Table 5.2. An intruder would find at least two individuals for each
key and cannot distinguish any more between individuals 1 \textendash{} 3,
individuals 4 and 6, individuals 5 and 7 and individuals 8 \textendash{} 10, based
on the selected key variables.

Table 5.2: Illustration of effect of recoding on frequency counts of
keys


\begin{savenotes}\sphinxattablestart
\centering
\begin{tabulary}{\linewidth}[t]{|T|T|T|T|T|}
\hline
\sphinxstartmulticolumn{5}%
\begin{varwidth}[t]{\sphinxcolwidth{5}{5}}
\sphinxstyletheadfamily Before recoding
\par
\vskip-\baselineskip\vbox{\hbox{\strut}}\end{varwidth}%
\sphinxstopmulticolumn
\\
\hline
\sphinxstyleemphasis{Individual}
&
\sphinxstyleemphasis{Region}
&
\sphinxstyleemphasis{Gender}
&
\sphinxstyleemphasis{Religion}
&
f\_k
\\
\hline
1
&
Region 1
&
Female
&
Catholic
&
1
\\
\hline
2
&
Region 2
&
Female
&
Catholic
&
2
\\
\hline
3
&
Region 2
&
Female
&
Catholic
&
2
\\
\hline
4
&
Region 3
&
Female
&
Protestant
&
2
\\
\hline
5
&
Region 3
&
Male
&
Protestant
&
1
\\
\hline
6
&
Region 3
&
Female
&
Protestant
&
2
\\
\hline
7
&
Region 3
&
Male
&
Protestant
&
2
\\
\hline
8
&
Region 4
&
Male
&
Muslim
&
2
\\
\hline
9
&
Region 4
&
Male
&
Muslim
&
2
\\
\hline
10
&
Region 5
&
Male
&
Muslim
&
1
\\
\hline\sphinxstartmulticolumn{5}%
\begin{varwidth}[t]{\sphinxcolwidth{5}{5}}
After recoding
\par
\vskip-\baselineskip\vbox{\hbox{\strut}}\end{varwidth}%
\sphinxstopmulticolumn
\\
\hline
\sphinxstyleemphasis{Individual}
&
\sphinxstyleemphasis{Region}
&
\sphinxstyleemphasis{Gender}
&
\sphinxstyleemphasis{Religion}
&
f\_k
\\
\hline
1
&
North
&
Female
&
Catholic
&
3
\\
\hline
2
&
North
&
Female
&
Catholic
&
3
\\
\hline
3
&
North
&
Female
&
Catholic
&
3
\\
\hline
4
&
Central
&
Female
&
Protestant
&
2
\\
\hline
5
&
Central
&
Male
&
Protestant
&
2
\\
\hline
6
&
Central
&
Female
&
Protestant
&
2
\\
\hline
7
&
Central
&
Male
&
Protestant
&
2
\\
\hline
8
&
South
&
Male
&
Muslim
&
3
\\
\hline
9
&
South
&
Male
&
Muslim
&
3
\\
\hline
10
&
South
&
Male
&
Muslim
&
3
\\
\hline
\end{tabulary}
\par
\sphinxattableend\end{savenotes}

Recoding is commonly the first step in an anonymization process. It can
be used to reduce the number of unique combinations of values of key
variables. This generally increases the frequency counts for most keys
and reduces the risk of disclosure. The reduction in the number of
possible combinations is illustrated in Table 5.3 with the
quasi-identifiers “region”, “marital status” and “age”. Table 5.3 shows
the number of categories of each variable and the number of
theoretically possible combinations, which is the product of the number
of categories of each quasi-identifier, before and after recoding. “Age”
is interpreted as a semi-continuous variable and treated as a
categorical variable. The number of possible combinations and hence the
risk for re-identification are reduced greatly by recoding. One should
bear in mind that the number of possible combinations is a theoretical
number; in practice, these may include very unlikely combinations such
as age = 3 and marital status = widow and the actual number of
combinations in a dataset may be lower.

Table 5.3: Illustration of the effect of recoding on the theoretically
possible number of combinations an a dataset


\begin{savenotes}\sphinxattablestart
\centering
\begin{tabular}[t]{|*{5}{\X{1}{5}|}}
\hline
\sphinxstyletheadfamily \begin{quote}

Number of
\end{quote}

categories
&\sphinxstyletheadfamily 
Region
&\sphinxstyletheadfamily \begin{quote}

Marital
\end{quote}

status
&\sphinxstyletheadfamily 
Age
&\sphinxstyletheadfamily \begin{quote}

Possible
\end{quote}

combination
s
\\
\hline
before
recoding
&
20
&
8
&
100
&
16,000
\\
\hline
after
recoding
&
6
&
6
&
15
&
540
\\
\hline
\end{tabular}
\par
\sphinxattableend\end{savenotes}

The main parameters for global recoding are the size of the new groups,
as well as defining which values are grouped together in new categories.
\sphinxstylestrong{NOTE: Care should be taken to choose new categories in line with the
data use of the end users and to minimize information loss as a result
of recoding.} We illustrate this with three examples:
\begin{itemize}
\item {} 
Age variable: The categories of age should be chosen so that they
still allow data users to make calculations relevant for the subject
being studied. For example, if indicators need to be calculated for
children of school going ages 6 \textendash{} 11 and 12 \textendash{} 17, and age needs to be
grouped to reduce risk, then care should be taken to create age
intervals that still allow the calculations to be made. A
satisfactory grouping could be, for example, 0 \textendash{} 5, 6 \textendash{} 11, 12 \textendash{} 17,
etc., whereas a grouping 0 \textendash{} 10, 11 \textendash{} 15, 16 \textendash{} 18 would destroy the
data utility for these users. While it is common practice to create
intervals (groups) of equal width (size), it is also possible (if
data users require this) to recode only part of the variables and
leave some values as they were originally. This could be done, for
example, by recoding all ages above 20, but leaving those below 20 as
they are. If SDC methods other than recoding will be used later or in
a next step, then care should be taken when applying recoding to only
part of the distribution, as this might increase the information loss
due to the other methods, since the grouping does not protect the
ungrouped variables. Partial recoding followed by suppression methods
such as local suppression may, for instance, leads to a higher number
of suppressions than desired or necessary in case the recoding is
done for the entire value range (see the next section on local
suppression). In the example above, the number of suppressions of
values below 20 will likely be higher than for values in the recoded
range. The disproportionately high number of suppressions in this
range of values that are not recoded can lead to higher utility loss
for these groups.

\item {} 
Geographic variables: If the original data specify administrative
level information in detail, e.g., down to municipality level, then
potentially those lower levels could be recoded or aggregated into
higher administrative levels, e.g., province, to reduce risk. In
doing so, the following should be noted: Grouping municipalities into
abstract levels that intersect different provinces would make data
analysis at the municipal or provincial level challenging. Care
should be taken to understand what the user requires and the
intention of the study. If a key component of the survey is to
conduct analysis at the municipal level, then aggregating up to
provincial level could damage the utility of the data for the user.
Recoding should be applied if the level of detail in the data is not
necessary for most data users and to avoid an extensive number of
suppressions when using other SDC methods subsequently. If the users
need information at a more detailed level, other methods such as
perturbative methods might provide a better solution than recoding.

\item {} 
Toilet facility: An example of a situation where a high level of
detail might not be necessary and recoding may do very little harm to
utility is the case of a detailed household toilet facility variable
that lists responses for 20 types of toilets. Researchers may only
need to distinguish between improved and unimproved toilet facilities
and may not require the exact classification of up to 20 types.
Detailed information of toilet types can be used to re-identify
households, while recoding to two categories \textendash{} improved and
unimproved facilities \textendash{} reduces the re-identification risk and in
this context, hardly reduces data utility. This approach can be
applied to any variable with many categories where data users are not
interested in detail, but rather in some aggregate categories.
Recoding addresses aggregation for the data users and at the same
time protects the microdata. Important is to take stock of the
aggregations used by data users.

\end{itemize}

Recoding should be applied only if removing the detailed information in
the data will not harm most data users. If the users need information at
a more detailed level, then recoding is not appropriate and other
methods such as perturbative methods might work better.

In \sphinxstyleemphasis{sdcMicro} there are different options for global recoding. In the
following paragraphs, we give examples of global recoding with the
functions groupVars() and globalRecode(). The function groupVars() is
generally used for categorical variables and the function globalRecode()
for continuous variables. Finally, we discuss the use of rounding to
reduce the detail in continuous variables.

\sphinxstyleemphasis{Recoding a categorical variable using the sdcMicro function
groupVars()}

Assume that an object of class \sphinxstyleemphasis{sdcMicro} was created, which is called
“sdcInitial” %
\begin{footnote}[2]\sphinxAtStartFootnote
Here the \sphinxstyleemphasis{sdcMicro} object “sdcIntial“ contains a dataset with 2,500
individuals and 103 variables. We selected five quasi-identifiers:
“sizeRes”, “age”, “gender”, “region”, and “ethnicity”.
%
\end{footnote} (see Section 7.5 how to create
objects of class \sphinxstyleemphasis{sdcMicro}). In Example 5.1, the variable “sizeRes” has
four different categories: ‘capital, large city’, ‘small city’, town’,
and ‘countryside’). The first three are recoded or regrouped as ‘urban’
and the category ‘countryside’ is renamed ‘rural’. In the function
arguments, we specify the categories to be grouped (before) and the
names of the categories after recoding (after). It is important that the
vectors “before” and “after” have the same length. Therefore, we have to
repeat ‘urban’ three times in the “after” vector to match the three
different values that are recoded to ‘urban’. \sphinxstylestrong{NOTE: the function
groupVars() works only for variables of class factor.} We refer to
Section 7.4 on classes in \sphinxstyleemphasis{R} and how to change the class of a variable.

Example 5.1: Using the sdcMicro function groupVars() to recode a
categorical variable

\fvset{hllines={, ,}}%
\begin{sphinxVerbatim}[commandchars=\\\{\}]
\PYG{c+c1}{\PYGZsh{} Frequencies of sizeRes before recoding}
\PYG{k+kp}{table}\PYG{p}{(}sdcInitial\PYG{o}{@}manipKeyVars\PYG{o}{\PYGZdl{}}sizeRes\PYG{p}{)}
\PYG{c+c1}{\PYGZsh{}\PYGZsh{} capital, large city          small city             town       countryside}
\PYG{c+c1}{\PYGZsh{}\PYGZsh{}                 686                 310              146              1358}

\PYG{c+c1}{\PYGZsh{} Recode urban}
sdcInitial  \PYG{o}{\PYGZlt{}\PYGZhy{}}  groupVars\PYG{p}{(}obj \PYG{o}{=} sdcInitial\PYG{p}{,} var \PYG{o}{=} \PYG{k+kt}{c}\PYG{p}{(}\PYG{l+s}{\PYGZdq{}}\PYG{l+s}{sizeRes\PYGZdq{}}\PYG{p}{)}\PYG{p}{,} before \PYG{o}{=} \PYG{k+kt}{c}\PYG{p}{(}\PYG{l+s}{\PYGZdq{}}\PYG{l+s}{capital, large city\PYGZdq{}}\PYG{p}{,} \PYG{l+s}{\PYGZdq{}}\PYG{l+s}{small city\PYGZdq{}}\PYG{p}{,} \PYG{l+s}{\PYGZdq{}}\PYG{l+s}{town\PYGZdq{}}\PYG{p}{)}\PYG{p}{,} after \PYG{o}{=} \PYG{k+kt}{c}\PYG{p}{(}\PYG{l+s}{\PYGZdq{}}\PYG{l+s}{urban\PYGZdq{}}\PYG{p}{,} \PYG{l+s}{\PYGZdq{}}\PYG{l+s}{urban\PYGZdq{}}\PYG{p}{,} \PYG{l+s}{\PYGZdq{}}\PYG{l+s}{urban\PYGZdq{}}\PYG{p}{)}\PYG{p}{)}

\PYG{c+c1}{\PYGZsh{} Recode rural}
sdcInitial  \PYG{o}{\PYGZlt{}\PYGZhy{}}  groupVars\PYG{p}{(}obj \PYG{o}{=} sdcInitial\PYG{p}{,} var \PYG{o}{=} \PYG{k+kt}{c}\PYG{p}{(}\PYG{l+s}{\PYGZdq{}}\PYG{l+s}{sizeRes\PYGZdq{}}\PYG{p}{)}\PYG{p}{,} before \PYG{o}{=} \PYG{k+kt}{c}\PYG{p}{(}\PYG{l+s}{\PYGZdq{}}\PYG{l+s}{countryside\PYGZdq{}}\PYG{p}{)}\PYG{p}{,} after \PYG{o}{=} \PYG{k+kt}{c}\PYG{p}{(}\PYG{l+s}{\PYGZdq{}}\PYG{l+s}{rural\PYGZdq{}}\PYG{p}{)}\PYG{p}{)}

\PYG{c+c1}{\PYGZsh{} Frequencies of sizeRes before recoding}
\PYG{k+kp}{table}\PYG{p}{(}sdcInitial\PYG{o}{@}manipKeyVars\PYG{o}{\PYGZdl{}}sizeRes\PYG{p}{)}
\PYG{c+c1}{\PYGZsh{}\PYGZsh{} urban rural}
\PYG{c+c1}{\PYGZsh{}\PYGZsh{}  1142  1358}
\end{sphinxVerbatim}

Figure 5.1 illustrates the effect of recoding the variable “sizeRes” and
show respectively the frequency counts before and after recoding. We see
that the number of categories has reduced from 4 to 2 and the small
categories (‘small city’ and ‘town’) have disappeared.

\noindent\sphinxincludegraphics[width=6.5in,height=3.25556in]{{image3}.png}

Figure 5.1 Effect of recoding \textendash{} frequency counts before and after
recoding

\sphinxstyleemphasis{Recoding a continuous variable using the sdcMicro function:
globalRecode()}

Global recoding of numerical (continuous) variables can be achieved in
\sphinxstyleemphasis{sdcMicro} by using the function globalRecode(), which allows specifying
a vector with the break points between the intervals. Recoding a
continuous variable changes it into a categorical variable. One can
additionally specify a vector of labels for the new categories. By
default, the labels are the intervals, e.g., “(0, 10{]}”. Example 5.2
shows how to recode the variable age in 10-year intervals for age values
between 0 and 100. \sphinxstylestrong{NOTE: Values that fall outside the specified
intervals are assigned a missing value (NA).} Therefore, the intervals
should cover the entire value range of the variable.

Example 5.2: Using the \sphinxstyleemphasis{sdcMicro} function globalRecode() to recode a
continuous variable (age)

\fvset{hllines={, ,}}%
\begin{sphinxVerbatim}[commandchars=\\\{\}]
sdcInitial \PYG{o}{\PYGZlt{}\PYGZhy{}} globalRecode\PYG{p}{(}sdcInitial\PYG{p}{,} column \PYG{o}{=} \PYG{k+kt}{c}\PYG{p}{(}\PYG{l+s}{\PYGZsq{}}\PYG{l+s}{age\PYGZsq{}}\PYG{p}{)}\PYG{p}{,} breaks \PYG{o}{=} \PYG{l+m}{10} \PYG{o}{*} \PYG{k+kt}{c}\PYG{p}{(}\PYG{l+m}{0}\PYG{o}{:}\PYG{l+m}{10}\PYG{p}{)}\PYG{p}{)}

\PYG{c+c1}{\PYGZsh{} Frequencies of age after recoding}
\PYG{k+kp}{table}\PYG{p}{(}sdcInitial\PYG{o}{@}manipKeyVars\PYG{o}{\PYGZdl{}}age\PYG{p}{)}
\PYG{c+c1}{\PYGZsh{}\PYGZsh{}   (0,10]  (10,20]  (20,30]  (30,40]  (40,50]  (50,60]  (60,70]  (70,80]  (80,90]  (90,100]}
\PYG{c+c1}{\PYGZsh{}\PYGZsh{}      462      483      344      368      294      214      172       94           26         3}
\end{sphinxVerbatim}

Figure 5.2 shows the effect of recoding the variable “age”.

\noindent\sphinxincludegraphics[width=6.5in,height=3.25556in]{{image4}.png}

Figure 5.2 Age variable before and after recoding

Instead of creating intervals of equal width, we can also create
intervals of unequal width. This is illustrated in Example 5.3, where we
use the age groups 1-5, 6-11, 12-17, 18-21, 22-25, 26-49, 50-64 and 65+.
In this example, this is a useful step, since even after recoding in
10-year intervals, the categories with high age values have low
frequencies. We chose the intervals by respecting relevant school age
and employment age values (e.g., retirement age is 65 in this example)
such that the data can still be used for common research on education
and employment. Figure 5.3 shows the effect of recoding the variable
“age”.

Example 5.3: Using globalRecode() to create intervals of unequal width

\fvset{hllines={, ,}}%
\begin{sphinxVerbatim}[commandchars=\\\{\}]
sdcInitial \PYG{o}{\PYGZlt{}\PYGZhy{}} globalRecode\PYG{p}{(}sdcInitial\PYG{p}{,} column \PYG{o}{=} \PYG{k+kt}{c}\PYG{p}{(}\PYG{l+s}{\PYGZsq{}}\PYG{l+s}{age\PYGZsq{}}\PYG{p}{)}\PYG{p}{,} breaks \PYG{o}{=} \PYG{k+kt}{c}\PYG{p}{(}\PYG{l+m}{0}\PYG{p}{,} \PYG{l+m}{5}\PYG{p}{,} \PYG{l+m}{11}\PYG{p}{,} \PYG{l+m}{17}\PYG{p}{,} \PYG{l+m}{21}\PYG{p}{,} \PYG{l+m}{25}\PYG{p}{,} \PYG{l+m}{49}\PYG{p}{,} \PYG{l+m}{65}\PYG{p}{,} \PYG{l+m}{100}\PYG{p}{)}\PYG{p}{)}

\PYG{c+c1}{\PYGZsh{} Frequencies of age after recoding}
\PYG{k+kp}{table}\PYG{p}{(}sdcInitial\PYG{o}{@}manipKeyVars\PYG{o}{\PYGZdl{}}age\PYG{p}{)}
\PYG{c+c1}{\PYGZsh{}\PYGZsh{}    (0,5]   (5,11]  (11,17]  (17,21]  (21,25]  (25,49]  (49,65] (65,100]}
\PYG{c+c1}{\PYGZsh{}\PYGZsh{}      192      317      332      134      142      808      350      185}
\end{sphinxVerbatim}

\noindent\sphinxincludegraphics[width=6.5in,height=3.25556in]{{image5}.png}

Figure 5.3 Age variable before and after recoding

Caution about using the globalRecode() function in \sphinxstyleemphasis{sdcMicro}: In the
current implementation of \sphinxstyleemphasis{sdcMicro}, the intervals are defined as
\sphinxstylestrong{left-open}. In mathematical terms, this means that, in our example,
age 0 is excluded from the specified intervals. In interval notation,
this is denoted as (0, 5{]} (as in x-axis labels in Figure 5.2 and Figure
5.3 graph labels for the recoded variable). The interval (0, 5{]} is
interpreted as from 0 to 5 and does not include 0, but does include 5.
\sphinxstyleemphasis{R} recodes values that are not contained in any of the intervals as
missing (NA). This implementation would set in our example all age
values 0 (children under 1 year) to missing and could potentially mean a
large data loss. The globalRecode() function allows only constructing
intervals, which are left-open. This may not be a desirable result and
the loss of the zero ages from the data is clearly problematic for a
real-world dataset.

To construct \sphinxstylestrong{right-open} intervals, e.g., in our example, for age
intervals {[}0,14), {[}15, 65), {[}66, 100), we present two alternatives for
global recoding:
\begin{itemize}
\item {} 
A work-around for semi-continuous variables %
\begin{footnote}[3]\sphinxAtStartFootnote
This approach works only for semi-continuous variables, because in
the case of continuous variables, there might be values that are
between the lower interval boundary and the lower interval boundary
minus the small number. For example, using this for income, we would
have an interval (9999, 19999{]} and the value 9999.5 would be
misclassified as belonging to the interval {[}10000, 19999{]}.
%
\end{footnote}
that would allow for the globalRecode() to be used would be
subtracting a small number from the boundary intervals, thus allowing
the desired intervals to be created. In the following example,
subtracting 0.1 from each interval forces globalRecode() to include 0
in the lowest interval and allow for breaks where we want them. We
set the upper interval boundary to be larger than the maximum value
for the “age” variable. We can use the option \sphinxstyleemphasis{labels} to define
clear labels for the new categories. This is illustrated in Example
5.4.

\end{itemize}

Example 5.4: Constructing right-open intervals for semi-continuous
variables using built-in \sphinxstyleemphasis{sdcMicro} function globalRecode()

\fvset{hllines={, ,}}%
\begin{sphinxVerbatim}[commandchars=\\\{\}]
sdcInitial \PYG{o}{\PYGZlt{}\PYGZhy{}} globalRecode\PYG{p}{(}sdcInitial\PYG{p}{,} column \PYG{o}{=} \PYG{k+kt}{c}\PYG{p}{(}\PYG{l+s}{\PYGZsq{}}\PYG{l+s}{age\PYGZsq{}}\PYG{p}{)}\PYG{p}{,} breaks \PYG{o}{=} \PYG{k+kt}{c}\PYG{p}{(}\PYG{l+m}{\PYGZhy{}0.1}\PYG{p}{,} \PYG{l+m}{14.9}\PYG{p}{,} \PYG{l+m}{64.9}\PYG{p}{,} \PYG{l+m}{99.9}\PYG{p}{)}\PYG{p}{,} labels \PYG{o}{=} \PYG{k+kt}{c}\PYG{p}{(}\PYG{l+s}{\PYGZsq{}}\PYG{l+s}{[0,15)\PYGZsq{}}\PYG{p}{,} \PYG{l+s}{\PYGZsq{}}\PYG{l+s}{[15,65)\PYGZsq{}}\PYG{p}{,} \PYG{l+s}{\PYGZsq{}}\PYG{l+s}{[65,100)\PYGZsq{}}\PYG{p}{)}\PYG{p}{)}
\end{sphinxVerbatim}
\begin{itemize}
\item {} 
It is also possible to use \sphinxstyleemphasis{R} code to manually recode the variables
without using \sphinxstyleemphasis{sdcMicro} functions. When using the built-in
\sphinxstyleemphasis{sdcMicro} functions, the change in risk after recoding is
automatically recalculated, but if recoded manually it is not. In
this case, we need to take an extra step and recalculate the risk
after manually changing the variables in the \sphinxstyleemphasis{sdcMicro} object. This
approach is also valid for continuous variables and is illustrated in
Example 5.5.

\end{itemize}

Example 5.5: Constructing intervals for semi-continuous and continuous
variables using manual recoding in \sphinxstyleemphasis{R}

\fvset{hllines={, ,}}%
\begin{sphinxVerbatim}[commandchars=\\\{\}]
  \PYG{c+c1}{\PYGZsh{} Group age 0\PYGZhy{}14}
sdcInitial\PYG{o}{@}manipKeyVars\PYG{o}{\PYGZdl{}}age\PYG{p}{[}sdcInitial\PYG{o}{@}manipKeyVars\PYG{o}{\PYGZdl{}}age \PYG{o}{\PYGZgt{}=} \PYG{l+m}{0} \PYG{o}{\PYGZam{}}
sdcInitial\PYG{o}{@}manipKeyVars\PYG{o}{\PYGZdl{}}age \PYG{o}{\PYGZlt{}} \PYG{l+m}{15}\PYG{p}{]} \PYG{o}{\PYGZlt{}\PYGZhy{}} \PYG{l+m}{0}

\PYG{c+c1}{\PYGZsh{} Group age 15\PYGZhy{}64}
sdcInitial\PYG{o}{@}manipKeyVars\PYG{o}{\PYGZdl{}}age\PYG{p}{[}sdcInitial\PYG{o}{@}manipKeyVars\PYG{o}{\PYGZdl{}}age \PYG{o}{\PYGZgt{}=} \PYG{l+m}{15} \PYG{o}{\PYGZam{}}
sdcInitial\PYG{o}{@}manipKeyVars\PYG{o}{\PYGZdl{}}age \PYG{o}{\PYGZlt{}} \PYG{l+m}{65}\PYG{p}{]} \PYG{o}{\PYGZlt{}\PYGZhy{}} \PYG{l+m}{1}

\PYG{c+c1}{\PYGZsh{} Group age 65\PYGZhy{}100}
sdcInitial\PYG{o}{@}manipKeyVars\PYG{o}{\PYGZdl{}}age\PYG{p}{[}sdcInitial\PYG{o}{@}manipKeyVars\PYG{o}{\PYGZdl{}}age \PYG{o}{\PYGZgt{}=} \PYG{l+m}{65} \PYG{o}{\PYGZam{}}
sdcInitial\PYG{o}{@}manipKeyVars\PYG{o}{\PYGZdl{}}age \PYG{o}{\PYGZlt{}=} \PYG{l+m}{100}\PYG{p}{]} \PYG{o}{\PYGZlt{}\PYGZhy{}} \PYG{l+m}{2}

\PYG{c+c1}{\PYGZsh{} Add labels for the new values}
sdcInitial\PYG{o}{@}manipKeyVars\PYG{o}{\PYGZdl{}}age \PYG{o}{\PYGZlt{}\PYGZhy{}}\PYG{k+kp}{ordered}\PYG{p}{(}sdcInitial\PYG{o}{@}manipKeyVars\PYG{o}{\PYGZdl{}}age\PYG{p}{,}
levels \PYG{o}{=} \PYG{k+kt}{c}\PYG{p}{(}\PYG{l+m}{0}\PYG{p}{,}\PYG{l+m}{1}\PYG{p}{,}\PYG{l+m}{2}\PYG{p}{)}\PYG{p}{,} labels \PYG{o}{=} \PYG{k+kt}{c}\PYG{p}{(}\PYG{l+s}{\PYGZdq{}}\PYG{l+s}{0\PYGZhy{}14\PYGZdq{}}\PYG{p}{,} \PYG{l+s}{\PYGZdq{}}\PYG{l+s}{15\PYGZhy{}64\PYGZdq{}}\PYG{p}{,} \PYG{l+s}{\PYGZdq{}}\PYG{l+s}{65\PYGZhy{}100\PYGZdq{}}\PYG{p}{)}\PYG{p}{)}

\PYG{c+c1}{\PYGZsh{} Recalculate risk after manual manipulation}
sdcInitial \PYG{o}{\PYGZlt{}\PYGZhy{}} calcRisks\PYG{p}{(}sdcInitial\PYG{p}{)}
\end{sphinxVerbatim}


\subsubsection{Top and bottom coding}
\label{\detokenize{anon_methods:top-and-bottom-coding}}
Top and bottom coding are similar to global recoding, but instead of
recoding all values, only the top and/or bottom values of the
distribution or categories are recoded. This can be applied only to
ordinal categorical variables and (semi-)continuous variables, since the
values have to be at least ordered. Top and bottom coding is especially
useful if the bulk of the values lies in the center of the distribution
with the peripheral categories having only few observations (outliers).
Examples are age and income; for these variables, there will often be
only a few observations above certain thresholds, typically at the tails
of the distribution. The fewer the observations within a category, the
higher the identification risk. One solution could be grouping the
values at the tails of the distribution into one category. This reduces
the risk for those observations, and, importantly, does so without
reducing the data utility for the other observations in the
distribution.

Deciding where to apply the threshold and what observations should be
grouped requires:
\begin{itemize}
\item {} 
Reviewing the overall distribution of the variable to identify at
which point the frequencies drop below the desired number of
observations and identify outliers in the distribution. Figure 5.4
shows the distribution of the age variable and suggests 65 (red
vertical line) for the top code age.

\item {} 
Taking into account the intended use of the data and the purpose for
which the survey was conducted. For example, if the data are
typically used to measure labor force participation for those aged 15
to 64, then top and bottom coding should not interfere with the
categories 15 to 64. Otherwise the analyst would find it impossible
to create the desired measures for which the data were intended. In
the example, we consider this and code all age larger than 64.

\end{itemize}

\noindent\sphinxincludegraphics[width=6.5in,height=3.25556in]{{image6}.png}

Figure 5.4: Utilizing the frequency distribution of variable age to
determine threshold for top coding

Top and bottom coding can be easily done with the function
topBotCoding() in \sphinxstyleemphasis{sdcMicro}. Top coding and bottom coding cannot be
done simultaneously in \sphinxstyleemphasis{sdcMicro}. Example 5.6 illustrates how to recode
values of age higher than 64 and values of age lower than 5; 65 and 5
replace the values respectively. To construct several top or bottom
coding categories, e.g., age 65 \textendash{} 80 and higher than age 80, one can use
the groupVars() function in \sphinxstyleemphasis{sdcMicro} or manual recoding as described
in the previous subsection.

Example 5.6: Top coding and bottom coding in \sphinxstyleemphasis{sdcMicro} using
topBotCoding() function

\fvset{hllines={, ,}}%
\begin{sphinxVerbatim}[commandchars=\\\{\}]
\PYG{c+c1}{\PYGZsh{} Top coding at age 65}
sdcInitial \PYG{o}{\PYGZlt{}\PYGZhy{}} topBotCoding\PYG{p}{(}obj \PYG{o}{=} sdcInitial\PYG{p}{,} value \PYG{o}{=} \PYG{l+m}{65}\PYG{p}{,} replacement \PYG{o}{=} \PYG{l+m}{65}\PYG{p}{,} kind \PYG{o}{=} \PYG{l+s}{\PYGZsq{}}\PYG{l+s}{top\PYGZsq{}}\PYG{p}{,} column \PYG{o}{=} \PYG{l+s}{\PYGZsq{}}\PYG{l+s}{age\PYGZsq{}}\PYG{p}{)}

\PYG{c+c1}{\PYGZsh{} Bottom coding at age 5}
sdcInitial \PYG{o}{\PYGZlt{}\PYGZhy{}} topBotCoding\PYG{p}{(}obj \PYG{o}{=} sdcInitial\PYG{p}{,} value \PYG{o}{=} \PYG{l+m}{5}\PYG{p}{,} replacement \PYG{o}{=} \PYG{l+m}{5}\PYG{p}{,} kind \PYG{o}{=} \PYG{l+s}{\PYGZsq{}}\PYG{l+s}{bottom\PYGZsq{}}\PYG{p}{,} column \PYG{o}{=} \PYG{l+s}{\PYGZsq{}}\PYG{l+s}{age\PYGZsq{}}\PYG{p}{)}
\end{sphinxVerbatim}


\subsubsection{Rounding}
\label{\detokenize{anon_methods:rounding}}
Rounding is similar to grouping, but used for continuous variables.
Rounding is useful to prevent exact matching with external data sources.
In addition, it can be used to reduce the level of detail in the data.
Examples are removing decimal figures or rounding to the nearest 1,000.

The next section discusses the method local suppression. Recoding is
often used before local suppression to reduce the number of necessary
suppressions.

\begin{sphinxadmonition}{note}{Recommended Reading Material on Recoding}

Hundepool, Anco, Josep Domingo-Ferrer, Luisa Franconi, Sarah Giessing,
Rainer Lenz, Jane Naylor, Eric Schulte Nordholt, Giovanni Seri, and
Peter Paul de Wolf. 2006. \sphinxstyleemphasis{Handbook on Statistical Disclosure Control.}
ESSNet SDC. \sphinxurl{http://neon.vb.cbs.nl/casc/handbook.htm}.

Hundepool, Anco, Josep Domingo-Ferrer, Luisa Franconi, Sarah Giessing,
Eric Schulte Nordholt, Keith Spicer, and Peter Paul de Wolf. 2012.
\sphinxstyleemphasis{Statistical Disclosure Control.} Chichester: John Wiley \& Sons Ltd.
doi:10.1002/9781118348239.

Templ, Matthias, Bernhard Meindl, Alexander Kowarik, and Shuang Chen.
2014. Statistical Disclosure Control (SDCMicro).
\sphinxurl{http://www.ihsn.org/home/software/disclosure-control-toolbox}. (accessed
November 13, 2014).

De Waal, A.G., and Willenborg, L.C.R.J. 1999. \sphinxstyleemphasis{Information loss through
global recoding and local suppression}. Netherlands Official Statistics,
14:17-20, 1999. Special issue on SDC
\end{sphinxadmonition}


\subsection{Local suppression}
\label{\detokenize{anon_methods:local-suppression}}
It is common in surveys to encounter values for certain variables or
combinations of quasi-identifiers (keys) that are shared by very few
individuals. When this occurs, the risk of re-identification for those
respondents is higher than the rest of the respondents (see
\(k\)-anonymity in Section 4.5.2). Often local suppression is used
after reducing the number of keys in the data by recoding the
appropriate variables. Recoding reduces the number of necessary
suppressions as well as the computation time needed for suppression.
Suppression of values means that values of a variable are replaced by a
missing value (NA in \sphinxstyleemphasis{R}). Section 4.5.2 on \(k\)-anonymity
discusses how missing values influence frequency counts and
\(k\)-anonymity. It is important to note that not all values for all
individuals of a certain variable are suppressed, which would be the
case when removing a direct identifier, such as “name”; only certain
values for a particular variable and a particular respondent or set of
respondents are suppressed. This is illustrated in the following example
and Table 5.4.

Table 5.4 presents a dataset with seven respondents and three
quasi-identifiers. The combination \{‘female’, ‘rural’, ‘higher’\} for the
variables “gender”, “region” and “education” is an unsafe combination,
since it is unique in the sample. By suppressing either the value
‘female’ or ‘higher’, the respondent cannot be distinguished from the
other respondents anymore, since that respondent shares the same
combination of key variables with at least three other respondents. Only
the value in the unsafe combination of the single respondent at risk is
suppressed, not the values for the same variable of the other
respondents. The freedom to choose which value to suppress can be used
to minimize the total number of suppressions and hence the information
loss. In addition, if one variable is very important to the user, we can
choose not to suppress values of this variable, unless strictly
necessary. In the example, we can choose between suppressing the value
‘female’ or ‘higher’ to achieve a safe data file; we chose to suppress
‘higher’. This choice should be made taking into account the needs of
data users. In this example we find “gender” more important than
“education”.

Table 5.4: Local suppression illustration - sample data before and after
suppression


\begin{savenotes}\sphinxattablestart
\centering
\begin{tabular}[t]{|*{7}{\X{1}{7}|}}
\hline
\sphinxstyletheadfamily 
Variable
&\sphinxstartmulticolumn{3}%
\begin{varwidth}[t]{\sphinxcolwidth{3}{7}}
\sphinxstyletheadfamily Before local suppression
\par
\vskip-\baselineskip\vbox{\hbox{\strut}}\end{varwidth}%
\sphinxstopmulticolumn
&\sphinxstartmulticolumn{3}%
\begin{varwidth}[t]{\sphinxcolwidth{3}{7}}
\sphinxstyletheadfamily After local suppression
\par
\vskip-\baselineskip\vbox{\hbox{\strut}}\end{varwidth}%
\sphinxstopmulticolumn
\\
\hline
ID
&
Gender
&
Region
&
Educat
ion*
&
Gender
&
Region
&
Educat
ion
\\
\hline
1
&
female
&
rural
&
higher
&
female
&
rural
&\begin{quote}

NA/miss
ing
\end{quote}

%
\begin{footnote}[5]\sphinxAtStartFootnote
In \sphinxstyleemphasis{R} suppressed values are recoded NA, the standard missing value
code.
%
\end{footnote}
\\
\hline
2
&
male
&
rural
&
higher
&
male
&
rural
&
higher
\\
\hline
3
&
male
&
rural
&
higher
&
male
&
rural
&
higher
\\
\hline
4
&
male
&
rural
&
higher
&
male
&
rural
&
higher
\\
\hline
5
&
female
&
rural
&
lower
&
female
&
rural
&
lower
\\
\hline
6
&
female
&
rural
&
lower
&
female
&
rural
&
lower
\\
\hline
7
&
female
&
rural
&
lower
&
female
&
rural
&
lower
\\
\hline
\end{tabular}
\par
\sphinxattableend\end{savenotes}

Since continuous variables have a high number of unique values (e.g.,
income in dollars or age in years), \(k\)-anonymity and local
suppression are not suitable for continuous variables or variables with
a very high number of categories. A possible solution in those cases
might be to first recode to produce fewer categories (e.g., recoding age
in 10-year intervals or income in quintiles). Always keep in mind,
though, what effect any recoding will have on the utility of the data.

The \sphinxstyleemphasis{sdcMicro} package includes two functions for local suppression:
localSuppression() and localSupp(). The function localSuppression() is
most commonly used and allows the use of suppression on specified
quasi-identifiers to achieve a certain level of \(k\)-anonymity for
these quasi-identifiers. The algorithm used seeks to minimize the total
number of suppressions while achieving the required \(k\)-anonymity
threshold. By default, the algorithm is more likely to suppress values
of variables with many different categories or values, and less likely
to suppress variables with fewer categories. For example, the values of
a geographical variable, with 12 different areas, are more likely to be
suppressed than the values of the variable “gender”, which has typically
only two categories. If variables with many different values are
important for data utility and suppression is not desired for them, it
is possible to rank variables by importance in the localSuppression()
function and thus specify the order in which the algorithm will seek to
suppress values within quasi-identifiers to achieve \(k\)-anonymity.
The algorithm seeks to apply fewer suppressions to variables of high
importance than to variables with lower importance. Nevertheless,
suppressions in the variables with high importance might be inevitable
to achieve the required level of \(k\)-anonymity.

In Example 5.7, local suppression is applied to achieve the
\(k\)-anonymity threshold of 5 on the quasi-identifiers “gender”,
“region”, “religion”, “age” and “ethnicity” %
\begin{footnote}[6]\sphinxAtStartFootnote
Here the \sphinxstyleemphasis{sdcMicro} object “sdcIntial“ contains a dataset with 2,500
individuals and 103 variables. We selected five quasi-identifiers:
“sizeRes”, “age”, “gender”, “region”, and “ethnicity”.
%
\end{footnote}.
Without ranking the importance of the variables, the value of the
variable “age” is more likely to be suppressed, since this is the
variable with most categories. The variable “age” has 10 categories
after recoding. The variable “gender” is least likely to be suppressed,
since it has only two different values: ‘male’ and ‘female’. The other
variables have 4 (“sizeRes”), 2 (“region”), and 8 (“ethnicity”)
categories. After applying the localSuppression() function, we display
the number of suppressions per variable with the built-in print()
function with the option ‘ls’ for the local suppression output. As
expected, the variable “age” has most suppressions (80). In fact, only
the variable “ethnicity” of the other variables also needed suppressions
(8) to achieve the \(k\)-anonymity threshold of 5. The variable
“ethnicity” is the variable with the second highest number of
suppressions. Subsequently, we undo and redo local suppression on the
same data and reduce the number of suppressions on “age” by specifying
the importance vector with high importance (little suppression) on the
quasi-identifier “age”. We also assign importance to the variable
“gender”. This is done by specifying an importance vector. The values in
the importance vector can range from 1 to \(k\), the number of
quasi-identifiers. In our example \(k\) is equal to 5. Variables
with lower values in the importance vectors have high importance and,
when possible, receive fewer suppressions than variables with higher
values.

To assign high importance to the variables “age” and “gender”, we
specify the importance vector as c(5, 1, 1, 5, 5), with the order
according to the order of the specified variables in the \sphinxstyleemphasis{sdcMicro}
object. The effect is clear: there are no suppressions in the variables
“age” and “gender”. For that, the other variables, especially “sizeRes”
and “ethnicity”, received many more suppressions. The total number of
suppressed values has increased from 88 to 166. \sphinxstylestrong{NOTE: Fewer
suppressions in one variable increase the number of necessary
suppressions in other variables (cf.} \sphinxstylestrong{Example 5.7).} Generally, the
total number of suppressed values needed to achieve the required level
of \(k\)-anonymity increases when specifying an importance vector,
since the importance vector prevents to use the optimal suppression
pattern. The importance vector should be specified only in cases where
the variables with many categories play an important role in data
utility for the data users %
\begin{footnote}[7]\sphinxAtStartFootnote
This can be assessed with utility measures.
%
\end{footnote}.

Example 5.7: Application of local suppression with and without
importance vector

\fvset{hllines={, ,}}%
\begin{sphinxVerbatim}[commandchars=\\\{\}]
\PYG{c+c1}{\PYGZsh{} local suppression without importance vector}
sdcInitial \PYG{o}{\PYGZlt{}\PYGZhy{}} localSuppression\PYG{p}{(}sdcInitial\PYG{p}{,} k \PYG{o}{=} \PYG{l+m}{5}\PYG{p}{)}

\PYG{k+kp}{print}\PYG{p}{(}sdcInitial\PYG{p}{,} \PYG{l+s}{\PYGZsq{}}\PYG{l+s}{ls\PYGZsq{}}\PYG{p}{)}
\PYG{c+c1}{\PYGZsh{}\PYGZsh{}     KeyVar \textbar{} Suppressions (\PYGZsh{}) \textbar{} Suppressions (\PYGZpc{})}
\PYG{c+c1}{\PYGZsh{}\PYGZsh{}    sizeRes \textbar{}                0 \textbar{}            0.000}
\PYG{c+c1}{\PYGZsh{}\PYGZsh{}        age \textbar{}               80 \textbar{}            3.200}
\PYG{c+c1}{\PYGZsh{}\PYGZsh{}     gender \textbar{}                0 \textbar{}            0.000}
\PYG{c+c1}{\PYGZsh{}\PYGZsh{}     region \textbar{}                0 \textbar{}            0.000}
\PYG{c+c1}{\PYGZsh{}\PYGZsh{}  ethnicity \textbar{}                8 \textbar{}            0.320}

\PYG{c+c1}{\PYGZsh{} Undoing the supressions}
sdcInitial \PYG{o}{\PYGZlt{}\PYGZhy{}} undolast\PYG{p}{(}sdcInitial\PYG{p}{)}

\PYG{c+c1}{\PYGZsh{} Local suppression with importance vector to avoid suppressions in the first (gender) and fourth (age) variables}
sdcInitial \PYG{o}{\PYGZlt{}\PYGZhy{}} localSuppression\PYG{p}{(}sdcInitial\PYG{p}{,} importance \PYG{o}{=} \PYG{k+kt}{c}\PYG{p}{(}\PYG{l+m}{5}\PYG{p}{,} \PYG{l+m}{1}\PYG{p}{,} \PYG{l+m}{1}\PYG{p}{,} \PYG{l+m}{5}\PYG{p}{,} \PYG{l+m}{5}\PYG{p}{)}\PYG{p}{,} k \PYG{o}{=} \PYG{l+m}{5}\PYG{p}{)}
\PYG{k+kp}{print}\PYG{p}{(}sdcInitial\PYG{p}{,} \PYG{l+s}{\PYGZsq{}}\PYG{l+s}{ls\PYGZsq{}}\PYG{p}{)}
\PYG{c+c1}{\PYGZsh{}\PYGZsh{}     KeyVar \textbar{} Suppressions (\PYGZsh{}) \textbar{} Suppressions (\PYGZpc{})}
\PYG{c+c1}{\PYGZsh{}\PYGZsh{}    sizeRes \textbar{}               87 \textbar{}            3.480}
\PYG{c+c1}{\PYGZsh{}\PYGZsh{}        age \textbar{}                0 \textbar{}            0.000}
\PYG{c+c1}{\PYGZsh{}\PYGZsh{}     gender \textbar{}                0 \textbar{}            0.000}
\PYG{c+c1}{\PYGZsh{}\PYGZsh{}     region \textbar{}               17 \textbar{}            0.680}
\PYG{c+c1}{\PYGZsh{}\PYGZsh{}  ethnicity \textbar{}               62 \textbar{}            2.480}
\end{sphinxVerbatim}

Figure 5.5 demonstrates the effect of the required \(k\)-anonymity
threshold and the importance vector on the data utility by using several
labor market-related indicators from an I2D2 %
\begin{footnote}[8]\sphinxAtStartFootnote
I2D2 is a dataset with data related to the labor market.
%
\end{footnote}
dataset before and after anonymization. Figure 5.5 displays the relative
changes as a percentage of the initial value after re-computing the
indicators with the data to which local suppression was applied. The
indicators are the proportion of active females and males, and the
number of females and males of working age. The values computed from the
raw data were, respectively, 68\%, 12\%, 8,943 and 9,702. The vertical
line at 0 is the benchmark of no change. The numbers indicate the
required k-anonymity threshold (3 or 5) and the colors indicate the
importance vector: red (no symbol) is no importance vector, blue (with
* symbol) is high importance on the variable with the employment status
information and dark green (with + symbol) is high importance on the age
variable.

A higher \(k\)-anonymity threshold leads to greater information loss
(i.e., larger deviations from the original values of the indicators, the
5’s are further away from the benchmark of no change than the
corresponding 3’s) caused by local suppression. Reducing the number of
suppressions on the employment status variable by specifying an
importance vector does not improve the indicators. Instead, reducing the
number of suppressions on age greatly reduces the information loss.
Since specific age groups have a large influence on the computation of
these indicators (the rare cases are in the extremes and will be
suppressed), high suppression rates on age distort the indicators. It is
generally useful to compare utility measures (see Chapter 6) to specify
the importance vector, since the effects can be unpredictable.

\noindent\sphinxincludegraphics[width=6.5in,height=3.25556in]{{image7}.png}

Figure 5.5: Changes in labor market indicators after anonymization of
I2D2 data

The threshold of \(k\)-anonymity to be set depends on several
factors, which are amongst others: 1) the legal requirements for a safe
data file; 2) other methods that will be applied to the data; 3) the
number of suppressions and related information loss resulting from
higher thresholds; 4) the type of variable; 5) the sample weights and
sample size; and 6) the release type (see Chapter 3). Commonly applied
levels for the \(k\)-anonymity threshold are 3 and 5.

Table 5.5 illustrates the influence of the importance vector and
\(k\)-anonymity threshold on the running time, global risk after
suppression and total number of suppressions required to achieve this
\(k\)-anonymity threshold. The dataset contains about 63,000
individuals. The higher the \(k\)-anonymity threshold, the more
suppressions are needed and the lower the risk after local suppression
(expected number of re-identifications). In this particular example, the
computation time is shorter for higher thresholds. This is due the
higher number of necessary suppressions, which reduces the difficulty of
the search for an optimal suppression pattern.

The age variable is recoded in five-year intervals and has 20 age
categories. This is the variable with the highest number of categories.
Prioritizing the suppression of other variables leads to a higher total
number of suppressions and a longer computation time.

Table 5.5: How importance vectors and k-anonymity thresholds affect
running time and total number of suppressions


\begin{savenotes}\sphinxattablestart
\centering
\begin{tabulary}{\linewidth}[t]{|T|T|T|T|T|T|}
\hline
\sphinxstyletheadfamily 
Threshold
ld

k-anony
mity
&\sphinxstyletheadfamily 
Importance
nce
vector**
&\sphinxstyletheadfamily 
Total
number of
suppressi
ons
&\sphinxstyletheadfamily 
Number
of
suppressi
ons
age
&\sphinxstyletheadfamily 
Global
risk
measure
&\sphinxstyletheadfamily 
Running
time
(hours)
\\
\hline
3
&
none
(default)
&
6,676
&
5,387
&
293.0
&
11.8
\\
\hline
3
&
employmen
t
status
&
7,254
&
5,512
&
356.5
&
13.1
\\
\hline
3
&
age
variable
&
8,175
&
60
&
224.6
&
4.5
\\
\hline
5
&
none
(default)
&
9,971
&
7,894
&
164.6
&
8.5
\\
\hline
5
&
employmen
t
status
&
11,668
&
8,469
&
217.0
&
10.2
\\
\hline
5
&
age
variable
&
13,368
&
58
&
123.1
&
3.8
\\
\hline
\end{tabulary}
\par
\sphinxattableend\end{savenotes}

In cases where there are a large number of quasi-identifiers and the
variables have many categories, the number of possible combinations
increases rapidly (see \(k\)-anonymity). If the number of variables
and categories is very large, the computation time of the
localSuppression() algorithm can be very long (see Section 7.7 on
computation time). Also, the algorithm may not reach a solution, or may
come to a solution that will not meet the specified level of
\(k\)-anonymity. Therefore, reducing the number of quasi-identifiers
and/or categories before applying local suppression is recommended. This
can be done by recoding variables or selecting some variables for other
(perturbative) methods, such as PRAM. This is to ensure that the number
of suppressions is limited and hence the loss of data is limited to only
those values that pose most risk.

In some datasets, it might prove difficult to reduce the number of
quasi-identifiers and even after reducing the number of categories by
recoding, the local suppression algorithm takes a long time to compute
the required suppressions. A solution in such cases can be the so-called
‘all-\(m\) approach’ (see de Wolf, 2015). The all-\(m\)
approach consists of applying the local suppression algorithm as
described above to all possible subsets of size \sphinxstyleemphasis{m} of the total set of
quasi-identifiers. The advantage of this approach is that the partial
problems are easier to solve and computation time will be slower.
Caution should be applied since this method does not necessarily lead to
\(k\)-anonymity in the complete set of quasi-identifiers. There are
two possibilities to reach the same level of protection: 1) to choose a
higher threshold for \sphinxstyleemphasis{k} or 2) to re-apply the local suppression
algorithm on the complete set of quasi-identifiers after using the
all-\(m\) approach to achieve the required threshold. In the
second case, the all-\(m\) approach leads to a shorter computation
time at the cost of a higher total number of suppressions. \sphinxstylestrong{NOTE: The
required level is not achieved automatically on the entire set of
quasi-identifiers if the all-m approach is used.} Therefore, it is
important to evaluate the risk measures carefully after using the
all-\(m\) approach.

In \sphinxstyleemphasis{sdcMicro} the all-\(m\) approach is implemented in the ‘combs’
argument in the localSuppression() function. The value for \sphinxstyleemphasis{m} is
specified in the ‘combs’ argument and can also take on several values.
The subsets of different sizes are then used sequentially in the local
suppression algorithm. For example if ‘combs’ is set to c(3,9), first
all subsets of size 3 are considered and subsequently all subsets of
size 9. Setting the last value in the combs argument to the total number
of key variables guarantees the achievement of \(k\)-anonymity for
the complete dataset. It is also possible to specify different values
for \sphinxstyleemphasis{k} for each subset size in the ‘k’ argument. If we would want to
achieve 5-anonimity on the subsets of size 3 and subsequently
3-anonimity on the subsets of size 9, we would set the ‘k’ argument to
c(5,3). Example 5.8 illustrates the use of the all-\(m\) approach
in \sphinxstyleemphasis{sdcMicro}.

Example 5.8 The all-\(\mathbf{m}\) approach in sdcMicro

\fvset{hllines={, ,}}%
\begin{sphinxVerbatim}[commandchars=\\\{\}]
\PYG{c+c1}{\PYGZsh{} Apply k\PYGZhy{}anonymity with threshold 5 to all subsets of two key variables and subsequently to the complete dataset}
sdcInitial \PYG{o}{\PYGZlt{}\PYGZhy{}} localSuppression\PYG{p}{(}sdcInitial\PYG{p}{,} k \PYG{o}{=} \PYG{l+m}{5}\PYG{p}{,} combs \PYG{o}{=} \PYG{k+kt}{c}\PYG{p}{(}\PYG{l+m}{2}\PYG{p}{,} \PYG{l+m}{5}\PYG{p}{)}\PYG{p}{)}
\PYG{c+c1}{\PYGZsh{} Apply k\PYGZhy{}anonymity with threshold 5 to all subsets of three key variables and subsequently with threshold 2 to the complete dataset}
sdcInitial \PYG{o}{\PYGZlt{}\PYGZhy{}} localSuppression\PYG{p}{(}sdcInitial\PYG{p}{,} k \PYG{o}{=} \PYG{k+kt}{c}\PYG{p}{(}\PYG{l+m}{3}\PYG{p}{,} \PYG{l+m}{5}\PYG{p}{)}\PYG{p}{,} combs \PYG{o}{=} \PYG{k+kt}{c}\PYG{p}{(}\PYG{l+m}{5}\PYG{p}{,} \PYG{l+m}{2}\PYG{p}{)}\PYG{p}{)}
\end{sphinxVerbatim}

Table 5.6 presents the results of using the all-\(m\) approach of
a test dataset with 9 key variables and 4,000 records. The table shows
the arguments ‘k’ and ‘combs’ of the localSuppression() function, the
number of \(k\)\sphinxstyleemphasis{-}anonymity violators for different levels of
\(k\) as well as the total number of suppressions. We observe that
the different combinations do not always lead to the required level of
\(k\)-anonimity. For example, when setting \(k = 3\), and combs
3 and 7, there are still 15 records in the dataset (with a total of 9
quasi-identifiers) that violate 3-anonimity after local suppression. Due
to the smaller sample size, the gains in running time are not yet
apparent in this example, since the rerunning algorithm several times
takes up time. A larger dataset would benefit more from the all-\sphinxstyleemphasis{m}
approach, as the algorithm would take longer in the first place.

Table 5.6 Effect of the all-\sphinxstyleemphasis{m} approach on k-anonymity


\begin{savenotes}\sphinxattablestart
\centering
\begin{tabular}[t]{|*{7}{\X{1}{7}|}}
\hline
\sphinxstyletheadfamily 
Argumen
ts
&\sphinxstartmulticolumn{4}%
\begin{varwidth}[t]{\sphinxcolwidth{4}{7}}
\sphinxstyletheadfamily {\color{red}\bfseries{}**}Numbe
r
of
violato
rs
for
differe
nt
levels
of
k-anony
mity
on
complet
e
set
\par
\vskip-\baselineskip\vbox{\hbox{\strut}}\end{varwidth}%
\sphinxstopmulticolumn
&\sphinxstyletheadfamily 
Total
number
of
suppres
sio ns
&\sphinxstyletheadfamily 
Running

time
(second
s)
\\
\hline
\sphinxstyleemphasis{k}
&
\sphinxstyleemphasis{combs}
&
\sphinxstyleemphasis{k = 2}
&
\sphinxstyleemphasis{k = 3}
&
\sphinxstyleemphasis{k = 5}
&&\\
\hline
\sphinxstyleemphasis{before
local
suppres
sion}
&
2,464
&
3,324
&
3,877
&
0
&
0.00
&\\
\hline
3
&\begin{itemize}
\item {} 
\end{itemize}
&
0
&
0
&
1,766
&
2,264
&
17.08
\\
\hline
5
&\begin{itemize}
\item {} 
\end{itemize}
&
0
&
0
&
0
&
3,318
&
10.57
\\
\hline
3
&
3
&
2,226
&
3,202
&
3,819
&
3,873
&
13.39
\\
\hline
3
&
3, 7
&
15
&
108
&
1,831
&
6,164
&
46.84
\\
\hline
3
&
3, 9
&
0
&
0
&
1,794
&
5,982
&
31.38
\\
\hline
3
&
5, 9
&
0
&
0
&
1,734
&
6,144
&
62.30
\\
\hline
5
&
3
&
2,047
&
3,043
&
3,769
&
3,966
&
12.88
\\
\hline
5
&
3, 7
&
0
&
6
&
86
&
7,112
&
46.57
\\
\hline
5
&
3, 9
&
0
&
0
&
0
&
7,049
&
24.13
\\
\hline
5
&
5, 9
&
0
&
0
&
0
&
7,129
&
54.76
\\
\hline
5, 3
&
3, 7
&
11
&
108
&
1,859
&
6,140
&
45.60
\\
\hline
5, 3
&
3, 9
&
0
&
0
&
1,766
&
2,264
&
30.07
\\
\hline
5, 3
&
5, 9
&
0
&
0
&
0
&
3,318
&
51.25
\\
\hline
\end{tabular}
\par
\sphinxattableend\end{savenotes}

Often the dataset contains variables that are related to the key
variables used for local suppression. Examples are rural/urban to
regions in case regions are completely rural or urban or variables that
are only answered for specific categories (e.g., sector for those
working, schooling related variables for certain age ranges). In those
cases, the variables rural/urban or sector might not be
quasi-identifiers themselves, but could allow the intruder to
reconstruct suppressed values in the quasi-identifiers region or
employment status. For example, if region 1 is completely urban, and all
other regions are only semi-urban or rural, a suppression in the
variable region for a record in region 1 can be simply reconstructed by
the rural/urban variable. Therefore, it is useful to suppress the values
corresponding to the suppressions in those linked variables. Example 5.9
illustrates how to suppress the values in the variable “rururb”
corresponding to the suppressions in the region variable. All values of
“rururb”, which correspond to a suppressed value (NA) in the variable
“region” are suppressed (set to NA).

Example 5.9: Manually suppressing values in linked variables

\fvset{hllines={, ,}}%
\begin{sphinxVerbatim}[commandchars=\\\{\}]
\PYG{c+c1}{\PYGZsh{} Suppress values of rururb in file if region is suppressed}
\PYG{k+kp}{file}\PYG{p}{[}\PYG{k+kp}{is.na}\PYG{p}{(}sdcInitial\PYG{o}{@}manipKeyVars\PYG{o}{\PYGZdl{}}region\PYG{p}{)} \PYG{o}{\PYGZam{}} \PYG{o}{!}\PYG{k+kp}{is.na}\PYG{p}{(}sdcInitial\PYG{o}{@}origData\PYG{o}{\PYGZdl{}}region\PYG{p}{)}\PYG{p}{,}\PYG{l+s}{\PYGZsq{}}\PYG{l+s}{sizRes\PYGZsq{}}\PYG{p}{]} \PYG{o}{\PYGZlt{}\PYGZhy{}} \PYG{k+kc}{NA}
\end{sphinxVerbatim}

Alternatively, the linked variables can be specified when creating the
\sphinxstyleemphasis{sdcMicro} object. The linked variables are called ghost variables. Any
suppression in the key variable will lead to a suppression in the
variables linked to that key variable. Example 5.10 shows how to specify
the linkage between “region” and “rururb” with ghost variables.

Example 5.10: Suppressing values in linked variables by specifying ghost
variables

\fvset{hllines={, ,}}%
\begin{sphinxVerbatim}[commandchars=\\\{\}]
\PYG{c+c1}{\PYGZsh{} Ghost (linked) variables are specified as a list of linkages}
ghostVars \PYG{o}{\PYGZlt{}\PYGZhy{}} \PYG{k+kt}{list}\PYG{p}{(}\PYG{p}{)}

\PYG{c+c1}{\PYGZsh{} Each linkage is a list, with the first element the key variable and the second element the linked variable(s)}
ghostVars\PYG{p}{[[}\PYG{l+m}{1}\PYG{p}{]]} \PYG{o}{\PYGZlt{}\PYGZhy{}} \PYG{k+kt}{list}\PYG{p}{(}\PYG{p}{)}
ghostVars\PYG{p}{[[}\PYG{l+m}{1}\PYG{p}{]]}\PYG{p}{[[}\PYG{l+m}{1}\PYG{p}{]]} \PYG{o}{\PYGZlt{}\PYGZhy{}} \PYG{l+s}{\PYGZdq{}}\PYG{l+s}{region\PYGZdq{}}
ghostVars\PYG{p}{[[}\PYG{l+m}{1}\PYG{p}{]]}\PYG{p}{[[}\PYG{l+m}{2}\PYG{p}{]]} \PYG{o}{\PYGZlt{}\PYGZhy{}} \PYG{k+kt}{c}\PYG{p}{(}\PYG{l+s}{\PYGZdq{}}\PYG{l+s}{sizeRes\PYGZdq{}}\PYG{p}{)}

\PYG{c+c1}{\PYGZsh{}\PYGZsh{} Create the sdcMicroObj}
sdcInitial \PYG{o}{\PYGZlt{}\PYGZhy{}} createSdcObj\PYG{p}{(}\PYG{k+kp}{file}\PYG{p}{,} keyVars \PYG{o}{=} keyVars\PYG{p}{,} numVars \PYG{o}{=} numVars\PYG{p}{,} weightVar \PYG{o}{=} weight\PYG{p}{,} ghostVars \PYG{o}{=} ghostVars\PYG{p}{)}

\PYG{c+c1}{\PYGZsh{} The manipulated ghost variables are in the slot manipGhostVars}
sdcInitial\PYG{o}{@}manipGhostVars
\end{sphinxVerbatim}

The simpler alternative for the localSuppression() function in
\sphinxstyleemphasis{sdcMicro} is the localSupp() function. The localSupp() function can be
used to suppress values of certain key variables of individuals with
risks above a certain threshold. In this case, all values of the
specified variable for respondents with a risk higher than the specified
threshold will be suppressed. The risk measure used is the individual
risk (see Section 4.5). This is useful if one variable has sensitive
values that should not be released for individuals with high risks of
re-identification. What is considered high re-identification probability
depends on legal requirements. In the following example, the values of
the variable “education” are suppressed for all individuals whose
individual risk is higher than 0.1, which is illustrated in Example
5.11. For an overview of the individual risk values, it can be useful to
look at the summary statistics of the individual risk values as well as
the number of suppressions.

Example 5.11: Application of built-in \sphinxstyleemphasis{sdcMicro} function localSupp()

\fvset{hllines={, ,}}%
\begin{sphinxVerbatim}[commandchars=\\\{\}]
\PYG{c+c1}{\PYGZsh{} Summary statistics}
\PYG{k+kp}{summary}\PYG{p}{(}sdcInitial\PYG{o}{@}risk\PYG{o}{\PYGZdl{}}individual\PYG{p}{[}\PYG{p}{,}\PYG{l+m}{1}\PYG{p}{]}\PYG{p}{)}
\PYG{c+c1}{\PYGZsh{}\PYGZsh{}    Min. 1st Qu.  Median    Mean 3rd Qu.    Max.}
\PYG{c+c1}{\PYGZsh{}\PYGZsh{} 0.05882 0.10000 0.14290 0.26480 0.33330 1.00000}

\PYG{c+c1}{\PYGZsh{} Number of individuals with individual risk higher than 0.1}
\PYG{k+kp}{sum}\PYG{p}{(}sdcInitial\PYG{o}{@}risk\PYG{o}{\PYGZdl{}}individual\PYG{p}{[}\PYG{p}{,}\PYG{l+m}{1}\PYG{p}{]} \PYG{o}{\PYGZgt{}} \PYG{l+m}{0.1}\PYG{p}{)}
\PYG{c+c1}{\PYGZsh{}\PYGZsh{} [1] 1863}

\PYG{c+c1}{\PYGZsh{} local suppression}
sdcInitial \PYG{o}{\PYGZlt{}\PYGZhy{}} localSupp\PYG{p}{(}sdcInitial\PYG{p}{,} threshold \PYG{o}{=} \PYG{l+m}{0.1}\PYG{p}{,} keyVar \PYG{o}{=} \PYG{l+s}{\PYGZsq{}}\PYG{l+s}{education\PYGZsq{}}\PYG{p}{)}
\end{sphinxVerbatim}


\section{Perturbative methods}
\label{\detokenize{anon_methods:perturbative-methods}}
Perturbative methods do not suppress values in the dataset, but perturb
(alter) values to limit disclosure risk by creating uncertainty around
the true values. An intruder is uncertain whether a match between the
microdata and an external file is correct or not. Most perturbative
methods are based on the principle of matrix masking, i.e., the altered
dataset Z is computed as
\begin{equation*}
\begin{split}Z = \text{AXB} + C\end{split}
\end{equation*}
where X is the original data, A is a matrix used to transform the
records, B is a matrix to transform the variables and C is a matrix with
additive noise.

\sphinxstylestrong{NOTE: Risk measures based on frequency counts of keys are no longer
valid after applying perturbative methods.} This can be seen in Table
5.7, which displays the same data before and after swapping some values.
The swapped values are in grey. Both before and after perturbing the
data, all observations violate \(k\)-anonymity at the level 3 (i.e.,
each key does not appear more than twice in the dataset). Nevertheless,
the risk of \sphinxstylestrong{correct} re-identification of the records is reduced and
hence information contained in other (sensitive) variables possibly not
disclosed. With a certain probability, a match of the microdata with an
external data file will be wrong. For example, an intruder would find
one individual with the combination \{‘male’, ‘urban’, ‘higher’\}, which
is a sample unique. However, this match is not correct, since the
original dataset did not contain any individual with these
characteristics and hence the matched individual cannot be a correct
match. The intruder cannot know with certainty whether the information
disclosed from other variables for that record is correct.

Table 5.7: Sample data before and after perturbation


\begin{savenotes}\sphinxattablestart
\centering
\begin{tabulary}{\linewidth}[t]{|T|T|T|T|T|T|T|}
\hline
\sphinxstyletheadfamily 
Variable
&\sphinxstartmulticolumn{3}%
\begin{varwidth}[t]{\sphinxcolwidth{3}{7}}
\sphinxstyletheadfamily Original data
\par
\vskip-\baselineskip\vbox{\hbox{\strut}}\end{varwidth}%
\sphinxstopmulticolumn
&\sphinxstartmulticolumn{3}%
\begin{varwidth}[t]{\sphinxcolwidth{3}{7}}
\sphinxstyletheadfamily After perturbing the data
\par
\vskip-\baselineskip\vbox{\hbox{\strut}}\end{varwidth}%
\sphinxstopmulticolumn
\\
\hline
ID
&
Gender
&
Region
&
Education
&
Gender
&
Region
&
Education
\\
\hline
1
&
female
&
rural
&
higher
&
female
&
rural
&
higher
\\
\hline
2
&
female
&
rural
&
higher
&
female
&
rural
&
lower
\\
\hline
3
&
male
&
rural
&
lower
&
male
&
rural
&
lower
\\
\hline
4
&
male
&
rural
&
lower
&
female
&
rural
&
lower
\\
\hline
5
&
female
&
urban
&
lower
&
male
&
urban
&
higher
\\
\hline
6
&
female
&
urban
&
lower
&
female
&
urban
&
lower
\\
\hline
\end{tabulary}
\par
\sphinxattableend\end{savenotes}

One advantage of perturbative methods is that the information loss is
reduced, since no values will be suppressed, depending on the level of
perturbation. One disadvantage is that data users might have the
impression that the data was not anonymized before release and will be
less willing to participate in future surveys. Therefore, there is a
need for reporting both for internal and external use (see Section
8.11).

An alternative to perturbative methods is the generation of synthetic
data files with the same characteristics as the original data files.
Synthetic data files are not discussed in these guidelines. For more
information and an overview of the use of synthetic data as SDC method,
we refer to Drechsler (2011) and Section 3.8 in Hundepool et al. (2012).
We discuss here five perturbative methods: Post Randomization Method
(PRAM), microaggregation, noise addition, shuffling and rank swapping.


\subsection{PRAM (Post RAndomization Method)}
\label{\detokenize{anon_methods:pram-post-randomization-method}}
PRAM is a perturbative method for categorical data. This method
reclassifies the values of one or more variables, such that intruders
that attempt to re-identify individuals in the data do so, but with
positive probability, the re-identification made is with the wrong
individual. This means that the intruder might be able to match several
individuals between external files and the released data files, but
cannot be sure whether these matches are to the correct individual.

PRAM is defined by the transition matrix \(P\), which specifies the
transition probabilities, i.e., the probability that a value of a
certain variable stays unchanged or is changed to any of the other
\(k - 1\) values. \(k\) is the number of categories or factor
levels within the variable to be PRAMmed. For example, if the variable
region has 10 different regions, \(k\) equals 10. In case of PRAM
for a single variable, the transition matrix is size \(k*k\). We
illustrate PRAM with an example of the variable “region”, which has
three different values: ‘capital’, ‘rural1’ and ‘rural2’. The transition
matrix for applying PRAM to this variable is size 3*3:
\begin{equation*}
\begin{split}P = \begin{bmatrix}
1 & 0 & 0 \\
0.05 & 0.8 & 0.15 \\
0.05 & 0.15 & 0.8 \\
\end{bmatrix}\end{split}
\end{equation*}
The values on the diagonal are the probabilities that a value in the
corresponding category is not changed. The value 1 at position (1,1) in
the matrix means that all values ‘capital’ stay ‘capital’; this might be
a useful decision, since most individuals live in the capital and no
protection is needed. The value 0.8 at position (2,2) means that an
individual with value ‘rural1’ will stay with probability 0.8 ‘rural1’.
The values 0.05 and 0.15 in the second row of the matrix indicate that
the value ‘rural1’ will be changed to ‘capital’ or ‘rural2’ with
respectively probability 0.05 and 0.15. If in the initial file we had
5,000 individuals with value ‘capital’ and resp. 500 and 400 with values
‘rural1’ and ‘rural2’, we expect after applying PRAM to have 5,045
individuals with capital, 460 with rural1 and 395 with
rural2 %
\begin{footnote}[9]\sphinxAtStartFootnote
The 5,045 is the expectation computed as 5,000 * 1 + 500 * 0.05 +
400 * 0.05.
%
\end{footnote}. The recoding is done independently for
each individual. We see that the tabulation of the variable “region”
yields different results before and after PRAM, which are shown in Table
5.8. The deviation from the expectation is due to the fact that PRAM is
a probabilistic method, i.e., the results depend on a
probability-generating mechanism; consequently, the results can differ
every time we apply PRAM to the same variables of a dataset. \sphinxstylestrong{NOTE: The
number of changed values is larger than one might think when inspecting
the tabulations in} \sphinxstylestrong{Table 5.8. Not all 5,000 individuals with value
captial after PRAM had this value before PRAM and the 457 individuals in
rural1 after PRAM are not all included in the 500 individuals before
PRAM. The number of changes is larger than the differences in the
tabulation (cf. transition matrix).} Given that the transition matrix
is known to the end users, there are several ways to correct statistical
analysis of the data for the distortions introduced by PRAM.

Table 5.8: Tabulation of variable “region” before and after PRAM


\begin{savenotes}\sphinxattablestart
\centering
\begin{tabulary}{\linewidth}[t]{|T|T|T|}
\hline
\sphinxstyletheadfamily 
\sphinxstylestrong{Value}
&\sphinxstyletheadfamily 
\sphinxstylestrong{Tabulation before PRAM}
&\sphinxstyletheadfamily 
\sphinxstylestrong{Tabulation after PRAM}
\\
\hline
capital
&
5,000
&
5,052
\\
\hline
rural1
&
500
&
457
\\
\hline
rural2
&
400
&
391
\\
\hline
\end{tabulary}
\par
\sphinxattableend\end{savenotes}

One way to guarantee consistency between the tabulations before and
after PRAM is to choose the transition matrix so that, in expectation,
the tabulations before and after applying PRAM are the same for all
variables.{[}\#foot43{]}\_ This method is called invariant PRAM
and is implemented in \sphinxstyleemphasis{sdcMicro} in the function pram(). The method
pram() determines the transition matrix that satisfies the requirements
for invariant PRAM. \sphinxstylestrong{NOTE: Invariant does not guarantee that
cross-tabulations of variables (unlike univariate tabulations) stay the
same.}

In Example 5.12, we give an example of invariant PRAM using
\sphinxstyleemphasis{sdcMicro}. %
\begin{footnote}[11]\sphinxAtStartFootnote
In this example and the following examples in this section, the
\sphinxstyleemphasis{sdcMicro} object “sdcIntial“ contains a dataset with 2,000
individuals and 39 variables. We selected five categorical
quasi-identifiers and 9 variables for PRAM: “ROOF”, “TOILET”,
“WATER”, “ELECTCON”, “FUELCOOK”, “OWNMOTORCYCLE”, “CAR”, “TV”, and
“LIVESTOCK”. These PRAM variabels were selected according to the
requirements of this particular dataset and for illustrative
purposes.
%
\end{footnote} PRAM is a probabilistic method and the
results can differ every time we apply PRAM to the same variables of a
dataset. To overcome this and make the results reproducible, it is good
practice to set a seed for the random number generator in \sphinxstyleemphasis{R}, so the
same random numbers will be generated every time. %
\begin{footnote}[12]\sphinxAtStartFootnote
The PRAM method in \sphinxstyleemphasis{sdcMicro} sometimes produces the following
error: Error in factor(xpramed, labels = lev) : invalid ‘labels’;
length 6 should be 1 or 5. Under some circumstances, changing the
seed can solve this error.
%
\end{footnote}
The number of changed records per variable is also shown.

Example 5.12: Producing reproducible PRAM results by using set.seed()

\fvset{hllines={, ,}}%
\begin{sphinxVerbatim}[commandchars=\\\{\}]
\PYG{c+c1}{\PYGZsh{} Set seed for random number generator}
\PYG{k+kp}{set.seed}\PYG{p}{(}\PYG{l+m}{123}\PYG{p}{)}

\PYG{c+c1}{\PYGZsh{} Apply PRAM to all selected variables}
sdcInitial \PYG{o}{\PYGZlt{}\PYGZhy{}} pram\PYG{p}{(}obj \PYG{o}{=} sdcInitial\PYG{p}{)}
\PYG{c+c1}{\PYGZsh{}\PYGZsh{} Number of changed observations:}
\PYG{c+c1}{\PYGZsh{}\PYGZsh{} \PYGZhy{} \PYGZhy{} \PYGZhy{} \PYGZhy{} \PYGZhy{} \PYGZhy{} \PYGZhy{} \PYGZhy{} \PYGZhy{} \PYGZhy{} \PYGZhy{}}
\PYG{c+c1}{\PYGZsh{}\PYGZsh{} ROOF != ROOF\PYGZus{}pram : 75 (3.75\PYGZpc{})}
\PYG{c+c1}{\PYGZsh{}\PYGZsh{} TOILET != TOILET\PYGZus{}pram : 200 (10\PYGZpc{})}
\PYG{c+c1}{\PYGZsh{}\PYGZsh{} WATER != WATER\PYGZus{}pram : 111 (5.55\PYGZpc{})}
\PYG{c+c1}{\PYGZsh{}\PYGZsh{} ELECTCON != ELECTCON\PYGZus{}pram : 99 (4.95\PYGZpc{})}
\PYG{c+c1}{\PYGZsh{}\PYGZsh{} FUELCOOK != FUELCOOK\PYGZus{}pram : 152 (7.6\PYGZpc{})}
\PYG{c+c1}{\PYGZsh{}\PYGZsh{} OWNMOTORCYCLE != OWNMOTORCYCLE\PYGZus{}pram : 42 (2.1\PYGZpc{})}
\PYG{c+c1}{\PYGZsh{}\PYGZsh{} CAR != CAR\PYGZus{}pram : 168 (8.4\PYGZpc{})}
\PYG{c+c1}{\PYGZsh{}\PYGZsh{} TV != TV\PYGZus{}pram : 170 (8.5\PYGZpc{})}
\PYG{c+c1}{\PYGZsh{}\PYGZsh{} LIVESTOCK != LIVESTOCK\PYGZus{}pram : 52 (2.6\PYGZpc{})}
\end{sphinxVerbatim}

Table 5.9 shows the tabulation of the variable after applying invariant
PRAM. We can see that the deviations from the initial tabulations, which
are in expectation 0, are smaller than with the transition matrix that
does not fulfill the invariance property. The remaining deviations are
due to the randomness.

Table 5.9: Tabulation of variable “region” before and after (invariant)
PRAM


\begin{savenotes}\sphinxattablestart
\centering
\begin{tabulary}{\linewidth}[t]{|T|T|T|T|}
\hline
\sphinxstyletheadfamily 
\sphinxstylestrong{Value}
&\sphinxstyletheadfamily 
\sphinxstylestrong{Tabulation
before PRAM}
&\sphinxstyletheadfamily 
\sphinxstylestrong{Tabulation
after PRAM}
&\sphinxstyletheadfamily 
\sphinxstylestrong{Tabulation
after invariant
PRAM}
\\
\hline
capital
&
5,000
&
5,052
&
4,998
\\
\hline
rural1
&
500
&
457
&
499
\\
\hline
rural2
&
400
&
391
&
403
\\
\hline
\end{tabulary}
\par
\sphinxattableend\end{savenotes}

Table 5.10 presents the cross-tabulations with the variable gender.
Before applying invariant PRAM, the share of males in the city is much
higher than the share of females (about 60\%). This property is not
maintained after invariant PRAM (the shares of males and females in the
city are roughly equal), although the univariate tabulations are
maintained. One solution is to apply PRAM separately for the males and
females in this example %
\begin{footnote}[13]\sphinxAtStartFootnote
This can also be achieved with multidimensional transition matrices.
In that case, the probability is not specified for ‘male’ -\textgreater{}
‘female’, but for ‘male’ + ‘rural’ -\textgreater{} ‘female’ + ‘rural’ and for
‘male’ + ‘urban’ -\textgreater{} ‘female’ + ‘urban’. This is not implemented in
sdcMicro but can be achieved by PRAMming the males and females
separately. In the example here, this could be done by specifying
gender as strata variable in the pram() function in \sphinxstyleemphasis{sdcMicro}.
%
\end{footnote}. This can be done by
specifying the strata argument in the pram() function in \sphinxstyleemphasis{sdcMicro} (see
below).

Table 5.10: Cross-tabulation of variable “region” and variable “gender”
before and after invariant PRAM


\begin{savenotes}\sphinxattablestart
\centering
\begin{tabulary}{\linewidth}[t]{|T|T|T|T|T|}
\hline
\sphinxstyletheadfamily &\sphinxstartmulticolumn{2}%
\begin{varwidth}[t]{\sphinxcolwidth{2}{5}}
\sphinxstyletheadfamily Tabulation before PRAM
\par
\vskip-\baselineskip\vbox{\hbox{\strut}}\end{varwidth}%
\sphinxstopmulticolumn
&\sphinxstartmulticolumn{2}%
\begin{varwidth}[t]{\sphinxcolwidth{2}{5}}
\sphinxstyletheadfamily Tabulation after invariant
PRAM
\par
\vskip-\baselineskip\vbox{\hbox{\strut}}\end{varwidth}%
\sphinxstopmulticolumn
\\
\hline
Value
&
male
&
female
&
male
&
female
\\
\hline
capital
&
3,056
&
1,944
&
2,623
&
2,375
\\
\hline
rural1
&
157
&
343
&
225
&
274
\\
\hline
rural2
&
113
&
287
&
187
&
216
\\
\hline
\end{tabulary}
\par
\sphinxattableend\end{savenotes}

The pram() function in \sphinxstyleemphasis{sdcMicro} has several options. \sphinxstylestrong{NOTE: If no
options are set and the PRAM method is applied to an sdcMicro object,
all PRAM variables selected in the sdcMicro object are automatically
used for PRAM and PRAM is applied within the selected strata} (see
Section 7.5 on \sphinxstyleemphasis{sdcMicro} objects for more details). Alternatively, PRAM
can also be applied to variables that are not specified in the
\sphinxstyleemphasis{sdcMicro} object as PRAM variables, such as key variables, which is
shown in Example 5.13. In that case, however, the risk measures that are
automatically computed will not be correct anymore, since the variables
are perturbed. Therefore, if during the SDC process PRAM will be applied
to some key variables, it is recommended to create a new \sphinxstyleemphasis{sdcMicro}
object where the variables to be PRAMmed are selected as PRAM variables
in the function createSdcObj().

Example 5.13: Selecting the variable “toilet” to apply PRAM

\fvset{hllines={, ,}}%
\begin{sphinxVerbatim}[commandchars=\\\{\}]
\PYG{c+c1}{\PYGZsh{} Set seed for random number generator}
\PYG{k+kp}{set.seed}\PYG{p}{(}\PYG{l+m}{123}\PYG{p}{)}
\PYG{c+c1}{\PYGZsh{} Apply PRAM only to the variable TOILET}
sdcInitial \PYG{o}{\PYGZlt{}\PYGZhy{}} pram\PYG{p}{(}obj \PYG{o}{=} sdcInitial\PYG{p}{,} variables \PYG{o}{=} \PYG{k+kt}{c} \PYG{p}{(}\PYG{l+s}{\PYGZdq{}}\PYG{l+s}{TOILET\PYGZdq{}}\PYG{p}{)}\PYG{p}{)}
\PYG{c+c1}{\PYGZsh{}\PYGZsh{} Number of changed observations:}
\PYG{c+c1}{\PYGZsh{}\PYGZsh{} \PYGZhy{} \PYGZhy{} \PYGZhy{} \PYGZhy{} \PYGZhy{} \PYGZhy{} \PYGZhy{} \PYGZhy{} \PYGZhy{} \PYGZhy{} \PYGZhy{}}
\PYG{c+c1}{\PYGZsh{}\PYGZsh{} TOILET != TOILET\PYGZus{}pram : 115 (5.75\PYGZpc{})}
\end{sphinxVerbatim}

The results for PRAM differ if applied simultaneously to several
variables or subsequently to each variable separately. It is not
possible to specify the entire transition matrix in \sphinxstyleemphasis{sdcMicro}, but we
can set minimum values (between 0 and 1) for the diagonal entries. The
diagonal entries specify the probability that a certain value stays the
same after applying PRAM. Setting the minimum value to 1 will yield no
changes to this category. By default, this value is 0.8, which applies
for all categories. It is also possible to specify a vector with value
for each diagonal element of the transformation matrix/category. In
Example 5.14 values of the first region are less likely to change than
values of the other regions. \sphinxstylestrong{NOTE: The invariant PRAM method requires
that the transition matrix has a unit eigenvalue.} Not all sets of
restrictions can therefore be used (e.g., the minimum value 1 on any of
the categories).

Example 5.14: Specifying minimum values for diagonal entries in PRAM
transition matrix

\fvset{hllines={, ,}}%
\begin{sphinxVerbatim}[commandchars=\\\{\}]
sdcInitial \PYG{o}{\PYGZlt{}\PYGZhy{}} pram\PYG{p}{(}obj \PYG{o}{=} sdcInitial\PYG{p}{,} variables \PYG{o}{=} \PYG{k+kt}{c}\PYG{p}{(}\PYG{l+s}{\PYGZdq{}}\PYG{l+s}{TOILET\PYGZdq{}}\PYG{p}{)}\PYG{p}{,} pd \PYG{o}{=} \PYG{k+kt}{c}\PYG{p}{(}\PYG{l+m}{0.9}\PYG{p}{,} \PYG{l+m}{0.5}\PYG{p}{,} \PYG{l+m}{0.5}\PYG{p}{,} \PYG{l+m}{0.5}\PYG{p}{)}\PYG{p}{)}
\PYG{c+c1}{\PYGZsh{}\PYGZsh{} Number of changed observations:}
\PYG{c+c1}{\PYGZsh{}\PYGZsh{} \PYGZhy{} \PYGZhy{} \PYGZhy{} \PYGZhy{} \PYGZhy{} \PYGZhy{} \PYGZhy{} \PYGZhy{} \PYGZhy{} \PYGZhy{} \PYGZhy{}}
\PYG{c+c1}{\PYGZsh{}\PYGZsh{} TOILET != TOILET\PYGZus{}pram : 496 (24.8\PYGZpc{})}
\end{sphinxVerbatim}

In the invariant PRAM method, we can also specify the amount of
perturbation by specifying the parameter alpha. This choice is reflected
in the transition matrix. By default, the alpha value is 0.5. The larger
alpha, the larger the perturbations. Alpha equal to zero leads to no
changes. The maximum value for alpha is 1.

PRAM is especially useful when a dataset contains many variables and
applying other anonymization methods, such as recoding and local
suppression, would lead to significant information loss. Checks on risk
and utility are important after PRAM.

To do statistical inference on variables to which PRAM was applied, the
researcher needs knowledge about the PRAM method as well as about the
transition matrix. The transition matrix, together with the random
number seed, can, however, lead to disclosure through reconstruction of
the non-perturbed values. Therefore, publishing the transition matrix
but not the random seed is recommended.

A disadvantage of using PRAM is that very unlikely combinations can be
generated, such as a 63-year- old who goes to school. Therefore, the
PRAMmed variables need to be audited to prevent such combinations from
happening in the released data file. In principal, the transition matrix
can be designed in such a way that certain transitions are not possible
(probability 0). For instance, for those that go to school, the age must
range within 6 to 18 years and only such changes are allowed. In
\sphinxstyleemphasis{sdcMicro} the transition matrix cannot be exactly specified. A useful
alternative is constructing strata and applying PRAM within the strata.
In this way, the changes between variables will only be applied within
the strata. Example 5.15 illustrates this by applying PRAM to the
variable “toilet” within the strata generated by the “region” education.
This prevents changes in the variable “toilet”, where toilet types in a
particular region are exchanged with those in other regions. For
instance, in the capital region certain types of unimproved toilet types
are not in use and therefore these combinations should not occur after
PRAMming. Values are only changed with those that are available in the
same strata. Strata can be formed by any categorical variable, e.g.,
gender, age groups, education level.

Example 5.15: Minimizing unlikely combinations by applying PRAM within
strata

\fvset{hllines={, ,}}%
\begin{sphinxVerbatim}[commandchars=\\\{\}]
\PYG{c+c1}{\PYGZsh{} Applying PRAM within the strata generated by the variable region}
sdcInitial \PYG{o}{\PYGZlt{}\PYGZhy{}} pram\PYG{p}{(}obj \PYG{o}{=} sdcInitial\PYG{p}{,} variables \PYG{o}{=} \PYG{k+kt}{c}\PYG{p}{(}\PYG{l+s}{\PYGZdq{}}\PYG{l+s}{TOILET\PYGZdq{}}\PYG{p}{)}\PYG{p}{,} strata\PYGZus{}variables \PYG{o}{=} \PYG{k+kt}{c}\PYG{p}{(}\PYG{l+s}{\PYGZdq{}}\PYG{l+s}{REGION\PYGZdq{}}\PYG{p}{)}\PYG{p}{)}
\PYG{c+c1}{\PYGZsh{}\PYGZsh{} Number of changed observations:}
\PYG{c+c1}{\PYGZsh{}\PYGZsh{} \PYGZhy{} \PYGZhy{} \PYGZhy{} \PYGZhy{} \PYGZhy{} \PYGZhy{} \PYGZhy{} \PYGZhy{} \PYGZhy{} \PYGZhy{} \PYGZhy{}}
\PYG{c+c1}{\PYGZsh{}\PYGZsh{} TOILET != TOILET\PYGZus{}pram : 179 (8.95\PYGZpc{})}
\end{sphinxVerbatim}

\begin{sphinxadmonition}{note}{Recommended Reading Material on PRAM}

Gouweleeuw, J. M, P Kooiman, L.C.R.J Willenborg, and P.P de Wolf. “Post
Randomization for Statistical Disclosure Control: Theory and
Implementation.\sphinxstyleemphasis{” Journal of Official Statistics} 14, no. 4 (1998a):
463-478. Available at
\sphinxurl{http://www.jos.nu/articles/abstract.asp?article=144463}

Gouweleeuw, J. M, P Kooiman, L.C.R.J Willenborg, and Peter Paul de Wolf.
“The Post Randomization Method for Protecting Microdata\sphinxstyleemphasis{.” Qüestiió,
Quaderns d’Estadística i Investigació Operativa 22,} no. 1 (1998b):
145-156. Available at
\sphinxurl{http://www.raco.cat/index.php/Questiio/issue/view/2250}

Marés, Jordi, and Vicenç Torra. 2010.”PRAM Optimization Using an
Evolutionary Algorithm.” \sphinxstyleemphasis{In Privacy in Statistical Databases}, by Josep
Domingo-Ferrer and Emmanouil Magkos, 97-106. Corfú, Greece: Springer.

Warner, S.L. “Randomized Response: A Survey Technique for Eliminating
Evasive Answer Bias.” \sphinxstyleemphasis{Journal of American Statistical Association} 57
(1965): 622-627.
\end{sphinxadmonition}


\subsection{Microaggregation}
\label{\detokenize{anon_methods:id14}}
Microaggregation is most suitable for continuous variables, but can be
extended in some cases to categorical variables. {[}\#foot47{]}\_\_
It is most useful where confidentiality rules have been predetermined
(e.g., a certain threshold for \(k\)-anonymity has been set) that
permit the release of data only if combinations of variables are shared
by more than a predetermined threshold number of respondents
(\(k\)). The first step in microaggregation is the formation of
small groups of individuals that are homogeneous with respect to the
values of selected variables, such as groups with similar income or age.
Subsequently, the values of the selected variables of all group members
are replaced with a common value, e.g., the mean of that group.
Microaggregation methods differ with respect to (i) how the homogeneity
of groups is defined, (ii) the algorithms used to find homogeneous
groups, and (iii) the determination of replacement values. In practice,
microaggregation works best when the values of the variables in the
groups are more homogeneous. When this is the case, then the information
loss due to replacing values with common values for the group will be
smaller than in cases where groups are less homogeneous.

In the univariate case, and also for ordinal categorical variables,
formation of homogeneous groups is straightforward: groups are formed by
first ordering the values of the variable and then creating \(g\)
groups of size \(n_{i}\) for all groups \(i\) in
\(1,\ \ldots,\ g\). This maximizes the within-group homogeneity,
which is measured by the within-groups sum of squares (SSE)
\begin{equation*}
\begin{split}SSE = \sum_{i = 1}^{g}{\sum_{j = 1}^{n_{i}}{\left( x_{ij} - {\overline{x}}_{i} \right)^{T}\left( x_{ij} - {\overline{x}}_{i} \right)}}\end{split}
\end{equation*}
The lower the SSE, the higher the within-group homogeneity. The group
sizes can differ amongst groups, but often groups of equal size are used
to simplify the search %
\begin{footnote}[15]\sphinxAtStartFootnote
Here all groups can have different sizes (i.e., number of
individuals in a group). In practice, the search for homogeneous
groups is simplified by imposing equal group sizes for all groups.
%
\end{footnote}.

The function microaggregation() in \sphinxstyleemphasis{sdcMicro} can be used for univariate
microaggregation. The argument ‘aggr’ specifies the group size. Forming
groups is easier if all groups \textendash{} except maybe the last group of
remainders \textendash{} have the same size. This is the case in the implementation
in \sphinxstyleemphasis{sdcMicro} as it is not possible to have groups of different sizes.
Example 5.16 shows how to use the function microaggregation() in
\sphinxstyleemphasis{sdcMicro}. %
\begin{footnote}[16]\sphinxAtStartFootnote
In this example and the following examples in this section, the
\sphinxstyleemphasis{sdcMicro} object “sdcIntial“ contains a dataset with 2,000
individuals and 39 variables. We selected five categorical
quasi-identifiers and three continuous quasi-identifiers: “INC”,
“EXP” and “WEALTH”.
%
\end{footnote} The default group size is 3 but the
user can specify any desired group size. Choice of group size depends on
the homogeneity within the groups and the required level of protection.
In general it holds that the larger the group, the higher the
protection. A disadvantage of groups of equal sizes is that the data
might be unsuitable for this. For instance, if two individuals have a
low income (e.g., 832 and 966) and four individuals have a high income
(e.g., 3,313, 3,211, 2,987, 3,088), the mean of two groups of size three
(e.g., (832 + 966 + 2,987) / 3 = 1,595 and (3,088 + 3,211 + 3,313) / 3 =
3,204) would represent neither the low nor the high income.

Example 5.16: Applying univariate microaggregation with \sphinxstyleemphasis{sdcMicro}
function microaggregation()

\fvset{hllines={, ,}}%
\begin{sphinxVerbatim}[commandchars=\\\{\}]
sdcInitial \PYG{o}{\PYGZlt{}\PYGZhy{}} \PYG{o}{*}\PYG{o}{*}microaggregation\PYG{o}{*}\PYG{o}{*}\PYGZbs{} \PYG{p}{(}obj \PYG{o}{=} sdcInitial\PYG{p}{,} variables \PYG{o}{=}
\PYG{l+s}{\PYGZsq{}}\PYG{l+s}{INC\PYGZsq{}}\PYG{p}{,} aggr \PYG{o}{=} \PYG{l+m}{3}\PYG{p}{,} method \PYG{o}{=} mafast\PYG{p}{,} measure \PYG{o}{=} \PYG{l+s}{\PYGZdq{}}\PYG{l+s}{mean\PYGZdq{}}\PYG{p}{)}
\end{sphinxVerbatim}

By default, the microaggregation function replaces values with the group
mean. An alternative, more robust approach is to replace group values
with the median. This can be specified in the argument \sphinxstyleemphasis{measure} of the
function microaggregation(). In cases where the median is chosen, one
individual in every group keeps the same value if groups have odd sizes.
In cases where there is a high degree of heterogeneity within the groups
(this is often the case for larger groups), the median is preferred to
preserve the information in the data. An example is income, where one
outlier can lead to multiple outliers being created when using
microaggregation. This is illustrated in Table 5.11. If we choose the
mean as replacement for all values, which are grouped with the outlier
(6,045 in group 2), these records will be assigned values far from their
original values. If we chose the median, the incomes of individuals 1
and 2 are not perturbed, but no value is an outlier. Of course, this
might in itself present problems. \sphinxstylestrong{NOTE: If microaggregation alters
outlying values, this can have a significant impact on the computation
of some measures sensitive to outliers, such as the GINI index.} In the
case where microaggregation is applied to categorical variables, the
median is used to calculate the replacement value for the group.

Table 5.11: Illustrating the effect of choosing mean vs. median for
microaggregation where outliers are concerned


\begin{savenotes}\sphinxattablestart
\centering
\begin{tabulary}{\linewidth}[t]{|T|T|T|T|T|}
\hline
\sphinxstyletheadfamily 
\sphinxstylestrong{ID}
&\sphinxstyletheadfamily 
\sphinxstylestrong{Group}
&\sphinxstyletheadfamily 
\sphinxstylestrong{Income}
&\sphinxstyletheadfamily 
\sphinxstylestrong{Microaggr
egation
(mean)}
&\sphinxstyletheadfamily 
\sphinxstylestrong{Microaggr
egation
(median)}
\\
\hline
1
&
1
&
2,300
&
2,245
&
2,300
\\
\hline
2
&
2
&
2,434
&
3,608
&
2,434
\\
\hline
3
&
1
&
2,123
&
2,245
&
2,300
\\
\hline
4
&
1
&
2,312
&
2,245
&
2,300
\\
\hline
5
&
2
&
6,045
&
3,608
&
2,434
\\
\hline
6
&
2
&
2,345
&
3,608
&
2,434
\\
\hline
\end{tabulary}
\par
\sphinxattableend\end{savenotes}

In case of multiple variables that are candidates for microaggregation,
one possibility is to apply univariate microaggregation to each of the
variables separately. The advantage of univariate microaggregation is
minimal information loss, since the changes in the variables are
limited. The literature shows, however, that disclosure risk can be very
high if univariate microaggregation is applied to several variables
separately and no additional anonymization techniques are applied
(Domingo-Ferrer et al., 2002). To overcome this shortcoming, an
alternative to univariate microaggregation is multivariate
microaggregation.

Multivariate microaggregation is widely used in official statistics. The
first step in multivariate aggregation is the creation of homogeneous
groups based on several variables. Groups are formed based on
multivariate distances between the individuals. Subsequently, the values
of all variables for all group members are replaced with the same
values. Table 5.12 illustrates this with three variables. We see that
the grouping by income, expenditure and wealth leads to a different
grouping, as in the case in Table 5.11, where groups were formed based
only on income.

Table 5.12: Illustration of multivariate microaggregation


\begin{savenotes}\sphinxattablestart
\centering
\begin{tabulary}{\linewidth}[t]{|T|T|T|T|T|T|T|T|}
\hline
\sphinxstyletheadfamily 
ID
&\sphinxstyletheadfamily 
Group
&\sphinxstartmulticolumn{3}%
\begin{varwidth}[t]{\sphinxcolwidth{3}{8}}
\sphinxstyletheadfamily Before microaggregation
\par
\vskip-\baselineskip\vbox{\hbox{\strut}}\end{varwidth}%
\sphinxstopmulticolumn
&\sphinxstartmulticolumn{3}%
\begin{varwidth}[t]{\sphinxcolwidth{3}{8}}
\sphinxstyletheadfamily After microaggregation
\par
\vskip-\baselineskip\vbox{\hbox{\strut}}\end{varwidth}%
\sphinxstopmulticolumn
\\
\hline&&
\sphinxstyleemphasis{Incom
e}
&
\sphinxstyleemphasis{Exp}
&
\sphinxstyleemphasis{Wealt
h}
&
\sphinxstyleemphasis{Incom
e}
&
\sphinxstyleemphasis{Exp}
&
\sphinxstyleemphasis{Wealt
h}
\\
\hline
1
&
1
&
2,300
&
1,714
&
5.3
&
2,285.
7
&
1,846.
3
&
6.3
\\
\hline
2
&
1
&
2,434
&
1,947
&
7.4
&
2,285.
7
&
1,846.
3
&
6.3
\\
\hline
3
&
1
&
2,123
&
1,878
&
6.3
&
2,285.
7
&
1,846.
3
&
6.3
\\
\hline
4
&
2
&
2,312
&
1,950
&
8.0
&
3,567.
3
&
2,814.
0
&
8.3
\\
\hline
5
&
2
&
6,045
&
4,569
&
9.2
&
3,567.
3
&
2,814.
0
&
8.3
\\
\hline
6
&
2
&
2,345
&
1,923
&
7.8
&
3,567.
3
&
2,814.
0
&
8.3
\\
\hline
\end{tabulary}
\par
\sphinxattableend\end{savenotes}

There are several multivariate microaggregation methods that differ with
respect to the algorithm used for creating groups of individuals. There
is a trade-off between speed of the algorithm and within-group
homogeneity, which is directly related to information loss. For large
datasets, this is especially challenging. We discuss the Maximum
Distance to Average Vector (MDAV) algorithm here in more detail. The
MDAV algorithm was first introduced by Domingo-Ferrer and Torra (2005)
and represents a good choice with respect to the trade-off between
computation time and the group homogeneity, computed by the within-group
SSE. The MDAV algorithm is implemented in \sphinxstyleemphasis{sdcMicro}.

The algorithm computes an average record or centroid C, which contains
the average values of all included variables. We select an individual A
with the largest squared Euclidean distance from C, and build a group of
\(k\) records around A. The group of \(k\) records is made up of
A and the \(k\)-1 records closest to A measured by the Euclidean
distance. Next, we select another individual B, with the largest squared
Euclidean distance from individual A. With the remaining records, we
build a group of \(k\) records around B. In the same manner, we
select an individual D with the largest distance from B and, with the
remaining records, build a new group of \(k\) records around D. The
process is repeated until we have fewer than 2\(k\) records
remaining. The MDAV algorithm creates groups of equal size with the
exception of maybe one last group of remainders. The microaggregated
dataset is then computed by replacing each record in the original
dataset by the average values of the group to which it belongs. Equal
group sizes, however, may not be ideal for data characterized by greater
variability. In \sphinxstyleemphasis{sdcMicro} multivariate microaggregation is also
implemented in the function microaggregation(). Example 5.17 shows how
to choose the MDAV algorithm in \sphinxstyleemphasis{sdcMicro}.

Example 5.17: Multivariate microaggregation with the Maximum Distance to
Average Vector (MDAV) algorithm in \sphinxstyleemphasis{sdcMicro}

\fvset{hllines={, ,}}%
\begin{sphinxVerbatim}[commandchars=\\\{\}]
sdcInitial \PYG{o}{\PYGZlt{}\PYGZhy{}} microaggregation\PYG{p}{(}obj \PYG{o}{=} sdcInitial\PYG{p}{,} variables \PYG{o}{=} \PYG{k+kt}{c}\PYG{p}{(}\PYG{l+s}{\PYGZdq{}}\PYG{l+s}{INC\PYGZdq{}}\PYG{p}{,} \PYG{l+s}{\PYGZdq{}}\PYG{l+s}{EXP\PYGZdq{}}\PYG{p}{,} \PYG{l+s}{\PYGZdq{}}\PYG{l+s}{WEALTH\PYGZdq{}}\PYG{p}{)}\PYG{p}{,} method \PYG{o}{=} \PYG{l+s}{\PYGZdq{}}\PYG{l+s}{mdav\PYGZdq{}}\PYG{p}{)}
\end{sphinxVerbatim}

It is also possible to group variables only within strata. This reduces
the computation time and adds an extra layer of protection to the data,
because of the greater uncertainty produced %
\begin{footnote}[17]\sphinxAtStartFootnote
Also the homogeneity in the groups will be generally lower, leading
to larger changes, higher protection, but also more information loss,
unless the strata variable correlates with the microaggregation
variable.
%
\end{footnote}. In
\sphinxstyleemphasis{sdcMicro} this can be achieved by specifying the strata variables, as
shown in Example 5.18.

Example 5.18: Specifying strata variables for microaggregation

\fvset{hllines={, ,}}%
\begin{sphinxVerbatim}[commandchars=\\\{\}]
sdcInitial \PYG{o}{\PYGZlt{}\PYGZhy{}} microaggregation\PYG{p}{(}obj \PYG{o}{=} sdcInitial\PYG{p}{,} variables \PYG{o}{=} \PYG{k+kt}{c}\PYG{p}{(}\PYG{l+s}{\PYGZdq{}}\PYG{l+s}{INC\PYGZdq{}}\PYG{p}{,} \PYG{l+s}{\PYGZdq{}}\PYG{l+s}{EXP\PYGZdq{}}\PYG{p}{,} \PYG{l+s}{\PYGZdq{}}\PYG{l+s}{WEALTH\PYGZdq{}}\PYG{p}{)}\PYG{p}{,} method \PYG{o}{=} \PYG{l+s}{\PYGZdq{}}\PYG{l+s}{mdav\PYGZdq{}}\PYG{p}{,} strata\PYGZus{}variables \PYG{o}{=} \PYG{k+kt}{c}\PYG{p}{(}\PYG{l+s}{\PYGZdq{}}\PYG{l+s}{strata\PYGZdq{}}\PYG{p}{)}\PYG{p}{)}
\end{sphinxVerbatim}

Besides the method MDAV, there are few other grouping methods
implemented in \sphinxstyleemphasis{sdcMicro} (Templ, Meindl and Kowarik, 2014). Table 5.13
gives an overview of these methods. Whereas the method ‘MDAV’ uses the
Euclidian distance, the method ‘rmd’ uses the Mahalanobis distance
instead. An alternative to these methods is sorting the respondents
based on the first principal component (PC), which is the projection of
all variables into a one-dimensional space maximizing the variance of
this projection. The performance of this method depends on the share of
the total variance in the data that is explained by the first PC. The
‘rmd’ method is computationally more intensive due to the computation of
Mahalanobis distances, but provides better results with respect to group
homogeneity. It is recommended for smaller datasets (ibid.).

Table 5.13: Grouping methods for microaggregation that are implemented
in \sphinxstyleemphasis{sdcMicro}


\begin{savenotes}\sphinxattablestart
\centering
\begin{tabulary}{\linewidth}[t]{|T|T|}
\hline
\sphinxstyletheadfamily 
\sphinxstylestrong{Method / option in sdcMicro}
&\sphinxstyletheadfamily 
\sphinxstylestrong{Description}
\\
\hline
mdav
&
grouping is based on classical
(Euclidean) distance measures
\\
\hline
rmd
&
grouping is based on robust
multivariate (Mahalanobis)
distance measures
\\
\hline
pca
&
grouping is based on principal
component analysis whereas the
data are sorted on the first
principal component
\\
\hline
clustpppca
&
grouping is based on clustering
and (robust) principal component
analysis for each cluster
\\
\hline
influence
&
grouping is based on clustering
and aggregation is performed
within clusters
\\
\hline
\end{tabulary}
\par
\sphinxattableend\end{savenotes}

In case of several variables to be used for microaggregation, looking
first at the covariance or correlation matrix of these variables is
recommended. If not all variables correlate well, but two or more sets
of variables show high correlation, less information loss will occur
when applying microaggregation separately to these sets of variables. In
general, less information loss will occur when applying multivariate
microaggregation, if the variables are highly correlated. The advantage
of replacing the values with the mean of the groups rather than other
replacement values has the advantage that the overall means of the
variables are preserved.

\begin{sphinxadmonition}{note}{Recommended Reading Material on Microaggregation}

Domingo-Ferrer, Josep, and Josep Maria Mateo-Sanz. 2002.”Practical
data-oriented microaggregation for statistical disclosure control.”
\sphinxstyleemphasis{IEEE Transactions on Knowledge and Data Engineering 14} (2002):
189-201.

Hansen, Stephen Lee, and Sumitra Mukherjee. 2003. “A polynomial
algorithm for univariate optimal.” \sphinxstyleemphasis{IEEE Transactions on Knowledge and
Data Engineering} 15 (2003): 1043-1044.

Hundepool, Anco, Josep Domingo-Ferrer, Luisa Franconi, Sarah Giessing,
Rainer Lenz, Jane Naylor, Eric Schulte Nordholt, Giovanni Seri, and
Peter Paul de Wolf. 2006. \sphinxstyleemphasis{Handbook on Statistical Disclosure Control.}
ESSNet SDC. \sphinxurl{http://neon.vb.cbs.nl/casc/handbook.htm}

Hundepool, Anco, Josep Domingo-Ferrer, Luisa Franconi, Sarah Giessing,
Eric Schulte Nordholt, Keith Spicer, and Peter Paul de Wolf. 2012.
\sphinxstyleemphasis{Statistical Disclosure Control.} Chichester: John Wiley \& Sons Ltd.
doi:10.1002/9781118348239.

Templ, Matthias, Bernhard Meindl, Alexander Kowarik, and Shuang Chen.
2014, August. “International Household Survey Network (IHSN).”
\sphinxurl{http://www.ihsn.org/home/software/disclosure-control-toolbox}. (accessed
November 13, 2014).
\end{sphinxadmonition}


\subsection{Noise addition}
\label{\detokenize{anon_methods:noise-addition}}
Noise addition, or noise masking, means adding or subtracting (small)
values to the original values of a variable, and is most suited to
protect continuous variables (see Brand (2002) for an overview). Noise
addition can prevent exact matching of continuous variables. The
advantages of noise addition are that the noise is typically continuous
with mean zero, and exact matching with external files will not be
possible. Depending on the magnitude of noise added, however,
approximate interval matching might still be possible.

When using noise addition to protect data, it is important to consider
the type of data, the intended use of the data and the properties of the
data before and after noise addition, i.e., the distribution \textendash{}
particularly the mean \textendash{} covariance and correlation between the perturbed
and original datasets.

Depending on the data, it may also be useful to check that the perturbed
values fall within a meaningful range of values. Figure 5.7 on page 68
illustrates the changes in data distribution with increasing levels of
noise. For data that has outliers, it is important to note that when the
perturbed data distribution is similar to the original data distribution
(e.g., at low noise levels), noise addition will not protect outliers.
After noise addition, these outliers can generally still be detected as
outliers and hence easily be identified. An example is a single very
high income in a certain region. After perturbing this income value, the
value will still be recognized as the highest income in that region and
can thus be used for re-identification. This is illustrated in Figure
5.6, where 10 original observations (open circles) and the anonymized
observations (red triangles) are plotted. The tenth observation is an
outlier. The values of the first nine observations are sufficiently
protected by adding noise: their magnitude and order has changed and
exact or interval matching can be successfully prevented. The outlier is
not sufficiently protected since, after noise addition, the outlier can
still be easily identified. The fact that the absolute value has changed
is not sufficient protection. On the other hand, at high noise levels,
protection is higher even for the outliers, but the data structure is
not preserved and the information loss is large, which is not an ideal
situation. One way to circumvent the outlier problem is to add noise of
larger magnitude to outliers than to the other values.

\noindent\sphinxincludegraphics[width=6.48958in,height=3.23958in]{{image8}.png}

Figure 5.6: Illustration of effect of noise addition to outliers

There are several noise addition algorithms. The simplest version of
noise addition is uncorrelated additive normally distributed noise,
where \(x_{j}\), the original values of variable
\(j\)are replaced by

\(z_{j} = x_{j} + \varepsilon_{j}\),

where
\(\varepsilon_{j}\ \sim\ N(0,\ \ \sigma_{\varepsilon_{j}}^{2})\ \)and
\(\sigma_{\varepsilon_{j}} = \alpha \bullet \sigma_{j}\) with
\(\sigma_{j}\) the standard deviation of the original data. In this
way, the mean and the covariances are preserved, but not the variances
and correlation coefficient. If the level of noise added,
\(\alpha\), is disclosed to the user, many statistics can be
consistently estimated from the perturbed data. The added noise is
proportional to the variance of the original variable. The magnitude of
the noise added is specified by the parameter \(\alpha\), which
specifies this proportion. The standard deviation of the perturbed data
is \(1 + \alpha\) times the standard deviation of the perturbed
data. A decision on the magnitude of noise added should be informed by
the legal situation regarding data privacy, data sensitivity and the
acceptable levels of disclosure risk and information loss. In general,
the level of noise is a function of the variance of the original
variables, the level of protection needed and the desired value range
after anonymization %
\begin{footnote}[18]\sphinxAtStartFootnote
Common values for \(\alpha\) are between 0.5 and 2. The default
value in the \sphinxstyleemphasis{sdcMicro} function addNoise() is 150, which is too
large for most datasets; the level of noise should be set in the
argument ‘noise’.
%
\end{footnote}. An \(\alpha\) value that
is too small will lead to insufficient protection, while an
\(\alpha\) value that is too high will make the data useless for
data users.

In \sphinxstyleemphasis{sdcMicro} noise addition is implemented in the function addNoise().
The algorithm and parameter can be specified as arguments in the
function addNoise(). Simple noise addition is implemented in the
function addNoise() with the value “additive” for the argument ‘method’.
Example 5.19 shows how to use \sphinxstyleemphasis{sdcMicro} to add uncorrelated noise to
expenditure variables, where the standard deviation of the added noise
equals half the standard deviation of the original
variables. %
\begin{footnote}[19]\sphinxAtStartFootnote
In this example and the following examples in this section, the
\sphinxstyleemphasis{sdcMicro} object “sdcIntial“ contains a dataset with 2,000
individuals and 39 variables. We selected five categorical
quasi-identifiers and 12 continuous quasi-identifiers. These are the
expenditure components “TFOODEXP”, “TALCHEXP”, “TCLTHEXP”,
“THOUSEXP”, “TFURNEXP”, “THLTHEXP”, “TTRANSEXP”, “TCOMMEXP”,
“TRECEXP”, “TEDUEXP”, “TRESTHOTEXP”, “TMISCEXP“.
%
\end{footnote} Noise is added to all selected
variables.

Example 5.19: Uncorrelated noise addition

\fvset{hllines={, ,}}%
\begin{sphinxVerbatim}[commandchars=\\\{\}]
sdcInitial \PYG{o}{\PYGZlt{}\PYGZhy{}} addNoise\PYG{p}{(}obj \PYG{o}{=} sdcInitial\PYG{p}{,} variables \PYG{o}{=} \PYG{k+kt}{c}\PYG{p}{(}\PYG{l+s}{\PYGZsq{}}\PYG{l+s}{TOTFOOD\PYGZsq{}}\PYG{p}{,} \PYG{l+s}{\PYGZsq{}}\PYG{l+s}{TOTHLTH\PYGZsq{}}\PYG{p}{,} \PYG{l+s}{\PYGZsq{}}\PYG{l+s}{TOTALCH\PYGZsq{}}\PYG{p}{,}     \PYG{l+s}{\PYGZsq{}}\PYG{l+s}{TOTCLTH\PYGZsq{}}\PYG{p}{,} \PYG{l+s}{\PYGZsq{}}\PYG{l+s}{TOTHOUS\PYGZsq{}}\PYG{p}{,} \PYG{l+s}{\PYGZsq{}}\PYG{l+s}{TOTFURN\PYGZsq{}}\PYG{p}{,} \PYG{l+s}{\PYGZsq{}}\PYG{l+s}{TOTTRSP\PYGZsq{}}\PYG{p}{,} \PYG{l+s}{\PYGZsq{}}\PYG{l+s}{TOTCMNQ\PYGZsq{}}\PYG{p}{,} \PYG{l+s}{\PYGZsq{}}\PYG{l+s}{TOTRCRE\PYGZsq{}}\PYG{p}{,} \PYG{l+s}{\PYGZsq{}}\PYG{l+s}{TOTEDUC\PYGZsq{}}\PYG{p}{,} \PYG{l+s}{\PYGZsq{}}\PYG{l+s}{TOTHOTL\PYGZsq{}}\PYG{p}{,} \PYG{l+s}{\PYGZsq{}}\PYG{l+s}{TOTMISC\PYGZsq{}}\PYG{p}{)}\PYG{p}{,} noise \PYG{o}{=} \PYG{l+m}{0.5}\PYG{p}{,} method \PYG{o}{=} \PYG{l+s}{\PYGZdq{}}\PYG{l+s}{additive\PYGZdq{}}\PYG{p}{)}
\end{sphinxVerbatim}

Figure 5.7 shows the frequency distribution of a numeric continuous
variable and the distribution before and after noise addition with
different levels of noise (0.1, 0.5, 1, 2 and 5). The first plot shows
the distribution of the original values. The histograms clearly show
that noise of large magnitudes (high values of alpha) lead to a
distribution of the data far from the original values. The distribution
of the data changes to a normal distribution when the magnitude of the
noise grows respective to the variance of the data. The mean in the data
is preserved, but, with an increased level of noise, the variance of the
perturbed data grows. After adding noise of magnitude 5, the
distribution of the original data is completely destroyed.

\noindent\sphinxincludegraphics[width=6.48958in,height=3.23958in]{{image9}.png}

Figure 5.7: Frequency distribution of a continuous variable before and
after noise addition

Figure 5.8 shows the value range of a variable before adding noise (no
noise) and after adding several levels of noise (\(\alpha\) from 0.1
to 1.5 with 0.1 increments). In the figure, the minimum value, the
20$^{\text{th}}$, 30$^{\text{th}}$, 40$^{\text{th}}$ percentiles, the median, the
60$^{\text{th}}$, 70$^{\text{th}}$, 80$^{\text{th}}$ and 90$^{\text{th}}$
percentiles and the maximum value are plotted. The median (50:sup:\sphinxtitleref{th}
percentile) is indicated with the red “+” symbol. From Figure 5.7 and
Figure 5.8, it is apparent that the range of values expands after noise
addition, and the median stays roughly at the same level, as does the
mean by construction. The larger the magnitude of noise added, the wider
the value range. In cases where the variable should stay in a certain
value range (e.g., only positive values, between 0 and 100), this can be
a disadvantage of noise addition. For instance, expenditure variables
typically have non-negative values, but adding noise to these variables
can generate negative values, which are difficult to interpret. One way
to get around this problem is to set any negative values to zero. This
truncation of values below a certain threshold, however, will distort
the distribution (mean and variance matrix) of the perturbed data. This
means that the characteristics that were preserved by noise addition,
such as the conservation of the mean and covariance matrix, are
destroyed and the user, even with knowledge of the magnitude of the
noise, can no longer use the data for consistent estimation.

Another way to avoid negative values is the application of
multiplicative rather than additive noise. In that case, variables are
multiplied by a random factor with expectation 1 and a positive
variance. This will also lead to larger perturbations (in absolute
value) of large initial values (outliers). If the variance of the noise
added is small, there will be no or few negative factors and thus fewer
sign changes than in case of additive noise masking. Multiplicative
noise masking is not implemented in \sphinxstyleemphasis{sdcMicro}, but can be relatively
easily implemented in base \sphinxstyleemphasis{R} by generating a vector of random numbers
and multiplying the data with this vector. For more information on
multiplicative noise masking and the properties of the data after
masking, we refer to Kim and Winkler (2003).

\noindent\sphinxincludegraphics[width=6.48958in,height=3.23958in]{{image10}.png}

Figure 5.8: Noise levels and the impact on the value range (percentiles)

If two or more variables are selected for noise addition, correlated
noise addition is preferred to preserve the correlation structure in the
data. In this case, the covariance matrix of noise
\(\Sigma_{\varepsilon}\ \)is proportional to the covariance matrix
of the original data \(\Sigma_{X}:\)
\begin{equation*}
\begin{split}\Sigma_{\varepsilon} = \alpha \Sigma_{X}\end{split}
\end{equation*}
In the addNoise() function of the \sphinxstyleemphasis{sdcMicro} package, correlated noise
addition can be used by specifying the methods ‘correlated’ or
‘correlated2’. The method “correlated” assumes that the variables are
approximately normally distributed. The method ‘correlated2’ is a
version of the method ‘correlated’, which is robust against the
normality assumption. Example 5.20 shows how to use the ‘correlated2’
method. The normality of variables can be investigated in \sphinxstyleemphasis{R}, with, for
instance, a Jarque-Bera or Shapiro-Wilk test %
\begin{footnote}[20]\sphinxAtStartFootnote
The Shapiro-Wilk test is implemented in the function shapiro.test()
from the package \sphinxstyleemphasis{stats} in \sphinxstyleemphasis{R}. The Jarque-Bera test has several
implementations in \sphinxstyleemphasis{R}, for example, in the function
jarque.bera.test() from the package \sphinxstyleemphasis{tseries}.
%
\end{footnote}.

Example 5.20: Correlated noise addition

\fvset{hllines={, ,}}%
\begin{sphinxVerbatim}[commandchars=\\\{\}]
sdcInitial \PYG{o}{\PYGZlt{}\PYGZhy{}} addNoise\PYG{p}{(}obj \PYG{o}{=} sdcInitial\PYG{p}{,} variables \PYG{o}{=} \PYG{k+kt}{c}\PYG{p}{(}\PYG{l+s}{\PYGZsq{}}\PYG{l+s}{TOTFOOD\PYGZsq{}}\PYG{p}{,} \PYG{l+s}{\PYGZsq{}}\PYG{l+s}{TOTHLTH\PYGZsq{}}\PYG{p}{,} \PYG{l+s}{\PYGZsq{}}\PYG{l+s}{TOTALCH\PYGZsq{}}\PYG{p}{,} \PYG{l+s}{\PYGZsq{}}\PYG{l+s}{TOTCLTH\PYGZsq{}}\PYG{p}{,} \PYG{l+s}{\PYGZsq{}}\PYG{l+s}{TOTHOUS\PYGZsq{}}\PYG{p}{,} \PYG{l+s}{\PYGZsq{}}\PYG{l+s}{TOTFURN\PYGZsq{}}\PYG{p}{,} \PYG{l+s}{\PYGZsq{}}\PYG{l+s}{TOTTRSP\PYGZsq{}}\PYG{p}{,} \PYG{l+s}{\PYGZsq{}}\PYG{l+s}{TOTCMNQ\PYGZsq{}}\PYG{p}{,} \PYG{l+s}{\PYGZsq{}}\PYG{l+s}{TOTRCRE\PYGZsq{}}\PYG{p}{,} \PYG{l+s}{\PYGZsq{}}\PYG{l+s}{TOTEDUC\PYGZsq{}}\PYG{p}{,} \PYG{l+s}{\PYGZsq{}}\PYG{l+s}{TOTHOTL\PYGZsq{}}\PYG{p}{,} \PYG{l+s}{\PYGZsq{}}\PYG{l+s}{TOTMISC\PYGZsq{}}\PYG{p}{)}\PYG{p}{,} noise \PYG{o}{=} \PYG{l+m}{0.5}\PYG{p}{,} method \PYG{o}{=} \PYG{l+s}{\PYGZdq{}}\PYG{l+s}{correlated2\PYGZdq{}}\PYG{p}{)}
\end{sphinxVerbatim}

In many cases, only the outliers have to be protected, or have to be
protected more. The method ‘outdect’ adds noise only to the outliers,
which is illustrated in Example 5.21. The outliers are identified with
univariate and robust multivariate procedures based on a robust
Mahalanobis distance calculated by the MCD estimator (Templ et al.,
2014). Nevertheless, noise addition is not the most suitable method for
outlier protection.

Example 5.21: Noise addition for outliers using the ‘outdect’ method

\fvset{hllines={, ,}}%
\begin{sphinxVerbatim}[commandchars=\\\{\}]
sdcInitial \PYG{o}{\PYGZlt{}\PYGZhy{}} addNoise\PYG{p}{(}obj \PYG{o}{=} sdcInitial\PYG{p}{,} variables \PYG{o}{=} \PYG{k+kt}{c}\PYG{p}{(}\PYG{l+s}{\PYGZsq{}}\PYG{l+s}{TOTFOOD\PYGZsq{}}\PYG{p}{,} \PYG{l+s}{\PYGZsq{}}\PYG{l+s}{TOTHLTH\PYGZsq{}}\PYG{p}{,} \PYG{l+s}{\PYGZsq{}}\PYG{l+s}{TOTALCH\PYGZsq{}}\PYG{p}{,} \PYG{l+s}{\PYGZsq{}}\PYG{l+s}{TOTCLTH\PYGZsq{}}\PYG{p}{,} \PYG{l+s}{\PYGZsq{}}\PYG{l+s}{TOTHOUS\PYGZsq{}}\PYG{p}{,} \PYG{l+s}{\PYGZsq{}}\PYG{l+s}{TOTFURN\PYGZsq{}}\PYG{p}{,} \PYG{l+s}{\PYGZsq{}}\PYG{l+s}{TOTTRSP\PYGZsq{}}\PYG{p}{,} \PYG{l+s}{\PYGZsq{}}\PYG{l+s}{TOTCMNQ\PYGZsq{}}\PYG{p}{,} \PYG{l+s}{\PYGZsq{}}\PYG{l+s}{TOTRCRE\PYGZsq{}}\PYG{p}{,} \PYG{l+s}{\PYGZsq{}}\PYG{l+s}{TOTEDUC\PYGZsq{}}\PYG{p}{,} \PYG{l+s}{\PYGZsq{}}\PYG{l+s}{TOTHOTL\PYGZsq{}}\PYG{p}{,} \PYG{l+s}{\PYGZsq{}}\PYG{l+s}{TOTMISC\PYGZsq{}}\PYG{p}{)}\PYG{p}{,} noise \PYG{o}{=} \PYG{l+m}{0.5}\PYG{p}{,} method \PYG{o}{=} \PYG{l+s}{\PYGZdq{}}\PYG{l+s}{outdect\PYGZdq{}}\PYG{p}{)}
\end{sphinxVerbatim}

If noise addition is applied to variables that are a ratio of an
aggregate, this structure can be destroyed by noise addition. Examples
are income and expenditure data with many income and expenditure
categories. The categories add up to total income or total expenditures.
In the original data, the aggregates match with the sum of the
components. After adding noise to their components (e.g., different
expenditure categories), however, their new aggregates will not
necessarily match the sum of the categories anymore. One way to keep
this structure is to add noise only to the aggregates and release the
components as ratio of the perturbed aggregates. Example 5.22
illustrates this by adding noise to the total of expenditures.
Subsequently, the ratios of the initial expenditure categories are used
for each individual to reconstruct the perturbed values for each
expenditure category.

Example 5.22: Noise addition to aggregates and their components

\fvset{hllines={, ,}}%
\begin{sphinxVerbatim}[commandchars=\\\{\}]
\PYG{c+c1}{\PYGZsh{} Add noise to totals (income / expenditures)}
sdcInital \PYG{o}{\PYGZlt{}\PYGZhy{}} addNoise\PYG{p}{(}noise \PYG{o}{=} \PYG{l+m}{0.5}\PYG{p}{,} obj \PYG{o}{=} sdcInitial\PYG{p}{,} variables\PYG{o}{=}\PYG{k+kt}{c}\PYG{p}{(}\PYG{l+s}{\PYGZdq{}}\PYG{l+s}{EXP\PYGZdq{}}\PYG{p}{,} \PYG{l+s}{\PYGZdq{}}\PYG{l+s}{INC\PYGZdq{}}\PYG{p}{)}\PYG{p}{,} method\PYG{o}{=}\PYG{l+s}{\PYGZdq{}}\PYG{l+s}{additive\PYGZdq{}}\PYG{p}{)}
\PYG{c+c1}{\PYGZsh{} Multiply anonymized totals with ratios to obtain anonymized components}
compExp \PYG{o}{\PYGZlt{}\PYGZhy{}}  \PYG{k+kt}{c}\PYG{p}{(}\PYG{l+s}{\PYGZdq{}}\PYG{l+s}{TOTFOOD\PYGZdq{}}\PYG{p}{,}  \PYG{l+s}{\PYGZdq{}}\PYG{l+s}{TOTALCH\PYGZdq{}}\PYG{p}{,}  \PYG{l+s}{\PYGZdq{}}\PYG{l+s}{TOTCLTH\PYGZdq{}}\PYG{p}{,}  \PYG{l+s}{\PYGZdq{}}\PYG{l+s}{TOTHOUS\PYGZdq{}}\PYG{p}{,}  \PYG{l+s}{\PYGZdq{}}\PYG{l+s}{TOTFURN\PYGZdq{}}\PYG{p}{,}  \PYG{l+s}{\PYGZdq{}}\PYG{l+s}{TOTHLTH\PYGZdq{}}\PYG{p}{,}                                \PYG{l+s}{\PYGZdq{}}\PYG{l+s}{TOTTRSP\PYGZdq{}}\PYG{p}{,}  \PYG{l+s}{\PYGZdq{}}\PYG{l+s}{TOTCMNQ\PYGZdq{}}\PYG{p}{,} \PYG{l+s}{\PYGZdq{}}\PYG{l+s}{TOTRCRE\PYGZdq{}}\PYG{p}{,}  \PYG{l+s}{\PYGZdq{}}\PYG{l+s}{TOTEDUC\PYGZdq{}}\PYG{p}{,}  \PYG{l+s}{\PYGZdq{}}\PYG{l+s}{TOTHOTL\PYGZdq{}}\PYG{p}{,}  \PYG{l+s}{\PYGZdq{}}\PYG{l+s}{TOTMISC\PYGZdq{}}\PYG{p}{)}

sdcInital\PYG{o}{@}manipNumVars\PYG{p}{[}\PYG{p}{,}compExp\PYG{p}{]} \PYG{o}{\PYGZlt{}\PYGZhy{}} sdcInital\PYG{o}{@}manipNumVars\PYG{p}{[}\PYG{p}{,}\PYG{l+s}{\PYGZdq{}}\PYG{l+s}{HHEXP\PYGZus{}N\PYGZdq{}}\PYG{p}{]} \PYG{o}{*}
                            sdcInital\PYG{o}{@}origData\PYG{p}{[}\PYG{p}{,}compExp\PYG{p}{]}\PYG{o}{/} sdcInital\PYG{o}{@}origData\PYG{p}{[}\PYG{p}{,}\PYG{l+s}{\PYGZdq{}}\PYG{l+s}{HHEXP\PYGZus{}N\PYGZdq{}}\PYG{p}{]}

\PYG{c+c1}{\PYGZsh{} Recalculate risks after manually changing values in sdcMicro object}
sdcInitial \PYG{o}{\PYGZlt{}\PYGZhy{}} calcRisks\PYG{p}{(}sdcInital\PYG{p}{)}
\end{sphinxVerbatim}

\begin{sphinxadmonition}{note}{Recommended Reading Material on Noise Addition}

Brand, Ruth. 2002. “Microdata Protection through Noise Addition.” In
\sphinxstyleemphasis{Inference Control in Statistical Databases - From Theory to Practice},
edited byJosep Domingo-Ferrer. Lecture Notes in Computer Science Series
2316, 97-116. Berlin Heidelberg: Springer.
\sphinxurl{http://link.springer.com/chapter/10.1007\%2F3-540-47804-3\_8}

Kim, Jay J, and William W Winkler. 2003. “Multiplicative Noise for
Masking Continuous Data.” \sphinxstyleemphasis{Research Report Series} (Statistical Research
Division. US Bureau of the Census).
\sphinxurl{https://www.census.gov/srd/papers/pdf/rrs2003-01.pdf}

Torra, Vicenç, and Isaac Cano. 2011. “Edit Constraints on
Microaggregation and Additive Noise.” In \sphinxstyleemphasis{Privacy and Security Issues in
Data Mining and Machine Learning}, edited by C. Dimitrakakis, A.
Gkoulalas-Divanis, A. Mitrokotsa, V. S. Verykios, Y. Saygin. Lecture
Notes in Computer Science Volume 6549, 1-14. Berlin Heidelberg:
Springer. \sphinxurl{http://link.springer.com/book/10.1007/978-3-642-19896-0}

Mivule, K. 2013. “Utilizing Noise Addition for Data Privacy, An
Overview.” \sphinxstyleemphasis{Proceedings of the International Conference on Information
and Knowledge Engineering (IKE 2012)}, (pp.65-71).Las Vegas, USA.
\sphinxurl{http://arxiv.org/ftp/arxiv/papers/1309/1309.3958.pdf}
\end{sphinxadmonition}


\subsection{Rank swapping}
\label{\detokenize{anon_methods:rank-swapping}}
Data swapping is based on interchanging values of a certain variable
across records. Rank swapping is one type of data swapping, which is
defined for ordinal and continuous variables. For rank swapping, the
values of the variable are first ordered. The possible number of values
for a variable to swap with is constrained by the values in a
neighborhood around the original value in the ordered values of the
dataset. The size of this neighborhood can be specified, e.g., as a
percentage of the total number of observations. This also means that a
value can be swapped with the same or very similar values. This is
especially the case if the neighborhood is small or there are only a few
different values in the variable (ordinal variable). An example is the
variables “education” with only few categories: (‘none’, ‘primary’,
‘secondary’, ‘tertiary’). In these cases, rank swapping is not a
suitable method.

If rank swapping is applied to several variables simultaneously, the
correlation structure between the variables is preserved. Therefore, it
is important to check whether the correlation structure in the data is
plausible. Rank swapping is implemented in the function rankSwap() in
\sphinxstyleemphasis{sdcMicro}. The variables, which have to be swapped, should be specified
in the argument ‘variables’. By default, values below the 5$^{\text{th}}$
percentile and above the 95$^{\text{th}}$ percentile are top and bottom
coded and replaced by their average value (see Section 5.2.1.2 on top
and bottom coding). By specifying the options ‘TopPercent’ and
‘BottomPercent’ we can choose these percentiles. The argument ‘P’
defines the size of the neighborhood as percentage of the sample size.
If the value ‘p’ is 0.05, the neighborhood will be of size 0.05 *
\(n\), where \(n\) is the sample size. Since rank swapping is a
probabilistic method, i.e., the swapping depends on a random number
generating mechanism, specifying a seed for the random number generator
before using rank swapping is recommended to guarantee reproducibility
of results. The seed can also be specified as a function argument in the
function rankSwap(). Example 5.23 shows how to apply rank swapping with
\sphinxstyleemphasis{sdcMicro}. If the variables contain missing values (NA in \sphinxstyleemphasis{R}), the
function rankSwap() will automatically recode those to the value
specified in the ‘missing’ argument. This value should not be in the
value range of any of the variables. After using the function
rankSwap(), these values should be recoded NA. This is shown in the
Example 5.23.

Example 5.23: Rank swapping using \sphinxstyleemphasis{sdcMicro}

\fvset{hllines={, ,}}%
\begin{sphinxVerbatim}[commandchars=\\\{\}]
\PYG{c+c1}{\PYGZsh{} Check correlation structure between the variables}
cor\PYG{p}{(}\PYG{k+kp}{file}\PYG{o}{\PYGZdl{}}TOTHOUS\PYG{p}{,} \PYG{k+kp}{file}\PYG{o}{\PYGZdl{}}TOTFOOD\PYG{p}{)}
\PYG{c+c1}{\PYGZsh{}\PYGZsh{} [1] 0.3811335}

\PYG{c+c1}{\PYGZsh{} Set seed for random number generator}
\PYG{k+kp}{set.seed}\PYG{p}{(}\PYG{l+m}{12345}\PYG{p}{)}

\PYG{c+c1}{\PYGZsh{} Apply rank swapping}
rankSwap\PYG{p}{(}sdcInitial\PYG{p}{,} variables \PYG{o}{=} \PYG{k+kt}{c}\PYG{p}{(}\PYG{l+s}{\PYGZdq{}}\PYG{l+s}{TOTHOUS\PYGZdq{}}\PYG{p}{,} \PYG{l+s}{\PYGZdq{}}\PYG{l+s}{TOTFOOD\PYGZdq{}}\PYG{p}{)}\PYG{p}{,} missing \PYG{o}{=} \PYG{k+kc}{NA}\PYG{p}{)}
\end{sphinxVerbatim}

Rank swapping has been found to yield good results with respect to the
trade-off between information loss and data protection (Domingo-Ferrer
and Torra, 2001). Rank swapping is not useful for variables with few
different values or many missing values, since the swapping in that case
will not result in altered values. Also, if the intruder knows to whom
the highest or lowest value of a specific variable belongs (e.g.,
income), the level of this variable will be disclosed after rank
swapping, because the values themselves are not altered and the original
values are all disclosed. This can be solved by top and bottom coding
the lowest and/or highest values.

\begin{sphinxadmonition}{note}{Recommended Reading Material on Rank Swapping}

Dalenius T. and Reiss S.P. 1978. Data-swapping: a technique for
disclosure control (extended abstract). In Proc. ASA Section on Survey
Research Methods. American Statistical Association, Washington DC,
191\textendash{}194.

Domingo-Ferrer J. and Torra V. 2001. “A Quantitative Comparison of
Disclosure Control Methods for Microdata.” In \sphinxstyleemphasis{Confidentiality,
Disclosure and Data Access: Theory and Practical Applications for
Statistical Agencies}, edited by P. Doyle, J.I. Lane, J.J.M. Theeuwes,
and L. Zayatz, 111\textendash{}134. Amsterdam, North-Holland.

Hundepool A., Van de Wetering A., Ramaswamy R., Franconi F., Polettini
S., Capobianchi A., De Wolf P.-P., Domingo-Ferrer J., Torra V., Brand R.
and Giessing S. 2007. \(\mu\)-Argus User’s Manual version 4.1.
\end{sphinxadmonition}


\subsection{Shuffling}
\label{\detokenize{anon_methods:shuffling}}
Shuffling as introduced by Muralidhar and Sarathy (2006) is similar to
swapping, but uses an underlying regression model for the variables to
determine which variables are swapped. Shuffling can be used for
continuous variables and is a deterministic method. Shuffling maintains
the marginal distributions in the shuffled data. Shuffling, however,
requires a complete ranking of the data, which can be computationally
very intensive for large datasets with several variables.

The method is explained in detail in Muralidhar and Sarathy (2006). The
idea is to rank the individuals based on their original variables. Then
fit a regression model with the variables to be protected as regressands
and a set of variables that predict this variable well (i.e., are
correlated with) as regressors. This regression model is used to
generate \(n\) synthetic (predicted) values for each variable that
has to be protected. These generated values are also ranked and each
original value is replaced with another original value with the rank
that corresponds to the rank of the generated value. This means that all
original values will be in the data. Table 5.14 presents a simplified
example of the shuffling method. The regressands are not specified in
this example.

Table 5.14: Simplified example of the shuffling method


\begin{savenotes}\sphinxattablestart
\centering
\begin{tabulary}{\linewidth}[t]{|T|T|T|T|T|T|}
\hline
\sphinxstyletheadfamily 
\sphinxstylestrong{ID}
&\sphinxstyletheadfamily 
\sphinxstylestrong{Income
(orig)}
&\sphinxstyletheadfamily 
\sphinxstylestrong{Rank
(orig)}
&\sphinxstyletheadfamily 
\sphinxstylestrong{Income
(pred)}
&\sphinxstyletheadfamily 
\sphinxstylestrong{Rank
(pred)}
&\sphinxstyletheadfamily 
\sphinxstylestrong{Shuffle
d
values}
\\
\hline
1
&
2,300
&
2
&
2,466.56
&
4
&
2,345
\\
\hline
2
&
2,434
&
6
&
2,583.58
&
7
&
2,543
\\
\hline
3
&
2,123
&
1
&
2,594.17
&
8
&
2,643
\\
\hline
4
&
2,312
&
3
&
2,530.97
&
6
&
2,434
\\
\hline
5
&
6,045
&
10
&
5,964.04
&
10
&
6,045
\\
\hline
6
&
2,345
&
4
&
2,513.45
&
5
&
2,365
\\
\hline
7
&
2,543
&
7
&
2,116.16
&
1
&
2,123
\\
\hline
8
&
2,854
&
9
&
2,624.32
&
9
&
2,854
\\
\hline
9
&
2,365
&
5
&
2,203.45
&
2
&
2,300
\\
\hline
10
&
2,643
&
8
&
2,358.29
&
3
&
2,312
\\
\hline
\end{tabulary}
\par
\sphinxattableend\end{savenotes}

The method ‘ds’ (the default method of data shuffling in \sphinxstyleemphasis{sdcMicro}) is
recommended for use (Templ et al., 2014) %
\begin{footnote}[21]\sphinxAtStartFootnote
In \sphinxstyleemphasis{sdcMicro}, there are several other methods for shuffling
implemented, including ‘ds’, ‘mvn’ and ‘mlm’. See the Help option for
the shuffle function in \sphinxstyleemphasis{sdcMicro} for details on methods ‘ds’, ‘mvm’
and ‘mlm’.
%
\end{footnote}. A
regression function with regressors for the variables to be protected
must be specified in the argument ‘form’. At least two regressands
should be specified and the regressors should have predictive power for
the variables to be predicted. This can be checked with goodness-of-fit
measures such as the \(R^{2}\) of the regression. The \(R^{2}\)
captures only linear relations, but these are also the only relations
that are captured by the linear regression model used for shuffling.
Following is an example for shuffling expenditure variables, which are
predicted by total household expenditures and household size.

Example 5.24: Shuffling using a specified regression equation

\fvset{hllines={, ,}}%
\begin{sphinxVerbatim}[commandchars=\\\{\}]
\PYG{c+c1}{\PYGZsh{} Evaluate R\PYGZhy{}squared (goodness\PYGZhy{}of\PYGZhy{}fit) of the regression model}
\PYG{k+kp}{summary}\PYG{p}{(}lm\PYG{p}{(}\PYG{k+kp}{file}\PYG{p}{,} form \PYG{o}{=} TOTFOOD  \PYG{o}{+} TOTALCH \PYG{o}{+} TOTCLTH \PYG{o}{+} TOTHOUS \PYG{o}{+} TOTFURN \PYG{o}{+} TOTHLTH  \PYG{o}{+} TOTTRSP \PYG{o}{+} TOTCMNQ \PYG{o}{+} TOTRCRE \PYG{o}{+} TOTEDUC \PYG{o}{+} TOTHOTL \PYG{o}{+} TOTMISC \PYG{o}{\PYGZti{}} EXP \PYG{o}{+} HHSIZE\PYG{p}{)}\PYG{p}{)}

\PYG{c+c1}{\PYGZsh{} Shuffling using the specified regression equation}
sdcInitial \PYG{o}{\PYGZlt{}\PYGZhy{}} shuffle\PYG{p}{(}sdcInitial\PYG{p}{,} method\PYG{o}{=}\PYG{l+s}{\PYGZsq{}}\PYG{l+s}{ds\PYGZsq{}}\PYG{p}{,} form \PYG{o}{=} TOTFOOD  \PYG{o}{+} TOTALCH \PYG{o}{+} TOTCLTH \PYG{o}{+} TOTHOUS \PYG{o}{+} TOTFURN \PYG{o}{+} TOTHLTH  \PYG{o}{+} TOTTRSP \PYG{o}{+} TOTCMNQ \PYG{o}{+} TOTRCRE \PYG{o}{+} TOTEDUC \PYG{o}{+} TOTHOTL \PYG{o}{+} TOTMISC \PYG{o}{\PYGZti{}} EXP \PYG{o}{+} HHSIZE\PYG{p}{)}
\end{sphinxVerbatim}

\begin{sphinxadmonition}{note}{Recommended Reading Material on Shuffling}

K. Muralidhar and R. Sarathy. 2006.”Data shuffling - A new masking
approach for numerical data,” \sphinxstyleemphasis{Management Science}, 52, 658-670.
\end{sphinxadmonition}


\subsection{Comparison of PRAM, rank swapping and shuffling}
\label{\detokenize{anon_methods:comparison-of-pram-rank-swapping-and-shuffling}}
PRAM, rank swapping and shuffling are all perturbative methods, i.e.,
they change the values for individual records and are mainly used for
continuous variables. After rank swapping and shuffling, the original
values are all contained in the treated dataset but might be assigned to
other records. This implies that univariate tabulations are not changed.
This also holds in expectation for PRAM, if a transition matrix is
chosen that has the invariant property.

Choosing a method is based on the structure to be preserved in the data.
In cases where the regression model fits the data well, data shuffling
would work very well, as there should be sufficient (continuous)
regressors available. Rank swapping works well if there are sufficient
categories in the variables. PRAM is preferred if the perturbation
method should be applied to only one or few variables; the advantage is
the possibility of specifying restrictions on the transition matrix and
applying PRAM only within strata, which can be user defined.


\section{Anonymization of geospatial variables}
\label{\detokenize{anon_methods:anonymization-of-geospatial-variables}}
Recently, geospatial data has become increasingly popular with
researchers and wide-spread. Georeferenced data identifies the
geographical location for each record with the help of a Geographical
Information System (GIS), that uses for instance GPS (Global Positioning
System) coordinates or address data. The advantages of geospatial data
are manifold: 1) researchers can create their own geographical areas,
such as the service area of a hospital; 2) it enables researchers to
measure the proximity to facilities, such as schools; 3) researchers can
use the data to extract geographical patterns; and 4) it enables linking
of data from different sources (see e.g., Burgert et al., 2015).
However, geospatial data, due to the precise reference to a location,
also pose a challenge to the privacy of the respondents.

One way to anonymize georeferenced data is removing the GIS variables
and instead leaving in or creating other geographical variables, such as
province, region. However, this approach also removes the benefits of
geospatial data. Another option is the geographical displacement of
areas and/or records. Burgert et al. (2013) describe a geographical
displacement procedure for a health dataset. This paper also includes
the code in Python. Hu and Drechsler (2015) propose three different
strategies for generating synthetic geocodes.

\begin{sphinxadmonition}{note}{Recommended Reading Material on Anonymization of Geospatial Data}

C.R. Burgert, J. Colston, T. Roy and B. Zachary. 2013. “DHS Spatial
Analysis Report No. 7 - Geographic Displacement Procedure and
Georeferenced Data Release Policy for the Demographic and Health
Surveys” (USAID). \sphinxurl{http://dhsprogram.com/pubs/pdf/SAR7/SAR7.pdf}

J. Hu and J. Drechsler. 2015. “Generating synthetic geocoding
information for public release.”
\sphinxurl{http://www.iab.de/389/section.aspx/Publikation/k150601301}
\end{sphinxadmonition}


\section{Anonymization of the quasi-identifier household size}
\label{\detokenize{anon_methods:anonymization-of-the-quasi-identifier-household-size}}
The size of a household is an important identifier, especially for large
households. %
\begin{footnote}[22]\sphinxAtStartFootnote
Even if the dataset does not contain an explicit variable with
household size, this information can be easily extracted from the
data and should be taken into account. Section 7.6 shows how to
create a variable “household size” based on the household IDs.
%
\end{footnote}  Suppression of the actual size
variable, if available (e.g., number of household members), however,
does not suffice to remove this information from the dataset, as a
simple count of the household members for a particular household will
allow this variable to be reconstructed as long as a household ID is in
the data. In any case, households of a very large size or with a unique
or special key (i.e., combination of values of quasi-identifiers) should
be checked manually. One way to treat them is to remove these households
from the dataset before release. Alternatively, the households can be
split, but care should be taken to suppress or change values for these
households to prevent an intruder from immediately understanding that
these households have been split and reconstructing them by combining
the two households with the same values.


\section{Special case: census data}
\label{\detokenize{anon_methods:special-case-census-data}}
Census microdata are a special case because the user (and intruder)
knows that all respondents are included in the dataset. Therefore, risk
measures that use the sample weights and are based on uncertainty of the
correctness of a match are no longer applicable. If an intruder has
identified a sample unique and successfully matched, there is no doubt
whether the match is correct, as it would be in the case of a sample.
One approach to release census microdata is to release a stratified
sample of the sample (1 \textendash{} 5\% of the total census). \sphinxstylestrong{NOTE: After
sampling, the anonymization process has to be followed; sampling alone
is not sufficient to guarantee confidentiality.}

Several statistical offices release microdata based on census data. A
few examples are:
\begin{itemize}
\item {} 
The British Office for National Statistics (ONS) released several

\end{itemize}

files based on the 2011 census: 1) A microdata teaching file for
educational purposes. This file is a 1\% sample of the total census with
a limited set of variables. 2) Two scientific use files with 5\% samples
are available for registered researchers who accept the terms and
conditions of their use. 3) Two 10\% samples are available in controlled
research data centers for approved researchers and research goals. All
these files have been anonymized prior to release. %
\begin{footnote}[23]\sphinxAtStartFootnote
More information on census microdata at ONS is available on their
website:
\sphinxurl{http://www.ons.gov.uk/ons/guide-method/census/2011/census-data/census-microdata/index.html}
%
\end{footnote}
\begin{itemize}
\item {} 
The U.S. Census Bureau released two samples of the 2000 census: a 5\%

\end{itemize}

sample on the national level and a 1\% sample on the state level. The
national level file is more detailed, but the most detailed geographical
area has at least 400,000 people. This, however, allows representation
of all states from the dataset. The state-level file has less detailed
variables but a more detailed geographical structure, which allows
representation of cities and larger counties from the dataset (the
minimum size of a geographical area is 100,000). Both files have been
anonymized by using data swapping, top coding, perturbation and reducing
detail by recoding.{[}\#foot57{]}\_


\chapter{Measuring Utility and Information Loss}
\label{\detokenize{utility:measuring-utility-and-information-loss}}\label{\detokenize{utility::doc}}
SDC is a trade-off between risk of disclosure and loss of data utility
and seeks to minimize the latter, while reducing the risk of disclosure
to an acceptable level. Data utility in this context means the
usefulness of the anonymized data for statistical analyses by end users
as well as the validity of these analyses when performed on the
anonymized data. Disclosure risk and its measurement are defined in
Chapter 4 of this guide. In order to make a trade-off between minimizing
disclosure risk and maximizing utility of data for end users, it is
necessary to measure the utility of the data after anonymization and
compare it with the utility of the original data. This chapter describes
measures that can be used to compare the data utility before and after
anonymization, or alternatively quantify the information loss.
Information loss is the inverse of data utility: the larger the data
utility after anonymization, the smaller the information loss. NOTE: If
the microdata to be anonymized is based on a sample, the data will incur
a sampling error. Also other errors may be present in the data, such as
nonresponse error. \sphinxstylestrong{NOTE: The methods discussed here only measure the
information loss caused by the anonymization process relative to the
original sample data and do not attempt to measure the error caused by
other sources.}

Ideally, the information loss is evaluated with respect to the needs and
uses of the end users of the microdata. However, different end users of
anonymized data may have very diverse uses for the released data and it
might not be possible to collect an exhaustive list of these uses. Even
if many uses can be identified, the characteristics in the data needed
for these uses can be contradictory (e.g., one user needs a detailed
geographical level whereas another is interested in a detailed age
structure and does not need a detailed geographical structure).
Nevertheless, as pointed out earlier, only one anonymized dataset can be
released for each dataset and every type of release to avoid unintended
disclosure. Releasing multiple anonymized datasets for different
purposes may lead to unintended disclosure. %
\begin{footnote}[1]\sphinxAtStartFootnote
It is possible to release data files for different groups of users,
e.g., PUF and SUF. All information in the less detailed file,
however, must also be included in the more detailed file to prevent
unintended disclosure. Datasets released in data enclaves can be
customized for the user, since the risk that they will be combined
with other version is zero.
%
\end{footnote}
Therefore, it is not possible to anonymize and release a file tailored
to each user’s needs.

Since collecting and taking into account all data uses is often
impossible, we also look at general (or generic) information loss
measures besides user- and data-specific information loss measures.
These measures do not take into account the specific data use, but can
be used as guiding measures for information loss and evaluating whether
a dataset is still analytically valid after anonymization. The main idea
for such measures is to compare records between the original and treated
datasets and compare statistics computed from both datasets (Hundepool
et al., 2012). Examples of such measures are the number of suppressions,
number of changed values, changes in contingency tables and changes in
mean and covariance matrices.

Many of the SDC methods discussed earlier are parametric, in the sense
that their outcome depends on parameters chosen by the user. Examples
are the cluster size for microaggregation (see Section 5.3.2) or the
importance vector in local suppression (see Section 5.2.2). Data utility
and information loss measures are useful for choosing these parameters
by comparing the impact of different parameters on the information loss.
Figure 6.1 illustrates this by showing the trade-off between the
disclosure risk and data utility of a hypothetical dataset. The triangle
represents the original data with full utility and a certain level of
disclosure risk, which is too high for disclosure. The square represents
no release of microdata. Although there is no risk of disclosure, there
is also no utility from the data for users since no data is released.
The points in between represent the result of applying different SDC
methods with different parameter specifications. We would select the SDC
method corresponding to the point, which maximizes the utility, while
keeping disclosure risk at an acceptable level.

\noindent\sphinxincludegraphics[width=6.5in,height=3.25556in]{{image11}.png}

Figure 6.1: The trade-off between risk and utility in a hypothetical
dataset

In the following sections, we first propose general utility measures
independent of data use, and later present an example of a specific
measure useful to measure information loss with respect to specific data
uses. Finally, we show how to visualize changes in the data caused by
anonymization and discuss the selection of utility measures for a
particular dataset.


\section{General utility measures for continuous and categorical variables}
\label{\detokenize{utility:general-utility-measures-for-continuous-and-categorical-variables}}
General or generic measures of information loss can be divided into
those comparing the actual values of the raw and anonymized data, and
those comparing statistics from both datasets. All measures are a
posteriori, since they measure utility after anonymization and require
both the data before and after the anonymization process. General
utility measures are different for categorical and continuous variables.


\subsection{General utility measures for categorical variables}
\label{\detokenize{utility:general-utility-measures-for-categorical-variables}}

\subsubsection{Number of missing values}
\label{\detokenize{utility:number-of-missing-values}}
An informative measure is to compare the number of missing values in the
data. Missing values are often introduced after suppression and more
suppressions indicate a higher degree of information loss. After using
the local suppression function on an \sphinxstyleemphasis{sdcMicro} object, the number of
suppressions for each categorical key variable can be retrieved with the
print() function, which is illustrated in Example
6.1{[}\#foot59{]}\_. The argument ‘ls’ in the print() function
stands for local suppression. The output shows both the absolute and
relative number of suppressions.

Example 6.1: Using the print() function to retrieve the total number of
suppressions for each categorical key variable

\fvset{hllines={, ,}}%
\begin{sphinxVerbatim}[commandchars=\\\{\}]
sdcInitial \PYG{o}{\PYGZlt{}\PYGZhy{}} localSuppression\PYG{p}{(}sdcInitial\PYG{p}{,} k \PYG{o}{=} \PYG{l+m}{5}\PYG{p}{,} importance \PYG{o}{=} \PYG{k+kc}{NULL}\PYG{p}{)}

\PYG{k+kp}{print}\PYG{p}{(}sdcInitial\PYG{p}{,} \PYG{l+s}{\PYGZsq{}}\PYG{l+s}{ls\PYGZsq{}}\PYG{p}{)}

\PYG{c+c1}{\PYGZsh{}\PYGZsh{} Local Suppression:}
\PYG{c+c1}{\PYGZsh{}\PYGZsh{} KeyVar \textbar{} Suppressions (\PYGZsh{}) \textbar{} Suppressions (\PYGZpc{})}
\PYG{c+c1}{\PYGZsh{}\PYGZsh{}   URBRUR \textbar{}                0 \textbar{}            0.000}
\PYG{c+c1}{\PYGZsh{}\PYGZsh{}   REGION \textbar{}               81 \textbar{}            4.050}
\PYG{c+c1}{\PYGZsh{}\PYGZsh{}    RELIG \textbar{}                0 \textbar{}            0.000}
\PYG{c+c1}{\PYGZsh{}\PYGZsh{}  MARITAL \textbar{}                0 \textbar{}            0.000}
\PYG{c+c1}{\PYGZsh{}\PYGZsh{} \PYGZhy{}\PYGZhy{}\PYGZhy{}\PYGZhy{}\PYGZhy{}\PYGZhy{}\PYGZhy{}\PYGZhy{}\PYGZhy{}\PYGZhy{}\PYGZhy{}\PYGZhy{}\PYGZhy{}\PYGZhy{}\PYGZhy{}\PYGZhy{}\PYGZhy{}\PYGZhy{}\PYGZhy{}\PYGZhy{}\PYGZhy{}\PYGZhy{}\PYGZhy{}\PYGZhy{}\PYGZhy{}\PYGZhy{}\PYGZhy{}\PYGZhy{}\PYGZhy{}\PYGZhy{}\PYGZhy{}\PYGZhy{}\PYGZhy{}\PYGZhy{}\PYGZhy{}\PYGZhy{}\PYGZhy{}\PYGZhy{}\PYGZhy{}\PYGZhy{}\PYGZhy{}\PYGZhy{}\PYGZhy{}\PYGZhy{}\PYGZhy{}\PYGZhy{}\PYGZhy{}\PYGZhy{}\PYGZhy{}\PYGZhy{}\PYGZhy{}\PYGZhy{}\PYGZhy{}\PYGZhy{}\PYGZhy{}\PYGZhy{}\PYGZhy{}\PYGZhy{}\PYGZhy{}\PYGZhy{}\PYGZhy{}\PYGZhy{}\PYGZhy{}\PYGZhy{}\PYGZhy{}\PYGZhy{}\PYGZhy{}\PYGZhy{}\PYGZhy{}\PYGZhy{}\PYGZhy{}\PYGZhy{}\PYGZhy{}\PYGZhy{}\PYGZhy{}}
\end{sphinxVerbatim}

More generally, it is possible to count and compare the number of
missing values in the original data and the treated data. This can be
useful to see the proportional increase in the number of missing values.
Missing values can also have other sources, such as nonresponse. Example
6.2 shows how to display the number of missing values for each of the
categorical key variables in an \sphinxstyleemphasis{sdcMicro} object. Here it is assumed
that all missing values are coded ‘NA’. If the missing values are not
coded ‘NA’, but instead another value, it is possible to use the
alternative missing values code. The results agree with the number of
missing values introduced by local suppression in the previous example,
but also shows that the variable “RELIG” has 1,000 missing values in the
original data.

Example 6.2: Displaying the number of missing values for each
categorical key variable in an \sphinxstyleemphasis{sdcMicro} object

\fvset{hllines={, ,}}%
\begin{sphinxVerbatim}[commandchars=\\\{\}]
\PYG{o}{\textbar{}} \PYG{o}{*}\PYG{c+c1}{\PYGZsh{} Store the names of all categorical key variables in a vector*}
\PYG{o}{\textbar{}} namesKeyVars \PYG{o}{\PYGZlt{}\PYGZhy{}} \PYG{o}{*}\PYG{o}{*}\PYG{k+kp}{names}\PYG{o}{*}\PYG{o}{*}\PYGZbs{} \PYG{p}{(}sdcInitial\PYG{o}{@}manipKeyVars\PYG{p}{)}

\PYG{o}{\textbar{}} \PYG{o}{*}\PYG{c+c1}{\PYGZsh{} Matrix to store the number of missing values (NA) before and after}
  anonymization\PYG{o}{*}
\PYG{o}{\textbar{}} NAcount \PYG{o}{\PYGZlt{}\PYGZhy{}} \PYG{o}{*}\PYG{o}{*}\PYG{k+kt}{matrix}\PYG{o}{*}\PYG{o}{*}\PYGZbs{} \PYG{p}{(}\PYG{k+kc}{NA}\PYG{p}{,} nrow \PYG{o}{=} \PYG{l+m}{2}\PYG{p}{,} ncol \PYG{o}{=}
  \PYG{o}{*}\PYG{o}{*}\PYG{k+kp}{length}\PYG{o}{*}\PYG{o}{*}\PYGZbs{} \PYG{p}{(}namesKeyVars\PYG{p}{)}\PYG{p}{)}

\PYG{o}{*}\PYG{o}{*}\PYG{k+kp}{colnames}\PYG{o}{*}\PYG{o}{*}\PYGZbs{} \PYG{p}{(}NAcount\PYG{p}{)} \PYG{o}{\PYGZlt{}\PYGZhy{}} \PYG{o}{*}\PYG{o}{*}\PYG{k+kt}{c}\PYG{o}{*}\PYG{o}{*}\PYGZbs{} \PYG{p}{(}\PYG{o}{*}\PYG{o}{*}\PYG{k+kp}{paste0}\PYG{o}{*}\PYG{o}{*}\PYGZbs{} \PYG{p}{(}\PYG{l+s}{\PYGZsq{}}\PYG{l+s}{NA\PYGZsq{}}\PYG{p}{,} namesKeyVars\PYG{p}{)}\PYG{p}{)} \PYG{o}{*}\PYG{c+c1}{\PYGZsh{}}
column \PYG{k+kp}{names}\PYG{o}{*}

\PYG{o}{*}\PYG{o}{*}\PYG{k+kp}{rownames}\PYG{o}{*}\PYG{o}{*}\PYGZbs{} \PYG{p}{(}NAcount\PYG{p}{)} \PYG{o}{\PYGZlt{}\PYGZhy{}} \PYG{o}{*}\PYG{o}{*}\PYG{k+kt}{c}\PYG{o}{*}\PYG{o}{*}\PYGZbs{} \PYG{p}{(}\PYG{l+s}{\PYGZsq{}}\PYG{l+s}{initial\PYGZsq{}}\PYG{p}{,} \PYG{l+s}{\PYGZsq{}}\PYG{l+s}{treated\PYGZsq{}}\PYG{p}{)} \PYG{o}{*}\PYG{c+c1}{\PYGZsh{} row names*}

\PYG{o}{\textbar{}} \PYG{o}{*}\PYG{c+c1}{\PYGZsh{} NA count in all key variables (NOTE: only those coded NA are}
  counted\PYG{p}{)}\PYG{o}{*}
\PYG{o}{\textbar{}} \PYG{k+kr}{for}\PYG{p}{(}i \PYG{k+kr}{in} \PYG{l+m}{1}\PYG{o}{:}\PYGZbs{} \PYG{o}{*}\PYG{o}{*}\PYG{k+kp}{length}\PYG{o}{*}\PYG{o}{*}\PYGZbs{} \PYG{p}{(}namesKeyVars\PYG{p}{)}\PYG{p}{)}
\PYG{o}{\textbar{}} \PYG{p}{\PYGZob{}}
\PYG{o}{\textbar{}} NAcount\PYG{p}{[}\PYG{l+m}{1}\PYG{p}{,} i\PYG{p}{]} \PYG{o}{\PYGZlt{}\PYGZhy{}}
  \PYG{o}{*}\PYG{o}{*}\PYG{k+kp}{sum}\PYG{o}{*}\PYG{o}{*}\PYGZbs{} \PYG{p}{(}\PYG{o}{*}\PYG{o}{*}\PYG{k+kp}{is.na}\PYG{o}{*}\PYG{o}{*}\PYGZbs{} \PYG{p}{(}sdcInitial\PYG{o}{@}origData\PYG{p}{[}\PYG{p}{,}namesKeyVars\PYG{p}{[}i\PYG{p}{]]}\PYG{p}{)}\PYG{p}{)}
\PYG{o}{\textbar{}} NAcount\PYG{p}{[}\PYG{l+m}{2}\PYG{p}{,} i\PYG{p}{]} \PYG{o}{\PYGZlt{}\PYGZhy{}} \PYG{o}{*}\PYG{o}{*}\PYG{k+kp}{sum}\PYG{o}{*}\PYG{o}{*}\PYGZbs{} \PYG{p}{(}\PYG{o}{*}\PYG{o}{*}\PYG{k+kp}{is.na}\PYG{o}{*}\PYG{o}{*}\PYGZbs{} \PYG{p}{(}sdcInitial\PYG{o}{@}manipKeyVars\PYG{p}{[}\PYG{p}{,}i\PYG{p}{]}\PYG{p}{)}\PYG{p}{)}
\PYG{o}{\textbar{}} \PYG{p}{\PYGZcb{}}

\PYG{o}{*}\PYG{c+c1}{\PYGZsh{} Show results*}

NAcount

\PYG{o}{\textbar{}} \PYG{l+s+sb}{{}`{}`}\PYG{c+c1}{\PYGZsh{}\PYGZsh{}         NAURBRUR NAREGION NARELIG NAMARITAL{}`{}`}
\PYG{o}{\textbar{}} \PYG{l+s+sb}{{}`{}`}\PYG{c+c1}{\PYGZsh{}\PYGZsh{} initial        0        0    1000        51{}`{}`}
\PYG{o}{\textbar{}} \PYG{l+s+sb}{{}`{}`}\PYG{c+c1}{\PYGZsh{}\PYGZsh{} treated        0       81    1000        51{}`{}`}
\end{sphinxVerbatim}


\subsubsection{Number of records changed}
\label{\detokenize{utility:number-of-records-changed}}
Another useful statistic is the number of records changed per variable.
These can be counted in a similar way as the missing values and include
suppressions (i.e., changes to missing/’NA’ in \sphinxstyleemphasis{R}). The number of
records changed gives a good indication of the impact of the
anonymization methods on the data. Example 6.3 illustrates how to
compute the number of records changed for the PRAMmed variables.

Example 6.3: Computing number of records changed per variable

\fvset{hllines={, ,}}%
\begin{sphinxVerbatim}[commandchars=\\\{\}]
\PYG{c+c1}{\PYGZsh{} Store the names of all pram variables in a vector}
namesPramVars \PYG{o}{\PYGZlt{}\PYGZhy{}} \PYG{o}{*}\PYG{o}{*}\PYG{k+kp}{names}\PYG{o}{*}\PYG{o}{*}\PYGZbs{} \PYG{p}{(}sdcInitial\PYG{o}{@}manipPramVars\PYG{p}{)}
\PYG{c+c1}{\PYGZsh{} Dataframe to save the number of records changed}
recChanged \PYG{o}{\PYGZlt{}\PYGZhy{}} \PYG{o}{*}\PYG{o}{*}\PYG{k+kp}{rep}\PYG{o}{*}\PYG{o}{*}\PYGZbs{} \PYG{p}{(}\PYG{l+m}{0}\PYG{p}{,} \PYG{o}{*}\PYG{o}{*}\PYG{k+kp}{length}\PYG{o}{*}\PYG{o}{*}\PYGZbs{} \PYG{p}{(}namesPramVars\PYG{p}{)}\PYG{p}{)}
\PYG{k+kp}{names}\PYG{p}{(}recChanged\PYG{p}{)} \PYG{o}{\PYGZlt{}\PYGZhy{}} \PYG{k+kt}{c}\PYG{p}{(}\PYG{k+kp}{paste0}\PYG{p}{(}\PYG{l+s}{\PYGZsq{}}\PYG{l+s}{RC\PYGZsq{}}\PYG{p}{,} namesPramVars\PYG{p}{)}\PYG{p}{)}
\PYG{c+c1}{\PYGZsh{} Count number of records changed*}
\PYG{k+kr}{for}\PYG{p}{(}j \PYG{k+kr}{in} \PYG{l+m}{1}\PYG{o}{:}\PYG{k+kp}{length}\PYG{p}{(}namesPramVars\PYG{p}{)}\PYG{p}{)} \PYG{c+c1}{\PYGZsh{} for all key variables}
\PYG{p}{\PYGZob{}}
comp \PYG{o}{\PYGZlt{}\PYGZhy{}} sdcInitial\PYG{o}{@}origData\PYG{p}{[}namesPramVars\PYG{p}{[}j\PYG{p}{]]} \PYG{o}{!=} sdcInitial\PYG{o}{@}manipPramVars\PYG{p}{[}namesPramVars\PYG{p}{[}j\PYG{p}{]]}
temp1 \PYG{o}{\PYGZlt{}\PYGZhy{}} \PYG{k+kp}{sum}\PYG{p}{(}comp\PYG{p}{,} na.rm \PYG{o}{=} \PYG{k+kc}{TRUE}\PYG{p}{)} \PYG{c+c1}{\PYGZsh{} all changed variables without NAs}
temp2 \PYG{o}{\PYGZlt{}\PYGZhy{}} \PYG{k+kp}{sum}\PYG{p}{(}\PYG{k+kp}{is.na}\PYG{p}{(}comp\PYG{p}{)}\PYG{p}{)} \PYG{c+c1}{\PYGZsh{} if NA, changed, unless NA initially}
temp3 \PYG{o}{\PYGZlt{}\PYGZhy{}} \PYG{k+kp}{sum}\PYG{p}{(}\PYG{k+kp}{is.na}\PYG{p}{(}sdcInitial\PYG{o}{@}origData\PYG{p}{[}namesPramVars\PYG{p}{[}j\PYG{p}{]]}\PYG{p}{)}
\PYG{o}{+} \PYG{o}{*}\PYG{o}{*}\PYG{k+kp}{is.na}\PYG{o}{*}\PYG{o}{*}\PYGZbs{} \PYG{p}{(}sdcInitial\PYG{o}{@}manipPramVars\PYG{p}{[}namesPramVars\PYG{p}{[}j\PYG{p}{]]}\PYG{p}{)}\PYG{o}{==}\PYG{l+m}{2}\PYG{p}{)}
\PYG{c+c1}{\PYGZsh{} both NA, no change, but counted in temp2}
recChanged\PYG{p}{[}j\PYG{p}{]} \PYG{o}{\PYGZlt{}\PYGZhy{}} temp1 \PYG{o}{+} temp2 \PYG{o}{\PYGZhy{}} temp3
\PYG{p}{\PYGZcb{}}

\PYG{c+c1}{\PYGZsh{} Show results}
recChanged

    \PYG{c+c1}{\PYGZsh{}\PYGZsh{}  RCWATER   RCROOF RCTOILET}
    \PYG{c+c1}{\PYGZsh{}\PYGZsh{}      125       86      180}
\end{sphinxVerbatim}


\subsubsection{Comparing contingency tables}
\label{\detokenize{utility:comparing-contingency-tables}}
A useful way to measure information loss in categorical variables is to
compare univariate tabulations and, more interestingly, contingency
tables (also cross tabulations or two-way tables) between pairs of
variables. To maintain the analytical validity of a dataset, the
contingency tables should stay approximately the same. The function
table() produces contingency tables of one or more variables. Example
6.4 creates a contingency table of the variables “REGION” and “URBRUR”.
We observe small differences between the tables before and after
anonymization.

Example 6.4: Comparing contingency tables of categorical variables

\fvset{hllines={, ,}}%
\begin{sphinxVerbatim}[commandchars=\\\{\}]
  \PYG{c+c1}{\PYGZsh{} Contingency table (cross tabulation) of the variables region and urban/rural}
      \PYG{k+kp}{table}\PYG{o}{*}\PYG{o}{*}\PYGZbs{} \PYG{p}{(}sdcInitial\PYG{o}{@}origData\PYG{p}{[}\PYG{p}{,} \PYG{o}{*}\PYG{o}{*}\PYG{k+kt}{c}\PYG{o}{*}\PYG{o}{*}\PYGZbs{} \PYG{p}{(}\PYG{l+s}{\PYGZsq{}}\PYG{l+s}{REGION\PYGZsq{}}\PYG{p}{,} \PYG{l+s}{\PYGZsq{}}\PYG{l+s}{URBRUR\PYGZsq{}}\PYG{p}{)}\PYG{p}{]}\PYG{p}{)} \PYG{o}{*}
before anonymization\PYG{o}{*}
\end{sphinxVerbatim}

\begin{DUlineblock}{0em}
\item[] \sphinxcode{\sphinxupquote{\#\#       URBRUR}}
\item[] \sphinxcode{\sphinxupquote{\#\# REGION   1   2}}
\item[] \sphinxcode{\sphinxupquote{\#\#      1 235  89}}
\item[] \sphinxcode{\sphinxupquote{\#\#      2 261  73}}
\item[] \sphinxcode{\sphinxupquote{\#\#      3 295  76}}
\item[] \sphinxcode{\sphinxupquote{\#\#      4 304  71}}
\item[] \sphinxcode{\sphinxupquote{\#\#      5 121 139}}
\item[] \sphinxcode{\sphinxupquote{\#\#      6 100 236}}
\end{DUlineblock}

\sphinxstylestrong{table}(sdcInitial@manipKeyVars{[}, \sphinxstylestrong{c}(‘REGION’, ‘URBRUR’){]}) \sphinxstyleemphasis{\#
after anonymization}

\begin{DUlineblock}{0em}
\item[] \sphinxcode{\sphinxupquote{\#\#       URBRUR}}
\item[] \sphinxcode{\sphinxupquote{\#\# REGION   1   2}}
\item[] \sphinxcode{\sphinxupquote{\#\#      1 235  89}}
\item[] \sphinxcode{\sphinxupquote{\#\#      2 261  73}}
\item[] \sphinxcode{\sphinxupquote{\#\#      3 295  76}}
\item[] \sphinxcode{\sphinxupquote{\#\#      4 304  71}}
\item[] \sphinxcode{\sphinxupquote{\#\#      5 105 130}}
\item[] \sphinxcode{\sphinxupquote{\#\#      6  79 201}}
\end{DUlineblock}

Domingo-Ferrer and Torra (2001) propose a Contingency Table-Based
Information Loss (CTBIL) measure, which quantifies the distance between
the contingency tables in the original and treated data. Alternatively,
visualizations of the contingency table with mosaic plots can be used to
compare the impact of anonymization methods on the tabulations and
contingency tables (see Section 6.4.3).


\subsection{General utility measures for continuous variables}
\label{\detokenize{utility:general-utility-measures-for-continuous-variables}}

\subsubsection{Statistics: mean, covariance, correlation}
\label{\detokenize{utility:statistics-mean-covariance-correlation}}
The statistics characterizing the dataset should not change after the
anonymization. Examples of such statistics are the mean, variance, and
covariance and correlation structure of the most important variables in
the dataset. Other statistics characterizing the data include the
principal components and the loadings. Domingo-Ferrer and Torra (2001)
give an overview of statistics that can be considered. In order to
evaluate the information loss caused by the anonymization, one should
compare the appropriate statistics for continuous variables computed
from the data before and after anonymization. There are several ways to
evaluate the loss of utility with respect to the changes in these
statistics, for instance, by comparing means and (co-)variances in the
data or comparing the (multivariate) distributions of the data.
Especially changes in the correlations gives valuable information on the
validity of the data for regressions. Functions from the \sphinxstyleemphasis{R} base
package or any other statistical package can be used to do this.
Following are a few examples in \sphinxstyleemphasis{R}.

To compute the mean of each numerical variable we use the function
colMeans(). To ignore missing values, it is necessary to use the option
na.rm = TRUE. “numVars” is a vector with the names of the numerical
variables. Example 6.5 shows how to compute the means for all numeric
variables. The untreated data is extracted from the ‘origData’ slot of
the \sphinxstyleemphasis{sdcMicro} object and the anonymized data from the ‘manipNumVars’
slot, which contains the manipulated numeric variables. We observe small
changes in each of the three variables.

Example 6.5: Comparing the means of continuous variables

\sphinxstyleemphasis{\# untreated data}

\sphinxstylestrong{colMeans}(sdcInitial@origData{[}, numVars{]}, na.rm = TRUE)

\begin{DUlineblock}{0em}
\item[] \sphinxcode{\sphinxupquote{\#\#       INC    INCRMT   INCWAGE}}
\item[] \sphinxcode{\sphinxupquote{\#\#  479.7710  961.0295 1158.1330}}
\end{DUlineblock}

\sphinxstyleemphasis{\# anonymized data}

\sphinxstylestrong{colMeans}(sdcInitial@manipNumVars{[}, numVars{]}, na.rm = TRUE)

\begin{DUlineblock}{0em}
\item[] \sphinxcode{\sphinxupquote{\#\#       INC    INCRMT   INCWAGE}}
\item[] \sphinxcode{\sphinxupquote{\#\#  489.6030  993.8512 1168.7561}}
\end{DUlineblock}

In the same way, one can compute the covariance and correlation matrices
of the numerical variables in the \sphinxstyleemphasis{sdcMicro} object from the untreated
and anonymized data. This is shown in Example 6.6. We observe that the
variance of each variable (the diagonal elements in the covariance
matrix) have increased by the anonymization. These functions also allow
computing confidence intervals in the case of samples. The means and
covariances of subsets in the data also should not differ. An example is
the mean of income by gender, by age group or by region. These
characteristics of the data are important for analysis.

Example 6.6: Comparing covariance structure and correlation matrices of
numeric variables

\begin{DUlineblock}{0em}
\item[] \sphinxstyleemphasis{\# untreated data}
\item[] \sphinxstylestrong{cov}(sdcInitial@origData{[}, numVars{]})
\end{DUlineblock}

\begin{DUlineblock}{0em}
\item[] \sphinxcode{\sphinxupquote{\#\#               INC    INCRMT  INCWAGE}}
\item[] \sphinxcode{\sphinxupquote{\#\# INC     1645926.1  586975.6  2378901}}
\item[] \sphinxcode{\sphinxupquote{\#\# INCRMT   586975.6 6984502.3  1664257}}
\item[] \sphinxcode{\sphinxupquote{\#\# INCWAGE 2378900.7 1664257.4 16169878}}
\end{DUlineblock}

\sphinxstylestrong{cor}(sdcInitial@origData{[}, numVars{]})

\begin{DUlineblock}{0em}
\item[] \sphinxcode{\sphinxupquote{\#\#               INC    INCRMT   INCWAGE}}
\item[] \sphinxcode{\sphinxupquote{\#\# INC     1.0000000 0.1731200 0.4611241}}
\item[] \sphinxcode{\sphinxupquote{\#\# INCRMT  0.1731200 1.0000000 0.1566028}}
\item[] \sphinxcode{\sphinxupquote{\#\# INCWAGE 0.4611241 0.1566028 1.0000000}}
\end{DUlineblock}

\begin{DUlineblock}{0em}
\item[] \sphinxstyleemphasis{\# anonymized data}
\item[] \sphinxstylestrong{cov}(sdcInitial@manipNumVars{[}, numVars{]})
\end{DUlineblock}

\begin{DUlineblock}{0em}
\item[] \sphinxcode{\sphinxupquote{\#\#               INC    INCRMT  INCWAGE}}
\item[] \sphinxcode{\sphinxupquote{\#\# INC     2063013.1  649937.5  2382447}}
\item[] \sphinxcode{\sphinxupquote{\#\# INCRMT   649937.5 8566169.1  1778985}}
\item[] \sphinxcode{\sphinxupquote{\#\# INCWAGE 2382447.4 1778985.1 19925870}}
\end{DUlineblock}

\sphinxstylestrong{cor}(sdcInitial@manipNumVars{[}, numVars{]})

\begin{DUlineblock}{0em}
\item[] \sphinxcode{\sphinxupquote{\#\#               INC    INCRMT   INCWAGE}}
\item[] \sphinxcode{\sphinxupquote{\#\# INC     1.0000000 0.1546063 0.3715897}}
\item[] \sphinxcode{\sphinxupquote{\#\# INCRMT  0.1546063 1.0000000 0.1361665}}
\item[] \sphinxcode{\sphinxupquote{\#\# INCWAGE 0.3715897 0.1361665 1.0000000}}
\end{DUlineblock}

Domingo-Ferrer and Torra (2001) propose several measures for the
discrepancy between the covariance and correlation matrices. These
measures are based on the mean squared error, the mean absolute error or
the mean variation of the individual cells. We refer to Domingo-Ferrer
and Torra (2001) for a complete overview of these measures.


\subsubsection{IL1s information loss measure}
\label{\detokenize{utility:il1s-information-loss-measure}}
Alternatively, we can also compare the actual data and quantify the
distance between the original dataset \(X\) and the treated dataset
\(Z\). Here \(X\) and \(Z\) contain only continuous
variables. Yancey, Winkler and Creecy (2002) introduce the distance
measure IL1s, which is the sum of the absolute distances between the
corresponding observations in the raw and anonymized datasets, which are
standardized by the standard deviation of the variables in the original
data. For the continuous variables in the dataset, the IL1s measure is
defined as

\(IL1s = \frac{1}{\text{pn}}\sum_{j = 1}^{p}{\sum_{i = 1}^{n}\frac{\left| x_{\text{ij}} - z_{\text{ij}} \right|}{\sqrt{2}S_{j}}}\)
,

where \(p\) is the number of continuous variables; \(n\) is the
number of records in the dataset; \(x_{\text{ij}}\) and
\(z_{\text{ij}}\), respectively, are the values before and after
anonymization for variable \(j\) and individual \(i\); and
\(S_{j}\) is the standard deviation of variable \(j\) in the
original data (Yancey, Winkler and Creecy, 2002).

When using \sphinxstyleemphasis{sdcMicro}, the IL1s data utility measure can be computed for
all numerical quasi-identifiers with the function dUtility(), which is
illustrated in Example 6.7. If required, the measure can also be
computed on subsets of the complete set of numerical quasi-identifiers.
The function is called dUtility(), but returns a measure of information
loss. The result is saved in the utility slot of the \sphinxstyleemphasis{sdcMicro} object.
Example 6.7 also illustrates how to call the result.

Example 6.7: Using dUtility() to compute IL1s data utility measure in
\sphinxstyleemphasis{sdcMicro}

\begin{DUlineblock}{0em}
\item[] \sphinxstyleemphasis{\# Evaluating IL1s measure for all variables in the sdcMicro object
sdcInitial}
\item[] sdcInitial \textless{}- \sphinxstylestrong{dUtility}(sdcInitial)
\item[] \sphinxstyleemphasis{\# Calling the result of IL1s}
\item[] \sphinxhref{mailto:sdcInitial@utility\$il1}{sdcInitial@utility\$il1}
\end{DUlineblock}

\sphinxcode{\sphinxupquote{\#\# {[}1{]} 0.2203791}}

\begin{DUlineblock}{0em}
\item[] \sphinxstyleemphasis{\# IL1s for a subset of the numerical quasi-identifiers}
\item[] subset \textless{}- \sphinxstylestrong{c}(‘INCRMT’, ‘INCWAGE’, ‘INCFARMBSN’)
\item[] \sphinxstylestrong{dUtility}(obj = sdcInitial@origData{[},subset{]}, xm =
sdcInitial@manipNumVars{[},subset{]}, method = ‘IL1’)
\end{DUlineblock}

\sphinxcode{\sphinxupquote{\#\# {[}1{]} 0.5641103}}

The measure is useful for comparing different methods. The smaller the
value of the measure, the closer the values are to the original values
and the higher the utility. \sphinxstylestrong{NOTE: This measure is related to risk
measures based on distance and intervals (see Section} \sphinxstylestrong{4.7).} The
greater the distance between the original and anonymized values, the
lower the data utility. Greater distance, however, also reduces the risk
of re-identification.


\subsubsection{Eigenvalues}
\label{\detokenize{utility:eigenvalues}}
Another way to evaluate the information loss is to compare the robust
eigenvalues of the data before and after anonymization. Example 6.8
illustrates how to use this approach with \sphinxstyleemphasis{sdcMicro}. Here “contVars” is
a vector with the names of the continuous variables in which we are
interested. “obj” is the argument that specifies the untreated data and
“xm” is the argument that specifies the anonymized data. The function’s
output is the difference in eigenvalues. Therefore, the minimum value is
0. Again, the main use is to compare different methods. The greater the
value, the greater the changes in the data and the information loss.

Example 6.8: Using dUtility() to compute eigenvalues in \sphinxstyleemphasis{sdcMicro}

\begin{DUlineblock}{0em}
\item[] \sphinxstyleemphasis{\# Comparison of eigenvalues of continuous variables}
\item[] \sphinxstylestrong{dUtility}(obj = sdcInitial@origData{[},contVars{]}, xm =
sdcInitial@manipNumVars{[},contVars{]}, method = ‘eigen’)
\end{DUlineblock}

\sphinxcode{\sphinxupquote{\#\# {[}1{]} 2.482948}}

\begin{DUlineblock}{0em}
\item[] \sphinxstyleemphasis{\# Comparison of robust eigenvalues of continuous variables}
\item[] \sphinxstylestrong{dUtility}(obj = sdcInitial@origData{[},contVars{]}, xm =
sdcInitial@manipNumVars{[},contVars{]}, method = ‘robeigen’)
\end{DUlineblock}

\sphinxcode{\sphinxupquote{\#\# {[}1{]} -4.297621e+14}}


\section{Utility measures based on the end user’s needs}
\label{\detokenize{utility:utility-measures-based-on-the-end-users-needs}}
Not all needs and uses of a certain dataset can be inventoried.
Nevertheless, some types of data have similar uses or important
characteristics, which can be evaluated before and after anonymization.
Examples of such “benchmarking indicators” (Templ et al., 2014) are
different for each dataset. Examples include poverty measures for income
datasets and school attendance ratios. Often ideas for selecting such
indicators come from the reports data users publish based on previously
released microdata.

The approach is to compare the indicators calculated on the untreated
data and the data after anonymization with different methods. If the
differences between the indicators are not too large, the anonymized
dataset can be released for use by researchers. It should be taken into
account that indicators calculated on samples are estimates with a
certain variance and confidence interval. Therefore, for sample data, it
is more informative to compare the overlap of confidence intervals
and/or to evaluate whether the point estimate calculated after
anonymization is contained within the confidence interval of the
original estimate. Examples of benchmark indicators and their confidence
intervals and how to compute these in \sphinxstyleemphasis{R} are included in the case
studies in these guidelines. Here we give the example of the GINI
coefficient.

The GINI coefficient is a measure of statistical dispersion, which is
often used to measure inequality in income. A way to measure the
information loss in income data is to compare the income distribution,
which can be easily done by comparing the GINI coefficients. Several \sphinxstyleemphasis{R}
packages have functions to compute the GINI coefficient. We chose the
\sphinxstyleemphasis{laeken} package, which computes the GINI coefficient as the area
between the 45-degree line and the Lorenz curve. To use the gini()
function, we first have to install and load the \sphinxstyleemphasis{laeken} library. To
calculate the GINI coefficient for the variable income, we use the
sample weights in the data. This is shown in Example 6.9. The GINI
coefficient of sample data is a random variable. Therefore, it is useful
to construct a confidence interval around the coefficient to evaluate
the significance of any change in the coefficient after anonymization.
The gini() function computes a 1-alpha confidence interval for the GINI
coefficient by using bootstrap.

Example 6.9: Computing the GINI coefficient from the income variable to
determine income inequality

\begin{DUlineblock}{0em}
\item[] \sphinxstyleemphasis{\# Gini coefficient before anonymization}
\item[] \sphinxstylestrong{gini}(inc = sdcInitial@origData{[}selInc,’INC’{]}, weights =
curW{[}selInc{]}, na.rm = TRUE)\$value \sphinxstyleemphasis{\# before}
\end{DUlineblock}

\sphinxcode{\sphinxupquote{\#\# {[}1{]} 34.05928}}

\begin{DUlineblock}{0em}
\item[] \sphinxstyleemphasis{\# Gini coefficient after anonymization}
\item[] \sphinxstylestrong{gini}(inc = sdcInitial@manipNumVars{[}selInc,’INC’{]}, weights =
curW{[}selInc{]}, na.rm = TRUE)\$value \sphinxstyleemphasis{\# after}
\end{DUlineblock}

\sphinxcode{\sphinxupquote{\#\# {[}1{]} 67.13218}}
\begin{quote}

Regression
\end{quote}


\bigskip\hrule\bigskip


Besides comparing covariance and correlation matrices, regressions are a
useful tool to evaluate whether the structure in the data is maintained
after anonymization. By comparing regressions parameters, it is also
possible to compare relations between non-continuous variables (e.g., by
introducing dummy variables or regression with ordinal variables). If it
is known for what purpose and in what field the data is used, common
regressions can be used to compare the change in coefficients and
confidence intervals.

An example of using regression to evaluate the data utility in income
data is the Mincer equation. The Mincer equation explains earnings as a
function of education and experience while controlling for other
variables. The Mincer equation is often used to evaluate the gender pay
gap and gender wage inequality by including a gender dummy. Here we show
how to evaluate the impact of anonymization methods on the gender
coefficient. We regress the log income on a constant, a gender dummy,
years of education, years of experience, years of experience squared and
other factors influencing wage.
\begin{equation*}
\begin{split}\ln\left( \text{wage} \right) = \beta_{0} + \beta_{1}gender + \beta_{2}education + \beta_{3}experience + \beta_{3}\text{experience}^{2} + \beta X\end{split}
\end{equation*}
The parameter of interest here is \(\beta_{1}\), the effect of
gender on the log wage. X is a matrix with several other factors
influencing wage and \(\beta\) the coefficients of these factors.
Example 6.10 illustrates how to run a Mincer regression in \sphinxstyleemphasis{R} using the
function lm() and evaluate the coefficients and confidence intervals
around the coefficients. We run the regression as specified for paid
employees with a positive wage in the age groups 15 \textendash{} 65 years.

Example 6.10: Estimating the Mincer equation (regression) to evaluate
data utility before and after anonymization

\begin{DUlineblock}{0em}
\item[] \sphinxstyleemphasis{\# Mincer equation variables before anonymization}
\item[] Mlwage \textless{}- \sphinxstylestrong{log}(\sphinxhref{mailto:sdcMincer@origData\$wage}{sdcMincer@origData\$wage}) \sphinxstyleemphasis{\# log wage}
\item[] Mempstat \textless{}- \sphinxhref{mailto:sdcMincer@origData\$empstat=='Paid}{sdcMincer@origData\$empstat==’Paid} employee’ \sphinxstyleemphasis{\# TRUE if
‘paid employee’, else FALSE or NA}
\item[] Mage \textless{}- \sphinxhref{mailto:sdcMincer@origData\$age}{sdcMincer@origData\$age} \sphinxstyleemphasis{\# age in years}
\item[] Meducy \textless{}- \sphinxhref{mailto:sdcMincer@origData\$educy}{sdcMincer@origData\$educy} \sphinxstyleemphasis{\# education in years}
\item[] Mexp \textless{}- \sphinxhref{mailto:sdcMincer@origData\$exp}{sdcMincer@origData\$exp} \sphinxstyleemphasis{\# experience in years}
\item[] Mexp2 \textless{}- Mexp\textasciicircum{}2 \sphinxstyleemphasis{\# squared experience}
\item[] Mgender \textless{}- \sphinxhref{mailto:sdcMincer@origData\$gender}{sdcMincer@origData\$gender} \sphinxstyleemphasis{\# gender dummy}
\item[] Mwgt \textless{}- \sphinxhref{mailto:sdcMincer@origData\$wgt}{sdcMincer@origData\$wgt} \sphinxstyleemphasis{\# weight variable for regression}
\item[] MfileB \textless{}- \sphinxstylestrong{as.data.frame}(\sphinxstylestrong{cbind}(Mlwage, Mempstat, Mage,
Meducy, Mexp, Mexp2, Mgender, Mwgt))
\item[] \sphinxstyleemphasis{\# Mincer equation variables after anonymization}
\item[] Mlwage \textless{}- \sphinxstylestrong{log}(\sphinxhref{mailto:sdcMincer@manipNumVars\$wage}{sdcMincer@manipNumVars\$wage}) \sphinxstyleemphasis{\# log wage}
\item[] Mempstat \textless{}- \sphinxhref{mailto:sdcMincer@manipKeyVars\$empstat=='Paid}{sdcMincer@manipKeyVars\$empstat==’Paid} employee’
\end{DUlineblock}

\begin{DUlineblock}{0em}
\item[] \sphinxstyleemphasis{\# TRUE if ‘paid employee’, else FALSE or NA}
\item[] Mage \textless{}- \sphinxhref{mailto:sdcMincer@manipKeyVars\$age}{sdcMincer@manipKeyVars\$age} \sphinxstyleemphasis{\# age in years}
\item[] Meducy \textless{}- \sphinxhref{mailto:sdcMincer@manipKeyVars\$educy}{sdcMincer@manipKeyVars\$educy} \sphinxstyleemphasis{\# education in years}
\item[] Mexp \textless{}- \sphinxhref{mailto:sdcMincer@manipKeyVars\$exp}{sdcMincer@manipKeyVars\$exp} \sphinxstyleemphasis{\# experience in years}
\item[] Mexp2 \textless{}- Mexp\textasciicircum{}2 \sphinxstyleemphasis{\# squared experience}
\item[] Mgender \textless{}- \sphinxhref{mailto:sdcMincer@manipKeyVars\$gender}{sdcMincer@manipKeyVars\$gender} \sphinxstyleemphasis{\# gender dummy}
\item[] Mwgt \textless{}- \sphinxhref{mailto:sdcMincer@origData\$wgt}{sdcMincer@origData\$wgt} \sphinxstyleemphasis{\# weight variable for regression}
\item[] MfileA \textless{}- \sphinxstylestrong{as.data.frame}(\sphinxstylestrong{cbind}(Mlwage, Mempstat, Mage,
Meducy, Mexp, Mexp2, Mgender, Mwgt))
\item[] \sphinxstyleemphasis{\# Specify regression formula}
\item[] Mformula \textless{}- ‘Mlwage \textasciitilde{} Meducy + Mexp + Mexp2 + Mgender’
\item[] \sphinxstyleemphasis{\# Regression Mincer equation}
\item[] mincer1565B \textless{}- \sphinxstylestrong{lm}(Mformula, data = \sphinxstylestrong{subset}(MfileB,
MfileB\$Mage \textgreater{}= 15 \& MfileB\$Mage \textless{}= 65 \& MfileB\$Mempstat==TRUE \&
MfileB\$Mlwage != -Inf), na.action = na.exclude, weights = Mwgt) \sphinxstyleemphasis{\#
before}
\item[] mincer1565A \textless{}- \sphinxstylestrong{lm}(Mformula, data = \sphinxstylestrong{subset}(MfileA,
MfileA\$Mage \textgreater{}= 15 \& MfileA\$Mage \textless{}= 65 \& MfileA\$Mempstat==TRUE \&
MfileA\$Mlwage != -Inf), na.action = na.exclude, weights = Mwgt) \sphinxstyleemphasis{\#
after}
\item[] \sphinxstyleemphasis{\# The objects mincer1565B and mincer1565A contain the results of the
regressions before and after anonymization}
\item[] mincer1565B\$coefficients \sphinxstyleemphasis{\# before}
\end{DUlineblock}

\begin{DUlineblock}{0em}
\item[] \sphinxcode{\sphinxupquote{\#\#   (Intercept)        Meducy          Mexp         Mexp2       Mgender}}
\item[] \sphinxcode{\sphinxupquote{\#\#  3.9532064886  0.0212367075  0.0255962570 -0.0005682651 -0.4931289413}}
\end{DUlineblock}

mincer1565A\$coefficients \sphinxstyleemphasis{\# after}

\begin{DUlineblock}{0em}
\item[] \sphinxcode{\sphinxupquote{\#\#   (Intercept)        Meducy          Mexp         Mexp2       Mgender}}
\item[] \sphinxcode{\sphinxupquote{\#\#  4.0526250282  0.0141090329  0.0326711056 -0.0007605492 -0.5393641862}}
\end{DUlineblock}

\begin{DUlineblock}{0em}
\item[] \sphinxstyleemphasis{\# Compute the 95 percent confidence interval}
\item[] \sphinxstylestrong{confint}(obj = mincer1565B, level = 0.95) \sphinxstyleemphasis{\# before}
\end{DUlineblock}

\begin{DUlineblock}{0em}
\item[] \sphinxcode{\sphinxupquote{\#\#                    2.5 \%        97.5 \%}}
\item[] \sphinxcode{\sphinxupquote{\#\# (Intercept)  3.435759991  4.4706529860}}
\item[] \sphinxcode{\sphinxupquote{\#\# Meducy      -0.018860497  0.0613339120}}
\item[] \sphinxcode{\sphinxupquote{\#\# Mexp         0.004602597  0.0465899167}}
\item[] \sphinxcode{\sphinxupquote{\#\# Mexp2       -0.000971303 -0.0001652273}}
\item[] \sphinxcode{\sphinxupquote{\#\# Mgender     -0.658085143 -0.3281727396}}
\end{DUlineblock}

\sphinxstylestrong{confint}(obj = mincer1565A, level = 0.95) \sphinxstyleemphasis{\# after}

\begin{DUlineblock}{0em}
\item[] \sphinxcode{\sphinxupquote{\#\#                   2.5 \%        97.5 \%}}
\item[] \sphinxcode{\sphinxupquote{\#\# (Intercept)  3.46800378  4.6372462758}}
\item[] \sphinxcode{\sphinxupquote{\#\# Meducy      -0.03305743  0.0612754964}}
\item[] \sphinxcode{\sphinxupquote{\#\# Mexp         0.01024867  0.0550935366}}
\item[] \sphinxcode{\sphinxupquote{\#\# Mexp2       -0.00119162 -0.0003294784}}
\item[] \sphinxcode{\sphinxupquote{\#\# Mgender     -0.71564602 -0.3630823543}}
\end{DUlineblock}

If the new estimates fall within the original confidence interval and
the new and original confidence intervals are greatly overlapping, the
data can be considered valid for this type of regression after
anonymization. Figure 6.2 shows the point estimates and confidence
intervals for the gender coefficient in this trade-off for a sample
income dataset and several SDC methods and parameters. The red dot and
confidence bar (on the top) correspond to the estimates for the
untreated data, whereas the other confidence bars correspond to the
respective SC methods and different parameters. The anonymization
reduces the number of expected re-identifications in the data (left
axis) and the point estimates and confidence intervals vary greatly for
the different SDC methods. We would choose a method, which reduces the
expected number of identifications, while not changing the gender
coefficient and having a great overlap of the confidence interval with
the confidence interval estimated from the original data.

\noindent\sphinxincludegraphics[width=6.48958in,height=3.25in]{{image12}.png}

Figure 6.2: Effect of anonymization on the point estimates and
confidence interval of the gender coefficient in the Mincer equation


\section{Assessing data utility with the help of data visualizations (in \sphinxstyleemphasis{R})}
\label{\detokenize{utility:assessing-data-utility-with-the-help-of-data-visualizations-in-r}}
The use of graphs and other visualization techniques is a good way to
assess at a glance how much the data have changed after anonymization,
and can aid the selection of appropriate anonymization techniques for
the data. Visualizations can be a useful tool to assess the impact on
data utility of anonymization methods and helps choose among
anonymization methods. The software package \sphinxstyleemphasis{R} provides several
functions and packages that can help visualize the results of
anonymization. This section lists a few of these functions and packages
and provides code examples to illustrate how to implement them. We
present the following visualizations:
\begin{itemize}
\item {} 
histograms and density plots

\item {} 
boxplots

\item {} 
mosaic plots

\end{itemize}

To make appropriate visualizations, we need to use the raw data and the
anonymized data. When using an \sphinxstyleemphasis{sdcMicro} object for the anonymization
process, the raw data are stored in the “origData” slot of the object
and the anonymized variables are in the slots “manipKeyVars”,
“manipPramVars”, “manipNumVars” and “manipStrataVar” slots. See Section
7.5 for more information on \sphinxstyleemphasis{sdcMicro} objects, slots and how to access
slots.


\subsection{Histograms and density plots}
\label{\detokenize{utility:histograms-and-density-plots}}
Histograms and density plots are useful for quick comparisons of
variable distribution before and after anonymization. The advantage of
histograms is that the results are exact. Visualization depends on the
bin widths and the start point of the first bin, however. Histograms can
be used for continuous and semi-continuous variables. Density plots
display the kernel density of the data; therefore, the plot depends on
the kernel that is chosen and whether the data fits the kernel well.
Nevertheless, density plots are a good tool to illustrate the change of
values and value ranges of continuous variables.

Histograms can be plotted with function hist() and kernel densities with
the functions plot() and density() in \sphinxstyleemphasis{R}. Example 6.11 provides
examples of how to use these functions to illustrate the changes in the
variable ”INC”, an income variable. The function hist() needs as
argument the break points for the histogram. The results are shown in
Figure 6.3 and Figure 6.4. The histograms and density plots give a clear
indication how the values have changed: the variability of the data has
increased and the shape of the distribution has changed. \sphinxstylestrong{NOTE: The
vertical axes of the histograms have different scales.}

Example 6.11: Plotting histograms and kernel densities

\begin{DUlineblock}{0em}
\item[] \sphinxstyleemphasis{\# Plot histograms}
\item[] \sphinxstyleemphasis{\# Plot histogram before anonymization}
\item[] \sphinxstylestrong{hist}(\sphinxhref{mailto:sdcInitial@origData\$INC}{sdcInitial@origData\$INC}, breaks = (0:180)*1e2, main =
“Histogram income - original data”)
\end{DUlineblock}

\begin{DUlineblock}{0em}
\item[] \sphinxstyleemphasis{\# Plot histogram after anonymization (noise addition)}
\item[] \sphinxstylestrong{hist}(\sphinxhref{mailto:sdcInitial@manipNumVars\$INC}{sdcInitial@manipNumVars\$INC}, breaks = (-20:190)*1e2, main =
“Histogram income - anonymized data”)
\end{DUlineblock}

\begin{DUlineblock}{0em}
\item[] \sphinxstyleemphasis{\# Plot densities}
\item[] \sphinxstyleemphasis{\# Plot original density curve}
\item[] \sphinxstylestrong{plot}(\sphinxstylestrong{density}(\sphinxhref{mailto:sdcInitial@origData\$INC}{sdcInitial@origData\$INC}), xlim = \sphinxstylestrong{c}(0,
8000), ylim = \sphinxstylestrong{c}(0, 0.006), main = “Density income”, xlab =
“income”)
\item[] \sphinxstylestrong{par}(new = TRUE)
\item[] \sphinxstyleemphasis{\# Plot density curve after anonymization (noise addition)}
\item[] \sphinxstylestrong{plot}(\sphinxstylestrong{density}(\sphinxhref{mailto:sdcInitial@manipNumVars\$INC}{sdcInitial@manipNumVars\$INC}), xlim =
\sphinxstylestrong{c}(0, 8000), ylim = \sphinxstylestrong{c}(0, 0.006), main = “Density income”,
xlab = “income”)
\end{DUlineblock}

\noindent\sphinxincludegraphics[width=6.48958in,height=3.23958in]{{image13}.png}

Figure 6.3: Histograms of income before and after anonymization

\noindent\sphinxincludegraphics[width=6.48958in,height=3.23958in]{{image14}.png}

Figure 6.4: Density plots of income before and after anonymization


\subsection{Box plots}
\label{\detokenize{utility:box-plots}}
Box plots give a quick overview of the changes in the spread and
outliers of continuous variables before and after anonymization. Example
6.12 shows how to generate box plots in \sphinxstyleemphasis{R} with the function boxplot().
The result in Figure 6.5 shows an example for an expenditure variable
after adding noise. The box plot shows clearly that the variability in
the expenditure variable increased as a result of the anonymization
methods applied.

Example 6.12: Creating boxplots for continuous variables

\sphinxstylestrong{boxplot}(\sphinxhref{mailto:sdcObj@origData\$TOTFOOD}{sdcObj@origData\$TOTFOOD}, \sphinxhref{mailto:sdcObj@manipNumVars\$TOTFOOD}{sdcObj@manipNumVars\$TOTFOOD}, xaxt
= ‘n’, ylab = “Expenditure”)

\sphinxstylestrong{axis}(1, at = \sphinxstylestrong{c}(1,2), labels = \sphinxstylestrong{c}(‘before, ‘after’))

\noindent\sphinxincludegraphics[width=6.48958in,height=3.25in]{{image15}.png}

Figure 6.5: Example of box plots of an expenditure variable before and
after anonymization


\subsection{Mosaic plots}
\label{\detokenize{utility:mosaic-plots}}
Univariate and multivariate mosaic plots are useful for showing changes
in the tabulations of categorical variables, especially when comparing
several “scenarios” next to one another. A scenario here refers to the
choice of anonymization methods and their parameters. With mosaic plots
we can, for instance, quickly see the effect of different levels of
\(k\)-anonymity or differences in the importance vectors in the
local suppression algorithm (see Section 5.2.2).

We illustrate the changes in tabulations with an example of the variable
“WATER” before and after applying PRAM. We can use mosaic plots to
quickly see the changes for each category. Example 6.13 shows the code
in \sphinxstyleemphasis{R}. The function mosaicplot() is available in base \sphinxstyleemphasis{R}. To plot a
tabulation, first the tabulation must be made with the table() function.
To show the labels in the mosaicplot(), we change the class of the
variables to ‘factor’ (see Section 7.4 on classes in \sphinxstyleemphasis{R}). Looking at
the mosaic plot in Figure 6.6 we see invariant PRAM has virtually no
influence on the univariate distribution.

Example 6.13: Creating univariate mosaic plots

\begin{DUlineblock}{0em}
\item[] \sphinxstyleemphasis{\# Collecting data of variable WATER before and after anonymization,
assigning factor levels for labels in plot}
\item[] dataWater \textless{}-
\sphinxstylestrong{t}(\sphinxstylestrong{cbind}(\sphinxstylestrong{table}(\sphinxstylestrong{factor}(\sphinxhref{mailto:sdcHH@origData\$WATER}{sdcHH@origData\$WATER},
levels = \sphinxstylestrong{c}(1, 2, 3, 4, 5, 6, 7, 8, 9),
\item[] labels = \sphinxstylestrong{c}(“Pipe (own tap)”, “Public standpipe”, “Borehole”,
“Wells
\item[] (protected)”, “Wells (unprotected)”, “Surface water”, “Rain water”,
\item[] “Vendor/truck”, “Other”))),
\sphinxstylestrong{table}(\sphinxstylestrong{factor}(\sphinxhref{mailto:sdcHH@manipPramVars\$WATER}{sdcHH@manipPramVars\$WATER},
\item[] levels = \sphinxstylestrong{c}(1,2, 3, 4, 5, 6, 7, 8, 9), labels = \sphinxstylestrong{c}(“Pipe
(own tap)”,
\item[] “Public standpipe”, “Borehole”, “Wells (protected)”, “Wells
\item[] (unprotected)”, “Surface water”, “Rain water”, “Vendor/truck”,
\item[] “Other”)))))
\end{DUlineblock}

\sphinxstylestrong{rownames}(dataWater) \textless{}- \sphinxstylestrong{c}(“before”, “after”)

\begin{DUlineblock}{0em}
\item[] \sphinxstyleemphasis{\# Plotting mosaic plot}
\item[] \sphinxstylestrong{mosaicplot}(dataWater, main = “”, color = 2:10, las = 2)
\end{DUlineblock}

\noindent\sphinxincludegraphics[width=6.48958in,height=3.23958in]{{image16}.png}

Figure 6.6: Mosaic plot to illustrate the changes in the WATER variable

We use the variables “gender” and “relationship status” to illustrate
the use of mosaic plots for the illustration of changes in univariate
tabulations introduced by several sets of anonymization methods. Table
6.1 provides the methods applied in each scenario. Scenario 0, the base
scenario, shows the original categories of the gender and relationship
status variables, while scenarios 1 to 6 show shifts in the categories
after applying different anonymization techniques. Table 6.1 provides a
description of the anonymization methods used in each scenario. In total
we visualize the impact of six different sets of anonymization methods.
We can use mosaic plots to quickly see which set of methods has what
impact on the gender and relationship status variables, which can be
used to select the best scenario. Looking at the mosaic plots in Figure
6.7, we see that scenarios 2, 5 and 6 give the smallest changes for the
gender variable and scenarios 3 and 4 for the relationship status
variable.

Table 6.1: Description of anonymization methods by scenario


\begin{savenotes}\sphinxattablestart
\centering
\begin{tabulary}{\linewidth}[t]{|T|T|}
\hline
\sphinxstyletheadfamily 
\sphinxstylestrong{Scenario}
&\sphinxstyletheadfamily 
\sphinxstylestrong{Description of anonymization
methods applied}
\\
\hline
0 (base)
&
Original data, no treatment
\\
\hline
\end{tabulary}
\par
\sphinxattableend\end{savenotes}

\begin{DUlineblock}{0em}
\item[] 0 (base)                          \textbar{} Original data, no treatment       \textbar{}
\end{DUlineblock}
\begin{quote}

\begin{DUlineblock}{0em}
\item[] 
\end{DUlineblock}
\end{quote}
\begin{quote}

\begin{DUlineblock}{0em}
\item[] 
\end{DUlineblock}
\end{quote}


\begin{savenotes}\sphinxattablestart
\centering
\begin{tabulary}{\linewidth}[t]{|T|T|}
\hline

3
&
Recode age (five-year intervals),
plus local suppression (required
k = 3, high importance on
toilet), while also recoding
region, urban, education level
and occupation variables
\\
\hline
4
&
Recode age (five-year steps),
plus local suppression (required
k = 5, high importance on water,
toilet and literacy), while also
recoding region, urban, education
level and occupation variables
\\
\hline
5
&
Recode age (five-year intervals),
plus local suppression (required
k = 3, no importance vector),
microaggregation (wealth index),
while also recoding region,
urban, education level and
occupation variables
\\
\hline
6
&
Recode age (five-year intervals)
plus local suppression (required
k=3, no importance vector), PRAM
literacy, while also recoding
region, urban, education level
and occupation variables
\\
\hline
\end{tabulary}
\par
\sphinxattableend\end{savenotes}

\noindent\sphinxincludegraphics[width=6.5in,height=3.25556in]{{image17}.png}

Figure 6.7: Comparison of treated vs. untreated gender and relationship
status variables with mosaic plots

As we discussed in Section 5.3.1, invariant PRAM preserves the
univariate distributions. Therefore, in this case it is more interesting
to look at the multivariate mosaic plots. Mosaic plots are also a
powerful tool to show changes in cross-tabulations/contingency tables.
Example 6.14 shows how to generate mosaic plots for two variables. To
compare the changes, we need to compare two different plots. Figure 6.8
and Figure 6.9 illustrate that (invariant) PRAM does not preserve the
two-way tables in this case.

Example 6.14: Creating multivariate mosaic plots

\begin{DUlineblock}{0em}
\item[] \sphinxstyleemphasis{\# Before anonymization: contingency table and mosaic plot}
\item[] ROOFTOILETbefore \textless{}-
\sphinxstylestrong{t}(\sphinxstylestrong{table}(\sphinxstylestrong{factor}(\sphinxhref{mailto:sdcHH@origData\$ROOF}{sdcHH@origData\$ROOF}, levels =
\sphinxstylestrong{c}(1,2, 3, 4, 5, 9),
\item[] labels = \sphinxstylestrong{c}(“Concrete/cement/ \textbackslash{}n brick/stone”, “Wood”,
\item[] “Bamboo/thatch”, “Tiles/shingles”,
\item[] “Tin/metal sheets”, “Other”)),
\item[] \sphinxstylestrong{factor}(\sphinxhref{mailto:sdcHH@origData\$TOILET}{sdcHH@origData\$TOILET}, levels = \sphinxstylestrong{c}(1,2, 3, 4, 9),
\item[] labels = \sphinxstylestrong{c}(“Flush \textbackslash{}n toilet”, “Improved \textbackslash{}n pit \textbackslash{}n latrine”,
\item[] “Pit \textbackslash{}n latrine”, “No \textbackslash{}n facility”, “Other”))))
\item[] \sphinxstylestrong{mosaicplot}(ROOFTOILETbefore, main = “”, las = 2, color = 2:6)
\item[] \sphinxstyleemphasis{\# After anonymization: contingency table and mosaic plot}
\item[] ROOFTOILETafter \textless{}-
\sphinxstylestrong{t}(\sphinxstylestrong{table}(\sphinxstylestrong{factor}(\sphinxhref{mailto:sdcHH@manipPramVars\$ROOF}{sdcHH@manipPramVars\$ROOF}, levels =
\sphinxstylestrong{c}(1,2, 3, 4, 5, 9),
\item[] labels = \sphinxstylestrong{c}(“Concrete/cement/ \textbackslash{}n brick/stone”, “Wood”,
\item[] “Bamboo/thatch”, “Tiles/shingles”,
\item[] “Tin/metal sheets”, “Other”)),
\item[] \sphinxstylestrong{factor}(\sphinxhref{mailto:sdcHH@manipPramVars\$TOILET}{sdcHH@manipPramVars\$TOILET}, levels = \sphinxstylestrong{c}(1,2, 3, 4,
9),
\item[] labels = \sphinxstylestrong{c}(“Flush \textbackslash{}n toilet”, “Improved \textbackslash{}n pit \textbackslash{}n latrine”,
\item[] “Pit \textbackslash{}n latrine”, “No \textbackslash{}n facility”, “Other”))))
\item[] \sphinxstylestrong{mosaicplot}(ROOFTOILETafter, main = “”, las = 2, color = 2:6)
\end{DUlineblock}

\noindent\sphinxincludegraphics[width=6.5in,height=3.25in]{{image18}.png}

Figure 6.8: Mosaic plot of the variables ROOF and TOILET before
anonymization

\noindent\sphinxincludegraphics[width=6.5in,height=3.25in]{{image19}.png}

Figure 6.9: Mosaic plot of the variables ROOF and TOILET after
anonymization


\section{Choice of utility measure}
\label{\detokenize{utility:choice-of-utility-measure}}
Besides the users’ requirements on the data, the utility measures should
be chosen in accordance with the variable types and anonymization
methods employed. The employed utility measures can be a combination of
both general and user-specific measures. As discussed earlier, different
utility measures should be used for continuous and categorical data.
Furthermore, some utility measures are not informative after certain
anonymization methods have been applied. For example, after applying
perturbative methods that interchange data values, comparing values
directly is not useful because they will give the impression of high
levels of information loss. In such cases, it is more informative to
look at means, covariances and benchmarking indicators that can be
computed from the data. Furthermore, it is important not only to focus
on the characteristics of variables one by one, but also on the
interactions between variables. This can be done by cross-tabulations
and regressions. In general, when anonymizing sampled data, it is
advisable to compute confidence intervals around estimates to interpret
the magnitude of changes.

\begin{sphinxadmonition}{note}{Recommended Reading Material on Measuring Utility and Information Loss}

A.G. De Waal and L.C.R.J. Willenborg. 1999. “Information Loss through
Global Recoding and Local Suppression” In Netherlands Official
Statistics: Special Issue on SDC, 14, 17-10.

J. Domingo-Ferrer, J.M. Mateo-Sanz and V. Torra. 2001. “Comparing SDC
Methods for Microdata on the basis of Information Loss and Disclosure
Risk”. In Pre-proceedings of ETK-NTTS 2001 (vol. 2), 807-826.
\sphinxurl{http://neon.vb.cbs.nl/casc/NTTSJosep.pdf}

J. Domingo-Ferrer and V. Torra. 2001. “Disclosure Protection Methods and
Information Loss for Microdata”. In P. Doyle, J.I. Lane, J.J.M. Theeuwes
and L. Zayatz (eds.) \sphinxstyleemphasis{Theory and Practical Applications for Statistical
Agencies}, 91-110, Amsterdam.
\sphinxurl{http://crises-deim.urv.cat/webCrises/publications/bcpi/cliatpasa01Disclosure.pdf}
\end{sphinxadmonition}


\chapter{SDC with \sphinxstyleemphasis{sdcMicro} in \sphinxstyleemphasis{R}: Setting Up Your Data and more}
\label{\detokenize{sdcMicro:sdc-with-sdcmicro-in-r-setting-up-your-data-and-more}}\label{\detokenize{sdcMicro::doc}}

\section{Installing \sphinxstyleemphasis{R}, \sphinxstyleemphasis{sdcMicro} and other packages}
\label{\detokenize{sdcMicro:installing-r-sdcmicro-and-other-packages}}
This guide is based on the software package \sphinxstyleemphasis{sdcMicro}, which is an
add-on package for the statistical software \sphinxstyleemphasis{R}. Both \sphinxstyleemphasis{R} and
\sphinxstyleemphasis{sdcMicro}, as well as other \sphinxstyleemphasis{R} packages, are freely available from the
CRAN (Comprehensive R Archive Network) website for Linux, Mac and
Windows (\sphinxurl{http://cran.r-project.org}). This website also offers
descriptions of packages. Besides the standard version of \sphinxstyleemphasis{R}, there is
a more user-friendly user interface for \sphinxstyleemphasis{R}: \sphinxstyleemphasis{RStudio}. \sphinxstyleemphasis{RStudio} is
also freely available for Linux, Mac and Windows
(\sphinxurl{http://www.rstudio.com}). The \sphinxstyleemphasis{sdcMicro} package is dependent on (i.e.,
uses) other \sphinxstyleemphasis{R} packages that must be installed on your computer before
using \sphinxstyleemphasis{sdcMicro}. Those will automatically be installed when installing
\sphinxstyleemphasis{sdcMicro}. For some functionalities, we use still other packages (such
as \sphinxstyleemphasis{foreign} for reading data and some graphical packages). If so, this
is indicated in the appropriate section in this guide. \sphinxstyleemphasis{R}, \sphinxstyleemphasis{RStudio},
the \sphinxstyleemphasis{sdcMicro} package and its dependencies and other packages have
regular updates. It is strongly recommended to regularly check for
updates: this requires installing a new version for an update of \sphinxstyleemphasis{R;}
with the update.packages() command or using the menu options in \sphinxstyleemphasis{R} or
\sphinxstyleemphasis{RStudio} one can update the installed packages.

When starting \sphinxstyleemphasis{R} or \sphinxstyleemphasis{RStudio}, it is necessary to specify each time
which packages are being used by loading those. This loading of packages
can be done either with the library() or the require() function. Both
options are illustrated in Example 7.1.

Example 7.1: Loading required packages

\sphinxstylestrong{library}(sdcMicro) \sphinxstyleemphasis{\# loading the sdcMicro package}

\sphinxstylestrong{require}(sdcMicro) \sphinxstyleemphasis{\# loading the sdcMicro package}

All packages and functions are documented. The easiest way to access the
documentation of a specific function is to use the built-in help, which
generally gives an overview of the parameters of the functions as well
as some examples. The help of a specific function can be called by a
question mark followed by the function name without any arguments.
Example 7.2 shows how to call the help file for the microaggregation()
function of the \sphinxstyleemphasis{sdcMicro} package.{[}\#foot60{]}\_ The download
page of the each package on the CRAN website also provides a reference
manual with a complete overview of the functions in the package.

Example 7.2: Displaying help for functions

?microaggregation \sphinxstyleemphasis{\# help for microaggregation function}

When issues or bugs in the \sphinxstyleemphasis{sdcMicro} package are encountered, comments,
remarks or suggestions can be posted for the developers of \sphinxstyleemphasis{sdcMicro} on
their GitHub: \sphinxurl{https://github.com/alexkowa/sdcMicro/issues}


\section{Read functions in \sphinxstyleemphasis{R}}
\label{\detokenize{sdcMicro:read-functions-in-r}}
The first step in the SDC process when using \sphinxstyleemphasis{sdcMicro} is to read the
data into \sphinxstyleemphasis{R} and create a dataframe. %
\begin{footnote}[2]\sphinxAtStartFootnote
A dataframe is an object class in \sphinxstyleemphasis{R}, which is similar to a data
table or matrix.
%
\end{footnote} \sphinxstyleemphasis{R} is
compatible with most statistical data formats and provides read
functions for most types of data. For those read functions, it is
sometimes necessary to install additional packages and their
dependencies in \sphinxstyleemphasis{R}. An overview of data formats, functions and the
packages containing these functions is provided in Table 7.1. These
functions are also available as write (e.g., write.dta()) to save the
anonymized data in the required format. \#foot{[}62{]}\_

Table 7.1: Packages and functions for reading data in \sphinxstyleemphasis{R}


\begin{savenotes}\sphinxattablestart
\centering
\begin{tabulary}{\linewidth}[t]{|T|T|T|T|}
\hline
\sphinxstyletheadfamily 
{\color{red}\bfseries{}**}Type/software
**
&\sphinxstyletheadfamily 
\sphinxstylestrong{Extension}
&\sphinxstyletheadfamily 
\sphinxstylestrong{Package}
&\sphinxstyletheadfamily 
\sphinxstylestrong{Function}
\\
\hline
SPSS
&
.sav
&
\sphinxstyleemphasis{foreign}
&
read.spss()
\\
\hline
STATA (v. 5-12)
&
.dta
&
\sphinxstyleemphasis{foreign}
&
read.dta()
\\
\hline
STATA (v.
13) {[}\#foot64{]}
&
.dta
&
\sphinxstyleemphasis{readstata13}
&
read.dta13()
\\
\hline
SAS
&
.xpt
&
\sphinxstyleemphasis{foreign}
&
read.xport()
\\
\hline
EXCEL
&
.csv
&
\sphinxstyleemphasis{utils} (base
package)
&
read.csv()
\\
\hline
EXCEL
&
.xls
&
\sphinxstyleemphasis{xlsReadWrite}
&
read.xls()
\\
\hline
EXCEL
&
.xlsx
&
\sphinxstyleemphasis{xlsx}
&
read.xlsx()
\\
\hline
\end{tabulary}
\par
\sphinxattableend\end{savenotes}

Most of these functions have options that specify how to handle missing
values and variables with factor levels and value labels. Example 7.3,
Example 7.4 and Example 7.5 provide example code for reading in a
\sphinxstyleemphasis{STATA} (.dta) file, an \sphinxstyleemphasis{Excel} (.csv) file and a \sphinxstyleemphasis{SPSS} (.sav) file.

Example 7.3: Reading in a \sphinxstyleemphasis{STATA} file

\begin{DUlineblock}{0em}
\item[] \sphinxstylestrong{setwd}(“/Users/World Bank”) \sphinxstyleemphasis{\# working directory with data file}
\item[] fname = “data.dta” \sphinxstyleemphasis{\# name of data file}
\item[] \sphinxstylestrong{library}(foreign) \sphinxstyleemphasis{\# loads required package for read/write
function for STATA files}
\item[] file \textless{}- \sphinxstylestrong{read.dta}(fname, convert.factors = F)
\item[] \sphinxstyleemphasis{\# reads the data into the data frame called file, factor levels read
as numeric codes}
\end{DUlineblock}

Example 7.4: Reading in an \sphinxstyleemphasis{Excel} file

\begin{DUlineblock}{0em}
\item[] \sphinxstylestrong{setwd}(“/Users/World Bank”) \sphinxstyleemphasis{\# working directory with data file}
\item[] fname = “data.csv” \sphinxstyleemphasis{\# name of data file}
\item[] file \textless{}- \sphinxstylestrong{read.csv}(fname, header = TRUE, sep = “,”, dec = “.”)
\item[] \sphinxstyleemphasis{\# reads the data into the data frame called file, the first line
contains the variable names, fields are separated with commas, decimal
points are indicated with ‘.’}
\end{DUlineblock}

Example 7.5: Reading in an \sphinxstyleemphasis{SPSS} file

\begin{DUlineblock}{0em}
\item[] \sphinxstylestrong{setwd}(“/Users/World Bank”) \sphinxstyleemphasis{\# working directory with data file}
\item[] fname = “data.sav” \sphinxstyleemphasis{\# name of data file}
\item[] \sphinxstylestrong{library}(foreign) \sphinxstyleemphasis{\# loads required package for read/write
function for SPSS files}
\item[] file \textless{}- \sphinxstylestrong{read.spss}(fname, use.value.labels = FALSE)
\item[] \sphinxstyleemphasis{\# reads the data into the data frame called file, factor levels are
read as numeric codes}
\end{DUlineblock}

The maximum data size in \sphinxstyleemphasis{R} is technically restricted. The maximum size
depends on the \sphinxstyleemphasis{R} build (32-bit or 64-bit) and the operating system.
Some SDC methods require long computation times for large datasets (see
Section 7.7 on computation times).


\section{Missing values}
\label{\detokenize{sdcMicro:missing-values}}
The standard way missing values are represented in \sphinxstyleemphasis{R} is by the symbol
‘NA’, which is different to impossible values, such as division by zero
or the log of a negative number, which are represented by the symbol
‘NaN’. The value ‘NA’ is used for both numeric and categorical
variables.{[}\#foot65{]}\_ Values suppressed by the
localSuppression() routine are also replaced by the ‘NA’ symbol. Some
datasets and statistical software might use different values for missing
values, such as ‘999’ or strings. It is possible to include arguments in
read functions to specify how missing values in the dataset should be
treated and automatically recode missing values to ‘NA’. For instance,
the function read.table() has the ‘na.strings’ argument, which replaces
the specified strings with ‘NA’ values.

Missing values can also be recoded after reading the data into \sphinxstyleemphasis{R}. This
may be necessary if there are several different missing value codes in
the data, different missing value codes for different variables or the
read function for the datatype does not allow specifying the missing
value codes. When preparing data, it is important to recode any missing
values that are not coded as ‘NA’ to ‘NA’ in \sphinxstyleemphasis{R} before starting the
anonymization process to ensure the correct measurement of risk (e.g.,
\(k\)-anonymity), as well as to ensure that many of the methods are
correctly applied to the data. Example 7.6 shows how to recode the value
‘99’ to ‘NA’ for the variable “toilet”.

Example 7.6: Recoding missing values to NA

file{[}file{[},’toilet’{]} == 99,’toilet’{]} \textless{}- NA \sphinxstyleemphasis{\# Recode missing value code
99 to NA for variable toilet}


\section{Classes in \sphinxstyleemphasis{R}}
\label{\detokenize{sdcMicro:classes-in-r}}
All objects in \sphinxstyleemphasis{R} are of a specific class, such as integer, character,
matrix, factor or dataframe. The class of an object is an attribute from
which the object inherits. To find out the class of an object, one can
use the function class(). Functions in \sphinxstyleemphasis{R} might require objects or
arguments of certain classes or functions might have different
functionality depending on the class of the argument. Examples are the
write functions that require dataframes and most functions in the
\sphinxstyleemphasis{sdcMicro} package that require either dataframes or \sphinxstyleemphasis{sdcMicro} objects.
The functionality of the functions in the \sphinxstyleemphasis{sdcMicro} package differs for
dataframes and \sphinxstyleemphasis{sdcMicro} objects. It is easy to change the class
attribute of an object with functions that start with “as.”, followed by
the name of the class (e.g., as.factor(), as.matrix(), as.data.frame()).
Example 7.7 shows how to check the class of an object and change the
class to “data.frame”. Before changing the class attribute of the object
“file”, it was in the class “matrix”. An important class defined and
used in the \sphinxstyleemphasis{sdcMicro} package is the class named \sphinxstyleemphasis{sdcMicroObj}. This
class is described in the next section.

Example 7.7: Changing the class of an object in \sphinxstyleemphasis{R}

\begin{DUlineblock}{0em}
\item[] \sphinxstyleemphasis{\# Finding out the class of the object ‘file’}
\item[] \sphinxstylestrong{class}(file)
\end{DUlineblock}

“matrix”

\begin{DUlineblock}{0em}
\item[] \sphinxstyleemphasis{\# Changing the class to data frame}
\item[] file \textless{}- \sphinxstylestrong{as.data.frame}(file)
\end{DUlineblock}

\sphinxstyleemphasis{\# Checking the result}

\sphinxstylestrong{class}(file)

“data.frame”


\section{Objects of class \sphinxstyleemphasis{sdcMicroObj}}
\label{\detokenize{sdcMicro:objects-of-class-sdcmicroobj}}
The \sphinxstyleemphasis{sdcMicro} package is built around objects{[}\#foot66{]}\_ of
class \sphinxstyleemphasis{sdcMicroObj}, a class especially defined for the \sphinxstyleemphasis{sdcMicro}
package. Each member of this class has a certain structure with slots
that contain information regarding the anonymization process (see Table
7.2 on page 100 for a description of all slots). Before evaluating risk
and utility and applying SDC methods, creating an object of class
\sphinxstyleemphasis{sdcMicro} is recommended. All examples in this guide are based on these
objects. The function used to create an \sphinxstyleemphasis{sdcMicro} object is
createSdcObj(). Most functions in the \sphinxstyleemphasis{sdcMicro} package, such as
microaggregation() or localSuppression(), automatically use the required
information (e.g., quasi-identifiers, sample weights) from the
\sphinxstyleemphasis{sdcMicro} object if applied to an object of class \sphinxstyleemphasis{sdcMicro}.

The arguments of the function createSdcObj() allow one to specify the
original data file and categorize the variables in this data file before
the start of the anonymization process. \sphinxstylestrong{NOTE: For this, disclosure
scenarios must already have been evaluated and quasi-identifiers
selected. In addition, one must ensure there are no problems with the
data, such as variables containing only missing values.}

In Example 7.8, we show all arguments of the function createSdcObj(),
and first define vectors with the names of the different variables. This
practice gives a better overview and later allows for quick changes in
the variable choices if required. We choose the categorical
quasi-identifiers (keyVars); the variables linked to the categorical
quasi-identifiers that need the same suppression pattern (ghostVars, see
Section 5.2.2); the numerical quasi-identifiers (numVars); the variables
selected for applying PRAM (pramVars); a variable with sampling weights
(weightVar); the clustering ID (hhId, e.g., a household ID, see Section
4.9); a variable specifying the strata (strataVar) and the sensitive
variables specified for the computation of \(l\)-diversity
(sensibleVar , see Section 4.5.3). \sphinxstylestrong{NOTE: Most SDC methods in the
sdcMicro package are automatically applied within the strata, if the
‘strataVar’ argument is specified.} Examples are local suppression and
PRAM. Not all variables must be specified, e.g., if there is no
hierarchical (household) structure, the argument ‘hhId’ can be omitted.
The names of the variables correspond to the names of the variables in
the dataframe containing the microdata to be anonymized. The selection
of variables is important for the risk measures that are automatically
calculated. Furthermore, several methods are by default applied to all
variables of one sort, e.g., microaggregation to all key
variables.{[}\#foot67{]}\_ After selecting these variables, we can
create the \sphinxstyleemphasis{sdcMicro} object. To obtain a summary of the object, it is
sufficient to write the name of the object.

Example 7.8: Selecting variables and creating an object of class
\sphinxstyleemphasis{sdcMicroObj} for the SDC process in \sphinxstyleemphasis{R}

\begin{DUlineblock}{0em}
\item[] \sphinxstyleemphasis{\# Select variables for creating sdcMicro object}
\item[] \sphinxstyleemphasis{\# All variable names should correspond to the names in the data file}
\item[] \sphinxstyleemphasis{\# selected categorical key variables}
\item[] selectedKeyVars = \sphinxstylestrong{c}(‘region’, ‘age’, ‘gender’, ‘marital’,
‘empstat’)
\end{DUlineblock}

\begin{DUlineblock}{0em}
\item[] \sphinxstyleemphasis{\# selected linked variables (ghost variables)}
\item[] selectedGhostVars = \sphinxstylestrong{c}(‘urbrur’)
\item[] \sphinxstyleemphasis{\# selected categorical numerical variables}
\item[] selectedNumVar = \sphinxstylestrong{c}(‘wage’, ‘savings’)
\item[] \sphinxstyleemphasis{\# weight variable}
\item[] selectedWeightVar = \sphinxstylestrong{c}(‘wgt’)
\item[] \sphinxstyleemphasis{\# selected pram variables}
\item[] selectedPramVars = \sphinxstylestrong{c}(‘roof’, ‘wall’)
\item[] \sphinxstyleemphasis{\# household id variable (cluster)}
\item[] selectedHouseholdID = \sphinxstylestrong{c}(‘idh’)
\item[] \sphinxstyleemphasis{\# stratification variable}
\item[] selectedStrataVar = \sphinxstylestrong{c}(‘strata’)
\end{DUlineblock}

\begin{DUlineblock}{0em}
\item[] \sphinxstyleemphasis{\# sensitive variables for l-diversity computation}
\item[] selectedSensibleVar = \sphinxstylestrong{c}(‘health’)
\end{DUlineblock}

\begin{DUlineblock}{0em}
\item[] \sphinxstyleemphasis{\# creating the sdcMicro object with the assigned variables}
\item[] sdcInitial \textless{}- \sphinxstylestrong{createSdcObj}(dat = file,
\item[] keyVars = selectedKeyVars,
\item[] ghostVars = selectedGhostVars,
\item[] numVar = selectedNumVar,
\item[] weightVar = selectedWeightVar,
\item[] pramVars = selectedPramVars,
\item[] hhId = selectedHouseholdID,
\item[] strataVar = selectedStrataVar,
\item[] sensibleVar = selectedSensibleVar)
\end{DUlineblock}

\begin{DUlineblock}{0em}
\item[] \sphinxstyleemphasis{\# Summary of object}
\item[] sdcInitial
\end{DUlineblock}

\begin{DUlineblock}{0em}
\item[] \sphinxcode{\sphinxupquote{\#\# Data set with 4580 rows and 14 columns.}}
\item[] \sphinxcode{\sphinxupquote{\#\#  -{-}\textgreater{} Categorical key variables:}}\sphinxcode{\sphinxupquote{region, age, gender, marital, empstat}}
\item[] \sphinxcode{\sphinxupquote{\#\#  -{-}\textgreater{} Numerical key va}}\sphinxcode{\sphinxupquote{riables: wage, savings}}
\item[] \sphinxcode{\sphinxupquote{\#\#  -{-}\textgreater{}}}\sphinxcode{\sphinxupquote{Weight variable: wgt}}
\item[] \sphinxcode{\sphinxupquote{\#\# -{-}-{-}-{-}-{-}-{-}-{-}-{-}-{-}-{-}-{-}-{-}-{-}-{-}-{-}-{-}-{-}-{-}-{-}-{-}-{-}-{-}-{-}-{-}-{-}-{-}-{-}-{-}-{-}-{-}-{-}-{-}-{-}-{-}-{-}-{-}-{-}-{-}-}}
\item[] \sphinxcode{\sphinxupquote{\#\#}}
\item[] \sphinxcode{\sphinxupquote{\#\# Information on categorical Key-Variables:}}
\item[] \sphinxcode{\sphinxupquote{\#\#}}
\item[] \sphinxcode{\sphinxupquote{\#\# Reported is the number, mean size and size of the smallest category for recoded variables.}}
\item[] \sphinxcode{\sphinxupquote{\#\# In parenthesis, the same statistics are shown for the unmodified data.}}
\item[] \sphinxcode{\sphinxupquote{\#\# Note: NA (missings) are counted as seperate categories!}}
\item[] \sphinxcode{\sphinxupquote{\#\#}}
\item[] \sphinxcode{\sphinxupquote{\#\#  Key Variable Number of categories     Mean size}}
\item[] \sphinxcode{\sphinxupquote{\#\#}}\sphinxcode{\sphinxupquote{region}}\sphinxcode{\sphinxupquote{2 (2)  2290.000 (2290.000)}}
\item[] \sphinxcode{\sphinxupquote{\#\#           age}}\sphinxcode{\sphinxupquote{5 (5)   916.000  (916.000)}}
\item[] \sphinxcode{\sphinxupquote{\#\#        gender}}\sphinxcode{\sphinxupquote{3 (3)  1526.667 (1526.667)}}
\item[] \sphinxcode{\sphinxupquote{\#\#       marital}}\sphinxcode{\sphinxupquote{8 (8)   572.500  (572.500)}}
\item[] \sphinxcode{\sphinxupquote{\#\#       empstat}}\sphinxcode{\sphinxupquote{3 (3)  1526.667 (1526.667)}}
\item[] \#\#
\item[] \sphinxcode{\sphinxupquote{\#\#  Size of smallest}}
\item[] \sphinxcode{\sphinxupquote{\#\#               646  (646)}}
\item[] \sphinxcode{\sphinxupquote{\#\#                16   (16)}}
\item[] \sphinxcode{\sphinxupquote{\#\#                50   (50)}}
\item[] \sphinxcode{\sphinxupquote{\#\#                26   (26)}}
\item[] \sphinxcode{\sphinxupquote{\#\#               107  (107)}}
\item[] \sphinxcode{\sphinxupquote{\#\# -{-}-{-}-{-}-{-}-{-}-{-}-{-}-{-}-{-}-{-}-{-}-{-}-{-}-{-}-{-}-{-}-{-}-{-}-{-}-{-}-{-}-{-}-{-}-{-}-{-}-{-}-{-}-{-}-{-}-{-}-{-}-{-}-{-}-{-}-{-}-{-}-{-}-}}
\item[] \sphinxcode{\sphinxupquote{\#\#}}
\item[] \sphinxcode{\sphinxupquote{\#\# Infos on 2/3-Anonymity:}}
\item[] \sphinxcode{\sphinxupquote{\#\#}}
\item[] \sphinxcode{\sphinxupquote{\#\# Number of observations violating}}
\item[] \sphinxcode{\sphinxupquote{\#\#  - 2-anonymity: 157}}
\item[] \sphinxcode{\sphinxupquote{\#\#  - 3-anonymity: 281}}
\item[] \sphinxcode{\sphinxupquote{\#\#}}
\item[] \sphinxcode{\sphinxupquote{\#\# Percentage of observations violating}}
\item[] \sphinxcode{\sphinxupquote{\#\#  - 2-anonymity: 3.428 \%}}
\item[] \sphinxcode{\sphinxupquote{\#\#  - 3-anonymity: 6.135 \%}}
\item[] \sphinxcode{\sphinxupquote{\#\# -{-}-{-}-{-}-{-}-{-}-{-}-{-}-{-}-{-}-{-}-{-}-{-}-{-}-{-}-{-}-{-}-{-}-{-}-{-}-{-}-{-}-{-}-{-}-{-}-{-}-{-}-{-}-{-}-{-}-{-}-{-}-{-}-{-}-{-}-{-}-{-}-{-}-}}
\item[] \sphinxcode{\sphinxupquote{\#\#}}
\item[] \sphinxcode{\sphinxupquote{\#\# Numerical key variables:}}\sphinxcode{\sphinxupquote{wage, savings}}
\end{DUlineblock}

\begin{DUlineblock}{0em}
\item[] \sphinxcode{\sphinxupquote{\#\#}}
\item[] \sphinxcode{\sphinxupquote{\#\# Disclosure risk is currently between {[}0.00\%; 100.00{]}}}
\item[] \sphinxcode{\sphinxupquote{\#\#}}
\item[] \sphinxcode{\sphinxupquote{\#\# Current Information Loss:}}
\item[] \sphinxcode{\sphinxupquote{\#\#  IL1: 0.00}}
\item[] \sphinxcode{\sphinxupquote{\#\#  Difference of Eigenvalues: 0.000\%}}
\item[] \sphinxcode{\sphinxupquote{\#\# -{-}-{-}-{-}-{-}-{-}-{-}-{-}-{-}-{-}-{-}-{-}-{-}-{-}-{-}-{-}-{-}-{-}-{-}-{-}-{-}-{-}-{-}-{-}-{-}-{-}-{-}-{-}-{-}-{-}-{-}-{-}-{-}-{-}-{-}-{-}-{-}-{-}-}}
\end{DUlineblock}

Table 7.2 presents the names of the slots and their respective contents.
The slot names can be listed using the function slotNames(), which is
illustrated in Example 7.9. Not all slots are used in all cases. Some
slots are filled only after applying certain methods, e.g., evaluating a
specific risk measure. Certain slots of the objects can be accessed by
accessor functions (e.g., extractManipData for extracting the anonymized
data) or print functions (e.g., print()) with the appropriate arguments.
The content of a slot can also be accessed directly with the ‘@’
operator and the slot name. This is illustrated for the risk slot in
Example 7.9. This functionality can be practical to save intermediate
results and compare the outcomes of different methods. Also, for manual
changes to the data during the SDC process, such as changing missing
value codes or manual recoding, the direct accession of the data in the
slots with the manipulated data (i.e., slot names starting with ‘manip’)
is useful. Within each slot there are generally several elements. Their
names can be shown with the names() function and they can be accessed
with the ‘\$’ operator. This is shown for the element with the individual
risk in the risk slot.

Example 7.9: Displaying slot names and accessing slots of an S4 object

\begin{DUlineblock}{0em}
\item[] \sphinxstyleemphasis{\# List names of all slots of sdcMicro object}
\item[] \sphinxstylestrong{slotNames}(sdcInitial)
\end{DUlineblock}

\begin{DUlineblock}{0em}
\item[] \sphinxcode{\sphinxupquote{\#\#  {[}1{]} "origData"          "keyVars"           "pramVars"}}
\item[] \sphinxcode{\sphinxupquote{\#\#  {[}4{]} "numVars"           "ghostVars"         "weightVar"}}
\item[] \sphinxcode{\sphinxupquote{\#\#  {[}7{]} "hhId"              "strataVar"         "sensibleVar"}}
\item[] \sphinxcode{\sphinxupquote{\#\# {[}10{]} "manipKeyVars"      "manipPramVars"     "manipNumVars"}}
\item[] \sphinxcode{\sphinxupquote{\#\# {[}13{]} "manipGhostVars"    "manipStrataVar"    "originalRisk"}}
\item[] \sphinxcode{\sphinxupquote{\#\# {[}16{]} "risk"              "utility"           "pram"}}
\item[] \sphinxcode{\sphinxupquote{\#\# {[}19{]} "localSuppression"  "options"           "additionalResults"}}
\item[] \sphinxcode{\sphinxupquote{\#\# {[}22{]} "set"               "prev"              "deletedVars"}}
\end{DUlineblock}

\begin{DUlineblock}{0em}
\item[] \sphinxstyleemphasis{\# Accessing the risk slot}
\item[] \sphinxhref{mailto:sdcInitial@risk}{sdcInitial@risk}
\end{DUlineblock}

\begin{DUlineblock}{0em}
\item[] \sphinxstyleemphasis{\# List names within the risk slot}
\item[] names(\sphinxhref{mailto:sdcInitial@risk}{sdcInitial@risk})
\end{DUlineblock}

\sphinxcode{\sphinxupquote{\#\# {[}1{]} "global"}}{\color{red}\bfseries{}{}`{}`}\sphinxcode{\sphinxupquote{\textbackslash{} {}`{}`}}\sphinxcode{\sphinxupquote{"individual"}}{\color{red}\bfseries{}{}`{}`}\sphinxcode{\sphinxupquote{\textbackslash{} {}`{}`"numeric"}}

\sphinxstyleemphasis{\# Two ways to access the individual risk within the risk slot}

\sphinxhref{mailto:sdcInitial@risk\$individual}{sdcInitial@risk\$individual}

\sphinxstylestrong{get.sdcMicroObj}(sdcInitial, “risk”)\$individual

Table 7.2: Slot names and slot description of \sphinxstyleemphasis{sdcMicro} object


\begin{savenotes}\sphinxattablestart
\centering
\begin{tabulary}{\linewidth}[t]{|T|T|}
\hline
\sphinxstyletheadfamily 
\sphinxstylestrong{Slotname}
&\sphinxstyletheadfamily 
\sphinxstylestrong{Content}
\\
\hline
origData
&
original data as specified in the
dat argument of the
createSdcObj() function
\\
\hline
keyVars
&
indices of columns in origData
with specified categorical key
variables
\\
\hline
pramVars
&
indices of columns in origData
with specified PRAM variables
\\
\hline
numVars
&
indices of columns in origData
with specified numerical key
variables
\\
\hline
ghostVars
&
indices of columns in origData
with specified ghostVars
\\
\hline
weightVar
&
indices of columns in origData
with specified weight variable
\\
\hline
hhId
&
indices of columns in origData
with specified cluster variable
\\
\hline
strataVar
&
indices of columns in origData
with specified strata variable
\\
\hline
sensibleVar
&
indices of columns in origData
with specified sensitive
variables for lDiversity
\\
\hline
manipKeyVars
&
manipulated categorical key
variables after applying SDC
methods (cf. keyVars slot)
\\
\hline
manipPramVars
&
manipulated PRAM variables after
applying PRAM (cf. pramVars slot)
\\
\hline
manipNumVars
&
manipulated numerical key
variables after applying SDC
methods (cf. numVars slot)
\\
\hline
manipGhostVars
&
manipulated ghost variables (cf.
ghostVars slot)
\\
\hline
manipStrataVar
&
manipulated strata variables (cf.
strataVar slot)
\\
\hline
originalRisk
&
global and individual risk
measures before anonymization
\\
\hline
risk
&
global and individual risk
measures after applied SDC
methods
\\
\hline
utility
&
utility measures (il1 and eigen)
\\
\hline
pram
&
details on PRAM after applying
PRAM
\\
\hline
localSuppression
&
number of suppression per
variable after local suppression
\\
\hline
options
&
options specified
\\
\hline
additionalResults
&
additional results
\\
\hline
set
&
list of slots currently in use
(for internal use)
\\
\hline
prev
&
information to undo one step with
the undo() function
\\
\hline
deletedVars
&
variables deleted (direct
identifiers)
\\
\hline
\end{tabulary}
\par
\sphinxattableend\end{savenotes}

There are two options to save the results after applying SDC methods:
\begin{itemize}
\item {} 
Overwriting the existing \sphinxstyleemphasis{sdcMicro} object, or

\item {} 
Creating a new \sphinxstyleemphasis{sdcMicro} object. The original object will not be
altered and can be used for comparing results. This is especially
useful for comparing several methods and selecting the best option.

In both cases, the result of any function has to be re-assigned to an
object with the ‘\textless{}-‘ operator. Both methods are illustrated in
Example 7.10.

\end{itemize}

Example 7.10: Saving results of applying SDC methods

\begin{DUlineblock}{0em}
\item[] \sphinxstyleemphasis{\# Applying local suppression and reassigning the results to the same
sdcMicro object}
\item[] sdcInitial \textless{}- \sphinxstylestrong{localSuppression}(sdcInitial)
\end{DUlineblock}

\begin{DUlineblock}{0em}
\item[] \sphinxstyleemphasis{\# Applying local suppression and assigning the results to a new
sdcMicro object}
\item[] sdc1 \textless{}- \sphinxstylestrong{localSuppression}(sdcInitial)
\end{DUlineblock}

If the results are reassigned to the same \sphinxstyleemphasis{sdcMicro} object, it is
possible to undo the last step in the SDC process. This is useful when
changing parameters. The results of the last step, however, are lost
after undoing that step. \sphinxstylestrong{NOTE: The undolast() function can be used to
go only one step back, not several.} The result must also be reassigned
to the same object. This is illustrated in Example 7.11.

Example 7.11: Undo last step in SDC process

\begin{DUlineblock}{0em}
\item[] \sphinxstyleemphasis{\# Undo last step in SDC process}
\item[] sdcInitial \textless{}- \sphinxstylestrong{undolast}(sdcInitial)
\end{DUlineblock}


\section{\sphinxstylestrong{Household structure}}
\label{\detokenize{sdcMicro:household-structure}}
If the data has a hierarchical structure and some variables are measured
on the higher hierarchical level and others on the lower level, the SDC
process should be adapted accordingly (see also Sections 4.9 and 5.5). A
common example in social survey data is datasets with a household
structure. Variables that are measured on the household level are, for
example, household income, type of house and region. Variables measured
on the individual level are, for example, age, education level and
marital status. Some variables are measured on the individual level, but
are nonetheless the same for all household members in almost all
households. These variables should be treated as measured on the
household level from the SDC perspective. An example is the variable
religion for some countries.

The SDC process should be divided into two stages in cases where the
data have a household structure. First, the variables on the higher
(household) level should be anonymized; subsequently, the treated
higher-level variables should be merged with the individual variables
and anonymized jointly. In this section, we explain how to extract
household variables from a file and merge them with the individual
levels variables after treatment in \sphinxstyleemphasis{R}. We illustrate this process with
an example of household and individual-level variables.

These steps are illustrated in Example 7.12. We require both an
individual ID and a household ID in the dataset; if they are lacking,
they must be generated. The individual ID has to be unique for every
individual in the dataset and the household ID has to be unique across
households. The first step is to extract the household variables and
save them in a new dataframe. We specify the variables that are measured
at the household level in the string vector “HHVars” and subtract only
these variables from the dataset. This dataframe will have for each
household the same number of entries as it has household members (e.g.,
if a household has four members, this household will appear four times
in the file). We next apply the function unique() to select only one
record per household. This argument of the unique function is the
household ID, which is the same for all household members, but unique
across households.

Example 7.12: Create a household level file with unique records (remove
duplicates)

\sphinxstyleemphasis{\# Create subset of file with only variables measured at household
level}

\begin{DUlineblock}{0em}
\item[] HHVars \textless{}- \sphinxstylestrong{c}(‘region’, ‘hhincome’)
\item[] fileHH \textless{}- file{[},HHVars{]}
\item[] \sphinxstyleemphasis{\# Remove duplicated rows based on the household ID / only every
household once in fileHH}
\item[] fileHH \textless{}- \sphinxstylestrong{unique}(fileHH, by = \sphinxstylestrong{c}(‘HID’))
\end{DUlineblock}

\begin{DUlineblock}{0em}
\item[] \sphinxstyleemphasis{\# Dimensions of fileHH (number of households)}
\item[] dim(fileHH)
\end{DUlineblock}

After anonymizing the household variables based on the dataframe
“fileHH”, we recombine the anonymized household variables with the
original variables, which are measured on the individual level. We can
extract the individual-level variables from the original dataset using
“INDVars” \textendash{} a string vector with the individual-level variable names.
For extracting the anonymized data from the \sphinxstyleemphasis{sdcMicro} object, we can
use the extractManipData() function from the \sphinxstyleemphasis{sdcMicro} package. Next,
we merge the data using the merge function. The ‘by’ argument in the
merge function specifies the variable used for merging \textendash{} in this case
the household ID, which has the same variable name in both datasets. All
other variables should have different names in both datasets. These
steps are illustrated in Example 7.13.

Example 7.13 Merging anonymized household-level variables with
individual-level variables

\begin{DUlineblock}{0em}
\item[] \sphinxstyleemphasis{\# Extract manipulated household level variables from the SDC object}
\item[] HHmanip \textless{}- \sphinxstylestrong{extractManipData}(sdcHH)
\item[] \sphinxstyleemphasis{\# Create subset of file with only variables measured at individual
level}
\item[] fileIND \textless{}- file{[},INDVars{]}
\end{DUlineblock}

\begin{DUlineblock}{0em}
\item[] \sphinxstyleemphasis{\# Merge the file by using the household ID}
\item[] fileCombined \textless{}- \sphinxstylestrong{merge}(HHmanip, fileIND, by = \sphinxstylestrong{c}(‘HID’))
\end{DUlineblock}

The file \sphinxstyleemphasis{fileCombined} is used for the SDC process with the entire
dataset. How to deal with data with household structure is illustrated
in the case studies in Chapter 9.

As discussed in Section 5.5, the size of a household can also be a
quasi-identifier, even if the household size is not included in the
dataset as variable. For the purpose of evaluating the disclosure risk,
it might be necessary to create such a variable by a headcount of the
members of each household. Example 7.14 shows how to generate a variable
household size with values for each individual based on the household ID
(HID). Two cases are shown: 1) the file sorted by household ID and 2)
the file not sorted.

Example 7.14 Generating the variable household size

\sphinxstyleemphasis{\# Sorted by HID}

file\$hhsize \textless{}- \sphinxstylestrong{rep}(\sphinxstylestrong{unname}(\sphinxstylestrong{table}(file\$HID)),
\sphinxstylestrong{unname}(\sphinxstylestrong{table}(file\$HID)))

\sphinxstyleemphasis{\# Unsorted}

file\$hhsize \textless{}- \sphinxstylestrong{rep}(\sphinxstylestrong{diff}(c(1, 1 +
\sphinxstylestrong{which}(\sphinxstylestrong{diff}(file\$HID) != 0), \sphinxstylestrong{length}(b)+1)),
\sphinxstylestrong{diff}(c(1, 1 + \sphinxstylestrong{which}(\sphinxstylestrong{diff}(file\$HID) != 0),
\sphinxstylestrong{length}(file\$HID) + 1)))

\sphinxstylestrong{NOTE: In some cases, the order of the individuals within the
households can provide information that could lead to
re-identification.} An example is information on the relation to the
household head. In many countries, the first individual in the household
is the household head, the second the partner of the household head and
the next few are children. Therefore, the line number within the
household could correlate well with a variable that contains information
on the relation to the household head. One way to avoid this unintended
release of information is to change the order of the individuals within
each household at random. Example 7.15 illustrates a way to do this in
\sphinxstyleemphasis{R}.

Example 7.15 Changing the order of individuals within households

\begin{DUlineblock}{0em}
\item[] \sphinxstyleemphasis{\# List of household sizes by household}
\item[] hhsize \textless{}- \sphinxstylestrong{diff}(\sphinxstylestrong{c}(1, 1 + \sphinxstylestrong{which}(\sphinxstylestrong{diff}(file\$HID)
!= 0), \sphinxstylestrong{length}(file\$HID) + 1))
\end{DUlineblock}

\begin{DUlineblock}{0em}
\item[] \sphinxstyleemphasis{\# Line numbers randomly assigned within each household}
\item[] \sphinxstylestrong{set.seed}(123)
\item[] dataAnon\$INDID \textless{}- \sphinxstylestrong{unlist}(\sphinxstylestrong{lapply}(hhsize,
function(n)\{\sphinxstylestrong{sample}(1:n, n, replace = FALSE, prob =
\sphinxstylestrong{rep}(1/n, n))\}))
\end{DUlineblock}

\begin{DUlineblock}{0em}
\item[] \sphinxstyleemphasis{\# Order the file by HID and randomized INDID (line number)}
\item[] dataAnon \textless{}- dataAnon{[}\sphinxstylestrong{order}(dataAnon\$HID, dataAnon\$INDID),{]}
\end{DUlineblock}


\section{\sphinxstylestrong{Randomizing order and numbering of individuals or households}}
\label{\detokenize{sdcMicro:randomizing-order-and-numbering-of-individuals-or-households}}
Often the order and numbering of individuals, households, and also
geographical units contains information that could be used by an
intruder to re-identify records. For example, households with IDs that
are close to one another in the dataset are likely to be geographically
close as well. This is often the case in a census, but also in a
household survey households close to one another in the dataset likely
share the same low level geographical unit if the dataset is sorted in
that way. Another example is a dataset that is alphabetically sorted by
name. Here, removing the direct identifier name before release is not
sufficient to guarantee that the name information cannot be used (e.g.
first record has a name which likely starts with ‘a’). Therefore, it is
often recommended to randomize the order of records in a dataset before
release. Randomization can also be done within subsets of the dataset,
e.g., within regions. If suppressions were made in the geographical
variable used for creating the subsets, randomization within the
geographical subsets implies that the geographical variable is the same
for all records in the subset and the suppressed value can be easily
derived (for instance, in cases where the geographical unit is included
in the randomized ID). Therefore, if the variable used for the subsets
has suppressed values, randomization should be done at the dataset level
and not at the subset level.

Table 7.3 illustrates the need and process of randomizing the order of
records in a dataset. The first three columns in Table 7.3 show the
original dataset. Some suppressions were made in the variable
“district”, as shown in columns 4 to 6 (‘NA’ values). This dataset also
already shows the randomized household IDs. The order of the records in
the columns 1-3 and columns 4-6 is unchanged. By the order of the
records, it is easy to guess the values of the two suppressed values.
Both the record before and after have the same value for district as the
suppressed values, respectively 3 and 5. After reordering the dataset
based on the randomized household IDs, we see that it becomes impossible
to reconstruct the suppressed values based on the values of the
neighboring records. Note that in this example the randomization was
carried out within the regions and the region number is included in the
household ID (first digit).

Table 7.3 Illustration of randomizing order of records in a dataset


\begin{savenotes}\sphinxattablestart
\centering
\begin{tabular}[t]{|*{9}{\X{1}{9}|}}
\hline
\sphinxstyletheadfamily 
Ori
ginal
datas
et
&\sphinxstartmulticolumn{3}%
\begin{varwidth}[t]{\sphinxcolwidth{3}{9}}
\sphinxstyletheadfamily Dataset with randomize
household IDs
\par
\vskip-\baselineskip\vbox{\hbox{\strut}}\end{varwidth}%
\sphinxstopmulticolumn
&\sphinxstartmulticolumn{5}%
\begin{varwidth}[t]{\sphinxcolwidth{5}{9}}
\sphinxstyletheadfamily Dataset for release ordered by the new
randomized household ID
\par
\vskip-\baselineskip\vbox{\hbox{\strut}}\end{varwidth}%
\sphinxstopmulticolumn
\\
\hline\begin{quote}

Hou
\end{quote}

sehol
d
ID**
&\begin{quote}

Reg
\end{quote}

ion
&\begin{quote}

Dis
\end{quote}

trict
&\begin{quote}

Ran
\end{quote}

domiz
ed
house
hold
ID**
&\begin{quote}

Reg
\end{quote}

ion
&\begin{quote}

Dis
\end{quote}

trict
&\begin{quote}

Ran
\end{quote}

domiz
ed
house
hold
ID**
&\begin{quote}

Reg
\end{quote}

ion
&\begin{quote}

Dis
\end{quote}

trict
\\
\hline
101
&
1
&
1
&
108
&
1
&
1
&
101
&
1
&
4
\\
\hline
102
&
1
&
1
&
106
&
1
&
1
&
102
&
1
&
3
\\
\hline
103
&
1
&
2
&
104
&
1
&
2
&
103
&
1
&
5
\\
\hline
104
&
1
&
2
&
112
&
1
&
2
&
104
&
1
&
2
\\
\hline
105
&
1
&
2
&
105
&
1
&
2
&
105
&
1
&
2
\\
\hline
106
&
1
&
3
&
102
&
1
&
3
&
106
&
1
&
1
\\
\hline
107
&
1
&
3
&
109
&
1
&
NA
&
107
&
1
&
3
\\
\hline
108
&
1
&
3
&
107
&
1
&
3
&
108
&
1
&
1
\\
\hline
109
&
1
&
4
&
101
&
1
&
4
&
109
&
1
&
NA
\\
\hline
110
&
1
&
5
&
111
&
1
&
5
&
110
&
1
&
NA
\\
\hline
111
&
1
&
5
&
110
&
1
&
NA
&
111
&
1
&
5
\\
\hline
112
&
1
&
5
&
103
&
1
&
5
&
112
&
1
&
2
\\
\hline
201
&
2
&
6
&
203
&
2
&
6
&
201
&
2
&
6
\\
\hline
202
&
2
&
6
&
204
&
2
&
6
&
202
&
2
&
6
\\
\hline
203
&
2
&
6
&
201
&
2
&
6
&
203
&
2
&
6
\\
\hline
204
&
2
&
6
&
202
&
2
&
6
&
204
&
2
&
6
\\
\hline
\end{tabular}
\par
\sphinxattableend\end{savenotes}

The randomization is easiest if done before or after the anonymization
process with \sphinxstyleemphasis{sdcMicro} and directly on the dataset (data.frame in \sphinxstyleemphasis{R}).
To randomize the order, we need an ID, such as an individual ID,
household ID or geographical ID. If the dataset does not contain such
ID, this should be created first. Example 7.16 shows how to randomize
households. “HID” is the household ID and “regionid” is the region ID.
First the variable “HID” is replaced by a randomized variable
“HIDrandom”. Then the file is sorted by region and the randomized ID and
the actual order of the records in the dataset is changed. To make the
randomization reproducible, it is advisable to set a seed for the random
number generator.

Example 7.16: Randomize order of households

n \textless{}- \sphinxstylestrong{length}(file\$HID) \sphinxstyleemphasis{\# number of households}

\begin{DUlineblock}{0em}
\item[] \sphinxstylestrong{set.seed}(123) \sphinxstyleemphasis{\# set seed}
\item[] file\$HIDrandom \textless{}- \sphinxstylestrong{sample}(1:n, n, replace = FALSE, prob =
\sphinxstylestrong{rep}(1/n, n)) \sphinxstyleemphasis{\# generate random HID}
\end{DUlineblock}

file \textless{}- file1{[}\sphinxstylestrong{order}(file\$regionid, file\$HIDrandom),{]} \sphinxstyleemphasis{\# sort
file by regionid and random HID}

file\$HIDrandom \textless{}- 1:n \sphinxstyleemphasis{\# renumber the households in randomized order to
1-n}


\section{Computation time}
\label{\detokenize{sdcMicro:computation-time}}
Some SDC methods can take a very long time to evaluate in terms of
computation. For instance, local suppression with the function
localSuppression() of the \sphinxstyleemphasis{sdcMicro} package in \sphinxstyleemphasis{R} can take days to
execute on large datasets of more than 30,000 individuals that have many
categorical quasi-identifiers. Our experiments reveal that computation
time is a function of the following factors: the applied SDC method;
data size, i.e., number of observations, number of variables and the
number of categories or factor levels of each categorical variable; data
complexity (e.g., the number of different combinations of values of key
variables in the data); as well as the computer/server specifications.

Table 7.4 gives some indication of computation times for different
methods on datasets of different size and complexity based on findings
from our experiments. The selected quasi-identifiers and categories for
those variables in Table 7.3 are the same in both datasets being
compared. Because it is impossible to predict the exact computation
time, this table should be used to illustrate how long computations may
take. These methods have been executed on a powerful server. Given long
computation times for some methods, it is recommended, where possible,
to first test the SDC methods on a subset or sample of the microdata,
and then choose the appropriate SDC methods. \sphinxstyleemphasis{R} provides functions to
select subsets from a dataset. After setting up the code, it can then be
run on the entire dataset on a powerful computer or server.

Table 7.4: Computation times of different methods on datasets of
different sizes


\begin{savenotes}\sphinxattablestart
\centering
\begin{tabular}[t]{|*{4}{\X{1}{4}|}}
\hline
\sphinxstartmulticolumn{2}%
\begin{varwidth}[t]{\sphinxcolwidth{2}{4}}
\sphinxstyletheadfamily Dataset with 5,000 observations
\par
\vskip-\baselineskip\vbox{\hbox{\strut}}\end{varwidth}%
\sphinxstopmulticolumn
&\sphinxstartmulticolumn{2}%
\begin{varwidth}[t]{\sphinxcolwidth{2}{4}}
\sphinxstyletheadfamily Dataset with 45,000 obervations
\par
\vskip-\baselineskip\vbox{\hbox{\strut}}\end{varwidth}%
\sphinxstopmulticolumn
\\
\hline
Methods
&\begin{quote}

Computation
\end{quote}

time (hours)
&
Methods
&\begin{quote}

Computation
\end{quote}

time (hours)
\\
\hline
Top coding age,
local
suppression
(k=3)
&
11
&
Top coding age,
local
suppression
(k=3)
&
268
\\
\hline
Recoding age,
local
suppression
(k=3)
&
8
&
Recoding age,
local
suppression
(k=3)
&
143
\\
\hline
Recoding age,
local
suppression
(k=5)
&
10
&
Recoding age,
local
suppression
(k=5)
&
156
\\
\hline
\end{tabular}
\par
\sphinxattableend\end{savenotes}

The number of categories and the product of the number of categories of
all categorical quasi-identifiers give an idea of the number of
potential combinations (keys). This is only an indication of the actual
number of combinations, which influences the computation time to
compute, for example, the frequencies of each key in the dataset. If
there are many categories but not so many combinations (e.g., when the
variables correlate), the computation time will be shorter.

Table 7.5 shows the number of categories for seven datasets with the
same variables but of different complexities that were all processed
using the same script on 16 processors, in order of execution time. The
table also shows an approximation of the number of unique combinations
of quasi-identifiers, as indicated by the percentage of observations
violating \(k\)-anonymity in each dataset pre-anonymization in
relation to processing time. The results in the table clearly indicate
that both the number of observations (i.e., sample size) and the
complexity of the data play a role in the execution time. Also, using
the same script (and hence anonymization methods), the execution time
can vary greatly; the longest running time is about 10 times longer than
the shortest. Computer specifications also influence the computation
time. This includes the processor, RAM and storage media.

Table 7.5: Number of categories (complexity), record uniqueness and
computation times


\begin{savenotes}\sphinxattablestart
\centering
\begin{tabular}[t]{|*{10}{\X{1}{10}|}}
\hline
\sphinxstyletheadfamily \begin{quote}

Sa
\end{quote}

mple
Size
&\sphinxstartmulticolumn{6}%
\begin{varwidth}[t]{\sphinxcolwidth{6}{10}}
\sphinxstyletheadfamily \begin{quote}

Nu
\end{quote}

mber
of
cate
gori
es
per
quas
i-id
enti
fier
(com
plex
ity)
\par
\vskip-\baselineskip\vbox{\hbox{\strut}}\end{varwidth}%
\sphinxstopmulticolumn
&\sphinxstyletheadfamily \begin{quote}

Pe
\end{quote}

rcen
tage
of
obse
rvat
ions
viol
atin
g
k
anon
ymit
y
befo
re
anon
ymiz
atio
n
&\sphinxstartmulticolumn{2}%
\begin{varwidth}[t]{\sphinxcolwidth{2}{10}}
\sphinxstyletheadfamily Execution
time
in hours
\par
\vskip-\baselineskip\vbox{\hbox{\strut}}\end{varwidth}%
\sphinxstopmulticolumn
\\
\hline&
Wate
r
&
Toil
et
&
Occu
pati
on
&
Reli
gion
&
Ethn
icit
y
&
Regi
on
&
k3
&
k5
&\\
\hline
20,0
14
&
10
&
4
&
70
&
5
&
7
&
6
&
74
&
88
&
53.7
2
\\
\hline
66,2
85
&
15
&
6
&
39
&
4
&
0
&
24
&
40
&
49
&
67.1
9
\\
\hline
60,7
47
&
13
&
6
&
70
&
8
&
9
&
4
&
35
&
45
&
74.4
7
\\
\hline
26,6
01
&
19
&
6
&
84
&
10
&
10
&
10
&
77
&
87
&
108.
84
\\
\hline
38,0
89
&
17
&
6
&
30
&
5
&
56
&
9
&
70
&
81
&
198.
90
\\
\hline
35,8
20
&
19
&
7
&
67
&
6
&
NA
&
6
&
81
&
90
&
267.
60
\\
\hline
51,9
76
&
12
&
6
&
32
&
8
&
50
&
12
&
77
&
87
&
503.
58
\\
\hline
\end{tabular}
\par
\sphinxattableend\end{savenotes}

The large-scale experiment executed for this guide utilized 75 microdata
files from 52 countries, using surveys on topics including health,
labor, income and expenditure. By applying anonymization methods
available in the \sphinxstyleemphasis{sdcMicro} package, at least 20 different anonymization
scenarios %
\begin{footnote}[9]\sphinxAtStartFootnote
Here a scenario refers to a combination of SDC methods and their
parameters.
%
\end{footnote} were tested on each dataset. Most of the
processing was done using a powerful server %
\begin{footnote}[10]\sphinxAtStartFootnote
The server has 512 GB RAM and four processors each with 16 cores,
translating to 64 cores total.
%
\end{footnote} and up
to 16 \textendash{} 20 processors (cores) at a time. Other processing platforms
included a laptop and desktop computers, each using four processors.
Computation times were significantly shorter for datasets processed on
the server, compared to those processed on the laptop and desktop.

The use of parallelization can improve performance even on a single
computer with one processor with multiple cores. Since \sphinxstyleemphasis{R} does not use
multiple cores unless instructed to do so, our anonymization programs
allowed for parallelization such that jobs/scenarios in each dataset
could be processed simultaneously through efficient allocation of tasks
to different processors. Without parallelization, depending on the
server/computer, only one core is used when running the jobs
sequentially. Running the anonymization program without parallelization
leads to significantly longer execution time. Note however, that the
parallelization itself also causes overhead. Therefore, a summation of
the times it takes to run each task in parallel does not necessarily
amount to the time it may take to run them sequentially. The fact that
the RAM is shared might, however, slightly reduce the gains of
parallelization. If you want to compare the results of different methods
on large datasets that require long computation times, using parallel
computing can be a solution.{[}\#foot70{]}\_

Appendix D zooms in on seven selected datasets from a health survey that
were processed using the same parallelization program and anonymization
methods. Note that the computation times in the appendix are only meant
to create awareness for expected computation time, and may vary based on
the type of computer used. In our case, although all datasets were
anonymized using the parallelization program, computation times were
significantly shorter for datasets processed on the server, compared to
those processed on the laptop and desktop. Among those datasets
processed on the server using the same number of processors (datasets 1,
2 and 6), some variation also exists in the computation times. \sphinxstylestrong{NOTE:
Computation time in the table in} Appendix \sphinxstylestrong{D includes recalculating
the risk after applying the anonymization methods, which is
automatically done in sdcMicro when using standard methods/functions.}
Using the function groupVars(), for instance, is not computationally
intensive but can still take a long time if the dataset is large and
risk measures have to be recalculated.


\section{Common errors}
\label{\detokenize{sdcMicro:common-errors}}
In this section, we present a few common errors and their causes, which
might be encountered when using the \sphinxstyleemphasis{sdcMicro} package in \sphinxstyleemphasis{R} for
anonymization of microdata:
\begin{itemize}
\item {} 
The class of a certain variable is not accepted by the function,
e.g., a categorical variable of class numeric should be first recoded
to the required class (e.g., factor or data.frame). Section 7.4 shows
how to do this.

\item {} 
After manually making changes to variables the risk did not change,
since it is not updated automatically and has to be manually
recomputed by using the function calcRisks().

\end{itemize}


\chapter{The SDC Process}
\label{\detokenize{process:the-sdc-process}}\label{\detokenize{process::doc}}
This section presents the SDC process in a step-by-step representation
and can be used as guidance for the actual SDC process. It should be
noted, however, that jumping between steps and returning to previous
steps is often required during the actual SDC process, as it is not
necessarily a linear step-by-step process. This guidance brings together
the different parts of the SDC process as discussed in the previous
chapters and links to these chapters. The case studies in the next
chapter follow these steps. This presentation is adapted from Hundepool
et al. (2012). Figure 8.1 at the end of this chapter presents the entire
process in a schematic way.


\section{Step 1: Need for confidentiality protection}
\label{\detokenize{process:step-1-need-for-confidentiality-protection}}
Before starting the SDC process for a microdata set, the need for
confidentiality protection has to be determined. This is closely linked
to the interpretation of laws and regulations on this topic from the
country in which the data originates and thus country-specific. A first
step is to determine the statistical units in the dataset: if these are
individuals, households or legal entities, such as companies, a need for
disclosure control is likely. There are also examples of microdata for
which there is no need for disclosure control. Examples could be data
with climate and weather observations or data with houses as statistical
units. Even if the primary statistical units are not natural or legal
persons, however, the data can still contain confidential information on
natural or legal persons. For example, a dataset with houses as primary
statistical units can also contain information on the persons living in
these houses and their income or a dataset on hospitalizations can
include information about the hospitalized patients. In these cases,
there is likely still a need for confidentiality protection. One option
to solve this is to remove the information on the natural and legal
persons in the datasets for release.

One dataset can also contain more than one type of statistical unit. The
standard example here is a dataset containing both information on
individuals and households. Another example is data with employees in
enterprises. All types of statistical units present in the dataset have
to be considered for the need of SDC. This is especially important in
case the data has a hierarchical structure, such as individuals in
households or employees in enterprises.

In addition, one has to evaluate whether the variables contained in the
dataset are confidential or sensitive. Which variables are sensitive or
confidential depends again on the applicable legislation and can differ
substantially from country to country. In case the dataset includes
sensitive or confidential variables, SDC is likely required. The set of
sensitive variables and confidential variables together with the
statistical units in the dataset determine the need for statistical
disclosure control.


\section{Step 2: Data preparation and exploring data characteristics}
\label{\detokenize{process:step-2-data-preparation-and-exploring-data-characteristics}}
After assessing the need for statistical disclosure control, we should
prepare the data and, if there are multiple, combine and consider all
related data files. Then we explore the characteristics and structure in
the data, which are important for the users of the data. Compiling an
inventory of these characteristics is important for assessing the
utility of the data after anonymization and producing an anonymized
dataset, which is useful for end users.

The first step in data preparation is classifying the variables as
sensitive or non-sensitive, and removing direct identifiers such as full
names, passport numbers, addresses, phone numbers and GPS coordinates.
In case of survey data, an inspection of the survey questionnaire is
useful to classify the variables. Furthermore, it is necessary to select
the variables that contain relevant information for end users and should
be included in the dataset for release. At this point, it can also be
useful to remove variables other than direct identifiers from the
microdata set to be released. An example can be a variable with many
missing values, e.g., a variable recorded only for a select group of
individuals eligible for a particular survey module, and missing values
for the rest. Such variables can cause a high level of disclosure risk
while adding little information for end users. Examples are variables
relating to education (current grade), where a missing value indicates
that the individual is not currently in school, or variables relating to
childbirth, where a missing value indicates that the individual has not
delivered a child in the reference period. Missing values in themselves
can be disclosive, especially if they indicate that the variable is not
applicable. Often variables with the majority of values missing are
deleted at this stage already. Other variables that might be deleted at
this stage are those too sensitive to be anonymized and released or
those not important to data users and that could increase the risk of
disclosure.

Relationships may exist among variables in a dataset for a variety of
reasons. For instance, variables can be mutually exclusive in cases
where several binary variables are used for each category. An individual
not in the labor force will have a missing value for the sector in which
this person is employed (or more precisely not applicable).
Relationships may also exist if some variables are ratios, sums or other
mathematical functions of other variables. Examples are the variable
household size (as a count of individuals per household), or aggregate
expenditure (as a sum of all expenditure components). A certain value in
one variable may also reduce the number of possible or valid values for
another variable; for example, the age of an individual attending
primary education or the gender of an individual having delivered a
child. These relationships are important for two reasons: 1) they can be
used by intruders to reconstruct anonymized values. For example, if age
is suppressed but another variable indicates that they are in school,
then it is still possible to infer a likely age range for that
individual. Another example is if an individual is shown to be active in
Sector B of the economy. Even if the labor status of this individual is
suppressed, it can be inferred that this person is employed. 2) the
relationships in the original data should be maintained in the
anonymized dataset and inconsistencies should be avoided (e.g., SDC
methods should not create 58-year-old school boys, or married
3-year-olds), or the dataset will be invalid for analysis. Another
example is the case of expenditures per category, where it is important
that the sum of the categories adds up to the total. One way to
guarantee this is to anonymize the totals and then recalculate the
sub-categories according to the original shares of the anonymized
totals.

It is also useful at this stage to consolidate variables that provide
the same information where possible, so as reduce the number of
variables, reduce the likelihood of inconsistencies and minimize the
variables an intruder can use to reconstruct the data. This is
especially true if the microdata stems from an elaborate questionnaire
and each variable represents one (sub-) question leading to a dataset
with hundreds of variables. As an example, we take a survey with several
labor force variables indicating whether a person is in the labor force,
employed or unemployed, and if employed, in what sector. The data in
Table 8.1 illustrates this example. It is possible that each type of
sector has its own binary variable. In that case, this set of variables
can be summarized in two variables: one variable indicating whether a
person is in labor force and another indicating the employment status,
as well as the respective sector if a person is employed. These two
variables contain all information contained in the previous five
variables and make the anonymization process easier. If data users are
used to a certain release format where including all five variables has
been the norm, then it is possible to transform the variables back after
the anonymization process rather than complicating the anonymization
process by trying to treat more variables than is necessary. This
approach also guarantees that the relationships between the variables
are preserved (e.g., no individuals will be employed in several
sectors).

Table 8.1: Illustration of merging variables without information loss
for SDC process


\begin{savenotes}\sphinxattablestart
\centering
\begin{tabulary}{\linewidth}[t]{|T|T|T|T|T|T|T|}
\hline
\sphinxstartmulticolumn{5}%
\begin{varwidth}[t]{\sphinxcolwidth{5}{7}}
\sphinxstyletheadfamily Before
\par
\vskip-\baselineskip\vbox{\hbox{\strut}}\end{varwidth}%
\sphinxstopmulticolumn
&\sphinxstartmulticolumn{2}%
\begin{varwidth}[t]{\sphinxcolwidth{2}{7}}
\sphinxstyletheadfamily After
\par
\vskip-\baselineskip\vbox{\hbox{\strut}}\end{varwidth}%
\sphinxstopmulticolumn
\\
\hline
In
labor
force
&
Employed
&
Sector A
&
Sector B
&
Sector C
&
In
labor
force
&
Emplo
yed
\\
\hline
Yes
&
Yes
&
Missing
&
Yes
&
Missing
&
Yes
&
B
\\
\hline
No
&
No
&
Missing
&
Missing
&
Missing
&
No
&
No
\\
\hline
Yes
&
Yes
&
Yes
&
Missing
&
Missing
&
Yes
&
A
\\
\hline
Yes
&
Yes
&
Missing
&
Yes
&
Missing
&
Yes
&
B
\\
\hline
Yes
&
Yes
&
Missing
&
Missing
&
Yes
&
Yes
&
C
\\
\hline
Yes
&
No
&
Missing
&
Missing
&
Missing
&
Yes
&
No
\\
\hline
\end{tabulary}
\par
\sphinxattableend\end{savenotes}

Besides relationships between variables, we also gather information
about the survey methodology, such as strata, sampling methods, survey
design and sample weights. This information is important in later
stages, when estimating the disclosure risk and choosing the SDC
methods. It is important to distinguish between a full census and a
sample. For a full census, it is common practice to publish only a
sample, as the risk of disclosure for a full sample is too high, given
that we know that everyone in the country or institution is in the data
(see also Section 5.6). Strata and sample weights can disclose
information about the area or group to which an individual belongs
(e.g., the weights can be linked with the geographical area or specific
group in case of stratified sampling); this should be taken into account
in the SDC process and checked before release of the dataset.


\section{Step 3: Type of release}
\label{\detokenize{process:step-3-type-of-release}}
The type of release is an important factor for determining the required
level of anonymization as well as the requirements end users have for
the data (e.g., researchers need more detail than students for whom a
teaching file might be sufficient) and should be clarified before the
start of the anonymization process. Data release or dissemination by
statistical agencies and data producers is often guided by the
applicable law and dissemination strategies of the statistical agency,
which specify the type of data that should be disseminated as well as
the fashion.

Generally, there exist three types of data release methods for different
target groups (Chapter 3 provides more information on different release
types):
\begin{itemize}
\item {} 
PUF: The data is directly available to anyone interested, e.g., on
the website of the statistical agency

\item {} 
SUF: The data is available to accredited researchers, who have to
file their research proposals beforehand and have to sign a contract;
this is also known as licensed file or microdata under contract

\item {} 
Available in a controlled research data center: only on-site access
to data on special computers; this is also known as data enclave

\end{itemize}

There are other data access possibilities besides these, such as
teaching files or files for other specific purposes. Obviously, the
required level of protection depends on the type of release: a PUF file
must be protected to a greater extent than a SUF file, which in turn has
to be protected more than a file which is available only in an on-site
facility, since the options the intruder can use the data are limited in
the latter case.

Besides the applicable legislation, the choice of the type of release
depends on the type of the data and the content. \sphinxstylestrong{NOTE: Not every
microdata set is suitable for release in any release type, even after
SDC.} Some data cannot be protected sufficiently \textendash{} it might always
contain information that is too sensitive to be published as SUF or PUF.
In such cases, the data can be released in on-site facilities, or the
number of variables can be reduced by removing problematic variables.

Generally, the release of two or more anonymized datasets, e.g.,
tailored for different end users from the same original, is problematic
because it can lead to disclosure if the two were later obtained and
merged by the same user. The information contained in one dataset that
is not contained in the other can lead to unintended disclosure. An
exception is the simultaneous release and anonymization of a microdata
set as PUF and SUF files. In this case, the PUF file is constructed from
the SUF file by further anonymization. In that way, all information in
the PUF file is also contained in the SUF file and the PUF file does not
provide any additional information for users that have access to the SUF
file.

\sphinxstylestrong{NOTE: The anonymization process is an iterative process where steps
can be revisited, whereas the publication of an anonymized dataset is a
one-shot process.} Once the anonymized data is published, it is not
possible to revoke and publish another dataset of the same microdata
file. This would in fact mean publishing more than one anonymized file
from the same microdata set, since some users might have saved the
previous file.


\section{Step 4: Intruder scenarios and choice of key variables}
\label{\detokenize{process:step-4-intruder-scenarios-and-choice-of-key-variables}}
After determining the release type of the data, the possibilities of how
an individual in the microdata could (realistically) be identified by an
intruder under that release type should be examined. For PUF and SUF
release the focus is on the use of external datasets from various
source. These possibilities are described in disclosure or intruder
scenarios, which specify what data an intruder could possibly have
access to and how this auxiliary data can be used for identity
disclosure. This leads to the specification of quasi-identifiers, which
are a set of variables that are available both in the dataset to be
released and in auxiliary datasets and need protection. \sphinxstylestrong{NOTE: If the
number of quasi-identifiers is high, it is recommended to reduce the set
of quasi-identifiers by removing some variables from the dataset for
release.} This is especially true for PUF releases. Disclosure
scenarios can also help define the required level of anonymization.

Drafting disclosure scenarios requires the support of subject matter
specialists, assuming the subject specialist is not the same as the
person doing the anonymization. Auxiliary datasets may contain
information on the identity of the individuals and allow identity
disclosure. Examples of such auxiliary data files are population
registers and electoral rolls, as well as data collected by specialized
firms. \sphinxstylestrong{NOTE: External datasets can come from many sources (e.g., other
institutions, private companies) and it is sometimes difficult to make a
full list of external data sources.} In addition, not all external data
sources are in the public domain. Nevertheless, proprietary data can be
used by the owner to re-identify individuals and should be taken into
account in the SDC process, even if the exact content is not known.
Also, the variables or datasets may not coincide perfectly (e.g.,
different definitions, more or less detailed variables, different survey
period). Nevertheless, they should be considered in the SDC process.

Disclosure scenarios include both identity and inferential disclosure.
The disclosure depends on the type of release, i.e., different data
users have different data available and may use the data in a different
way for re-identification. For example, a user in a research data center
cannot match with large external datasets as (s)he is not permitted to
take these into the data center. A user of a SUF is bound by an
agreement specifying the use of the data and consequences if the
agreement is breached (see Chapter 3). Furthermore, it should be
evaluated whether, in case of a sample, possible intruders have
knowledge as to which individuals are in the sample. This can be the
case if it is known which schools were visited by the survey team, for
example. A few examples of disclosure scenarios are (see Section 4.3 for
more information):
\begin{itemize}
\item {} 
Matching: The intruder uses auxiliary data, e.g., data on region,
marital status and age from a population register, and matches them
to released microdata. Individuals from the two datasets that match
and are unique are successfully identified. This principle is used as
an assumption in several disclosure risk measures, such as
\(k\)-anonymity, individual and global risk, as described in
Chapter 4. This scenario can apply to both PUFs and SUFs.

\item {} 
Spontaneous recognition: This scenario should be considered for SUF
files, but is especially important for data available in research
data centers where outliers are present in the data and data is often
not strongly anonymized. The researcher might (unintentionally)
recognize some individuals he knows (e.g., his colleagues, neighbors,
family members, public figures, famous persons or large companies),
while inspecting the data. This is especially true for rare
combinations of values, such as outliers or unlikely combinations.

\end{itemize}


\section{Step 5: Data key uses and selection of utility measures}
\label{\detokenize{process:step-5-data-key-uses-and-selection-of-utility-measures}}
In this step, we analyze the main uses of the data by the end users of
the released microdata file. The data should be useful for the type of
statistical analysis for which the data was collected and for which it
is mostly used. The uses and requirements of data users will be
different for different release types. Contacting data users directly or
searching for scientific studies and papers that use similar data can be
useful when collecting this information and making this assessment.
Alternatively, this information can be collected from research proposals
by researchers when applying for microdata access (SUF) or user groups
can be set up. Furthermore, it is important to understand what level of
precision the data users need and what types of categories are used. For
instance, in the case of global recoding of age in years, one could
recode age in groups of 10 years, e.g., 0 \textendash{} 9, 10 \textendash{} 19, 20 \textendash{} 29, … Many
indicators relating to the labor market use categories that span the
range 15 \textendash{} 65, however. Therefore, constructing categories that coincide
with the categories used for the indicators keeps the data much more
useful while at the same time reducing the risk of disclosure in a
similar way. This knowledge is important for the selection of useful
utility measures, which in turn are used for selecting appropriate SDC
methods.

The uses of the data depend on the release type, too. Researchers using
SUF files require a higher level of detail in the data than PUF users.
\sphinxstylestrong{NOTE: Anonymization will always lead to information loss and a PUF
file will have reduced utility. If certain users require a high level of
detail, release types other than PUF should be considered, such as SUF
or release through a research data center.} In the case of SUFs, it is
easier to find the main uses of the data since access is documented. One
way to obtain information on the use of PUF files is to ask for a short
description of intended use of the data before supplying the data. This
is, however, useful only if microdata has been released previously.

Statistics computed from the anonymized and released microdata file
should produce analytical results that agree or almost agree with
previously published statistics from the original data. If, for
instance, a previously published primary school enrollment rate was
computed from these data and published, the released anonymized dataset
should produce a very similar result to the officially published result.
At the very least, the result should fall within the confidence region
of the published result. It might be the case that not all published
statistics can be generated from the published data. If this is the
case, a choice should be made on which indicators and statistics to
focus, and inform the users as to which ones have been selected and why.

As discussed in Chapter 6, it is necessary to compute general utility
measures that compare the raw and anonymized data, taking into
consideration the end user’s need for their analysis. In some cases the
utility measures can give contradicting results, for example, a certain
SDC method might lead to lower information loss for labor force figures
but greater information loss for ratios relating to education. In such
cases, the data uses might need to be ranked in order of importance and
it should be clearly documented for the user that the prioritization of
certain metrics over others means that certain metrics are no longer
valid. This may be necessary, as it is not possible to release multiple
files for different users. This problem occurs especially in
multi-purpose studies. For more details on utility measures, refer to
Chapter 6.

\sphinxstylestrong{Note on Steps 6 to 10}

The following Steps 6 through 10 should be repeated if the data has
quasi-identifiers that are on different hierarchical levels, e.g.,
individual and household. In that case, variables on the higher
hierarchical level should be anonymized first, and then merged with the
lower-level untreated variables. Subsequently, the merged dataset should
be anonymized. This approach guarantees consistency in the treated data.
If we neglect this procedure, the values of variables measured on the
higher hierarchical level could be treated differently for observations
of the same unit. For instance, the variable “region” is the same for
all household members. If the value ‘rural’ would be suppressed for two
members but not for the remaining three, this would lead to unintended
disclosure; with the household ID the variable region would be easy to
reconstruct for the two suppressed values. Sections 4.9 and 7.6 provide
more details on how to deal with data with household structure in \sphinxstyleemphasis{R}
and \sphinxstyleemphasis{sdcMicro}.


\section{Step 6: Assessing disclosure risk}
\label{\detokenize{process:step-6-assessing-disclosure-risk}}
The next step is to evaluate the disclosure risk of the raw data. Here
it is important to distinguish between sample data and census data. In
the case of census data, it is possible to directly calculate the risk
measures when assuming that the dataset covers the entire population. If
working with a sample, or a sample of the census (which is the more
common case when releasing sample data), we can use the models discussed
in Chapter 4 to estimate the risk in the population. The main inputs for
the risk measurement are the set of quasi-identifiers determined from
the disclosure scenarios in Step 4 and the thresholds for risk
calculations (e.g., the level of \(k\)-anonymity or the threshold
for which an individual is considered at risk). If the data has a
hierarchical structure (e.g., a household structure), the risk should be
measured taking into account this structure as described Section 4.9.

The different risk measures described in Chapter 4 each have advantages
and disadvantages. Generally,\(\text{\ k}\)-anonymity, individual
risk and global risk are used to produce an idea of the disclosure risk.
These values can initially be very high but can often very easily be
reduced after some simple but appropriate recoding (see Step 8). The
thresholds shall be determined according to the release type. Always
remember, though, that when using a sample, the risk measures based on
the models presented in the literature offer a worst-case risk scenario
and might therefore be an exaggeration of the real risks for some cases
(see Section 4.5).


\section{Step 7: Assessing utility measures}
\label{\detokenize{process:step-7-assessing-utility-measures}}
To quantify the information loss due to the anonymization, we first
compute the utility measures selected in Step 6 using the raw data. This
creates a base for comparison of results obtained when using the
anonymized data \textendash{} i.e., in Step 10. \sphinxstylestrong{NOTE: If the raw data is a sample,
the utility measures are an estimate with a variance and therefore it is
useful to construct confidence intervals in addition to the point
estimates for the utility measures.}


\section{Step 8: Choice and application of SDC methods}
\label{\detokenize{process:step-8-choice-and-application-of-sdc-methods}}
The choice of SDC methods depends on the need for data protection (as
measured by the disclosure risk), the structure of the data and the type
of variables. The influence of different methods on the characteristics
of the data important for the users or the data utility should also be
taken into account when selecting the SDC methods. In practice, the
choice of SDC methods is partially a trial-and-error process: after
applying a chosen method, disclosure risk and data utility are measured
and compared to other choices of methods and parameters. The choice of
methods is bound by legislation on the one hand, and a trade-off between
utility and risk on the other.

The classification of methods as presented in Table 5.1 gives a good
overview for choosing the appropriate methods. Methods should be chosen
according to the type of variable \textendash{} continuous or categorical \textendash{} the
requirements by the users and the type of release. The anonymization of
datasets with both continuous and categorical variables is discussed in
Section 4.2.

In general for anonymization of categorical variables, it is useful to
restrict the number of suppressions by first applying global recoding
and/or removing variables from the microdata set. When the required
number of suppressions to achieve the required level of risk is
sufficiently low, the few individuals at risk can be treated by
suppression. These are generally outliers. It should be noted that
possibly not all variables can be released and some must be removed from
the dataset (see Step 2). Recoding and minimal use of suppression
ensures that already published figures from the raw data can be
reproduced sufficiently well from the anonymized data. If suppression is
applied without sufficient recoding, the number of suppressions can be
very high and the structure of the data can change significantly. This
is because suppression mainly affects combinations that are rare in the
data.

If the results of recoding and suppression do not achieve the required
result, especially in cases where the number of select quasi-identifiers
is high, an alternative is using perturbative methods. These can be used
without prior recoding of variables. These methods, however, preserve
data structure only partially. The preferred method depends on the
requirements of the users. We refer to Chapter 5 and especially Section
5.3 for a discussion of perturbative methods implemented in \sphinxstyleemphasis{sdcMicro}.

Finally, the choice of SDC methods depends on the data used since the
same methods produce different results on different datasets. Therefore,
the comparison of results with respect to risk and utility (Steps 9 and
10) is key to the choice made. Most methods are implemented in the
\sphinxstyleemphasis{sdcMicro} package. Nevertheless, it is sometimes useful to use
custom-made solutions. A few examples are presented in Chapter 5.


\section{Step 9: Re-measure risk}
\label{\detokenize{process:step-9-re-measure-risk}}
In this step, we re-evaluate the disclosure risk with the risk measures
chosen under Step 6 after applying SDC methods. Besides these risk
measures, it is also important to look at individuals with high risk
and/or special characteristics, combinations of values or outliers in
the data. If the risk is not at an acceptable level, Steps 6 to 10
should be repeated with different methods and/or parameters. \sphinxstylestrong{NOTE:
Risk measures based on frequency counts
(}\(\mathbf{k}\)\sphinxstylestrong{-anonymity, individual risk, global risk and
household risk) cannot be used after applying perturbative methods since
their risk estimates are not valid.} These methods are based on
introducing uncertainty into the dataset and not on increasing the
frequencies of keys in the data and will hence overestimate the risk.


\section{Step 10: Re-measure utility}
\label{\detokenize{process:step-10-re-measure-utility}}
In this step, we re-measure the utility measures from Step 7 and compare
these with the results from the raw data. Also, it is useful here to
construct confidence intervals around the point estimates and compare
these confidence intervals. The importance of the absolute value of a
deviation can only be interpreted knowing the variance of the estimate.
Besides these specific utility measures, general utility measures, as
discussed in Section 6.1, should be evaluated. This is especially
important if perturbative methods have been applied. If the data does
not meet the user requirements and deviations are too large, repeat
Steps 6 to 10 with different methods and/or different parameters.
\sphinxstylestrong{NOTE: Anonymization will always lead to at least some information
loss.}


\section{Step 11: Audit and Reporting}
\label{\detokenize{process:step-11-audit-and-reporting}}
After anonymization, it is important to check whether all relationships
in the data as identified in Step 2, such as variables that are sums of
other variables or ratios, are preserved. Also, any unusual values
caused by the anonymization should be detected. Examples of such
anomalies are negative income or a pupil in the twentieth grade of
school. This can happen after applying perturbative SDC methods.
Furthermore, it is necessary to check whether previously published
indicators from the raw data are reproducible from the data to be
released. If this is not the case, data users might question the
credibility of the anonymized dataset.

An important step in the SDC process is reporting, both internal and
external. Internal reporting contains the exact description of
anonymization methods used, parameters but also the risk measures before
and after anonymization. This allows replication of the anonymized
dataset and is important for supervisory authorities/bodies to ensure
the anonymization process is sufficient to guarantee anonymity according
to the applicable legislation.

External reporting informs the user that the data has been anonymized,
provides information for valid analysis on the data and explains the
limitations to the data as a result of the anonymization. A brief
description of the methods used can be included. The release of
anonymized microdata should be accompanied by the usual metadata of the
survey (survey weight, strata, survey methodology) as well as
information on the anonymization methods that allow researchers to do
valid analysis (e.g., amount of noise added, transition matrix for
PRAM). \sphinxstylestrong{NOTE: Care should be taken that this information cannot be used
for re-identification (e.g., no release of random seed for PRAM).} The
metadata must be updated to comply with the anonymized data. Variable
descriptions or value labels might have changed as a result of the
anonymization process. In addition, the information loss due to the
anonymization process should be explained in detail to the users to make
them aware of the limits to the validity of the data and their analyses.


\section{Step 12: Data release}
\label{\detokenize{process:step-12-data-release}}
The last step in the SDC process is the actual release of the anonymized
data. This step depends on the type of release chosen in Step 3. Changes
to the variables made under Step 2, such a merging variables, can be
undone to generate a dataset useful for users.

\begin{sphinxadmonition}{note}{Recommended Reading Material on Risk Measurement}

Dupriez, Olivier, and Ernie Boyko. 2010. \sphinxstyleemphasis{Dissemination of Microdata
Files; Principles, Procedures and Practices.} IHSN Working Paper No.
005, International Household Survey Network (IHSN).
\sphinxurl{http://www.ihsn.org/HOME/sites/default/files/resources/IHSN-WP005.pdf}
\end{sphinxadmonition}

\noindent\sphinxincludegraphics[width=6.55208in,height=8.60417in]{{image20}.png}

Figure 8.1: Overview of the SDC process


\chapter{Appendices}
\label{\detokenize{appendices:appendices}}\label{\detokenize{appendices::doc}}

\section{\sphinxstylestrong{Appendix} \sphinxstylestrong{A: Overview of Case Study Variables}}
\label{\detokenize{appendices:appendix-a-overview-of-case-study-variables}}

\begin{savenotes}\sphinxatlongtablestart\begin{longtable}{|l|l|l|l|}
\hline
\sphinxstyletheadfamily &\sphinxstyletheadfamily 
\sphinxstylestrong{Variable}
&\sphinxstyletheadfamily 
\sphinxstylestrong{Description}
&\sphinxstyletheadfamily 
\sphinxstylestrong{Type}
\\
\hline
\endfirsthead

\multicolumn{4}{c}%
{\makebox[0pt]{\sphinxtablecontinued{\tablename\ \thetable{} -- continued from previous page}}}\\
\hline
\sphinxstyletheadfamily &\sphinxstyletheadfamily 
\sphinxstylestrong{Variable}
&\sphinxstyletheadfamily 
\sphinxstylestrong{Description}
&\sphinxstyletheadfamily 
\sphinxstylestrong{Type}
\\
\hline
\endhead

\hline
\multicolumn{4}{r}{\makebox[0pt][r]{\sphinxtablecontinued{Continued on next page}}}\\
\endfoot

\endlastfoot

1
&
REGION
&
Region
&
HH
\\
\hline
2
&
DIST
&
District
&
HH
\\
\hline
3
&
URBRUR
&
Area of
residence
&
HH
\\
\hline
4
&
WGTHH
&
Individual
weighting
coefficient
(Country-specif
ic.
Weighting
co-efficient to
derive
individual-leve
l
indicators.)
&
HH
\\
\hline
5
&
WGTPOP
&
Population
weighting
coefficient
(Weighting
co-efficient
to derive
population-leve
l
indicators.)
&
HH
\\
\hline
6
&
IDH
&
Household
unique
identification
&
HH
\\
\hline
7
&
IDP
&
Individual
identification
&
HH
\\
\hline
8
&
HHSIZE
&
Household
members
&
HH
\\
\hline
9
&
GENDER
&
Sex
&
IND
\\
\hline
10
&
REL
&
Relationship to
household head
&
IND
\\
\hline
11
&
MARITAL
&
Marital status
&
IND
\\
\hline
12
&
AGEYRS
&
Age in
completed years
&
IND
\\
\hline
13
&
AGEMTH
&
Age of child in
completed
months
&
IND
\\
\hline
14
&
RELIG
&
Religion of
household head
&
HH
\\
\hline
15
&
ETHNICITY
&
Ethnicity
&
IND
\\
\hline
16
&
LANGUAGE
&
Language
&
IND
\\
\hline
17
&
MORBID
&
Morbidity last
x weeks
&
IND
\\
\hline
18
&
MEASLES
&
Child immunized
against Measles
&
IND
\\
\hline
19
&
MEDATT
&
Sought medical
attention
&
IND
\\
\hline
20
&
CHWEIGHTKG
&
Weight of the
child (Kg)
&
IND
\\
\hline
21
&
CHHEIGHTCM
&
Height of the
child (cms)
&
IND
\\
\hline
22
&
ATSCHOOL
&
Current school
enrolment
&
IND
\\
\hline
23
&
EDUCY
&
Highest level
of education
completed
&
IND
\\
\hline
24
&
EDYRS
&
Years of
education
&
IND
\\
\hline
25
&
EDYRSCURRAT
&
Years of
education for
currently
enrolled
&
IND
\\
\hline
26
&
SCHTYP
&
Type of school
attending
&
IND
\\
\hline
27
&
LITERACY
&
Literacy status
&
IND
\\
\hline
28
&
EMPTYP1
&
Type of
employment,
Primary job
&
IND
\\
\hline
29
&
UNEMP1
&
Unemployed
&
IND
\\
\hline
30
&
INDUSTRY1
&
1 digit
industry
classification,
Primary job
&
IND
\\
\hline
31
&
EMPCAT1
&
Employment
categories,
Primary job
&
IND
\\
\hline
32
&
WHOURSWEEK1
&
Hours worked
last week,
Primary job
&
IND
\\
\hline
33
&
OWNHOUSE
&
Ownership of
dwelling unit
&
HH
\\
\hline
34
&
ROOF
&
Main material
used for roof
&
HH
\\
\hline
35
&
TOILET
&
Main toilet
facility
&
HH
\\
\hline
36
&
ELECTCON
&
Connection of
electricity in
dwelling
&
HH
\\
\hline
37
&
FUELCOOK
&
Main cooking
fuel
&
HH
\\
\hline
38
&
WATER
&
Main source of
water
&
HH
\\
\hline
39
&
OWNAGLAND
&
Ownership of
agricultural
land
&
HH
\\
\hline
40
&
LANDSIZEHA
&
Land size owned
by household
(ha)
&
HH
\\
\hline
41
&
OWNMOTORCYCLE
&
Ownership of
motorcycle
&
HH
\\
\hline
42
&
CAR
&
Ownership of
car
&
HH
\\
\hline
43
&
TV
&
Ownership of
television
&
HH
\\
\hline
44
&
LIVESTOCK
&
Number of
large-sized
livestock owned
&
HH
\\
\hline
45
&
INCRMT
&
Total amount of
remittances
received from
remittance
sending members
&
HH
\\
\hline
46
&
INCWAGE
&
Wage and
salaries(annual
)
&
HH
\\
\hline
47
&
INCBONSOCALL
&
Bonus and
social
allowance from
wage
job(annual)
&
HH
\\
\hline
48
&
INCFARMBSN
&
Gross income
from household
farm
businesses(annu
al)
&
HH
\\
\hline
49
&
INCNFARMBSN
&
Gross income
from household
non-farm
businesses(annu
al)
&
HH
\\
\hline
50
&
INCRENT
&
Rental
income(annual)
&
HH
\\
\hline
51
&
INCFIN
&
Financial
income from
savings, loans,
tax refunds,
maturity
payments on
insurance
&
HH
\\
\hline
52
&
INCPENSN
&
Pension and
other social
assistance(annu
al)
&
HH
\\
\hline
53
&
INCOTHER
&
Other
income(annual)
&
HH
\\
\hline
54
&
INCTOTGROSSHH
&
Total gross
household
income(annual)
&
HH
\\
\hline
55
&
FARMEMP
&
Farm employment
&
HH
\\
\hline
56
&
THOUSEXP
&
Total
expenditure on
housing
&
HH
\\
\hline
57
&
TFOODEXP
&
Total food
expenditure
&
HH
\\
\hline
58
&
TALCHEXP
&
Total alcohol
expenditure
&
HH
\\
\hline
59
&
TCLTHEXP
&
Total
expenditure on
clothing and
footwear
&
HH
\\
\hline
60
&
TFURNEXP
&
Total
expenditure on
furnishing
&
HH
\\
\hline
61
&
THLTHEXP
&
Total
expenditure on
health
&
HH
\\
\hline
62
&
TTRANSEXP
&
Total
expenditure on
transport
&
HH
\\
\hline
63
&
TCOMMEXP
&
Total
expenditure on
communications
&
HH
\\
\hline
64
&
TRECEXP
&
Total
expenditure on
recreation
&
HH
\\
\hline
65
&
TEDUEXP
&
Total
expenditure on
education
&
HH
\\
\hline
66
&
TRESTHOTEXP
&
Total
expenditure on
restaurants and
hotel
&
HH
\\
\hline
67
&
TMISCEXP
&
Total
miscellaneous
expenditure
&
HH
\\
\hline
68
&
TANHHEXP
&
Total annual
nominal
household
expenditures
&
HH
\\
\hline
\end{longtable}\sphinxatlongtableend\end{savenotes}


\section{\sphinxstylestrong{Appendix} \sphinxstylestrong{B: Example of Blanket Agreement for SUF}}
\label{\detokenize{appendices:appendix-b-example-of-blanket-agreement-for-suf}}
\sphinxstylestrong{Agreement between {[}providing agency{]} and {[}receiving agency{]} regarding
the deposit and use of microdata}

\sphinxstylestrong{A. This agreement relates to the following microdatasets:}
\begin{enumerate}
\item {} 
\_\_\_\_\_\_\_\_\_\_\_\_\_\_\_\_\_\_\_\_\_\_\_\_\_\_\_\_\_\_\_\_\_\_\_\_\_\_\_\_\_\_\_\_\_\_\_\_\_\_\_\_\_\_\_

\item {} 
\_\_\_\_\_\_\_\_\_\_\_\_\_\_\_\_\_\_\_\_\_\_\_\_\_\_\_\_\_\_\_\_\_\_\_\_\_\_\_\_\_\_\_\_\_\_\_\_\_\_\_\_\_\_\_

\item {} 
\_\_\_\_\_\_\_\_\_\_\_\_\_\_\_\_\_\_\_\_\_\_\_\_\_\_\_\_\_\_\_\_\_\_\_\_\_\_\_\_\_\_\_\_\_\_\_\_\_\_\_\_\_\_\_

\item {} 
\_\_\_\_\_\_\_\_\_\_\_\_\_\_\_\_\_\_\_\_\_\_\_\_\_\_\_\_\_\_\_\_\_\_\_\_\_\_\_\_\_\_\_\_\_\_\_\_\_\_\_\_\_\_\_

\item {} 
\_\_\_\_\_\_\_\_\_\_\_\_\_\_\_\_\_\_\_\_\_\_\_\_\_\_\_\_\_\_\_\_\_\_\_\_\_\_\_\_\_\_\_\_\_\_\_\_\_\_\_\_\_\_\_

\end{enumerate}
\begin{enumerate}
\setcounter{enumi}{1}
\item {} 
Terms of the agreement:

\end{enumerate}

As the owner of the copyright in the materials listed in section A, or
as duly authorized by the owner of the copyright in the materials, the
representative of {[}providing agency{]} grants the {[}receiving agency{]}
permission for the datasets listed in section A to be used by {[}receiving
agency{]} employees, subject to the following conditions:
\begin{enumerate}
\item {} 
Microdata (including subsets of the datasets) and copyrighted
materials provided by the {[}providing agency{]} will not be
redistributed or sold to other individuals, institutions or
organisations without the {[}providing agency{]}’s written agreement.
Non-copyrighted materials which do not contain microdata (such as
survey questionnaires, manuals, codebooks, or data dictionaries) may
be distributed without further authorization. The ownership of all
materials provided by the {[}providing agency{]} remains with the
{[}providing agency{]}.

\item {} 
Data will be used for statistical and scientific research purposes
only. They will be employed solely for reporting aggregated
information, including modeling, and not for investigating specific
individuals or organisations.

\item {} 
No attempt will be made to re-identify respondents, and there will
be no use of the identity of any person or establishment discovered
inadvertently. Any such discovery will be reported immediately to
the {[}providing agency{]}.

\item {} 
No attempt will be made to produce links between datasets provided
by the {[}providing agency{]} or between {[}providing agency{]} data and
other datasets that could identify individuals or organisations.

\item {} 
Any books, articles, conference papers, theses, dissertations,
reports or other publications employing data obtained from the
{[}providing agency{]} will cite the source, in line with the citation
requirement provided with the dataset.

\item {} 
An electronic copy of all publications based on the requested data
will be sent to the {[}providing agency{]}.

\item {} 
The {[}providing agency{]} and the relevant funding agencies bear no
responsibility the data’s use or for interpretation or inferences
based upon it.

\item {} 
An electronic copy of all publications based on the requested data
will be sent to the {[}providing agency{]}.

\item {} 
Data will be stored in a secure environment, with adequate access
restrictions. The {[}providing agency{]} may at any time request
information on the storage and dissemination facilities in place.

\item {} 
The {[}recipient agency{]} will provide an annual report on uses and
users of the listed microdatasets to the {[}providing agency{]}, with
information on the number of researchers having accessed each
dataset, and on the output of this research.

\item {} 
This access is granted for a period of {[}provide information on this
period, or state that the agreement is open ended{]}.

\end{enumerate}
\begin{enumerate}
\setcounter{enumi}{2}
\item {} 
Communications:

\end{enumerate}

The {[}receiving organisation{]} will appoint a contact person who

will act as unique focal person for this agreement. Should the focal
person be replaced, the {[}recipient agency{]} will immediately communicate
the name and coordinates of the new contact person to the {[}providing
agency{]}. Communications for administrative and procedural purposes may
be made by email, fax or letter as follows:

Communications made by {[}providing agency{]} to {[}recipient agency{]} will be
directed to:

Name of contact person:

Title of contact person:

Address of the recipient agency:

Email:

Tel:

Fax:

Communications made by {[}recipient agency{]} to {[}depositor agency{]}

will be directed to:

Name of contact person:

Title of contact person:

Address of the recipient agency:

Email:

Tel:

Fax:

\sphinxstylestrong{D. Signatories}

The following signatories have read and agree with the Agreement as
presented above:

\sphinxstylestrong{Representative of the {[}providing agency{]}}

Name \_\_\_\_\_\_\_\_\_\_\_\_\_\_\_\_\_\_\_\_\_\_\_\_\_\_\_\_\_\_\_\_\_\_\_\_\_\_\_\_\_\_\_\_\_\_\_\_\_\_\_\_

Signature \_\_\_\_\_\_\_\_\_\_\_\_\_\_\_\_\_\_\_\_\_\_\_\_\_\_\_\_\_\_\_ Date \_\_\_\_\_\_\_\_\_\_\_\_\_\_

\sphinxstylestrong{Representative of the {[}recipient agency{]}}

Name \_\_\_\_\_\_\_\_\_\_\_\_\_\_\_\_\_\_\_\_\_\_\_\_\_\_\_\_\_\_\_\_\_\_\_\_\_\_\_\_\_\_\_\_\_\_\_\_\_\_\_\_

Signature \_\_\_\_\_\_\_\_\_\_\_\_\_\_\_\_\_\_\_\_\_\_\_\_\_\_\_\_\_\_\_ Date \_\_\_\_\_\_\_\_\_\_\_\_\_\_

Source: Dupriez and Boyko, 2010


\section{\sphinxstylestrong{Appendix} \sphinxstylestrong{C:} \sphinxstylestrong{Internal and External Reports for Case Studies}}
\label{\detokenize{appendices:appendix-c-internal-and-external-reports-for-case-studies}}
This appendix provides example for internal and external reports on the
anonymization process for the case studies in Section 9.1. The internal
report consists of two parts: the first is on the anonymization of the
household-level variables and the second is on the anonymization of the
individual-level variables.

\sphinxstylestrong{Appendix} \sphinxstylestrong{D: Execution Times for Multiple Scenarios Tested using Selected Sample Data}
———————————-=———————————————————-

\noindent\sphinxincludegraphics[width=5.10448in,height=7.80597in]{{image22}.png}

\sphinxstylestrong{Description of anonymization scenarios}{\color{red}\bfseries{}\textbar{}image22\textbar{}}


\chapter{Bibliography}
\label{\detokenize{bibliography::doc}}\label{\detokenize{bibliography:bibliography}}

\chapter{Lists}
\label{\detokenize{lists:lists}}\label{\detokenize{lists::doc}}
May no longer be needed in this format or need to be automated


\section{List of Tables}
\label{\detokenize{lists:list-of-tables}}
Table 4.1: Example dataset showing sample frequencies, population
frequencies and individual disclosure risk 20

Table 4.2: Example dataset to illustrate the effect of missing values on
k-anonymity 25

Table 4.3: l-diversity illustration 26

Table 4.4: Sample uniques and special uniques 28

Table 4.5: Illustrating the calculation of SUDA and DIS-SUDA scores 30

Table 5.1: SDC methods and corresponding functions in \sphinxstyleemphasis{sdcMicro} 39

Table 5.3: Illustration of the effect of recoding on the theoretically
possible number of combinations an a dataset 40

Table 5.3: Illustration of the effect of recoding on the theoretically
possible number of combinations an a dataset 41

Table 5.4: Local suppression illustration - sample data before and after
suppression 49

Table 5.5: How importance vectors and k-anonymity thresholds affect
running time and total number of suppressions 52

Table 5.6 Effect of the all-\sphinxstyleemphasis{m} approach on k-anonymity 53

Table 5.7: Sample data before and after perturbation 56

Table 5.8: Tabulation of variable “region” before and after PRAM 57

Table 5.9: Tabulation of variable “region” before and after (invariant)
PRAM 58

Table 5.10: Cross-tabulation of variable “region” and variable “gender”
before and after invariant PRAM 59

Table 5.11: Illustrating the effect of choosing mean vs. median for
microaggregation where outliers are concerned 62

Table 5.12: Illustration of multivariate microaggregation 63

Table 5.13: Grouping methods for microaggregation that are implemented
in \sphinxstyleemphasis{sdcMicro} 64

Table 5.14: Simplified example of the shuffling method 72

Table 6.1: Description of anonymization methods by scenario 90

Table 7.1: Packages and functions for reading data in \sphinxstyleemphasis{R} 95

Table 7.2: Slot names and slot description of \sphinxstyleemphasis{sdcMicro} object 100

Table 7.3 Illustration of randomizing order of records in a dataset 104

Table 7.4: Computation times of different methods on datasets of
different sizes 106

Table 7.5: Number of categories (complexity), record uniqueness and
computation times 106

Table 8.1: Illustration of merging variables without information loss
for SDC process 111

Table 9.1 Overview of variables in dataset 122

Table 9.2: List of selected quasi-identifiers 126

Table 9.3: Overview of selected utility measures 127

Table 9.4: Number and proportion of households violating k-anonymity 129

Table 9.5: Percentiles 90-100 of the variable LANDSIZE 131

Table 9.6: 50 largest values of the variable LANDSIZE 131

Table 9.7: Frequencies of variable HHSIZE (household size) 132

Table 9.8: Number and proportion of households violating k-anonymity
after anonymization 137

Table 9.9: Univariate frequencies of the PRAMmed variable before and
after anonymization 139

Table 9.10: Multivariate frequencies of the variables WATER with RURURB
before and after anonymization 139

Table 9.11: GINI point estimates and bootstrapped confidence intervals
for sum of expenditure components 139

Table 9.12: Mean monthly expenditure and mean monthly income per capita
by rural/urban 140

Table 9.13 Shares of expenditures components 140

Table 9.14: k-anonymity violations 142

Table 9.15: Number of suppressions by variable for different variations
of local suppression 145

Table 9.16: k-anonymity violations 145

Table 9.17: Net enrollment in primary and secondary education by gender
145

Table 9.18: Overview of the variables in the dataset 149

Table 9.19: Overview of selected key variables for PUF file 152

\(k\)Table 9.20: Number and proportion of households violating
-anonymity 155

Table 9.21: Number of suppressions by variable after local suppression
with and without importance vector 158

Table 9.22: Comparison of utility measures 163

Table 9.23: Number of records violating k-anonimity 164

Table 9.24: Overview of recodes of categorical variables at individual
level 166

Table 9.25: k-anonymity violations 170


\section{List of Figures}
\label{\detokenize{lists:list-of-figures}}
Figure 2.1: Risk-utility trade-off 7

Figure 4.1 Classification of variables 17

Figure 4.2: Visualizations of DIS-SUDA scores 32

Figure 5.1 Effect of recoding \textendash{} frequency counts before and after
recoding 43

Figure 5.2 Age variable before and after recoding 44

Figure 5.3 Age variable before and after recoding 45

Figure 5.4: Utilizing the frequency distribution of variable age to
determine threshold for top coding 47

Figure 5.5: Changes in labor market indicators after anonymization of
I2D2 data 51

Figure 5.6: Illustration of effect of noise addition to outliers 66

Figure 5.7: Frequency distribution of a continuous variable before and
after noise addition 68

Figure 5.8: Noise levels and the impact on the value range (percentiles)
69

Figure 6.1: The trade-off between risk and utility in a hypothetical
dataset 77

Figure 6.2: Effect of anonymization on the point estimates and
confidence interval of the gender coefficient in the Mincer equation 86

Figure 6.3: Histograms of income before and after anonymization 87

Figure 6.4: Density plots of income before and after anonymization 88

Figure 6.5: Example of box plots of an expenditure variable before and
after anonymization 89

Figure 6.6: Mosaic plot to illustrate the changes in the WATER variable
90

Figure 6.7: Comparison of treated vs. untreated gender and relationship
status variables with mosaic plots 91

Figure 6.8: Mosaic plot of the variables ROOF and TOILET before
anonymization 92

Figure 6.9: Mosaic plot of the variables ROOF and TOILET after
anonymization 92

Figure 8.1: Overview of the SDC process 118

Figure 9.1: Lorenz curve based on positive total expenditures values 140


\section{List of Examples}
\label{\detokenize{lists:list-of-examples}}
\(f_{k}\)Example 4.1: Calculating using \sphinxstyleemphasis{sdcMicro} 21

Example 4.2: Calculating the sample and population frequencies using
\sphinxstyleemphasis{sdcMicro} 22

Example 4.3: The individual risk slot in the \sphinxstyleemphasis{sdcMicro} object 23

Example 4.4: Using the print() function to display observations
violating k-anonymity 24

Example 4.5: Computing k-anonymity violations for other values of k 25

\(l\)Example 4.6: -diversity function in \sphinxstyleemphasis{sdcMicro} 27

Example 4.7: Evaluating SUDA scores 31

Example 4.8: Histogram and density plots of DIS-SUDA scores 31

Example 4.9 Example with the function dRisk() 33

Example 4.10: Computing 90 \% percentile of variable income 34

Example 4.11: Computation of the global risk measure 34

Example 4.12: Computation of expected number of re-identifications 35

Example 4.13: Number of individuals with individual risk higher than the
threshold 0.05 35

Example 4.14: Computation of household risk and expected number of
re-identifications 36

Example 5.1: Using the sdcMicro function groupVars() to recode a
categorical variable 42

Example 5.2: Using the \sphinxstyleemphasis{sdcMicro} function globalRecode() to recode a
continuous variable (age) 43

Example 5.3: Using globalRecode() to create intervals of unequal width
44

Example 5.4: Constructing right-open intervals for semi-continuous
variables using built-in \sphinxstyleemphasis{sdcMicro} function globalRecode() 46

Example 5.5: Constructing intervals for semi-continuous and continuous
variables using manual recoding in \sphinxstyleemphasis{R} 46

Example 5.6: Top coding and bottom coding in \sphinxstyleemphasis{sdcMicro} using
topBotCoding() function 47

Example 5.7: Application of local suppression with and without
importance vector 50

\(m\)Example 5.8 The all- approach in sdcMicro 53

Example 5.9: Manually suppressing values in linked variables 54

Example 5.10: Suppressing values in linked variables by specifying ghost
variables 54

Example 5.11: Application of built-in \sphinxstyleemphasis{sdcMicro} function localSupp() 55

Example 5.12: Producing reproducible PRAM results by using set.seed() 58

Example 5.13: Selecting the variable “toilet” to apply PRAM 59

Example 5.14: Specifying minimum values for diagonal entries in PRAM
transition matrix 59

Example 5.15: Minimizing unlikely combinations by applying PRAM within
strata 60

Example 5.16: Applying univariate microaggregation with \sphinxstyleemphasis{sdcMicro}
function microaggregation() 62

Example 5.17: Multivariate microaggregation with the Maximum Distance to
Average Vector (MDAV) algorithm in \sphinxstyleemphasis{sdcMicro} 64

Example 5.18: Specifying strata variables for microaggregation 64

Example 5.19: Uncorrelated noise addition 67

Example 5.20: Correlated noise addition 69

Example 5.21: Noise addition for outliers using the ‘outdect’ method 70

Example 5.22: Noise addition to aggregates and their components 70

Example 5.23: Rank swapping using \sphinxstyleemphasis{sdcMicro} 71

Example 5.24: Shuffling using a specified regression equation 73

Example 6.1: Using the print() function to retrieve the total number of
suppressions for each categorical key variable 78

Example 6.2: Displaying the number of missing values for each
categorical key variable in an \sphinxstyleemphasis{sdcMicro} object 78

Example 6.3: Computing number of records changed per variable 79

Example 6.4: Comparing contingency tables of categorical variables 79

Example 6.5: Comparing the means of continuous variables 80

Example 6.6: Comparing covariance structure and correlation matrices of
numeric variables 81

Example 6.7: Using dUtility() to compute IL1s data utility measure in
\sphinxstyleemphasis{sdcMicro} 82

Example 6.8: Using dUtility() to compute eigenvalues in \sphinxstyleemphasis{sdcMicro} 83

Example 6.9: Computing the GINI coefficient from the income variable to
determine income inequality 83

Example 6.10: Estimating the Mincer equation (regression) to evaluate
data utility before and after anonymization 84

Example 6.11: Plotting histograms and kernel densities 87

Example 6.12: Creating boxplots for continuous variables 88

Example 6.13: Creating univariate mosaic plots 89

Example 6.14: Creating multivariate mosaic plots 91

Example 7.1: Loading required packages 94

Example 7.2: Displaying help for functions 94

Example 7.3: Reading in a \sphinxstyleemphasis{STATA} file 95

Example 7.4: Reading in an \sphinxstyleemphasis{Excel} file 95

Example 7.5: Reading in an \sphinxstyleemphasis{SPSS} file 96

Example 7.6: Recoding missing values to NA 96

Example 7.7: Changing the class of an object in \sphinxstyleemphasis{R} 97

Example 7.8: Selecting variables and creating an object of class
\sphinxstyleemphasis{sdcMicroObj} for the SDC process in \sphinxstyleemphasis{R} 98

Example 7.9: Displaying slot names and accessing slots of an S4 object
100

Example 7.10: Saving results of applying SDC methods 101

Example 7.11: Undo last step in SDC process 101

Example 7.12: Create a household level file with unique records (remove
duplicates) 102

Example 7.13 Merging anonymized household-level variables with
individual-level variables 103

Example 7.14 Generating the variable household size 103

Example 7.15 Changing the order of individuals within households 103

Example 7.16: Randomize order of households 105

Example 9.1: Loading required packages 120

Example 9.2: Loading the data 120

Example 9.3: Number of individuals and variables and variable names 120

Example 9.4: Tabulation of the variable ‘gender’ and summary statistics
for the variable ‘total annual expenditures’ in \sphinxstyleemphasis{R} 122

Example 9.5: Recoding missing value codes 124

Example 9.6: Dropping variables with only missing values 125

Example 9.7: Selecting the variables for the household-level
anonymization 127

Example 9.8: Taking a subset with only households 128

Example 9.9: Creating a \sphinxstyleemphasis{sdcMicro} object for the household variables
128

Example 9.10: Showing number of households violating k-anonymity for
levels 2,3 and 5 129

Example 9.11: Showing households that violate k-anonymity 130

Example 9.12: Printing global risk measures 130

Example 9.13: Observations with individual risk higher than 1\% 130

Example 9.14 Percentiles of LANDSIZE and listing the sizes of the
largest 50 plots 131

Example 9.15: Removing households with large (rare) household sizes 132

Example 9.16: Local suppression with and without importance vector 134

Example 9.17: Applying PRAM 135

Example 9.18: Anonymizing the variable LANDSIZEHA 135

Example 9.19: Anonymizing continuous variables 137

Example 9.20: Measuring risk of re-identification of continuous
variables 138

Example 9.21: Merging the files with household and individual-level
variables and creating an \sphinxstyleemphasis{sdcMicro} object for the anonymization of the
individual-level variables 141

Example 9.22: Global risk of the individual-level variables 142

Example 9.23: Recoding age in 10-year intervals in the range 15 \textendash{} 65 and
top code age over 65 years143

Example 9.24: Experimenting with different options in local suppression
144

Example 9.25: Using the report() function for internal and external
reports 146

Example 9.26: Exporting the anonymized dataset 147

Example 9.27: Loading required packages and datasets 148

Example 9.28 Number of individuals and variables and variable names 148

Example 9.29: Selecting the variables for the household-level
anonymization 153

Example 9.30: Taking a subset with only households 154

Example 9.31: Creating a \sphinxstyleemphasis{sdcMicro} object for the household variables
154

\(k\)Example 9.32: Showing number of households violating
-anonymity for levels 2, 3 and 5 155

\(k\)Example 9.33: Showing records of households that violate
-anonymity 156

Example 9.34: Printing global risk measures 156

Example 9.35 Determining the highest individual risk 156

Example 9.36: Local suppression with and without importance vector 158

Example 9.37: Applying PRAM 160

Example 9.38: Anonymization of income and expenditure variables 161

Example 9.39: Measuring risk of re-identification of continuous
variables 162

Example 9.40: Computation of decile dispersion ratio and share of total
consumption by the poorest decile 163

Example 9.41: Merging the files with household and individual-level
variables and creating an \sphinxstyleemphasis{sdcMicro} object for the anonymization of the
individual-level variables 163

Example 9.42: Risk measures before anonymization 165

Example 9.43: Recoding the categorical and continuous variables 166

Example 9.44: Local suppression to reach 5-anonimity 168

Example 9.45: Randomizing the order of records within regions 169

Example 9.46: Exporting the anonymized PUF file 171


\chapter{Case Study (Illustrating the SDC Process)}
\label{\detokenize{case_studies:case-study-illustrating-the-sdc-process}}\label{\detokenize{case_studies::doc}}
In order to evaluate the use of different SDC methods on different types
of survey datasets, we compared the results of the different methods
applied to 75 datasets from 52 countries representing six geographic
regions: Latin America and the Caribbean (LAC), Sub-Saharan Africa
(AFR), South Asia (SA), Europe and Central Asia (ECA), Middle East and
North Africa (MENA) as well as East Asia and the Pacific (EAP). The
datasets chosen were from a mix of datasets that are already publically
available, as well as data made available only to the World Bank for
official business. The surveys used included, amongst others, household,
demographic, and health surveys. The variables from these surveys used
for the experiments were selected based on their relevance for users
(e.g. for indicators, MDGs), their sensitivity, and their classification
with respect to the SDC process.

The following case studies draw from knowledge gained from these
experiments and try to incorporate the lessons learned. The case studies
use synthetic data that mimic the structure of the survey types we used
in our experiments and present the anonymization of a dataset similar to
many surveys designed to measure household income and consumption, labor
force participation and general demographic characteristics. The first
case study creates a SUF, whereas in the second case study we take this
SUF and treat it further to create a PUF.


\section{Case study 1- SUF}
\label{\detokenize{case_studies:case-study-1-suf}}
This case study shows an example of how the anonymization process might
be approached, particularly for a dataset with many continuous
variables. We also show how this can be achieved using the open source
and free \sphinxstyleemphasis{sdcMicro} package and \sphinxstyleemphasis{R}. A ready-to-run \sphinxstyleemphasis{R} script for this
case study and the dataset are also available to reproduce the results
and allow the user to adapt the code
(see \sphinxurl{http://ihsn.org/home/projects/sdc-practice}). Extracts of this code
are presented in this section to illustrate several steps of the
anonymization process. \sphinxstylestrong{NOTE: The choices of methods and parameters in
this case study are based on this particular dataset and the results and
choices might be different for other datasets.} The aim is to show the
process, not to compare methods per se.

This example uses a dataset with a similar structure to that of a
typical social survey with a focus on demographics, labor force
participation and income and expenditure patterns. The dataset has been
compiled using observations from several datasets from different
countries. They are considered synthetic data and as such are used only
for illustrative purposes. The source datasets were already treated for
disclosure control by their producers. This does not matter, as our
concern is to illustrate the process only. The data from which we
compiled our case study file was from surveys that contain many
variables, but pay particular attention to labor force variables as well
as household income and household expenditure variables. The variables
in the demo dataset have already been pre-selected from the total set of
variables available in the datasets. See Appendix A for the complete
overview of all variables.

This case study follows the steps of the SDC process outlined in Chapter
8.

\sphinxstylestrong{Step 1: Need for disclosure control}

The statistical units in this dataset are individuals and households.
The household structure provides a hierarchical structure in the data,
which should be taken into account when measuring risk and selecting
anonymization methods.

The data contains variables with demographic information, income,
expenditures, education variables and variables relating to the labor
status of the individual. These variables include sensitive and
confidential variables. The dataset is an example of a social survey
and, due to the nature of the statistical units and the variables,
disclosure control is needed before release of the microdata. This is
the case regardless of the legal framework, which is not specified here,
as this is a hypothetical dataset.

\sphinxstylestrong{Step 2: Data preparation and exploring data characteristics}

The first step is to explore the data. To analyze the data in \sphinxstyleemphasis{R} we
first have to read the data into \sphinxstyleemphasis{R}. In our case, the data is saved in
a \sphinxstyleemphasis{STATA} file (.dta). To read \sphinxstyleemphasis{STATA} files, we need to load the \sphinxstyleemphasis{R}
package \sphinxstyleemphasis{foreign} (see Section 7.2 on importing other data formats in
\sphinxstyleemphasis{R}). We also load the \sphinxstyleemphasis{sdcMicro} package and several other packages
used later for the computation of the utility measures. If these
packages are not yet installed, you should do so before trying to load
them. The \sphinxstyleemphasis{R} code for this case study demonstrates how to do this.

Example 9.1: Loading required packages

\begin{DUlineblock}{0em}
\item[] \sphinxstyleemphasis{\# Load required packages}
\item[] \sphinxstylestrong{library}(foreign) \sphinxstyleemphasis{\# for read/write function for STATA files}
\item[] \sphinxstylestrong{library}(sdcMicro) \sphinxstyleemphasis{\# sdcMicro package with functions for the SDC
process}
\end{DUlineblock}

\sphinxstylestrong{library}(laeken) \sphinxstyleemphasis{\# for GINI}

\sphinxstylestrong{library}(reldist) \sphinxstyleemphasis{\# for GINI}

\begin{DUlineblock}{0em}
\item[] \sphinxstylestrong{library}(bootstrap) \sphinxstyleemphasis{\# for bootstrapping}
\item[] \sphinxstylestrong{library}(ineq) \sphinxstyleemphasis{\# for Lorenz curves}
\end{DUlineblock}

After setting the working directory to the directory where the \sphinxstyleemphasis{STATA}
file is stored, we load the data into the object called \sphinxstyleemphasis{file}. All
output, unless otherwise specified, is saved in the working directory.

Example 9.2: Loading the data

\sphinxstylestrong{setwd}(“C:/WorldBank/CaseStudy/”) \sphinxstyleemphasis{\# Set working directory}

\begin{DUlineblock}{0em}
\item[] \sphinxstyleemphasis{\# Specify file name}
\item[] fname \textless{}- ” case\_1\_data.dta”
\item[] \sphinxstyleemphasis{\# Read-in file}
\item[] file \textless{}- \sphinxstylestrong{read.dta}(fname, convert.factors = F) \sphinxstyleemphasis{\# factors as
numeric code}
\end{DUlineblock}

We check the number of variables, number of observations and variable
names, as shown in Example 9.3.

Example 9.3: Number of individuals and variables and variable names

\sphinxstylestrong{dim}(file) \sphinxstyleemphasis{\# Dimensions of file (observations, variables)}

\#\# {[}1{]} 10574 68

\sphinxstylestrong{colnames}(file) \sphinxstyleemphasis{\# Variable names}

\begin{DUlineblock}{0em}
\item[] \sphinxcode{\sphinxupquote{\#\#  {[}1{]} "REGION"        "DIST"          "URBRUR"        "WGTHH"}}
\item[] \sphinxcode{\sphinxupquote{\#\#  {[}5{]} "WGTPOP"        "IDH"           "IDP"           "HHSIZE"}}
\item[] \sphinxcode{\sphinxupquote{\#\#  {[}9{]} "GENDER"        "REL"           "MARITAL"       "AGEYRS"}}
\item[] \sphinxcode{\sphinxupquote{\#\# {[}13{]} "AGEMTH"        "RELIG"         "ETHNICITY"     "LANGUAGE"}}
\item[] \sphinxcode{\sphinxupquote{\#\# {[}17{]} "MORBID"        "MEASLES"       "MEDATT"        "CHWEIGHTKG"}}
\item[] \sphinxcode{\sphinxupquote{\#\# {[}21{]} "CHHEIGHTCM"    "ATSCHOOL"      "EDUCY"         "EDYRS"}}
\item[] \sphinxcode{\sphinxupquote{\#\# {[}25{]} "EDYRSCURRAT"   "SCHTYP"        "LITERACY"      "EMPTYP1"}}
\item[] \sphinxcode{\sphinxupquote{\#\# {[}29{]} "UNEMP1"        "INDUSTRY1"     "EMPCAT1"       "WHOURSWEEK1"}}
\item[] \sphinxcode{\sphinxupquote{\#\# {[}33{]} "OWNHOUSE"      "ROOF"          "TOILET"        "ELECTCON"}}
\item[] \sphinxcode{\sphinxupquote{\#\# {[}37{]} "FUELCOOK"      "WATER"         "OWNAGLAND"     "LANDSIZEHA"}}
\item[] \sphinxcode{\sphinxupquote{\#\# {[}41{]} "OWNMOTORCYCLE" "CAR"           "TV"            "LIVESTOCK"}}
\item[] \sphinxcode{\sphinxupquote{\#\# {[}45{]} "INCRMT"        "INCWAGE"       "INCBONSOCALL"  "INCFARMBSN"}}
\item[] \sphinxcode{\sphinxupquote{\#\# {[}49{]} "INCNFARMBSN"   "INCRENT"       "INCFIN"        "INCPENSN"}}
\item[] \sphinxcode{\sphinxupquote{\#\# {[}53{]} "INCOTHER"      "INCTOTGROSSHH" "FARMEMP"       "THOUSEXP"}}
\item[] \sphinxcode{\sphinxupquote{\#\# {[}57{]} "TFOODEXP"      "TALCHEXP"      "TCLTHEXP"      "TFURNEXP"}}
\item[] \sphinxcode{\sphinxupquote{\#\# {[}61{]} "THLTHEXP"      "TTRANSEXP"     "TCOMMEXP"      "TRECEXP"}}
\item[] \sphinxcode{\sphinxupquote{\#\# {[}65{]} "TEDUEXP"       "TRESTHOTEXP"   "TMISCEXP"}}\sphinxcode{\sphinxupquote{"TANHHEXP"}}
\end{DUlineblock}

The dataset has 10,574 individuals in 2,000 households and contains 68
variables. The survey corresponds to a population of about 4.3 million
individuals, which means that the sample is relatively small and the
sample weights are high. This has an impact on the disclosure risk, as
we will see in Steps 6a and 6b.

To get an overview of the values of the variables, we use tabulations
and cross-tabulations for categorical variables and summary statistics
for continuous variables. To include the number of missing values (NA or
other), we use the option useNA = “ifany” in the table() function (see

Example 9.4).

In Table 9.1 the variables in the dataset are listed along with concise
descriptions of the variables, the level at which they are collected
(individual (IND), household (HH)), the measurement type (continuous,
semi-continuous, categorical) and value ranges. Note that the dataset
contains a selection of 68 variables (cf. Appendix A) of a total of 112
variables in the survey dataset. The variables have been preselected
based on their relevance for data users. This allows to reduce the total
numbers of variables to consider in the anonymization process and makes
the process easier. The numerical values for many of the categorical
variables are codes that refer to values, e.g., in the variable URBRUR,
1 stands for rural and 2 for urban. More information on the meanings of
coded values of the categorical variables is available in the \sphinxstyleemphasis{R} code
for this case study.

We identified the following sensitive variables in the data: ETHNICITY,
RELIGION, variables related to the labor force status of the individual
and the variables containing information on income and expenditures of
the household. Whether variables can be identified as sensitive may vary
across countries and datasets.

The case study dataset does not have any direct identifiers that, if
they were present, would need to be removed at this stage. Examples of
direct identifiers would be names, telephone numbers, geographical
location coordinates, etc.

Example 9.4: Tabulation of the variable ‘gender’ and summary statistics
for the variable ‘total annual expenditures’ in \sphinxstyleemphasis{R}

\sphinxstylestrong{table}(file\$GENDER, useNA = “ifany”) \sphinxstyleemphasis{\# tabulation of variable
GENDER (sex, categorical)}

\begin{DUlineblock}{0em}
\item[] \sphinxcode{\sphinxupquote{\#\#    0    1}}
\item[] \sphinxcode{\sphinxupquote{\#\# 5448 5126}}
\end{DUlineblock}

\sphinxstylestrong{summary}(file\$TANHHEXP) \sphinxstyleemphasis{\# summary statistics for variable TANHHEXP
(total annual household expenditures, continuous)}

\begin{DUlineblock}{0em}
\item[] \sphinxcode{\sphinxupquote{\#\#    Min. 1st Qu.  Median    Mean 3rd Qu.    Max.}}
\item[] \sphinxcode{\sphinxupquote{\#\#     498   15550   17290   28560   29720  353200}}
\end{DUlineblock}

Table 9.1 Overview of variables in dataset


\begin{savenotes}\sphinxatlongtablestart\begin{longtable}{|*{6}{\X{1}{6}|}}
\hline
\sphinxstyletheadfamily 
No.
&\sphinxstyletheadfamily 
Variable
name
&\sphinxstyletheadfamily 
Description
&\sphinxstyletheadfamily 
Level
&\sphinxstyletheadfamily 
Measurement
&\sphinxstyletheadfamily 
Values
\\
\hline
\endfirsthead

\multicolumn{6}{c}%
{\makebox[0pt]{\sphinxtablecontinued{\tablename\ \thetable{} -- continued from previous page}}}\\
\hline
\sphinxstyletheadfamily 
No.
&\sphinxstyletheadfamily 
Variable
name
&\sphinxstyletheadfamily 
Description
&\sphinxstyletheadfamily 
Level
&\sphinxstyletheadfamily 
Measurement
&\sphinxstyletheadfamily 
Values
\\
\hline
\endhead

\hline
\multicolumn{6}{r}{\makebox[0pt][r]{\sphinxtablecontinued{Continued on next page}}}\\
\endfoot

\endlastfoot

1
&
IDH
&
Household
ID
&
HH
&\begin{itemize}
\item {} 
\end{itemize}
&
1-2,000
\\
\hline
2
&
IDP
&
Individua
l
ID
&
IND
&\begin{itemize}
\item {} 
\end{itemize}
&
1-33
\\
\hline
3
&
REGION
&
Region
&
HH
&
categoric
al
&
1-6
\\
\hline
4
&
DISTRICT
&
District
&
HH
&
categoric
al
&
101-1105
\\
\hline
5
&
URBRUR
&
Area of
residence
&
HH
&
categoric
al
&
1, 2
\\
\hline
6
&
WGTHH
&
Individua
l
weighting
coefficie
nt
&
HH
&
weight
&
31.2-8495
.7
\\
\hline
7
&
WGTPOP
&
Populatio
n
weighting
coefficie
nt
&
IND
&
weight
&
45.8-9345
2.2
\\
\hline
8
&
HHSIZE
&
Household
size
&
HH
&
semi-cont
inuous
&
1-33
\\
\hline
9
&
GENDER
&
Gender
&
IND
&
categoric
al
&
0, 1
\\
\hline
10
&
REL
&
Relations
hip
to
household
head
&
IND
&
categoric
al
&
1-9
\\
\hline
11
&
MARITAL
&
Marital
status
&
IND
&
categoric
al
&
1-6
\\
\hline
12
&
AGEYRS
&
Age in
completed
years
&
IND
&
semi-cont
inuous
&
0-95
(under 1,
1/12 year
increment
s)
\\
\hline
13
&
AGEMTH
&
Age of
child in
completed
years
&
IND
&
semi-cont
inuous
&
1-1140
\\
\hline
14
&
RELIG
&
Religion
of
household
head
&
HH
&
categoric
al
&
1, 5-7, 9
\\
\hline
15
&
ETHNICITY
&
Ethnicity
of
household
head
&
HH
&
categoric
al
&
all
missing
values
\\
\hline
16
&
LANGUAGE
&
Language
of
household
head
&
HH
&
categoric
al
&
all
missing
values
\\
\hline
17
&
MORBID
&
Morbidity
last x
weeks
&
IND
&
categoric
al
&
0, 1
\\
\hline
18
&
MEASLES
&
Child
immunized
against
measles
&
IND
&
categoric
al
&
0, 1, 9
\\
\hline
19
&
MEDATT
&
Sought
medical
attention
&
IND
&
categoric
al
&
0, 1
\\
\hline
20
&
CHWEIGHTK
G
&
Weight of
the child
(Kg)
&
IND
&
continuou
s
&
2 \textendash{} 26.5
\\
\hline
21
&
CHHEIGHTC
M
&
Height of
child
(cms)
&
IND
&
continuou
s
&
7 - 140
\\
\hline
22
&
ATSCHOOL
&
Currently
enrolled
in school
&
IND
&
categoric
al
&
0, 1
\\
\hline
23
&
EDUCY
&
Highest
level of
education
attended
&
IND
&
categoric
al
&
1-6
\\
\hline
24
&
EDYEARS
&
Years of
education
&
IND
&
semi-cont
inuous
&
0-18
\\
\hline
25
&
EDYRSCURR
AT
&
Years of
education
for
currently
enrolled
&
IND
&
semi-cont
inuous
&
1-18
\\
\hline
26
&
SCHTYP
&
Type of
school
attending
&
IND
&
categoric
al
&
1-3, 9
\\
\hline
27
&
LITERACY
&
Literacy
&
IND
&
categoric
al
&
1-3
\\
\hline
28
&
EMPTYP1
&
Type of
employmen
t
&
IND
&
categoric
al
&
1-9
\\
\hline
29
&
UNEMP1
&
Unemploye
d
&
IND
&
categoric
al
&
0, 1
\\
\hline
30
&
INDUSTRY1
&
Industry
classific
ation
(1-digit)
&
IND
&
categoric
al
&
1-10
\\
\hline
31
&
EMPCAT1
&
Employmen
t
categorie
s
&
IND
&
categoric
al
&
11, 12,
13, 14,
21, 22
\\
\hline
32
&
WHOURSLAS
TWEEK1
&
Hours
worked
last week
&
IND
&
continuou
s
&
0-154
\\
\hline
33
&
OWNHOUSE
&
Ownership
of
dwelling
&
HH
&
categoric
al
&
0, 1
\\
\hline
34
&
ROOF
&
Main
material
used for
roof
&
IND
&
categoric
al
&
1-5, 9
\\
\hline
35
&
TOILET
&
Main
toilet
facility
&
HH
&
categoric
al
&
1-4, 9
\\
\hline
36
&
ELECTCON
&
Electrici
ty
&
HH
&
categoric
al
&
0-3
\\
\hline
37
&
FUELCOOK
&
Main
cooking
fuel
&
HH
&
categoric
al
&
1-5, 9
\\
\hline
38
&
WATER
&
Main
source of
water
&
HH
&
categoric
al
&
1-9
\\
\hline
39
&
OWNAGLAND
&
Ownership
of
agricultu
ral
land
&
HH
&
categoric
al
&
1-3
\\
\hline
40
&
LANDSIZEH
A
&
Land size
owned by
household
(ha)
(agric
and non
agric)
&
HH
&
continuou
s
&
0-1214
\\
\hline
41
&
OWNMOTORC
YCLE
&
Ownership
of
motorcycl
e
&
HH
&
categoric
al
&
0, 1
\\
\hline
42
&
CAR
&
Ownership
of car
&
HH
&
categoric
al
&
0, 1
\\
\hline
43
&
TV
&
Ownership
of
televisio
n
&
HH
&
categoric
al
&
0, 1
\\
\hline
44
&
LIFESTOCK
&
Number of
large-siz
ed
livestock
owned
&
HH
&
semi-cont
inuous
&
0-25
\\
\hline
45
&
INCRMT
&
Income \textendash{}
Remittanc
es
&
HH
&
continuou
s
&\begin{itemize}
\item {} 
\end{itemize}
\\
\hline
46
&
INCWAGE
&
Income -
Wages and
salaries
&
HH
&
continuou
s
&\begin{itemize}
\item {} 
\end{itemize}
\\
\hline
47
&
INCBONSOC
ALL
&
Income -
Bonuses
and
social
allowance
s
derived
from wage
jobs
&
HH
&
continuou
s
&\begin{itemize}
\item {} 
\end{itemize}
\\
\hline
48
&
INCFARMBS
N
&
Income -
Gross
income
from
household
farm
businesse
s
&
HH
&
continuou
s
&\begin{itemize}
\item {} 
\end{itemize}
\\
\hline
49
&
INCNFARMB
SN
&
Income
-Gross
income
from
household
nonfarm
businesse
s
&
HH
&
continuou
s
&\begin{itemize}
\item {} 
\end{itemize}
\\
\hline
50
&
INCRENT
&
Income -
Rent
&
HH
&
continuou
s
&\begin{itemize}
\item {} 
\end{itemize}
\\
\hline
51
&
INCFIN
&
Income -
Financial
&
HH
&
continuou
s
&\begin{itemize}
\item {} 
\end{itemize}
\\
\hline
52
&
INCPENSN
&
Income -
Pensions/
social
assistanc
e
&
HH
&
continuou
s
&\begin{itemize}
\item {} 
\end{itemize}
\\
\hline
53
&
INCOTHER
&
Income -
Other
&
HH
&
continuou
s
&\begin{itemize}
\item {} 
\end{itemize}
\\
\hline
54
&
INCTOTGRO
SSHH
&
Income -
Total
&
HH
&
continuou
s
&\begin{itemize}
\item {} 
\end{itemize}
\\
\hline
55
&
FARMEMP
&&&&\begin{itemize}
\item {} 
\end{itemize}
\\
\hline
56
&
TFOODEXP
&
Total
expenditu
re
on food
&
HH
&
continuou
s
&\begin{itemize}
\item {} 
\end{itemize}
\\
\hline
57
&
TALCHEXP
&
Total
expenditu
re
on
alcoholic
beverages
,
tobacco
and
narcotics
&
HH
&
continuou
s
&\begin{itemize}
\item {} 
\end{itemize}
\\
\hline
58
&
TCLTHEXP
&
Total
expenditu
re
on
clothing
&
HH
&
continuou
s
&\begin{itemize}
\item {} 
\end{itemize}
\\
\hline
59
&
THOUSEXP
&
Total
expenditu
re
on
housing
&
HH
&
continuou
s
&\begin{itemize}
\item {} 
\end{itemize}
\\
\hline
60
&
TFURNEXP
&
Total
expenditu
re
on
furnishin
g
&
HH
&
continuou
s
&\begin{itemize}
\item {} 
\end{itemize}
\\
\hline
61
&
THLTHEXP
&
Total
expenditu
re
on health
&
HH
&
continuou
s
&\begin{itemize}
\item {} 
\end{itemize}
\\
\hline
62
&
TTRANSEXP
&
Total
expenditu
re
on
transport
&
HH
&
continuou
s
&\begin{itemize}
\item {} 
\end{itemize}
\\
\hline
63
&
TCOMMEXP
&
Total
expenditu
re
on
communica
tion
&
HH
&
continuou
s
&\begin{itemize}
\item {} 
\end{itemize}
\\
\hline
64
&
TRECEXP
&
Total
expenditu
re
on
recreatio
n
&
HH
&
continuou
s
&\begin{itemize}
\item {} 
\end{itemize}
\\
\hline
65
&
TEDUEXP
&
Total
expenditu
re
on
education
&
HH
&
continuou
s
&\begin{itemize}
\item {} 
\end{itemize}
\\
\hline
66
&
TRESHOTEX
P
&
Total
expenditu
re
on
restauran
ts
and
hotels
&
HH
&
continuou
s
&\begin{itemize}
\item {} 
\end{itemize}
\\
\hline
67
&
TMISCEXP
&
Total
expenditu
re
on
miscellan
eous
spending
&
HH
&
continuou
s
&\begin{itemize}
\item {} 
\end{itemize}
\\
\hline
68
&
TANHHEXP
&
Total
annual
nominal
household
expenditu
res
&
HH
&
continuou
s
&\begin{itemize}
\item {} 
\end{itemize}
\\
\hline
\end{longtable}\sphinxatlongtableend\end{savenotes}

It is always important to ensure that the relationships between
variables in the data are preserved during the anonymization process and
to explore and take note of these relationships before beginning the
anonymization. In the final step in the anonymization process, an audit
should be conducted, using these initial results, to check that these
relationships are maintained in the anonymized dataset.

In our demo dataset, we identify several relationships between variables
that need to be preserved during the anonymization process. The
variables TANHHEXP and INCTOTGROSSHH represent the total annual nominal
household expenditure and the total gross annual household income,
respectively, and these variables are aggregations of existing income
and expenditure components in the dataset.

The variables related to education are available only for individuals in
the appropriate age groups and missing for other individuals. We make a
similar observation for variables relating to children, such as height,
weight and age in months. In addition, the household-level variables
(cf. fourth column of Table 9.1) have the same values for all members in
any particular household. The value of household size corresponds to the
actual number of individuals belonging to that household in the dataset.
As we proceed, we have to take care that these relationships and
structures are preserved in the anonymization process.

When tabulating the variables, we notice that the variables RELIG,
EMPTYP1 and LIVESTOCK have missing value codes different from the \sphinxstyleemphasis{R}
standard missing value code NA. Before proceeding, we need to recode
these to NA so \sphinxstyleemphasis{R} interprets them correctly. The missing value codes
are resp. 99999, 99 and 9999 for these three variables. These are
genuine missing value codes and not caused by the variables being not
applicable to the individual. Example 9.5 shows how to make these
changes. \sphinxstylestrong{NOTE: At the end of the anonymization process, and if desired
for users, it is relatively easy to change these values back to their
original missing value code.}

Example 9.5: Recoding missing value codes

\sphinxstyleemphasis{\# Set different NA codes to R missing value NA}

file{[},’RELIG’{]}{[}file{[},’RELIG’{]} == 99999{]} \textless{}- NA

\begin{DUlineblock}{0em}
\item[] file{[},’EMPTYP1’{]}{[}file{[},’EMPTYP1’{]} == 99{]} \textless{}- NA
\item[] file{[},’LIVESTOCK’{]}{[}file{[},’LIVESTOCK’{]} == 9999{]} \textless{}- NA
\end{DUlineblock}

We also take note that the variables LANGUAGE and ETHNICITY have only
missing values. Variables that contain only missing values should be
removed from the dataset at this stage and excluded from the
anonymization process. Removing these variables does not mean loss of
data or reduction of the data utility, since these variables did not
contain any information. It is, however, necessary to remove them,
because keeping them can lead to errors in some of the anonymization
methods in \sphinxstyleemphasis{R}. It is always possible to add these variables back into
the dataset to be released at the end of the anonymization process. It
is useful to reduce the dataset to those variables and records relevant
for the anonymization process. This guarantees the best results in \sphinxstyleemphasis{R}
and fewer errors. In Example 9.6 we drop the variables that contain all
missing values.

Example 9.6: Dropping variables with only missing values

\begin{DUlineblock}{0em}
\item[] \sphinxstyleemphasis{\# Drop variables containing only missings}
\item[] file \textless{}- file{[},!\sphinxstylestrong{names}(file) \%in\% \sphinxstylestrong{c}(‘LANGUAGE’,
‘ETHNICITY’){]}
\end{DUlineblock}

We assume that the data are collected in a survey that uses simple
sampling of households. The data contains two weight coefficients: WGTHH
and WGTPOP. The relationship between the weights is WGTPOP = WGTHH *
HHSIZE. WGTPOP is the sampling weight for the households and WGTHH is
the sampling weight for the individuals to be used for disclosure risk
calculations. WGTHH is used for computing individual-level indicators
(such as education) and WGTPOP is used for population level indicators
(such as income indicators). There are no strata variables available in
the data. We will use WGTPOP for the anonymization of the household
variables and WGTHH for the anonymization of the individual-level
variables.

\sphinxstylestrong{Step 3: Type of release}

In this case study, we assume that data will be released as a SUF, which
will be only available under license to accredited researchers with
approved research proposals (see Section 3.2 for more information of the
release of SUF). Therefore, the accepted risk level is higher and a
broader set of variables can be released than would be the case when
releasing a PUF. Since we do not have an overview of the requirements of
all users, we restrict the utility measures to a selected number of data
uses (see Step 5).

\sphinxstylestrong{Step 4: Intruder scenarios and choice of key variables}

Next, we analyze possible intruder scenarios and select
quasi-identifiers or key variables based on these scenarios. Since the
dataset used in this case study is a demo dataset that does not stem
from an existing country (and hence we do not have information on
external data sources available to possible intruders) and the original
data has already been anonymized, it is not possible to define exact
disclosure scenarios. Instead, we draft intruder scenarios for this demo
dataset based on some hypothetical assumptions about availability of
external data sources. We consider two types of disclosure scenarios: 1)
matching to other publicly available datasets and 2) spontaneous
recognition. The license under which the dataset will be distributed
(SUF) prohibits matching to external resources. Still this can happen.
However, the more important scenario is the one of spontaneous
recognition. We describe both scenarios in the following two paragraphs.

For the sake of illustration, we assume that population registers are
available with the demographic variables gender, age, place of residence
(region, urban/rural), religion and other variables such as marital
status and variables relating to education and professional status that
are also present in our dataset. In addition, we assume that there is a
publically available cadastral register on land ownership. Based on this
analysis of available data sources, we select the variables REGION,
URBRUR, HHSIZE, OWNAGLAND, RELIG, GENDER, REL (relationship to household
head), MARITAL (marital status), AGEYRS, INDUSTRY1 and two variables
relating to school attendance as categorical quasi-identifiers, the
expenditure and income variables as well as LANDSIZEHA as continuous
quasi-identifiers. According to our assessment, these variables might
enable an intruder to re-identify an individual or household in the
dataset by matching with other available datasets.

Table 9.2 gives an overview of the selected quasi-identifiers and their
levels of measurement.

The decision to release the dataset as a SUF means the level of
anonymization will be relatively low and consequently, the variables are
more detailed and a scenario of spontaneous recognition is our main
concern. Therefore, we should check for rare combinations or unusual
patterns in the variables. Variables that may lead to spontaneous
recognition in our sample are amongst others HHSIZE (household size),
LANDSIZEHA as well as income and expenditure variables. Large households
and large land ownership are easily identifiable. The same holds for
extreme outliers in wealth and expenditure variables, especially when
combined with other identifying variables such as region. There might be
only one or a few households in a certain region with a high income,
such as the local doctor. Variables that are easily observable and known
by neighbors such as ROOF, TOILET, WATER, ELECTCON, FUELCOOK,
OWNMOTORCYCLE, CAR, TV and LIVESTOCK may also need protection depending
on what stands out in the community, since a researcher might be able to
identify persons (s)he knows. This is called the nosy-neighbor scenario.

Table 9.2: List of selected quasi-identifiers


\begin{savenotes}\sphinxattablestart
\centering
\begin{tabulary}{\linewidth}[t]{|T|T|}
\hline
\sphinxstyletheadfamily 
\sphinxstylestrong{Name}
&\sphinxstyletheadfamily 
\sphinxstylestrong{Measurement}
\\
\hline
REGION (region)
&
Household, categorical
\\
\hline
URBRUR (area of residence)
&
Household, categorical
\\
\hline
HHSIZE (household size)
&
Household, categorical
\\
\hline
OWNAGLAND (agricultural land
ownership)
&
Household, categorical
\\
\hline
RELIG (religion of household
head)
&
Household, categorical
\\
\hline
LANDSIZEHA (size of agr. and
non-agr. land)
&
Household, continuous
\\
\hline
TANHHEXP (total expenditures)
&
Household, continuous
\\
\hline
T***EXP (expenditures in category
***)
&
Household, continuous
\\
\hline
INCTOTGROSSHH (total income)
&
Household, continuous
\\
\hline
INC*** (income in category ***)
&
Household, continuous
\\
\hline
GENDER (sex)
&
Individual, categorical
\\
\hline
REL (relationship to household
head)
&
Individual, categorical
\\
\hline
MARITAL (marital status)
&
Individual, categorical
\\
\hline
AGEYRS (age in completed years)
&
Individual, semi-continuous
\\
\hline
EDYRSCURATT (years of education
for currently enrolled)
&
Individual, semi-continuous
\\
\hline
EDUCY (highest level of education
completed)
&
Individual, categorical
\\
\hline
ATSCHOOL (currently enrolled in
school)
&
Individual, categorical
\\
\hline
INDUSTRY1 (industry
classification)
&
Individual, categorical
\\
\hline
\end{tabulary}
\par
\sphinxattableend\end{savenotes}

\sphinxstylestrong{Step 5: Data key uses and selection of utility measures}

In this case study, our aim is to create a SUF that provides sufficient
information for accredited researchers. We know that the primary use of
these data will be to evaluate indicators relating to income and
inequality. Examples are the GINI coefficient and indicators on what
share of income is spent on what type of expenditures. Furthermore, we
focus on some education indicators. Table 9.3 gives an overview of the
utility measures we selected. Besides these utility measures, which are
specific to the data uses, we also do standard checks, such as comparing
tabulations, cross-tabulations and summary statistics before and after
anonymization.

Table 9.3: Overview of selected utility measures


\begin{savenotes}\sphinxattablestart
\centering
\begin{tabulary}{\linewidth}[t]{|T|}
\hline
\sphinxstyletheadfamily 
Gini point estimates and confidence intervals for total expenditures
\\
\hline
Lorenz curves for total expenditures
\\
\hline
Mean monthly per capita total expenditures by area of residence
\\
\hline
Average share of components for expenditures
\\
\hline
Mean monthly per capita total income by area of residence
\\
\hline
Average share of components for income
\\
\hline
Net enrollment in primary education by gender
\\
\hline
\end{tabulary}
\par
\sphinxattableend\end{savenotes}

There are no published figures and statistics available that are
calculated from this dataset because it is a demo. In general, the
published figures should be re-computed based on the anonymized dataset
and compared to the published figures in Step 11. Large differences
would reduce the credibility of the anonymized dataset.

\sphinxstylestrong{Hierarchical (household) structure}

Our demo survey collects data on individuals in households. The
household structure is important for data users and should be considered
in the risk assessment. Since some variables are measured on the
household level and thus have identical values for each household
member, the values of the household variables should be treated in the
same way for each household member (see Section 5.5). Therefore, we
first anonymize only the household variables. After this, we merge them
with the individual-level variables and then anonymize the
individual-level and household-level variables jointly.

Since the data has a hierarchical structure, Steps 6 through 10 are
repeated twice: Steps 6a through 10a are for the household-level
variables and Steps 6b through 10b for the combined dataset. In this
way, we ensure that household-level variable values remain consistent
across household members for each household and the household structure
cannot be used to re-identify individuals. This is further explained in
Sections 4.4 and 7.6.

Before continuing to Step 6a, we select the categorical key variables,
continuous key variables and any variables selected for use in PRAM
routines, as well as household-level sampling weights. We extract these
selected household variables and the households from the dataset and
save them as \sphinxstyleemphasis{fileHH}. The choice of PRAM variables is further explained
in Step 8a. Example 9.7 illustrates how these steps are done in \sphinxstyleemphasis{R} (see
also Section 7.6). \sphinxstylestrong{NOTE: In our dataset, some of the categorical
variables when imported from the STATA file were not imported as
factors. sdcMicro requires that these be converted to factors before
proceeding.} Conversion of these variables to factors is also shown in
Example 9.7.

Example 9.7: Selecting the variables for the household-level
anonymization

\begin{DUlineblock}{0em}
\item[] \#\#\# Select variables (household level)
\item[] \sphinxstyleemphasis{\# Key variables (household level)}
\item[] selectedKeyVarsHH = \sphinxstylestrong{c}(‘URBRUR’, ‘REGION’, ‘HHSIZE’, ‘OWNHOUSE’,
‘OWNAGLAND’, ‘RELIG’)
\end{DUlineblock}

\sphinxstyleemphasis{\# Changing variables to class factor}

file\$URBRUR \textless{}- \sphinxstylestrong{as.factor}(file\$URBRUR)

file\$REGION \textless{}- \sphinxstylestrong{as.factor}(file\$REGION)

file\$OWNHOUSE \textless{}- \sphinxstylestrong{as.factor}(file\$OWNHOUSE)

file\$OWNAGLAND \textless{}- \sphinxstylestrong{as.factor}(file\$OWNAGLAND)

file\$RELIG \textless{}- \sphinxstylestrong{as.factor}(file\$RELIG)

\begin{DUlineblock}{0em}
\item[] \sphinxstyleemphasis{\# Numerical variables}
\item[] numVarsHH = \sphinxstylestrong{c}(‘LANDSIZEHA’, ‘TANHHEXP’, ‘TFOODEXP’, ‘TALCHEXP’,
‘TCLTHEXP’, ‘THOUSEXP’, ‘TFURNEXP’, ‘THLTHEXP’, ‘TTRANSEXP’,
‘TCOMMEXP’, ‘TRECEXP’, ‘TEDUEXP’, ‘TRESHOTEXP’, ‘TMISCEXP’,
‘INCTOTGROSSHH’, ‘INCRMT’, ‘INCWAGE’, ‘INCFARMBSN’, ‘INCNFARMBSN’,
‘INCRENT’, ‘INCFIN’, ‘INCPENSN’, ‘INCOTHER’)
\item[] \sphinxstyleemphasis{\# PRAM variables}
\item[] pramVarsHH = \sphinxstylestrong{c}(‘ROOF’, ‘TOILET’, ‘WATER’, ‘ELECTCON’,
‘FUELCOOK’, ‘OWNMOTORCYCLE’, ‘CAR’, ‘TV’, ‘LIVESTOCK’)
\end{DUlineblock}

\begin{DUlineblock}{0em}
\item[] \sphinxstyleemphasis{\# sample weight (WGTPOP) (household)}
\item[] weightVarHH = \sphinxstylestrong{c}(‘WGTPOP’)
\item[] \sphinxstyleemphasis{\# All household level variables}
\item[] HHVars \textless{}- \sphinxstylestrong{c}(‘HID’, selectedKeyVarsHH, pramVarsHH, numVarsHH,
weightVarHH)
\end{DUlineblock}

We then extract these variables from \sphinxstyleemphasis{file}, the dataframe in \sphinxstyleemphasis{R} that
contains all variables. Every household has the same number of entries
as it has members (e.g., a household of three will be repeated three
times in \sphinxstyleemphasis{fileHH}). Before analyzing the household-level variables, we
select only one entry per household, as illustrated in Example 9.8. This
is further explained in Section 7.6.

Example 9.8: Taking a subset with only households

\begin{DUlineblock}{0em}
\item[] \sphinxstyleemphasis{\# Create subset of file with households and HH variables}
\item[] fileHH \textless{}- file{[},HHVars{]}
\end{DUlineblock}

\begin{DUlineblock}{0em}
\item[] \sphinxstyleemphasis{\# Remove duplicated rows based on IDH, select uniques, one row per
household in fileHH}
\item[] fileHH \textless{}- fileHH{[}which(!duplicated(fileHH\$IDH)),{]}
\end{DUlineblock}

\sphinxstylestrong{dim}(fileHH)

\sphinxcode{\sphinxupquote{\#\# {[}1{]} 2000   39}}

The file \sphinxstyleemphasis{fileHH} contains 2,000 households and 39 variables. We are now
ready to create our \sphinxstyleemphasis{sdcMicro} object with the corresponding variables
we selected in Example 9.7. For our case study, we will create an
\sphinxstyleemphasis{sdcMicro} object called \sphinxstyleemphasis{sdcHH} based on the data in \sphinxstyleemphasis{fileHH}, which we
will use for steps 6a \textendash{} 10a (see Example 9.9). \sphinxstylestrong{NOTE: When the sdcMicro
object is created, the sdcMicro package automatically calculates and
stores the risk measures for the data.} This leads us to Step 6a.

Example 9.9: Creating a \sphinxstyleemphasis{sdcMicro} object for the household variables

\begin{DUlineblock}{0em}
\item[] \sphinxstyleemphasis{\# Create initial SDC object for household level variables}
\item[] sdcHH \textless{}- \sphinxstylestrong{createSdcObj}(dat = fileHH, keyVars = selectedKeyVarsHH,
pramVars = pramVarsHH,
\end{DUlineblock}

weightVar = weightVarHH, numVars = numVarsHH)

numHH \textless{}- \sphinxstylestrong{length}(fileHH{[},1{]}) \sphinxstyleemphasis{\# number of households}

\sphinxstylestrong{Step 6a: Assessing disclosure risk (household level)}

As a first measure, we evaluate the number of households violating
k-anonymity at the levels 2, 3 and 5.

Table 9.4 shows the number of violating households as well as the
percentage of the total number of households. Example 9.10 illustrates
how to find these values with \sphinxstyleemphasis{sdcMicro}. The print() function in
\sphinxstyleemphasis{sdcMicro} shows only the values for thresholds 2 and 3. Values for
other thresholds can be calculated manually by summing up the
frequencies smaller than the k-anonymity threshold, as shown in Example
9.10.

Table 9.4: Number and proportion of households violating k-anonymity


\begin{savenotes}\sphinxattablestart
\centering
\begin{tabulary}{\linewidth}[t]{|T|T|T|}
\hline
\sphinxstyletheadfamily 
\sphinxstylestrong{k-anonymity level}
&\sphinxstyletheadfamily 
\sphinxstylestrong{Number of HH
violating}
&\sphinxstyletheadfamily 
\sphinxstylestrong{Percentage of total
number of HH}
\\
\hline
2
&
103
&
5.15 \%
\\
\hline
3
&
229
&
11.45 \%
\\
\hline
5
&
489
&
24.45 \%
\\
\hline
\end{tabulary}
\par
\sphinxattableend\end{savenotes}

Example 9.10: Showing number of households violating k-anonymity for
levels 2,3 and 5

\begin{DUlineblock}{0em}
\item[] \sphinxstyleemphasis{\# Number of observations violating k-anonymity (thresholds 2 and 3)}
\item[] \sphinxstylestrong{print}(sdcHH)
\end{DUlineblock}

\begin{DUlineblock}{0em}
\item[] \sphinxcode{\sphinxupquote{\#\# Infos on 2/3-Anonymity:}}
\item[] \sphinxcode{\sphinxupquote{\#\#}}
\item[] \sphinxcode{\sphinxupquote{\#\# Number of observations violating}}
\item[] \sphinxcode{\sphinxupquote{\#\#  - 2-anonymity: 103}}
\item[] \sphinxcode{\sphinxupquote{\#\#  - 3-anonymity: 229}}
\item[] \sphinxcode{\sphinxupquote{\#\#}}
\item[] \sphinxcode{\sphinxupquote{\#\# Percentage of observations violating}}
\item[] \sphinxcode{\sphinxupquote{\#\#  - 2-anonymity: 5.150 \%}}
\item[] \sphinxcode{\sphinxupquote{\#\#  - 3-anonymity: 11.450 \%}}
\item[] \sphinxcode{\sphinxupquote{-{-}-{-}-{-}-{-}-{-}-{-}-{-}-{-}-{-}-{-}-{-}-{-}-{-}-{-}-{-}-{-}-{-}-{-}-{-}-{-}-{-}-{-}-{-}-{-}-{-}-{-}-{-}-{-}-{-}-{-}-{-}-{-}-{-}-{-}-{-}-{-}-{-}}}
\end{DUlineblock}

\begin{DUlineblock}{0em}
\item[] \sphinxstyleemphasis{\# Calculate sample frequencies and count number of obs. violating k
(5) - anonymity}
\item[] kAnon5 \textless{}- \sphinxstylestrong{sum}(sdcHH@risk\$individual{[},2{]} \textless{} 5)
\end{DUlineblock}

kAnon5

\#\# {[}1{]} 489

\begin{DUlineblock}{0em}
\item[] \sphinxstyleemphasis{\# As percentage of total}
\item[] kAnon5 / numHH
\end{DUlineblock}

\#\# {[}1{]} 0.2445

It is often useful to view the values for the household(s) that violate
k-anonymity. This might help clarify which variables cause the
uniqueness of these households; this can then be used later when
choosing appropriate SDC methods. Example 9.11 shows how to assess the
values of the households violating 3 and 5-anonymity. It seems that
among the categorical key variables, the variable HHSIZE is responsible
for many of the unique combinations and the origin of much of the risk.
Having determined this, we can flag HHSIZE as a possible first variable
to treat to obtain the required risk level. In practice, with a variable
like HHSIZE, this will likely involve removing large households from the
dataset to be released. As explained in Section 5.5, recoding and local
suppression are no valid options for the variable HHSIZE. The
frequencies of household size in Table 9.7 on page 132 show that there
are few households with more than 13 household members. This makes these
households easily identifiable based on the number of household members
and at high risk of re-identification, also in the context of the nosy
neighbor scenario.

Example 9.11: Showing households that violate k-anonymity

\begin{DUlineblock}{0em}
\item[] \sphinxstyleemphasis{\# Show values of key variable of records that violate k-anonymity}
\item[] fileHH{[}sdcHH@risk\$individual{[},2{]} \textless{} 3, selectedKeyVarsHH{]} \sphinxstyleemphasis{\# for
3-anonymity}
\end{DUlineblock}

fileHH{[}sdcHH@risk\$individual{[},2{]} \textless{} 5, selectedKeyVarsHH{]} \sphinxstyleemphasis{\# for
5-anonymity}

We also assess the disclosure risk of the categorical variables with the
individual and global risk measures as described in Sections 4.5 and
4.8. In \sphinxstyleemphasis{fileHH} every entry represents a household. Therefore, we use
the individual non-hierarchical risk here, where the individual refers
in this case to a household. \sphinxstyleemphasis{fileHH} contains only households and has
no hierarchical structure. In Step 6b, we evaluate the hierarchical risk
in \sphinxstyleemphasis{file}, the dataset containing both households and individuals. The
individual and global risk measures automatically take into
consideration the household weights, which we defined in Example 9.7. In
our file, the global risk measure calculated using the chosen key
variables is 0.05\%. This percentage is extremely low and corresponds to
1.03 expected re-identifications. The results are also shown in Example
9.12. This low figure can be explained by the relatively small sample
size of 0.25\% of the total population. Furthermore, one should keep in
mind that this risk measure is based only on the categorical
quasi-identifiers at the household level. Example 9.12 illustrates how
to print the global risk measure.

Example 9.12: Printing global risk measures

\sphinxstylestrong{print}(sdcHH, “risk”)

\begin{DUlineblock}{0em}
\item[] \sphinxcode{\sphinxupquote{\#\# Risk measures:}}
\item[] \sphinxcode{\sphinxupquote{\#\#}}
\item[] \sphinxcode{\sphinxupquote{\#\# Number of observations with higher risk than the main part of the data: 0}}
\item[] \sphinxcode{\sphinxupquote{\#\# Expected number of re-identifications: 1.03 (0.05 \%)}}
\end{DUlineblock}

The global risk measure does not provide information about the spread of
the individual risk measures. There might be a few households with
relatively high risk, while the global (average) risk is low. It is
therefore useful as a next step to inspect the observations with
relatively high risk. The highest risk is 5.5\% and only 14 households
have risk larger than 1\%. Example 9.13 shows how to display those
households with risk over a certain threshold. Here the threshold is
0.01 (1\%).

Example 9.13: Observations with individual risk higher than 1\%

\begin{DUlineblock}{0em}
\item[] \sphinxstyleemphasis{\# Observations with risk above certain threshold (0.01)}
\item[] fileHH{[}sdcHH@risk\$individual{[}, “risk”{]} \textgreater{} 0.01,{]}
\end{DUlineblock}

Since the selected key variables at the household level are both
categorical and numerical, the individual and global risk measures based
on frequency counts do not completely reflect the disclosure risk of the
entire dataset. Both categorical and continuous key variables are
important for the data users, thus options like recoding the continuous
variables (e.g., by creating quantiles of income and expenditure
variables) to make all of them categorical will likely not satisfy the
data user’s needs. We therefore avoid recoding continuous variables and
assess the disclosure risk of the categorical and continuous variables
separately. This approach can be partly justified by the fact that any
potential matching to external data sources for the continuous and
categorical variables are available from different external data sources
and as such will not be used simultaneously for matching.

\sphinxstylestrong{Continuous variables}

To measure the risk of the continuous variables, we use an interval
measure, which measures the number of anonymized values that are too
close to their original values. See Section 4.7.2 for more information
on interval-based risk measures for continuous variables. This measure
is an ex-post measure, meaning that the risk can be evaluated only after
anonymization and measures whether the perturbation is sufficiently
large. Since it is an ex-post measure, we can evaluate it only in Step
9a after the variables have been treated. Evaluating this measure based
on the original data would lead to a risk of 100\%; all values would be
too close to the original values since they would coincide with the
original values, no matter how small the chosen intervals would be.

We also look at the distribution of LANDSIZEHA. In the variable
LANDSIZEHA high values are rare and can lead to re-identification. An
example is a large landowner in a specific region. To evaluate the
distribution of the variable LANDSIZEHA, we look at the percentiles.
Every percentile represents approximately 20 households. In addition, we
look at the values of the largest 50 plots. Example 9.14 shows how to
use \sphinxstyleemphasis{R} to display the quantiles and the largest landplots. Table 9.5
shows the 90$^{\text{th}}$ \textendash{} 100$^{\text{th}}$ percentiles and Table 9.6
displays the largest 50 values for LANDSIZEHA. Based on these values, we
conclude that values of LANDSIZEHA over 40 make the household very
identifiable. These large households and households with large land
plots need extra protection, as discussed in Step 8a.

Example 9.14 Percentiles of LANDSIZE and listing the sizes of the
largest 50 plots

\begin{DUlineblock}{0em}
\item[] \sphinxstyleemphasis{\# 1st - 100th percentiles of land size}
\item[] \sphinxstylestrong{quantile}(fileHH\$LANDSIZEHA, probs = (1:100)/100, na.rm= TRUE)
\end{DUlineblock}

\begin{DUlineblock}{0em}
\item[] \sphinxstyleemphasis{\# Values of landsize for largest 50 plots}
\item[] \sphinxstylestrong{tail}(\sphinxstylestrong{sort}(fileHH\$LANDSIZEHA), n = 50)
\end{DUlineblock}

Table 9.5: Percentiles 90-100 of the variable LANDSIZE


\begin{savenotes}\sphinxattablestart
\centering
\begin{tabulary}{\linewidth}[t]{|T|T|T|T|T|T|T|}
\hline
\sphinxstyletheadfamily 
\sphinxstylestrong{Percentile}
&\sphinxstyletheadfamily 
90
&\sphinxstyletheadfamily 
91
&\sphinxstyletheadfamily 
92
&\sphinxstyletheadfamily 
93
&\sphinxstyletheadfamily 
94
&\sphinxstyletheadfamily 
95
\\
\hline
\sphinxstylestrong{Value}
&
6.00
&
8.00
&
8.09
&
10.12
&
10.12
&
10.12
\\
\hline
\sphinxstylestrong{Percentile}
&
96
&
97
&
98
&
99
&
100
&\\
\hline
\sphinxstylestrong{Value}
&
12.14
&
20.23
&
33.83
&
121.41
&
1,214.08
&\\
\hline
\end{tabulary}
\par
\sphinxattableend\end{savenotes}

Table 9.6: 50 largest values of the variable LANDSIZE


\begin{savenotes}\sphinxattablestart
\centering
\begin{tabulary}{\linewidth}[t]{|T|T|T|T|T|T|T|T|T|T|}
\hline
\sphinxstyletheadfamily 
12.14
&\sphinxstyletheadfamily 
15.00
&\sphinxstyletheadfamily 
15.37
&\sphinxstyletheadfamily 
15.78
&\sphinxstyletheadfamily 
16.19
&\sphinxstyletheadfamily 
20.00
&\sphinxstyletheadfamily 
20.23
&\sphinxstyletheadfamily 
20.23
&\sphinxstyletheadfamily 
20.23
&\sphinxstyletheadfamily 
20.23
\\
\hline
20.23
&
20.23
&
20.23
&
20.23
&
20.23
&
20.23
&
20.23
&
20.23
&
20.23
&
20.23
\\
\hline
20.23
&
20.23
&
20.50
&
30.35
&
32.38
&
40.47
&
40.47
&
40.47
&
40.47
&
40.47
\\
\hline
40.47
&
40.47
&
80.93
&
80.93
&
80.93
&
80.93
&
121.41
&
121.41
&
161.88
&
161.88
\\
\hline
161.88
&
182.11
&
246.86
&
263.05
&
283.29
&
404.69
&
404.69
&
607.04
&
809.39
&
1214.08
\\
\hline
\end{tabulary}
\par
\sphinxattableend\end{savenotes}

\sphinxstylestrong{Step 7a: Assessing utility measures (household level)}

The utility of the data does not only depend on the household level
variables, but on the combination of household-level and
individual-level variables. Therefore, it is not useful to evaluate all
the utility measures selected in Step 5 at this stage, i.e., before
anonymizing the individual level variables. We restrict the initial
measurement of utility to those measures that are solely based on the
household variables. In our dataset, these are the measures related to
income and expenditure and their distributions. The results are
presented in Step 10a, together with the results after anonymization,
which allow direct comparison. If after the next anonymization step it
appears that the data utility has been significantly decreased by the
suppression of some household level variables, we can return to this
step.

\sphinxstylestrong{Step 8a: Choice and application of SDC methods (household variables)}

This step is divided into the anonymization of the variable HHSIZE, as
this is a special case, the anonymization of the other selected
categorical quasi-identifiers and the anonymization of the selected
continuous quasi-identifiers.

\sphinxstylestrong{Variable HHSIZE}

The variable HHSIZE poses a problem for the anonymization of the file,
since suppressing it will not anonymize this variable: a simple
headcount based on the household ID would allow the reconstruction of
this variable. Table 9.7 shows the absolute frequencies of HHSIZE. The
number of households for each size larger than 13 is 6 or fewer and can
be considered outliers with a higher risk of re-identification, as
discussed in Step 6a. One way to deal with this is to remove all
households of size 14 or larger from the dataset %
\begin{footnote}[1]\sphinxAtStartFootnote
Other methods and guidance on treating datasets where household size
is a quasi-identifier are discussed in Section 5.5.
%
\end{footnote}.
Removing 29 households of size 14 or larger reduces the number of
2-anonymity violations by 18, of 3-anonymity violations by 26 and of
5-anonymity violations by 29. This means that all removed households
violated 5-anonymity due to the value of the variable HHSIZE and many of
them 2- or 3-anonymity. In addition, the average individual risk amongst
the 29 households is 0.15\%, which is almost three times higher than the
average individual risk of all households. The impact on the global risk
measure of removing these 29 households is, however, limited, due to the
relatively small number of removed households in comparison to the total
number of 2,000 households. Removing the households is primarily to
protect these specific households, not to reduce the global risk.
\sphinxstylestrong{NOTE: When using sdcMicro and manually removing households, the
sdcMicro object should be recreated based on the new, manually edited
dataset.} Changes, such as removing records, cannot be done in the
\sphinxstyleemphasis{sdcMicro} object. Example 9.15 illustrates the way to remove households
and recreate the \sphinxstyleemphasis{sdcMicro} object.

Table 9.7: Frequencies of variable HHSIZE (household size)


\begin{savenotes}\sphinxattablestart
\centering
\begin{tabulary}{\linewidth}[t]{|T|T|T|T|T|T|T|T|T|T|T|T|T|}
\hline
\sphinxstyletheadfamily 
\sphinxstylestrong{HHSIZE}
&\sphinxstyletheadfamily 
1
&\sphinxstyletheadfamily 
2
&\sphinxstyletheadfamily 
3
&\sphinxstyletheadfamily 
4
&\sphinxstyletheadfamily 
5
&\sphinxstyletheadfamily 
6
&\sphinxstyletheadfamily 
7
&\sphinxstyletheadfamily 
8
&\sphinxstyletheadfamily 
9
&\sphinxstyletheadfamily 
10
&\sphinxstyletheadfamily 
11
&\sphinxstyletheadfamily 
12
\\
\hline
\sphinxstylestrong{Frequency}
&
152
&
194
&
238
&
295
&
276
&
252
&
214
&
134
&
84
&
66
&
34
&
21
\\
\hline
\sphinxstylestrong{HHSIZE}
&
13
&
14
&
15
&
16
&
17
&
18
&
19
&
20
&
21
&
22
&
33
&\\
\hline
\sphinxstylestrong{Frequency}
&
11
&
6
&
6
&
5
&
4
&
2
&
1
&
2
&
1
&
1
&
1
&\\
\hline
\end{tabulary}
\par
\sphinxattableend\end{savenotes}

Example 9.15: Removing households with large (rare) household sizes

\begin{DUlineblock}{0em}
\item[] \sphinxstyleemphasis{\# Tabulation of variable HHSIZE}
\item[] \sphinxstylestrong{table}(\sphinxhref{mailto:sdcHH@manipKeyVars\$HHSIZE}{sdcHH@manipKeyVars\$HHSIZE})
\end{DUlineblock}

\begin{DUlineblock}{0em}
\item[] \sphinxstyleemphasis{\# Remove large households (14 or more household members) from file
and fileHH}
\item[] file \textless{}- file{[}!file{[},’HHSIZE’{]} \textgreater{}= 14,{]}
\end{DUlineblock}

fileHHnew \textless{}- fileHH{[}!fileHH{[},’HHSIZE’{]} \textgreater{}= 14,{]}

\begin{DUlineblock}{0em}
\item[] \sphinxstyleemphasis{\# Create new sdcMicro object based on the file without the removed
households}
\item[] sdcHH \textless{}- \sphinxstylestrong{createSdcObj}(dat=fileHHnew, keyVars=selectedKeyVarsHH,
pramVars=pramVarsHH, weightVar=weightVarHH, numVars = numVarsHH)
\end{DUlineblock}

\sphinxstylestrong{Categorical variables}

We are now ready to move on to the choice of SDC methods for the
categorical variables on the household level in our dataset. As noted in
our discussion of the methods, applying perturbative methods and local
suppression may lead to large loss of utility. The common approach is to
apply recoding to the largest possible extent as a first approach, to
reach a prescribed level of risk and reduce the number of suppressions
needed. Only after that should methods such as local suppression be
applied. If this approach does not already achieve the desired result,
we can consider perturbative methods.

Since the file is to be released as a SUF, we can keep a higher level of
detail in the data. The selected categorical key variables at the
household level are not suitable for recoding at this point. Due to the
relatively low risk of re-identification based on the five selected
categorical household level variables, it is possible in this case to
use an option like local suppression to achieve our desired level of
risk. Applying local suppression when initial risk is relatively low
will likely only lead to suppression of few observations and thus limit
the loss of utility. If, however, the data had been measured to have a
relatively high risk, then applying local suppression without previous
recoding would likely result in a large number of suppressions and
greater information loss. Efforts such a recoding should be taken first
before suppressing in cases where risk is initially measured as high.
Recoding will reduce risk with little information loss and thus the
number of suppressions, if local suppression is applied as an additional
step. We apply local suppression to reach 2-anonymity. The choice of the
low level of two is based on the overall low re-identification risk due
to the high sample weights and the release as SUF. High sample weights
mean, ceteris paribus, a low level of re-identification risk. Achieving
2-anonymity is the same as removing sample uniques. This leads to 42
suppressions in the variable HHSIZE and 4 suppressions in the variable
REGION. As explained earlier, suppression of the value of the variable
HHSIZE does not lead to actual suppression of this information.
Therefore, we redo the local suppression, but this time we tell
\sphinxstyleemphasis{sdcMicro} to, if possible, not suppress HHSIZE but one of the other
variables.

In \sphinxstyleemphasis{sdcMicro} it is possible to tell the algorithm which variables are
important and less important for making small changes (see also Section
5.2.2). To prevent HHSIZE being suppressed, we set the importance of
HHSIZE in the importance vectors to the highest (i.e., 1). Example 9.16
shows how to apply local suppression and put importance on the variable
HHSIZE. The variable REGION is the type of variable that should not have
any suppressions either. We also set the importance of REGION to 2 and
the importance of RURURB to 3. This leads to an order of the variables
to be considered for suppression by the algorithm. Instead of 42
suppressions in the variable HHSIZE, this leads one suppressed value in
the variable HHSIZE, and to 6, 1, 48 and 16 suppressions respectively
for the variables URBRUR, REGION, OWNAGLAND and RELIG (which we set as
less important). The importance is clearly reflected in the number of
suppression. The total number of suppressions is higher than without
importance vector (71 vs. 46), but 2-anonymity is achieved in the
dataset with fewer suppressions in the variables HHSIZE and REGION. We
remove the one household with the suppressed value of HHSIZE (13) to
protect this household. \sphinxstylestrong{NOTE: In} Example 9.16 \sphinxstylestrong{we use the
undolast() function in sdcMicro to go one step back after we had first
applied local suppression with no importance vector.} The undolast()
function restores the \sphinxstyleemphasis{sdcMicro} object back to the previous state
(i.e., before we applied local suppression), which allows us to rerun
the same command, but this time with an importance vector set. The
undolast() function can only be used to go one step back.

Example 9.16: Local suppression with and without importance vector

\begin{DUlineblock}{0em}
\item[] \sphinxstyleemphasis{\# Local suppression}
\item[] sdcHH \textless{}- \sphinxstylestrong{localSuppression}(sdcHH, k=2, importance = NULL) \sphinxstyleemphasis{\# no
importance vector}
\end{DUlineblock}

\sphinxstylestrong{print}(sdcHH, “ls”)

\begin{DUlineblock}{0em}
\item[] \sphinxcode{\sphinxupquote{\#\# Local Suppression:}}
\item[] \sphinxcode{\sphinxupquote{\#\#     KeyVar \textbar{} Suppressions (\#) \textbar{} Suppressions (\%)}}
\item[] \sphinxcode{\sphinxupquote{\#\#     URBRUR \textbar{}                0 \textbar{}            0.000}}
\item[] \sphinxcode{\sphinxupquote{\#\#     REGION \textbar{}                4 \textbar{}            0.203}}
\item[] \sphinxcode{\sphinxupquote{\#\#     HHSIZE \textbar{}               37 \textbar{}            1.877}}
\item[] \sphinxcode{\sphinxupquote{\#\#  OWNAGLAND \textbar{}                0 \textbar{}            0.000}}
\item[] \sphinxcode{\sphinxupquote{\#\#      RELIG \textbar{}                0 \textbar{}            0.000}}
\end{DUlineblock}

sdcHH \textless{}- \sphinxstylestrong{undolast}(sdcHH)

sdcHH \textless{}- \sphinxstylestrong{localSuppression}(sdcHH, k=2, importance = \sphinxstylestrong{c}(3, 2,
1, 5, 5)) \sphinxstyleemphasis{\# importance on HHSIZE (1), REGION (2) and URBRUR (3)}

\sphinxstylestrong{print}(sdcHH, “ls”)

\begin{DUlineblock}{0em}
\item[] \sphinxcode{\sphinxupquote{\#\# Local Suppression:}}
\item[] \sphinxcode{\sphinxupquote{\#\#     KeyVar \textbar{} Suppressions (\#) \textbar{} Suppressions (\%)}}
\item[] \sphinxcode{\sphinxupquote{\#\#     URBRUR \textbar{}                6 \textbar{}            0.304}}
\item[] \sphinxcode{\sphinxupquote{\#\#     REGION \textbar{}                1 \textbar{}            0.051}}
\item[] \sphinxcode{\sphinxupquote{\#\#     HHSIZE \textbar{}                1 \textbar{}            0.051}}
\item[] \sphinxcode{\sphinxupquote{\#\#  OWNAGLAND \textbar{}               43 \textbar{}            2.182}}
\item[] \sphinxcode{\sphinxupquote{\#\#      RELIG \textbar{}               16 \textbar{}            0.812}}
\end{DUlineblock}

The variables ROOF, TOILET, WATER, ELECTCON, FUELCOOK, OWNMOTORCYCLE,
CAR, TV and LIVESTOCK are not sensitive variables and were not selected
as quasi-identifiers because we assumed that there are no external data
sources containing this information that could be used for matching.
Values can be easily observed or be known to neighbors, however, and
therefore are important, together with other variables, for the
spontaneous recognition scenario and nosy neighbor scenario. To
anonymize these variables, we want to introduce a low level of
uncertainty in them. Therefore, we decide to use invariant PRAM for the
variables ROOF, TOILET, WATER, ELECTCON, FUELCOOK, OWNMOTORCYCLE, CAR,
TV and LIVESTOCK, where we treat LIVESTOCK as a semi-continuous variable
due to the low number of different values. Section 5.3.1 provides more
information on the PRAM method and its implementation in \sphinxstyleemphasis{sdcMicro}.
Example 9.17 illustrates how to apply PRAM. We choose the parameter
\sphinxstyleemphasis{pd}, the lower bound for the probability that a value is not changed,
to be relatively high at 0.8. We can choose a high value, because the
variables themselves are not sensitive and we only want to introduce a
low level of changes to minimize the utility loss. Because the
distribution of many of the variables chosen for PRAM depends on the
REGION, we decide to use the variable REGION as a strata variable. In
this way the transition matrix is computed for each region separately.
Because PRAM is a probabilistic method, we set a seed for the random
number generator before applying PRAM to ensure reproducibility of the
results. \sphinxstylestrong{NOTE: In practice, it is not advisable to set a seed of
12345, but rather a longer more complex and less easy to guess
sequence.} The seed should not be released, since it allows for
reconstructing the original values if combined with the transition
matrix. The transition matrix can be released: this allows for
consistent statistical inference by correcting the statistical methods
used if the researcher has knowledge about the PRAM method (at this
point \sphinxstyleemphasis{sdcMicro} does not allow the retrieval of the transition matrix).

Example 9.17: Applying PRAM

\begin{DUlineblock}{0em}
\item[] \sphinxstyleemphasis{\# Pram}
\item[] \sphinxstylestrong{set.seed}(12345)
\item[] sdcHH \textless{}- \sphinxstylestrong{pram}(sdcHH, strata\_variables = “REGION”, pd = 0.8)
\end{DUlineblock}

\begin{DUlineblock}{0em}
\item[] \sphinxcode{\sphinxupquote{\#\# Number of changed observations:}}
\item[] \sphinxcode{\sphinxupquote{\#\# - - - - - - - - - - -}}
\item[] \sphinxcode{\sphinxupquote{\#\# ROOF != ROOF\_pram : 98 (4.97\%)}}
\item[] \sphinxcode{\sphinxupquote{\#\# TOILET != TOILET\_pram : 151 (7.66\%)}}
\item[] \sphinxcode{\sphinxupquote{\#\# WATER != WATER\_pram : 167 (8.47\%)}}
\item[] \sphinxcode{\sphinxupquote{\#\# ELECTCON != ELECTCON\_pram : 90 (4.57\%)}}
\item[] \sphinxcode{\sphinxupquote{\#\# FUELCOOK != FUELCOOK\_pram : 113 (5.73\%)}}
\item[] \sphinxcode{\sphinxupquote{\#\# OWNMOTORCYCLE != OWNMOTORCYCLE\_pram : 41 (2.08\%)}}
\item[] \sphinxcode{\sphinxupquote{\#\# CAR != CAR\_pram : 172 (8.73\%)}}
\item[] \sphinxcode{\sphinxupquote{\#\# TV != TV\_pram : 137 (6.95\%)}}
\item[] \sphinxcode{\sphinxupquote{\#\# LIVESTOCK != LIVESTOCK\_pram : 149 (7.56\%)}}
\end{DUlineblock}

PRAM has changed values within the variables according to the invariant
transition matrices. Since we used the invariant PRAM method (see
Section 5.3.1), the absolute univariate frequencies remain unchanged.
This is not the case for the multivariate frequencies. In Step 10a we
compare the changes in the multivariate frequencies for the PRAMmed
variables.

\sphinxstylestrong{Continuous variables}

We have selected income and expenditures variables and the variable
LANDSIZEHA as numerical quasi-identifiers, as discussed in Step 4. In
Step 5 we identified variables having high interest for the users of our
data: many users use the data for measuring inequality and expenditure
patterns.

Based on the risk evaluation in Step 6a, we decide to anonymize the
variable LANDSIZEHA by top coding at the value 40 (cf. Table 9.5 and
Table 9.6) and round values smaller than 1 to one digit, and values
larger than 1 to zero digits. Rounding the values prevents exact
matching with the available cadastral register. Furthermore, we group
the values between 5 and 40 in the groups 5 \textendash{} 19 and 20 \textendash{} 39. After
these steps, no household has a unique plot size and the number of
households in the sample with the same plot size was increased to at
least 7. This is shown by the tabulation of the variable LANDSIZEHA
after manipulation in the last line of Example 9.18. In addition, all
outliers have been removed by top coding the values. This has reduced
the risk of spontaneous recognition as discussed in Step 6. How to
recode values in \sphinxstyleemphasis{R} is introduced in Section 5.2.1 and, for this
particular case, shown in Example 9.18.

Example 9.18: Anonymizing the variable LANDSIZEHA

\begin{DUlineblock}{0em}
\item[] \sphinxstyleemphasis{\# Rounding values of LANDSIZEHA to 1 digit for plots smaller than 1
and to 0 digits for plots larger than 1}
\item[] sdcHH@manipNumVars\$LANDSIZEHA{[}\sphinxhref{mailto:sdcHH@manipNumVars\$LANDSIZEHA}{sdcHH@manipNumVars\$LANDSIZEHA} \textless{}= 1 \&
!\sphinxstylestrong{is.na}(\sphinxhref{mailto:sdcHH@manipNumVars\$LANDSIZEHA}{sdcHH@manipNumVars\$LANDSIZEHA}){]} \textless{}-
\sphinxstylestrong{round}(sdcHH@manipNumVars\$LANDSIZEHA{[}\sphinxhref{mailto:sdcHH@manipNumVars\$LANDSIZEHA}{sdcHH@manipNumVars\$LANDSIZEHA}
\textless{}= 1 \& !\sphinxstylestrong{is.na}(\sphinxhref{mailto:sdcHH@manipNumVars\$LANDSIZEHA}{sdcHH@manipNumVars\$LANDSIZEHA}){]}, digits = 1)
\end{DUlineblock}

sdcHH@manipNumVars\$LANDSIZEHA{[}\sphinxhref{mailto:sdcHH@manipNumVars\$LANDSIZEHA}{sdcHH@manipNumVars\$LANDSIZEHA} \textgreater{} 1 \&
!\sphinxstylestrong{is.na}(\sphinxhref{mailto:sdcHH@manipNumVars\$LANDSIZEHA}{sdcHH@manipNumVars\$LANDSIZEHA}){]} \textless{}-
\sphinxstylestrong{round}(sdcHH@manipNumVars\$LANDSIZEHA{[}\sphinxhref{mailto:sdcHH@manipNumVars\$LANDSIZEHA}{sdcHH@manipNumVars\$LANDSIZEHA}
\textgreater{} 1 \& !\sphinxstylestrong{is.na}(\sphinxhref{mailto:sdcHH@manipNumVars\$LANDSIZEHA}{sdcHH@manipNumVars\$LANDSIZEHA}){]}, digits = 0)

\begin{DUlineblock}{0em}
\item[] \sphinxstyleemphasis{\# Grouping values of LANDSIZEHA into intervals 5-19, 20-39}
\item[] sdcHH@manipNumVars\$LANDSIZEHA{[}\sphinxhref{mailto:sdcHH@manipNumVars\$LANDSIZEHA}{sdcHH@manipNumVars\$LANDSIZEHA} \textgreater{}= 5 \&
\sphinxhref{mailto:sdcHH@manipNumVars\$LANDSIZEHA}{sdcHH@manipNumVars\$LANDSIZEHA} \textless{} 20 \&
!\sphinxstylestrong{is.na}(\sphinxhref{mailto:sdcHH@manipNumVars\$LANDSIZEHA}{sdcHH@manipNumVars\$LANDSIZEHA}){]} \textless{}- 13
\end{DUlineblock}

sdcHH@manipNumVars\$LANDSIZEHA{[}\sphinxhref{mailto:sdcHH@manipNumVars\$LANDSIZEHA}{sdcHH@manipNumVars\$LANDSIZEHA} \textgreater{}= 20 \&
\sphinxhref{mailto:sdcHH@manipNumVars\$LANDSIZEHA}{sdcHH@manipNumVars\$LANDSIZEHA} \textless{} 40 \&
!\sphinxstylestrong{is.na}(\sphinxhref{mailto:sdcHH@manipNumVars\$LANDSIZEHA}{sdcHH@manipNumVars\$LANDSIZEHA}){]} \textless{}- 30

\begin{DUlineblock}{0em}
\item[] \sphinxstyleemphasis{\# Topcoding values of LANDSIZEHA larger than 40 (also recomputes risk
after manual changes)}
\item[] sdcHH \textless{}- \sphinxstylestrong{topBotCoding}(sdcHH, value = 40, replacement = 40, kind
= ‘top’, column = ‘LANDSIZEHA’)
\end{DUlineblock}

\begin{DUlineblock}{0em}
\item[] \sphinxstyleemphasis{\# Results for LANDSIZEHA}
\item[] \sphinxstylestrong{table}(\sphinxhref{mailto:sdcHH@manipNumVars\$LANDSIZEHA}{sdcHH@manipNumVars\$LANDSIZEHA})
\end{DUlineblock}

\begin{DUlineblock}{0em}
\item[] \sphinxcode{\sphinxupquote{\#\#   0 0.1 0.2 0.3 0.4 0.5 0.6 0.7 0.8 0.9   1   2   3   4  13  30  40}}
\item[] \sphinxcode{\sphinxupquote{\#\# 188 109  55  30  24  65  22   7  31  16 154 258  53  60 113  18  25}}
\end{DUlineblock}

For the expenditure and income variables we compared, \sphinxstylestrong{based on the
actual case study data}, several methods. As mentioned earlier, the
main use of the data is to compute inequality measures, such as the Gini
coefficient. Recoding these variables into percentiles creates
difficulties computing these measures or changes these measures to a
large extent and is hence not a suitable method. Often, income and
expenditure variables that are released in a SUF are anonymized by
top-coding. This protects the outliers, which are the values that are
the most at risk. Top-coding, however, destroys the inequality
information in the data, by removing high (and low) incomes. Therefore,
we decide to use noise addition. To take into account the higher risk of
outliers, we add a higher level of noise to those.

Adding noise can lead to a transformation of the shape of the
distribution. Depending on the magnitude of the noise (see Section 5.3.3
for the definition of the magnitude of noise), the values of income can
also become negative. One way to solve this would be to cut off the
values below zero and set them to zero. This would, however, destroy the
properties conserved by noise addition (amongst others the value of the
expected mean, see also Section 5.3.3) and we chose to keep the negative
values.

As mentioned before, the aggregates of income and expenditures are the
sums of the components. Adding noise to each of the components might
lead to violation of this condition. Therefore, one solution is to add
noise to the aggregates and remove the components. We prefer to keep the
components in the data and apply noise addition to each component
separately. This allows to apply a lower level of noise than when
applying noise only to the aggregates. A noise level of 0.01 seems to be
sufficient with extra noise of 0.05 added to the outliers. The outliers
are defined by a robust Mahalanobis distance (see Section 5.3.3). After
adding noise to the components, we recomputed the aggregates as the sum
of the perturbed components. \sphinxstylestrong{NOTE: This result is only based on the
actual case study dataset and is not necessarily true for other
datasets.} The noise addition is shown in Example 9.19. Before applying
probabilistic methods such as noise addition, we set a seed for the
random number generator. This allows us to reproduce the results.

Example 9.19: Anonymizing continuous variables

\sphinxstyleemphasis{\# Add noise to income and expenditure variables by category}

\begin{DUlineblock}{0em}
\item[] \sphinxstyleemphasis{\# Anonymize components}
\item[] compExp \textless{}- \sphinxstylestrong{c}(“TFOODEXP”, “TALCHEXP”, “TCLTHEXP”, “THOUSEXP”,
“TFURNEXP”, “THLTHEXP”, “TTRANSEXP”, “TCOMMEXP”, “TRECEXP”, “TEDUEXP”,
“TRESHOTEXP”, “TMISCEXP”)
\item[] \sphinxstylestrong{set.seed}(123)
\end{DUlineblock}

\sphinxstyleemphasis{\# Add noise to expenditure variables}

sdcHH \textless{}- \sphinxstylestrong{addNoise}(noise = 0.01, obj = sdcHH, variables = compExp,
method = “additive”)

\sphinxstyleemphasis{\# Add noise to outliers}

sdcHH \textless{}- \sphinxstylestrong{addNoise}(noise = 0.05, obj = sdcHH, variables = compExp,
method = “outdect”)

\sphinxstyleemphasis{\# Sum over expenditure categories to obtain consistent totals}

sdcHH@manipNumVars{[},’TANHHEXP’{]} \textless{}-
\sphinxstylestrong{rowSums}(sdcHH@manipNumVars{[},compExp{]})

\begin{DUlineblock}{0em}
\item[] compInc \textless{}- \sphinxstylestrong{c}(‘INCRMT’, ‘INCWAGE’, ‘INCFARMBSN’, ‘INCNFARMBSN’,
‘INCRENT’, ‘INCFIN’, ‘INCPENSN’, ‘INCOTHER’)
\item[] \sphinxstyleemphasis{\# Add noise to income variables}
\end{DUlineblock}

sdcHH \textless{}- \sphinxstylestrong{addNoise}(noise = 0.01, obj = sdcHH, variables = compInc,
method = “additive”)

\sphinxstyleemphasis{\# Add noise to outliers}

sdcHH \textless{}- \sphinxstylestrong{addNoise}(noise = 0.05, obj = sdcHH, variables = compInc,
method = “outdect”)

\sphinxstyleemphasis{\# Sum over income categories to obtain consistent totals}

sdcHH@manipNumVars{[},’INCTOTGROSSHH’{]} \textless{}-
\sphinxstylestrong{rowSums}(sdcHH@manipNumVars{[},compInc{]})

\sphinxstyleemphasis{\# recalculate risks after manually changing values in sdcMicro object}

\sphinxstylestrong{calcRisks}(sdcHH)

\sphinxstylestrong{Step 9a: Re-measure risk}

For the categorical variables, we conclude that we have achieved
2-anonymity in the data with local suppression. Only 104 households, or
about 5\% of the total number, violate 3-anonymity. Table 9.8 gives an
overview of these risk measures. The global risk is reduced to 0.02\%
(expected number of re-identifications 0.36), which is extremely low.
Therefore, we conclude that based on the categorical variables, the data
has been sufficiently anonymized. No household has a risk of
re-identification higher than 0.01 (1\%). By removing households with
rare values (or outliers) of the variable HHSIZE, we have reduced the
risk of spontaneous recognition of these households. This reasoning can
also be applied to the result of the risk of recoding the variable
LANDSIZEHA and PRAMming the variables identified to be important in the
nosy neighbor scenario. An intruder cannot know with certainty whether a
household that he recognizes in the data is the correct household, due
to the noise.

Table 9.8: Number and proportion of households violating k-anonymity
after anonymization


\begin{savenotes}\sphinxattablestart
\centering
\begin{tabulary}{\linewidth}[t]{|T|T|T|}
\hline
\sphinxstyletheadfamily 
\sphinxstylestrong{k-anonymity}
&\sphinxstyletheadfamily 
\sphinxstylestrong{Number HH violating}
&\sphinxstyletheadfamily 
\sphinxstylestrong{Percentage}
\\
\hline
2
&
0
&
0 \%
\\
\hline
3
&
104
&
5.28 \%
\\
\hline
5
&
374
&
18.70 \%
\\
\hline
\end{tabulary}
\par
\sphinxattableend\end{savenotes}

These measures refer only to the categorical variables. To evaluate the
risk of the continuous variables we could use an interval measure or
closest neighbor algorithm. These risk measures are discussed in Section
4.7. We chose to use an interval measure, since exact value matching is
not our largest concern based on the assumed scenarios and external data
sources. Instead, datasets with similar values but not the exact same
values could be used for matching. Here the main concern is that the
values are sufficiently far from the original values, which is measured
with an interval measure.

Example 9.20 shows how to evaluate the interval measure for each of the
expenditure variables, which are contained in the vector
\sphinxstyleemphasis{compExp} %
\begin{footnote}[2]\sphinxAtStartFootnote
For illustrative purposes, we only show this evaluation for the
expenditure variables. It can be easily copied for the income
variables. The results are similar.
%
\end{footnote}. The different values of the parameter
\sphinxstyleemphasis{k} in the function dRisk() define the size of the interval around the
original value, as explained in Section 4.7.2\sphinxstyleemphasis{.} The larger \sphinxstyleemphasis{k}, the
larger the intervals, the higher the probability that a perturbed value
is in the interval around the original value and the higher the risk
measure. The result is satisfactory with relatively small intervals (k =
0.01), but not when increasing the size of the intervals. In our case, k
= 0.01 is sufficiently large, since we are looking at the components,
not the aggregates. We have to pay special attention to the outliers.
Here the value 0.01 for k is too small to assume that they are protected
when outside this small interval. It would be necessary to check
outliers and their perturbed values and there might be a need for a
higher level of perturbation for outliers. We conclude that, from a risk
perspective and based on the interval measure, the chosen levels of
noise are acceptable. In the next step, we will look at the impact on
the data utility of the noise addition.

Example 9.20: Measuring risk of re-identification of continuous
variables

\sphinxstylestrong{dRisk}(sdcHH@origData{[},compExp{]}, xm = sdcHH@manipNumVars{[},compExp{]},
k = 0.01)

{[}1{]} 0.0608828

\sphinxstylestrong{dRisk}(sdcHH@origData{[},compExp{]}, xm = sdcHH@manipNumVars{[},compExp{]},
k = 0.02)

\sphinxstylestrong{{[}1{]} 0.9025875}

\sphinxstylestrong{dRisk}(sdcHH@origData{[},compExp{]}, xm = sdcHH@manipNumVars{[},compExp{]},
k = 0.05)

{[}1{]} 1

\sphinxstylestrong{Step 10a: Re-measure utility}

None of the variables has been recoded and the original level of detail
in the data is kept, except for the variable LANDSIZEHA. As described in
Step 8a, local suppression has only removed a few values in the other
variables, which has not greatly reduced the validity of the data.

The univariate frequency distributions of the variables ROOF, TOILET,
WATER, ELECTCON, FUELCOOK, OWNMOTORCYCLE, CAR, TV and LIVESTOCK did not,
by definition of the invariant PRAM method (see Section 5.3.1), change
to a large extent. The tabulations are presented in Table 9.9 (the
values 1 \textendash{} 9 and NA in the first row are the values of the variables and
.m after the variable name refers to the values after anonymization).
\sphinxstylestrong{NOTE: Although the frequencies are almost the same, this does not mean
that the values of particular households did not change.} Values have
been swapped between households. This becomes apparent when looking at
the multivariate frequencies of the WATER with the variable URBRUR in
Table 9.10. The multivariate frequencies of the PRAMmed with the
variable URBRUR could be of interest for users, but these are not
preserved. Since we applied PRAM within the regions, the multivariate
frequencies of the PRAMmed variables with REGION are preserved.

Table 9.9: Univariate frequencies of the PRAMmed variable before and
after anonymization


\begin{savenotes}\sphinxattablestart
\centering
\begin{tabulary}{\linewidth}[t]{|T|T|T|T|T|T|T|T|T|T|T|T|}
\hline
\sphinxstyletheadfamily &\sphinxstyletheadfamily 
\sphinxstylestrong{0}
&\sphinxstyletheadfamily 
\sphinxstylestrong{1}
&\sphinxstyletheadfamily 
\sphinxstylestrong{2}
&\sphinxstyletheadfamily 
\sphinxstylestrong{3}
&\sphinxstyletheadfamily 
\sphinxstylestrong{4}
&\sphinxstyletheadfamily 
\sphinxstylestrong{5}
&\sphinxstyletheadfamily 
\sphinxstylestrong{6}
&\sphinxstyletheadfamily 
\sphinxstylestrong{7}
&\sphinxstyletheadfamily 
\sphinxstylestrong{8}
&\sphinxstyletheadfamily 
\sphinxstylestrong{9}
&\sphinxstyletheadfamily 
\sphinxstylestrong{NA}
\\
\hline
ROOF
&&
27
&
1
&
914
&
307
&
711
&&&&
10
&
1
\\
\hline
ROOF.m
&&
25
&
1
&
907
&
319
&
712
&&&&
6
&
1
\\
\hline
TOILET
&&
76
&
594
&
817
&
481
&&&&&
3
&\\
\hline
TOILET.m
&&
71
&
597
&
816
&
483
&&&&&
4
&\\
\hline
WATER
&&
128
&
323
&
304
&
383
&
562
&
197
&
18
&
21
&
35
&\\
\hline
WATER.m
&&
134
&
319
&
308
&
378
&
573
&
188
&
16
&
21
&
34
&\\
\hline
ELECTCON
&
768
&
216
&
8
&
2
&&&&&&&
977
\\
\hline
ELECTCON.m
&
761
&
218
&
8
&
3
&&&&&&&
981
\\
\hline
FUELCOOK
&&
1289
&
21
&
376
&
55
&
36
&&&&
139
&
55
\\
\hline
FUELCOOK.m
&&
1284
&
22
&
383
&
50
&
39
&&&&
143
&
50
\\
\hline
OWNMOTORCYCLE
&
1883
&
86
&&&&&&&&&
2
\\
\hline
OWNMOTORCYCLE.m
&
1882
&
86
&&&&&&&&&
2
\\
\hline
CAR
&
963
&
31
&&&&&&&&&
977
\\
\hline
CAR.m
&
966
&
25
&&&&&&&&&\\
\hline
TV
&
1216
&
264
&&&&&&&&&
491
\\
\hline
TV.m
&
1203
&
272
&&&&&&&&&
496
\\
\hline
\end{tabulary}
\par
\sphinxattableend\end{savenotes}

Table 9.10: Multivariate frequencies of the variables WATER with RURURB
before and after anonymization


\begin{savenotes}\sphinxattablestart
\centering
\begin{tabulary}{\linewidth}[t]{|T|T|T|T|T|T|T|T|T|T|}
\hline
\sphinxstyletheadfamily &\sphinxstyletheadfamily 
\sphinxstylestrong{1}
&\sphinxstyletheadfamily 
\sphinxstylestrong{2}
&\sphinxstyletheadfamily 
\sphinxstylestrong{3}
&\sphinxstyletheadfamily 
\sphinxstylestrong{4}
&\sphinxstyletheadfamily 
\sphinxstylestrong{5}
&\sphinxstyletheadfamily 
\sphinxstylestrong{6}
&\sphinxstyletheadfamily 
\sphinxstylestrong{7}
&\sphinxstyletheadfamily 
\sphinxstylestrong{8}
&\sphinxstyletheadfamily 
\sphinxstylestrong{9}
\\
\hline
WATER/URB
&
11
&
49
&
270
&
306
&
432
&
183
&
12
&
15
&
21
\\
\hline
WATER/RUR
&
114
&
274
&
32
&
76
&
130
&
14
&
6
&
6
&
14
\\
\hline
WATER/URB.m
&
79
&
220
&
203
&
229
&
402
&
125
&
10
&
12
&
19
\\
\hline
WATER/RUR.m
&
54
&
98
&
105
&
147
&
169
&
63
&
6
&
9
&
15
\\
\hline
\end{tabulary}
\par
\sphinxattableend\end{savenotes}

For conciseness, we restrict ourselves to the analysis of the
expenditure variables. The analysis of the income variables can be done
in the same way and leads to similar results.

We look at the effect of anonymization on some indicators as discussed
in Step 5. Table 9.11 presents the point estimates and bootstrapped
confidence interval of the GINI coefficient %
\begin{footnote}[3]\sphinxAtStartFootnote
To compute the GINI coefficient, bootstrap to construct the
confidence intervals and plot the Lorenz curve we used the \sphinxstyleemphasis{R}
packages \sphinxstyleemphasis{laeken, reldist, bootstrap} and \sphinxstyleemphasis{ineq}.
%
\end{footnote} for
the sum of the expenditure components. The calculation of the GINI
coefficient and the confidence interval are based on the positive
expenditure values. We observe very small changes in the Gini
coefficient, that are statistically negligible. We use a visualization
to illustrate the impact on utility of the anonymization. Visualizations
are discussed in Section 6.4 and the specific \sphinxstyleemphasis{R} code for this case
study is available in the \sphinxstyleemphasis{R} script. The change in the inequality
measures is illustrated in Figure 9.1, which shows the Lorenz curves
based on the positive expenditure values before and after anonymization.

Table 9.11: GINI point estimates and bootstrapped confidence intervals
for sum of expenditure components


\begin{savenotes}\sphinxattablestart
\centering
\begin{tabulary}{\linewidth}[t]{|T|T|T|}
\hline
\sphinxstyletheadfamily &\sphinxstyletheadfamily 
\sphinxstylestrong{before}
&\sphinxstyletheadfamily 
\sphinxstylestrong{after}
\\
\hline
Point estimate
&
0.510
&
0.508
\\
\hline
Left bound of CI
&
0.476
&
0.476
\\
\hline
Right bound of CI
&
0.539
&
0.538
\\
\hline
\end{tabulary}
\par
\sphinxattableend\end{savenotes}

\noindent\sphinxincludegraphics[width=6.5in,height=3.25in]{{image21}.png}

Figure 9.1: Lorenz curve based on positive total expenditures values

We compare the mean monthly expenditures (MME) and mean monthly income
(MMI) for rural, urban and total population. The results are shown in
Table 9.12. We observe that the chosen levels of noise add only small
distortions to the MME and slightly larger changes to the MMI.

Table 9.12: Mean monthly expenditure and mean monthly income per capita
by rural/urban


\begin{savenotes}\sphinxattablestart
\centering
\begin{tabulary}{\linewidth}[t]{|T|T|T|}
\hline
\sphinxstyletheadfamily &\sphinxstyletheadfamily 
\sphinxstylestrong{before}
&\sphinxstyletheadfamily 
\sphinxstylestrong{after}
\\
\hline
MME rural
&
400.5
&
398.5
\\
\hline
MME urban
&
457.3
&
459.9
\\
\hline
MME total
&
412.6
&
412.6
\\
\hline
MMI rural
&
397.1
&
402.2
\\
\hline
MMI urban
&
747.6
&
767.8
\\
\hline
MMI total
&
472.1
&
478.5
\\
\hline
\end{tabulary}
\par
\sphinxattableend\end{savenotes}

Table 9.13 shows the share of each of the components of the expenditure
variables before and after anonymization.

Table 9.13 Shares of expenditures components


\begin{savenotes}\sphinxattablestart
\centering
\begin{tabulary}{\linewidth}[t]{|T|T|T|T|T|T|T|}
\hline
\sphinxstyletheadfamily &\sphinxstyletheadfamily 
\sphinxstylestrong{TFOOD
EXP}
&\sphinxstyletheadfamily 
\sphinxstylestrong{TALCH
EXP}
&\sphinxstyletheadfamily 
\sphinxstylestrong{TCLTH
EXP}
&\sphinxstyletheadfamily 
\sphinxstylestrong{THOUS
EXP}
&\sphinxstyletheadfamily 
\sphinxstylestrong{TFURN
EXP}
&\sphinxstyletheadfamily 
\sphinxstylestrong{THLTH
EXP}
\\
\hline
before
&
0.58
&
0.01
&
0.03
&
0.09
&
0.02
&
0.03
\\
\hline
after
&
0.59
&
0.01
&
0.03
&
0.09
&
0.02
&
0.03
\\
\hline&
\sphinxstylestrong{TTRAN
SEXP}
&
\sphinxstylestrong{TCOMM
EXP}
&
\sphinxstylestrong{TRECE
XP}
&
\sphinxstylestrong{TEDUE
XP}
&
\sphinxstylestrong{TRESH
OTEXP}
&
\sphinxstylestrong{TMISC
EXP}
\\
\hline
before
&
0.04
&
0.02
&
0.00
&
0.08
&
0.03
&
0.05
\\
\hline
after
&
0.04
&
0.02
&
0.00
&
0.08
&
0.03
&
0.05
\\
\hline
\end{tabulary}
\par
\sphinxattableend\end{savenotes}

Anonymization for the creation of a SUF will inevitably lead to some
degree of utility loss. It is important to describe this loss in the
external report, so that users are aware of the changes in the data.
This is described in Step 11 and presented in Appendix A. Appendix A
also shows summary statistics and tabulations of the household level
variables before and after anonymization.

\sphinxstylestrong{Merging the household- and individual-level variables}

The next step is to merge the treated household variables with the
untreated individual variables for the anonymization of the individual
level variables. Example 9.21 shows the steps to merge these files. This
also includes the selection of variables used in the anonymization of
the individual-level variables. We create the \sphinxstyleemphasis{sdcMicro} object for the
anonymization of the individual variables in the same way as for the
household variable in Example 9.9. Subsequently, we repeat Steps 6-10
for the individual-level variables.

Example 9.21: Merging the files with household and individual-level
variables and creating an \sphinxstyleemphasis{sdcMicro} object for the anonymization of the
individual-level variables

\sphinxstyleemphasis{\#\#\# Select variables (individual level)}

\begin{DUlineblock}{0em}
\item[] \sphinxstyleemphasis{\# Key variables (individual level)}
\item[] selectedKeyVarsIND = \sphinxstylestrong{c}(‘GENDER’, ‘REL’, ‘MARITAL’, ‘AGEYRS’,
‘EDUCY’, ‘ATSCHOOL’, ‘INDUSTRY1’) \sphinxstyleemphasis{\# list of selected key variables}
\item[] \sphinxstyleemphasis{\# Sample weight (WGTHH, individual weight)}
\item[] selectedWeightVarIND = \sphinxstylestrong{c}(‘WGTHH’)
\item[] \sphinxstyleemphasis{\# Household ID}
\item[] selectedHouseholdID = \sphinxstylestrong{c}(‘IDH’)
\item[] \sphinxstyleemphasis{\# No strata}
\item[] \sphinxstyleemphasis{\# Recombining anonymized HH datasets and individual level variables}
\item[] indVars \textless{}- \sphinxstylestrong{c}(“IDH”, “IDP”, selectedKeyVarsIND, “WGTHH”) \sphinxstyleemphasis{\# HID
and all non HH variables}
\item[] fileInd \textless{}- file{[}indVars{]} \sphinxstyleemphasis{\# subset of file without HHVars}
\end{DUlineblock}

HHmanip \textless{}- \sphinxstylestrong{extractManipData}(sdcHH) \sphinxstyleemphasis{\# manipulated variables HH}

HHmanip \textless{}- HHmanip{[}HHmanip{[},’IDH’{]} != 1782,{]}

fileCombined \textless{}- \sphinxstylestrong{merge}(HHmanip, fileInd, by.x=\sphinxstylestrong{c}(‘IDH’))

fileCombined \textless{}- fileCombined{[}\sphinxstylestrong{order}(fileCombined{[},’IDH’{]},
fileCombined{[},’IDP’{]}),{]}

\sphinxstylestrong{dim}(fileCombined)

\begin{DUlineblock}{0em}
\item[] \sphinxstyleemphasis{\# SDC objects with all variables and treated HH vars for
anonymization of individual level variables}
\item[] sdcCombined \textless{}- \sphinxstylestrong{createSdcObj}(dat = fileCombined, keyVars =
selectedKeyVarsIND, weightVar = selectedWeightVarIND, hhId =
selectedHouseholdID)
\end{DUlineblock}

\sphinxstylestrong{Step 6b: Assessing disclosure risk (individual level)}

All key variables at the individual level are categorical. Therefore, we
can use k-anonymity and the individual and global risk measures (see
Sections 4.5 and 4.8). The hierarchical risk is now of interest, given
the household structure in the dataset \sphinxstyleemphasis{fileCombined}, which includes
both household- and individual-level variables. The number of
individuals (absolute and relative) that violate k-anonymity at the
levels 2, 3 and 5 are shown in Table 9.14. \sphinxstylestrong{NOTE: k-anonymity does not
consider the household structure and therefore underestimates the risk.
Therefore, we are more interested in the individual and global
hierarchical risk measures.}

Table 9.14: k-anonymity violations


\begin{savenotes}\sphinxattablestart
\centering
\begin{tabulary}{\linewidth}[t]{|T|T|T|}
\hline
\sphinxstyletheadfamily 
\sphinxstylestrong{k-anonymity}
&\sphinxstyletheadfamily 
\sphinxstylestrong{Number HH violating}
&\sphinxstyletheadfamily 
\sphinxstylestrong{Percentage}
\\
\hline
2
&
998
&
9.91\%
\\
\hline
3
&
1,384
&
13.75\%
\\
\hline
5
&
2,194
&
21.79\%
\\
\hline
\end{tabulary}
\par
\sphinxattableend\end{savenotes}

The global risk measures can be found using \sphinxstyleemphasis{R} as illustrated in
Example 9.22. The global risk is 0.24\%, which corresponds to 24 expected
re-identifications. Accounting for the hierarchical structure, this
rises to 1.26\%, or 127 expected re-identifications. The global risk
measures are low compared to the number of k-anonymity violators due to
the low sampling weights. The high number of k-anonymity violators is
mainly due to the very detailed age variable. The risk measures are
based only on the individual level variables, since we assume that the
individual and household level variables are be used simultaneously by
an intruder. If we would consider an intruder scenario where these
variables are used simultaneously by an intruder to re-identify
individuals, the household level variables should also be taken into
account here. This would results in a high number of key variables.

Example 9.22: Global risk of the individual-level variables

\sphinxstylestrong{print}(sdcCombined, ‘risk’)

\begin{DUlineblock}{0em}
\item[] \sphinxcode{\sphinxupquote{\#\# Risk measures:}}
\item[] \sphinxcode{\sphinxupquote{\#\#}}
\item[] \sphinxcode{\sphinxupquote{\#\# Number of observations with higher risk than the main part of the data: 0}}
\item[] \sphinxcode{\sphinxupquote{\#\# Expected number of re-identifications: 23.98 (0.24 \%)}}
\item[] \sphinxcode{\sphinxupquote{\#\#}}
\item[] \sphinxcode{\sphinxupquote{\#\# Information on hierarchical risk:}}
\item[] \sphinxcode{\sphinxupquote{\#\# Expected number of re-identifications: 127.12 (1.26 \%)}}
\end{DUlineblock}

\sphinxstylestrong{Step 7b: Assessing utility (individual level)}

We evaluate the utility measures selected in Step 5 besides some general
utility measures. The values computed from the raw data are presented in
step 10b to allow for direct comparison with the values computed from
the anonymized data.

\sphinxstylestrong{Step 8b: Choice and application of SDC methods (individual level)}

We use the same approach for the anonymization of the individual-level
categorical key variables as for the household level categorical
variables described earlier: first use global recoding to limit the
necessary number of suppressions, then apply local suppressions and
finally, if necessary, use of perturbative methods.

The variable AGEYRS (i.e., age in years) has many different values (age
in months for children 0 \textendash{} 1 years and age in years for individuals over
1 year). This level of detail leads to a high level of re-identification
risk, given external datasets with exact age as well as knowledge of the
exact age of close relatives. We have to reduce the level of detail in
the age variables by recoding the age values (see Section 5.2.1 on
recoding). First, we recode the values from 15 to 65 in ten-year
intervals. Since some indicators related to education are computed from
the survey dataset, our first approach is not to recode the age range 0
\textendash{} 15 years. For children under the age of 1 year, we reduce the level of
detail and recode these to 0 years. These recodes are shown in Example
9.23. We also top-code age at the age of 65 years. This protects
individuals with high (rare) age values.

Example 9.23: Recoding age in 10-year intervals in the range 15 \textendash{} 65 and
top code age over 65 years

\sphinxstyleemphasis{\# Recoding age and top coding age (top code 65), below that 10 year age
groups, children aged under 1 are recoded 0 (previously in months)}

sdcCombined@manipKeyVars\$AGEYRS{[}\sphinxhref{mailto:sdcCombined@manipKeyVars\$AGEYRS}{sdcCombined@manipKeyVars\$AGEYRS} \textgreater{}= 0 \&
\sphinxhref{mailto:sdcCombined@manipKeyVars\$AGEYRS}{sdcCombined@manipKeyVars\$AGEYRS} \textless{} 1{]} \textless{}- 0

sdcCombined@manipKeyVars\$AGEYRS{[}\sphinxhref{mailto:sdcCombined@manipKeyVars\$AGEYRS}{sdcCombined@manipKeyVars\$AGEYRS} \textgreater{}= 15 \&
\sphinxhref{mailto:sdcCombined@manipKeyVars\$AGEYRS}{sdcCombined@manipKeyVars\$AGEYRS} \textless{} 25{]} \textless{}- 20

…

sdcCombined@manipKeyVars\$AGEYRS{[}\sphinxhref{mailto:sdcCombined@manipKeyVars\$AGEYRS}{sdcCombined@manipKeyVars\$AGEYRS} \textgreater{}= 55 \&
\sphinxhref{mailto:sdcCombined@manipKeyVars\$AGEYRS}{sdcCombined@manipKeyVars\$AGEYRS} \textless{} 66{]} \textless{}- 60

\begin{DUlineblock}{0em}
\item[] \sphinxstyleemphasis{\#topBotCoding also recalculates risk based on manual recoding above}
\item[] sdcCombined \textless{}- \sphinxstylestrong{topBotCoding}(obj = sdcCombined, value = 65,
replacement = 65, kind = ‘top’, column = ‘AGEYRS’)
\end{DUlineblock}

\begin{DUlineblock}{0em}
\item[] \sphinxstylestrong{table}(\sphinxhref{mailto:sdcCombined@manipKeyVars\$AGEYRS}{sdcCombined@manipKeyVars\$AGEYRS}) \sphinxstyleemphasis{\# check results}
\item[] \sphinxcode{\sphinxupquote{\#\#    0    1    2    3    4    5    6    7    8    9   10   11   12   13   14}}
\item[] \sphinxcode{\sphinxupquote{\#\#  311  367  340  332  260  334  344  297  344  281  336  297  326  299  263}}
\item[] \sphinxcode{\sphinxupquote{\#\#   20   30   40   50   60   65}}
\item[] \sphinxcode{\sphinxupquote{\#\# 1847 1220  889  554  314  325}}
\end{DUlineblock}

These recodes already reduce the risk to 531 individuals violating
3-anonymity. We could recode the values of age in the lower range
according to the age categories users require (e.g., 8 \textendash{} 11 for
education). There are many different categories for different
indicators, however, including education indicators. This would reduce
the utility of the data for some users. Therefore, we decide to look
first at the number of suppressions needed in local suppression after
this limited recoding. If the number of suppressions is too high, we can
go back and recode age in the range 1 \textendash{} 14 years.

In Example 9.24 we demonstrate how one might experiment with local
suppression to find the best option. We use local suppression to achieve
3-anonymity (see Section 5.2.2 on local suppression). On the first
attempt, we do not specify any importance vector; this leads to many
suppressions in the variable AGEYRS (see Table 9.15 below, first row),
however. This is undesirable from a utility point of view. Therefore, we
decide to specify an importance vector to prevent suppressions in the
variable AGEYRS. Suppressing the variable GENDER is also undesirable
from the utility point of view. The variable GENDER is a type of
variable that should not have suppressions. We set GENDER as variable
with the second highest importance. After specifying the importance
vector to prevent suppressions of the age variable, there are no age
suppressions (see Table 9.15, second row). The total number of
suppressions in the other variables increased, however, from 253 to 323
because of the importance vector. This is to be expected because the
algorithm without the importance vector minimizes the total number of
suppressions by first suppressing values in variables with many
categories \textendash{} in this case, age and gender. Specifying an importance
vector prevents reaching this optimality and hence leads to a higher
total number of suppressions. There is a trade-off between which
variables are suppressed and the total number of suppressions. After
specifying an importance vector, the variable REL has many suppressions
(see Table 9.15, second row). We choose this second option.

Example 9.24: Experimenting with different options in local suppression

\begin{DUlineblock}{0em}
\item[] \sphinxstyleemphasis{\# Copy of sdcMicro object to later undo steps}
\item[] sdcCopy \textless{}- sdcCombined
\end{DUlineblock}

\begin{DUlineblock}{0em}
\item[] \sphinxstyleemphasis{\# Importance vectors for local suppression (depending on utility
measures)}
\item[] impVec1 \textless{}- NULL \sphinxstyleemphasis{\# for optimal suppression}
\item[] impVec2 \textless{}- \sphinxstylestrong{rep}(\sphinxstylestrong{length}(selectedKeyVarsIND),
\sphinxstylestrong{length}(selectedKeyVarsIND))
\item[] impVec2{[}\sphinxstylestrong{match}(‘AGEYRS’, selectedKeyVarsIND){]} \textless{}- 1 \sphinxstyleemphasis{\# AGEYRS}
\item[] impVec2{[}\sphinxstylestrong{match}(‘GENDER’, selectedKeyVarsIND){]} \textless{}- 2 \sphinxstyleemphasis{\# GENDER}
\end{DUlineblock}

\begin{DUlineblock}{0em}
\item[] \sphinxstyleemphasis{\# Local suppression without importance vector}
\item[] sdcCombined \textless{}- \sphinxstylestrong{localSuppression}(sdcCombined, k = 2, importance =
impVec1)
\end{DUlineblock}

\begin{DUlineblock}{0em}
\item[] \sphinxstyleemphasis{\# Number of suppressions per variable}
\item[] \sphinxstylestrong{print}(sdcCombined, “ls”)
\end{DUlineblock}

\begin{DUlineblock}{0em}
\item[] \sphinxcode{\sphinxupquote{\#\# Local Suppression:}}
\item[] \sphinxcode{\sphinxupquote{\#\#       KeyVar \textbar{} Suppressions (\#) \textbar{} Suppressions (\%)}}
\item[] \sphinxcode{\sphinxupquote{\#\#       GENDER \textbar{}                0 \textbar{}            0.000}}
\item[] \sphinxcode{\sphinxupquote{\#\#          REL \textbar{}               34 \textbar{}            0.338}}
\item[] \sphinxcode{\sphinxupquote{\#\#      MARITAL \textbar{}                0 \textbar{}            0.000}}
\item[] \sphinxcode{\sphinxupquote{\#\#       AGEYRS \textbar{}              195 \textbar{}            1.937}}
\item[] \sphinxcode{\sphinxupquote{\#\#        EDUCY \textbar{}                0 \textbar{}            0.000}}
\item[] \sphinxcode{\sphinxupquote{\#\#  EDYRSCURRAT \textbar{}                3 \textbar{}            0.030}}
\item[] \sphinxcode{\sphinxupquote{\#\#     ATSCHOOL \textbar{}                0 \textbar{}            0.000}}
\item[] \sphinxcode{\sphinxupquote{\#\#    INDUSTRY1 \textbar{}               21 \textbar{}            0.209}}
\end{DUlineblock}

\begin{DUlineblock}{0em}
\item[] \sphinxstyleemphasis{\# Number of suppressions per variable for each value of AGEYRS}
\item[] \sphinxstylestrong{table}(\sphinxhref{mailto:sdcCopy@manipKeyVars\$AGEYRS}{sdcCopy@manipKeyVars\$AGEYRS}) -
\sphinxstylestrong{table}(\sphinxhref{mailto:sdcCombined@manipKeyVars\$AGEYRS}{sdcCombined@manipKeyVars\$AGEYRS})
\end{DUlineblock}

\begin{DUlineblock}{0em}
\item[] \sphinxcode{\sphinxupquote{\#\#  0  1  2  3  4  5  6  7  8  9 10 11 12 13 14 20 30 40 50 60 65}}
\item[] \sphinxcode{\sphinxupquote{\#\#  0  0  0  0  0  0  2  0  2  1  0  1  4  1  5 25 53 37 36 15 13}}
\end{DUlineblock}

\begin{DUlineblock}{0em}
\item[] \sphinxstyleemphasis{\# Undo local suppression}
\item[] sdcCombined \textless{}- \sphinxstylestrong{undolast}(sdcCombined)
\end{DUlineblock}

\begin{DUlineblock}{0em}
\item[] \sphinxstyleemphasis{\# Local suppression with importance vector on AGEYRS and GENDER}
\item[] sdcCombined \textless{}- \sphinxstylestrong{localSuppression}(sdcCombined, k = 2, importance =
impVec2)
\end{DUlineblock}

\begin{DUlineblock}{0em}
\item[] \sphinxstyleemphasis{\# Number of suppressions per variable}
\item[] \sphinxstylestrong{print}(sdcCombined, “ls”)
\end{DUlineblock}

\begin{DUlineblock}{0em}
\item[] \sphinxcode{\sphinxupquote{\#\# Local Suppression:}}
\item[] \sphinxcode{\sphinxupquote{\#\#       KeyVar \textbar{} Suppressions (\#) \textbar{} Suppressions (\%)}}
\item[] \sphinxcode{\sphinxupquote{\#\#       GENDER \textbar{}                0 \textbar{}            0.000}}
\item[] \sphinxcode{\sphinxupquote{\#\#          REL \textbar{}              323 \textbar{}            3.208}}
\item[] \sphinxcode{\sphinxupquote{\#\#      MARITAL \textbar{}                0 \textbar{}            0.000}}
\item[] \sphinxcode{\sphinxupquote{\#\#       AGEYRS \textbar{}                0 \textbar{}            0.000}}
\item[] \sphinxcode{\sphinxupquote{\#\#        EDUCY \textbar{}                0 \textbar{}            0.000}}
\item[] \sphinxcode{\sphinxupquote{\#\#  EDYRSCURRAT \textbar{}                0 \textbar{}            0.000}}
\item[] \sphinxcode{\sphinxupquote{\#\#     ATSCHOOL \textbar{}                0 \textbar{}            0.000}}
\item[] \sphinxcode{\sphinxupquote{\#\#    INDUSTRY1 \textbar{}                0 \textbar{}            0.000}}
\end{DUlineblock}

\begin{DUlineblock}{0em}
\item[] \sphinxstyleemphasis{\# Number of suppressions for each value of the variable AGEYRS}
\item[] \sphinxstylestrong{table}(\sphinxhref{mailto:sdcCopy@manipKeyVars\$AGEYRS}{sdcCopy@manipKeyVars\$AGEYRS}) -
\sphinxstylestrong{table}(\sphinxhref{mailto:sdcCombined@manipKeyVars\$AGEYRS}{sdcCombined@manipKeyVars\$AGEYRS})
\end{DUlineblock}

\begin{DUlineblock}{0em}
\item[] \sphinxcode{\sphinxupquote{\#\#  0  1  2  3  4  5  6  7  8  9 10 11 12 13 14 20 30 40 50 60 65}}
\item[] \sphinxcode{\sphinxupquote{\#\#  0  0  0  0  0  0  0  0  0  0  0  0  0  0  0  0  0  0  0  0  0}}
\end{DUlineblock}

Table 9.15: Number of suppressions by variable for different variations
of local suppression


\begin{savenotes}\sphinxattablestart
\centering
\begin{tabular}[t]{|*{9}{\X{1}{9}|}}
\hline
\sphinxstyletheadfamily 
Loc
al
suppr
essio
n
optio
ns
&\sphinxstyletheadfamily 
GENDER
&\sphinxstyletheadfamily 
REL
&\sphinxstyletheadfamily 
MARITAL
&\sphinxstyletheadfamily 
AGEYRS
&\sphinxstyletheadfamily 
EDUCY
&\sphinxstyletheadfamily \begin{description}
\item[{EDYRS}] \leavevmode
CURRAT

\end{description}
&\sphinxstyletheadfamily 
ATS
CHOOL
&\sphinxstyletheadfamily 
IND
USTRY1
\\
\hline
k = 2,
no imp
&
0
&
34
&
0
&
195
&
0
&
3
&
0
&
21
\\
\hline
k = 2,
imp
on
AGEYRS
&
0
&
323
&
0
&
0
&
0
&
0
&
0
&
0
\\
\hline
\end{tabular}
\par
\sphinxattableend\end{savenotes}

\sphinxstylestrong{Step 9b: Re-measure risk (individual level)}

We re-evaluate the risk measures selected in Step 6b. Table 9.16 shows
that local suppression, not surprisingly, has reduced the number of
individuals violating 2-anonymity to 0.

Table 9.16: k-anonymity violations


\begin{savenotes}\sphinxattablestart
\centering
\begin{tabulary}{\linewidth}[t]{|T|T|T|}
\hline
\sphinxstyletheadfamily 
\sphinxstylestrong{k-anonymity}
&\sphinxstyletheadfamily 
\sphinxstylestrong{Number HH violating}
&\sphinxstyletheadfamily 
\sphinxstylestrong{Percentage}
\\
\hline
2
&
0
&
0.00 \%
\\
\hline
3
&
197
&
1.96 \%
\\
\hline
5
&
518
&
5.15 \%
\\
\hline
\end{tabulary}
\par
\sphinxattableend\end{savenotes}

The hierarchical global risk was reduced to 0.11\%, which corresponds to
11.3 expected re-identifications. The highest individual hierarchical
re-identification risk is 1.21\%. These risk levels would seem acceptable
for a SUF.

\sphinxstylestrong{Step 10b: Re-measure utility (individual level)}

We selected two utility measures for the individual variables: primary
and secondary education enrollment, both also by gender. These two
measures are sensitive to changes in the variables gender (GENDER), age
(AGEYRS) and education (EDUCY and EDYRSATCURR), and therefore give a
good overview of the impact of the anonymization. As shown in Table 9.17
the anonymization did not change the results. The results of the
tabulations in Appendix A confirm these results.

Table 9.17: Net enrollment in primary and secondary education by gender


\begin{savenotes}\sphinxattablestart
\centering
\begin{tabulary}{\linewidth}[t]{|T|T|T|T|T|T|T|}
\hline
\sphinxstyletheadfamily &\sphinxstartmulticolumn{3}%
\begin{varwidth}[t]{\sphinxcolwidth{3}{7}}
\sphinxstyletheadfamily Primary education
\par
\vskip-\baselineskip\vbox{\hbox{\strut}}\end{varwidth}%
\sphinxstopmulticolumn
&\sphinxstartmulticolumn{3}%
\begin{varwidth}[t]{\sphinxcolwidth{3}{7}}
\sphinxstyletheadfamily Secondary education
\par
\vskip-\baselineskip\vbox{\hbox{\strut}}\end{varwidth}%
\sphinxstopmulticolumn
\\
\hline&
Total
&
Male
&
Female
&
Total
&
Male
&
Female
\\
\hline
Before
&
72.6\%
&
74.2\%
&
70.9\%
&
42.0\%
&
44.8\%
&
39.1\%
\\
\hline
After
&
72.6\%
&
74.2\%
&
70.9\%
&
42.0\%
&
44.8\%
&
39.1\%
\\
\hline
\end{tabulary}
\par
\sphinxattableend\end{savenotes}

\sphinxstylestrong{Step 11: Audit and reporting}

In the audit step, we check whether the data allow for reproduction of
published figures from the original dataset and relationships between
variables and other data characteristics are preserved in the
anonymization process. In short, we check whether the dataset is valid
for analytical purposes. There are no figures available that were
published from the dataset and need to be reproducible from the
anonymized data.

In Step 2, we explored the data characteristics and relationships
between variables. These data characteristics and relationships have
been mainly preserved, since we took them into account when choosing the
appropriate anonymization methods. The variables TANHHEXP and
INCTOTGROSSHH are the sums of the individual components, because we
added noise to the components and reconstructed the aggregates by
summing over the components. Initially, the income variables were all
positive. This characteristic has been violated, as a result of noise
addition. Since values of the variable AGEYRS were not perturbed, but
only recoded and suppressed, we did not introduce unlikely combinations,
such as a 60-year-old individual enrolled in primary education. Also, by
separating the anonymization process into two parts, one for
household-level variables and one for individual-level variables, the
values of variables measured at the household level agree for all
members of each household.

Furthermore, we drafted two reports, internal and external, on the
anonymization of the case study dataset. The internal report includes
the methods used, the risk before and after anonymization as well as the
reasons for the selected methods and their parameters. The external
report focuses on the changes in the data and the loss in utility. Focus
here should be on the number of suppressions as well as the perturbative
methods (PRAM). This is described in the previous steps. \sphinxstylestrong{NOTE: When
creating a SUF, it is inevitable that there will be a loss of
information and it is very important for the users to be aware of these
changes and release them in a report that accompanies the data}.
Appendix A provides examples of an internal and external report of the
anonymization process of this dataset. Depending on the users and
readers of the reports, the content may differ. The code to this case
study shows how to obtain the information for the reports. Some measures
are also available in the standard reports generated with the report()
function. This is shown in Example 9.25. The report() function will only
use the data available in the \sphinxstyleemphasis{sdcMicro} object, which does not contain
all households for sdcHH.

Example 9.25: Using the report() function for internal and external
reports

\begin{DUlineblock}{0em}
\item[] \sphinxstyleemphasis{\# Create reports with sdcMicro report() function}
\item[] \sphinxstylestrong{report}(sdcHH, internal = F) \sphinxstyleemphasis{\# external (brief) report}
\end{DUlineblock}

\sphinxstylestrong{report}(sdcHH, internal = T) \sphinxstyleemphasis{\# internal (extended) report}

\begin{DUlineblock}{0em}
\item[] \sphinxstyleemphasis{\# Create reports with sdcMicro report() function}
\item[] \sphinxstylestrong{report}(sdcCombined, internal = F) \sphinxstyleemphasis{\# external (brief) report}
\end{DUlineblock}

\sphinxstylestrong{report}(sdcCombined, internal = T) \sphinxstyleemphasis{\# internal (extended) report}

\sphinxstylestrong{Step 12: Data release}

The final step is the release of the anonymized dataset together with
the external report. Example 9.26 shows how to collect the data from the
\sphinxstyleemphasis{sdcMicro} object with the extractManipData() function. Before releasing
the file, we add an individual ID to the file (line number in
household). We export the anonymized dataset in as \sphinxstyleemphasis{STATA} file. Section
7.2 presents functions for exporting files in other data formats.

Example 9.26: Exporting the anonymized dataset

\begin{DUlineblock}{0em}
\item[] \sphinxstyleemphasis{\# Anonymized dataset}
\item[] \sphinxstyleemphasis{\# Household variables and individual variables}
\item[] dataAnon \textless{}- \sphinxstylestrong{extractManipData}(sdcCombined, ignoreKeyVars = F,
ignorePramVars = F, ignoreNumVars = F, ignoreStrataVar = F) \sphinxstyleemphasis{\#
extracts all variables, not just the manipulated ones}
\end{DUlineblock}

\begin{DUlineblock}{0em}
\item[] \sphinxstyleemphasis{\# Create STATA file}
\item[] \sphinxstylestrong{write.dta}(dataframe = dataAnon, file= ‘Case1DataAnon.dta’,
convert.dates=TRUE)
\end{DUlineblock}


\section{Case study 2 - PUF}
\label{\detokenize{case_studies:case-study-2-puf}}
This case study is a continuation of case study 1 in Section 9.1. Case
study 1 produces a SUF file. In this case study we use this SUF file to
produce a PUF file of the same dataset, which can be freely distributed.
The structure of the SUF and PUF releases will be the same. However, the
PUF will contain fewer variables and less (detailed) information than
the SUF. We refer to Section 9.1 for a description of the dataset.
\sphinxstylestrong{NOTE}: \sphinxstylestrong{It is also possible to directly produce a PUF from a dataset
without first creating a SUF.}

As in case study 1, we show how the creation of a PUF can be achieved
using the open source and free \sphinxstyleemphasis{sdcMicro} package and \sphinxstyleemphasis{R}. A
ready-to-run \sphinxstyleemphasis{R} script for this case study and the dataset are also
available to reproduce the results and allow the user to adapt the code
(see \sphinxurl{http://ihsn.org/home/projects/sdc-practice}). Extracts of this code
are presented in this section to illustrate several steps of the
anonymization process. \sphinxstylestrong{NOTE: The choices of methods and parameters in
this case study are based on this particular dataset and the results and
choices might be different for other datasets.}

This case study follows the steps of the SDC process outlined in Chapter
8.

\sphinxstylestrong{Step 1: Need for disclosure control}

The same reasoning as in case study 1 applies: the SUF dataset produced
in case study 1 contains data on individuals and households and some
variables are confidential and/or sensitive. The decisions made in case
study 1 are based on the disclosure scenarios for a SUF release. The
anonymization applied for the SUF does not provide sufficient protection
for the release as PUF and the SUF file cannot be released as PUF
without further treatment. Therefore, we have to repeat the SDC process
with a different set of disclosure scenarios based on the
characteristics of a PUF release (see Step 4). This leads to different
risk measures, lower accepted risk levels and different SDC methods.

\sphinxstylestrong{Step 2: Data preparation and exploring data characteristics}

In order to guarantee consistency between the released PUF and SUF
files, which is required to prevent intruders from using the datasets
together (SUF users have also access to the PUF file), we have to use
the anonymized SUF file to create the PUF file (see also Section 8.3).
In this way all information in the PUF file is also contained in the
SUF, and the PUF does not provide additional information to an intruder
with access to the SUF. We load the required packages to read the data
(\sphinxstyleemphasis{foreign} package for \sphinxstyleemphasis{STATA} files) and load the SUF dataset into
“file” as illustrated in Example 9.27. We also load the original data
file (raw data) as “fileOrig”. We need the raw data to undo perturbative
methods used in case study 1 (see Step 8) and to compare data utility
measures (see Step 5). To evaluate the utility loss in the PUF, we have
to compare the information in the anonymized PUF file with the
information in the raw data. For an overview of the data characteristics
and a description of the variables in both files, we refer to Step 2 of
case study 1 in Section 9.1.

Example 9.27: Loading required packages and datasets

\begin{DUlineblock}{0em}
\item[] \sphinxstyleemphasis{\# Load required packages}
\item[] \sphinxstylestrong{library}(foreign) \sphinxstyleemphasis{\# for read/write function for STATA}
\item[] \sphinxstylestrong{library}(sdcMicro) \sphinxstyleemphasis{\# sdcMicro package}
\end{DUlineblock}

\begin{DUlineblock}{0em}
\item[] \sphinxstyleemphasis{\# Set working directory - set to the path on your machine}
\item[] \sphinxstylestrong{setwd}(“/Users/CaseStudy2”)
\end{DUlineblock}

\begin{DUlineblock}{0em}
\item[] \sphinxstyleemphasis{\# Specify file name of SUF file from case study 1}
\item[] fname \textless{}- “CaseDataAnon.dta”
\end{DUlineblock}

\begin{DUlineblock}{0em}
\item[] \sphinxstyleemphasis{\# Specify file name of original dataset (raw data)}
\item[] fnameOrig \textless{}- “CaseA.dta”
\end{DUlineblock}

\begin{DUlineblock}{0em}
\item[] \sphinxstyleemphasis{\# Read-in files}
\item[] file \textless{}- \sphinxstylestrong{read.dta}(fname, convert.factors = TRUE) \sphinxstyleemphasis{\# SUF file case
study 1
*fileOrig \textless{}- **read.dta*}(fnameOrig, convert.factors = TRUE) \sphinxstyleemphasis{\#
original data}
\end{DUlineblock}

We check the number of variables and number of observations of both
files and the variable names of the SUF file, as shown in Example 9.28.
The PUF file has fewer records and fewer variables than the original
data file, since we removed large households and several variables to
generate the SUF file.

Example 9.28 Number of individuals and variables and variable names

\sphinxstyleemphasis{\# Dimensions of file (observations, variables)}

\sphinxstylestrong{dim}(file)

\sphinxcode{\sphinxupquote{\#\# {[}1{]} 10068    49}}

\sphinxstylestrong{dim}(fileOrig)

\sphinxcode{\sphinxupquote{\#\# {[}1{]} 10574    68}}

\sphinxstylestrong{colnames}(file) \sphinxstyleemphasis{\# Variable names}

\begin{DUlineblock}{0em}
\item[] \sphinxcode{\sphinxupquote{\#\#  {[}1{]} "IDH"           "URBRUR"        "REGION"        "HHSIZE"}}
\item[] \sphinxcode{\sphinxupquote{\#\#  {[}5{]} "OWNAGLAND"     "RELIG"         "ROOF"          "TOILET"}}
\item[] \sphinxcode{\sphinxupquote{\#\#  {[}9{]} "WATER"         "ELECTCON"      "FUELCOOK"      "OWNMOTORCYCLE"}}
\item[] \sphinxcode{\sphinxupquote{\#\# {[}13{]} "CAR"           "TV"            "LIVESTOCK"     "LANDSIZEHA"}}
\item[] \sphinxcode{\sphinxupquote{\#\# {[}17{]} "TANHHEXP"      "TFOODEXP"      "TALCHEXP"      "TCLTHEXP"}}
\item[] \sphinxcode{\sphinxupquote{\#\# {[}21{]} "THOUSEXP"      "TFURNEXP"      "THLTHEXP"      "TTRANSEXP"}}
\item[] \sphinxcode{\sphinxupquote{\#\# {[}25{]} "TCOMMEXP"      "TRECEXP"       "TEDUEXP"       "TRESTHOTEXP"}}
\item[] \sphinxcode{\sphinxupquote{\#\# {[}29{]} "TMISCEXP"      "INCTOTGROSSHH" "INCRMT"        "INCWAGE"}}
\item[] \sphinxcode{\sphinxupquote{\#\# {[}33{]} "INCFARMBSN"    "INCNFARMBSN"   "INCRENT"       "INCFIN"}}
\item[] \sphinxcode{\sphinxupquote{\#\# {[}37{]} "INCPENSN"      "INCOTHER"      "WGTPOP"        "IDP"}}
\item[] \sphinxcode{\sphinxupquote{\#\# {[}41{]} "GENDER"        "REL"           "MARITAL"       "AGEYRS"}}
\item[] \sphinxcode{\sphinxupquote{\#\# {[}45{]} "EDUCY"         "EDYRSCURRAT"   "ATSCHOOL"      "INDUSTRY1"}}
\item[] \sphinxcode{\sphinxupquote{\#\# {[}49{]} "WGTHH"}}
\end{DUlineblock}

To get an overview of the values of the variables, we use tabulations
and cross-tabulations for categorical variables and summary statistics
for continuous variables. To include the number of missing values (‘NA’
or other), we use the option useNA = “ifany” in the table() function.
For some variables, these tabulations differ from the tabulations of the
raw data, due to the anonymization of the SUF file.

In Table 9.18 the variables in the dataset “file” are listed along with
concise descriptions of the variables, the level at which they are
collected (individual level (IND) or household level (HH)), the
measurement type (continuous, semi-continuous or categorical) and value
ranges. Note that the dataset contains a selection of 49 variables of
the 68 variable selected for the SUF release. The variables have been
preselected based on their relevance for data users and some variables
were removed while creating a SUF file. The numerical values for many of
the categorical variables are codes that refer to values, e.g., in the
variable “URBRUR”, 1 stands for ‘rural’ and 2 for ‘urban’. More
information on the meanings of coded values of the categorical variables
is available in the \sphinxstyleemphasis{R} code for this case study.

Any data cleaning, such as recoding missing value codes and removing
empty variables, was already done in case study 1. The same holds for
removing any direct identifiers. Direct identifiers are not released in
the SUF file.

We identified the following sensitive variables in the dataset:
variables related to schooling and labor force status as well as income
and expenditure related variables. These variables need protection.
Whether a variable is considered sensitive may depend on the release
type, country and the dataset itself.

Table 9.18: Overview of the variables in the dataset


\begin{savenotes}\sphinxatlongtablestart\begin{longtable}{|*{6}{\X{1}{6}|}}
\hline
\sphinxstyletheadfamily 
No.
&\sphinxstyletheadfamily \begin{description}
\item[{Variable}] \leavevmode
name

\end{description}
&\sphinxstyletheadfamily 
Description
&\sphinxstyletheadfamily 
Level
&\sphinxstyletheadfamily 
Measurement
&\sphinxstyletheadfamily 
Values
\\
\hline
\endfirsthead

\multicolumn{6}{c}%
{\makebox[0pt]{\sphinxtablecontinued{\tablename\ \thetable{} -- continued from previous page}}}\\
\hline
\sphinxstyletheadfamily 
No.
&\sphinxstyletheadfamily \begin{description}
\item[{Variable}] \leavevmode
name

\end{description}
&\sphinxstyletheadfamily 
Description
&\sphinxstyletheadfamily 
Level
&\sphinxstyletheadfamily 
Measurement
&\sphinxstyletheadfamily 
Values
\\
\hline
\endhead

\hline
\multicolumn{6}{r}{\makebox[0pt][r]{\sphinxtablecontinued{Continued on next page}}}\\
\endfoot

\endlastfoot

1
&
IDH
&
Household
ID
&
HH
&\begin{itemize}
\item {} 
\end{itemize}
&
1-2,000
\\
\hline
2
&
IDP
&
Individua
l
ID
&
IND
&\begin{itemize}
\item {} 
\end{itemize}
&
1-13
\\
\hline
3
&
REGION
&
Region
&
HH
&
categoric
al
&
1-6
\\
\hline
4
&
URBRUR
&
Area of
residence
&
HH
&
categoric
al
&
1, 2
\\
\hline
5
&
WGTHH
&
Individua
l
weighting
coefficie
nt
&
HH
&
weight
&
31.2-8495
.7
\\
\hline
6
&
WGTPOP
&
Populatio
n
weighting
coefficie
nt
&
IND
&
weight
&
45.8-9345
2.2
\\
\hline
7
&
HHSIZE
&
Household
size
&
HH
&
semi-cont
inuous
&
1-33
\\
\hline
8
&
GENDER
&
Gender
&
IND
&
categoric
al
&
0, 1
\\
\hline
9
&
REL
&
Relations
hip
to
household
head
&
IND
&
categoric
al
&
1-9
\\
\hline
10
&
MARITAL
&
Marital
status
&
IND
&
categoric
al
&
1-6
\\
\hline
11
&
AGEYRS
&
Age in
completed
years
&
IND
&
semi-cont
inuous
&
0-65
\\
\hline
12
&
RELIG
&
Religion
of
household
head
&
HH
&
categoric
al
&
1, 5-7, 9
\\
\hline
13
&
ATSCHOOL
&
Currently
enrolled
in school
&
IND
&
categoric
al
&
0, 1
\\
\hline
14
&
EDUCY
&
Highest
level of
education
attended
&
IND
&
categoric
al
&
1-6
\\
\hline
15
&
EDYRSCURR
AT
&
Years of
education
for
currently
enrolled
&
IND
&
semi-cont
inuous
&
1-18
\\
\hline
16
&
INDUSTRY1
&
Industry
classific
ation
(1-digit)
&
IND
&
categoric
al
&
1-10
\\
\hline
17
&
ROOF
&
Main
material
used for
roof
&
IND
&
categoric
al
&
1-5, 9
\\
\hline
18
&
TOILET
&
Main
toilet
facility
&
HH
&
categoric
al
&
1-4, 9
\\
\hline
19
&
ELECTCON
&
Electrici
ty
&
HH
&
categoric
al
&
0-3
\\
\hline
20
&
FUELCOOK
&
Main
cooking
fuel
&
HH
&
categoric
al
&
1-5, 9
\\
\hline
21
&
WATER
&
Main
source of
water
&
HH
&
categoric
al
&
1-9
\\
\hline
22
&
OWNAGLAND
&
Ownership
of
agricultu
ral
land
&
HH
&
categoric
al
&
1-3
\\
\hline
23
&
LANDSIZEH
A
&
Land size
owned by
household
(ha)
(agric
and non
agric)
&
HH
&
continuou
s
&
0-40
\\
\hline
24
&
OWNMOTORC
YCLE
&
Ownership
of
motorcycl
e
&
HH
&
categoric
al
&
0, 1
\\
\hline
25
&
CAR
&
Ownership
of car
&
HH
&
categoric
al
&
0, 1
\\
\hline
26
&
TV
&
Ownership
of
televisio
n
&
HH
&
categoric
al
&
0, 1
\\
\hline
27
&
LIVESTOCK
&
Number of
large-siz
ed
livestock
owned
&
HH
&
semi-cont
inuous
&
0-25
\\
\hline
28
&
INCRMT
&
Income \textendash{}
Remittanc
es
&
HH
&
continuou
s
&\begin{itemize}
\item {} 
\end{itemize}
\\
\hline
29
&
INCWAGE
&
Income -
Wages and
salaries
&
HH
&
continuou
s
&\begin{itemize}
\item {} 
\end{itemize}
\\
\hline
30
&
INCFARMBS
N
&
Income -
Gross
income
from
household
farm
businesse
s
&
HH
&
continuou
s
&\begin{itemize}
\item {} 
\end{itemize}
\\
\hline
31
&
INCNFARMB
SN
&
Income
-Gross
income
from
household
nonfarm
businesse
s
&
HH
&
continuou
s
&\begin{itemize}
\item {} 
\end{itemize}
\\
\hline
32
&
INCRENT
&
Income -
Rent
&
HH
&
continuou
s
&\begin{itemize}
\item {} 
\end{itemize}
\\
\hline
33
&
INCFIN
&
Income -
Financial
&
HH
&
continuou
s
&\begin{itemize}
\item {} 
\end{itemize}
\\
\hline
34
&
INCPENSN
&
Income -
Pensions/
social
assistanc
e
&
HH
&
continuou
s
&\begin{itemize}
\item {} 
\end{itemize}
\\
\hline
35
&
INCOTHER
&
Income -
Other
&
HH
&
continuou
s
&\begin{itemize}
\item {} 
\end{itemize}
\\
\hline
36
&
INCTOTGRO
SSHH
&
Income -
Total
&
HH
&
continuou
s
&\begin{itemize}
\item {} 
\end{itemize}
\\
\hline
37
&
TFOODEXP
&
Total
expenditu
re
on food
&
HH
&
continuou
s
&\begin{itemize}
\item {} 
\end{itemize}
\\
\hline
38
&
TALCHEXP
&
Total
expenditu
re
on
alcoholic
beverages
,
tobacco
and
narcotics
&
HH
&
continuou
s
&\begin{itemize}
\item {} 
\end{itemize}
\\
\hline
39
&
TCLTHEXP
&
Total
expenditu
re
on
clothing
&
HH
&
continuou
s
&\begin{itemize}
\item {} 
\end{itemize}
\\
\hline
40
&
THOUSEXP
&
Total
expenditu
re
on
housing
&
HH
&
continuou
s
&\begin{itemize}
\item {} 
\end{itemize}
\\
\hline
41
&
TFURNEXP
&
Total
expenditu
re
on
furnishin
g
&
HH
&
continuou
s
&\begin{itemize}
\item {} 
\end{itemize}
\\
\hline
42
&
THLTHEXP
&
Total
expenditu
re
on health
&
HH
&
continuou
s
&\begin{itemize}
\item {} 
\end{itemize}
\\
\hline
43
&
TTRANSEXP
&
Total
expenditu
re
on
transport
&
HH
&
continuou
s
&\begin{itemize}
\item {} 
\end{itemize}
\\
\hline
44
&
TCOMMEXP
&
Total
expenditu
re
on
communica
tion
&
HH
&
continuou
s
&\begin{itemize}
\item {} 
\end{itemize}
\\
\hline
45
&
TRECEXP
&
Total
expenditu
re
on
recreatio
n
&
HH
&
continuou
s
&\begin{itemize}
\item {} 
\end{itemize}
\\
\hline
46
&
TEDUEXP
&
Total
expenditu
re
on
education
&
HH
&
continuou
s
&\begin{itemize}
\item {} 
\end{itemize}
\\
\hline
47
&
TRESHOTEX
P
&
Total
expenditu
re
on
restauran
ts
and
hotels
&
HH
&
continuou
s
&\begin{itemize}
\item {} 
\end{itemize}
\\
\hline
48
&
TMISCEXP
&
Total
expenditu
re
on
miscellan
eous
spending
&
HH
&
continuou
s
&\begin{itemize}
\item {} 
\end{itemize}
\\
\hline
49
&
TANHHEXP
&
Total
annual
nominal
household
expenditu
res
&
HH
&
continuou
s
&\begin{itemize}
\item {} 
\end{itemize}
\\
\hline
\end{longtable}\sphinxatlongtableend\end{savenotes}

It is always important to ensure that the relationships between
variables in the data are preserved during the anonymization process and
to explore and take note of these relationships before beginning the
anonymization. At the end of the anonymization process before the
release of the data, an audit should be conducted, using these initial
results, to check that these relationships are maintained in the
anonymized dataset (see Step 11).

In our dataset, we identify several relationships between variables that
need to be preserved during the anonymization process. The variables
“TANHHEXP” and “INCTOTGROSSHH” represent the total annual nominal
household expenditure and the total gross annual household income,
respectively, and these variables are aggregations of existing income
and expenditure components in the dataset.

The variables related to education are available only for individuals in
the appropriate age groups and missing for other individuals. In
addition, the household-level variables (cf. fourth column of Table
9.18) have the same values for all members in any particular household.
The value of household size corresponds to the actual number of
individuals belonging to that household in the dataset. As we proceed,
we have to take care that these relationships and structures are
preserved in the anonymization process.

We assume that the data are collected in a survey that uses simple
sampling of households. The data contains two weight coefficients:
“WGTHH” and “WGTPOP”. The relationship between the weights is
\(WGTPOP\  = \ WGTHH\ *\ HHSIZE\). “WGTPOP” is the sampling weight
for the households and “WGTHH” is the sampling weight for the
individuals to be used for disclosure risk calculations. “WGTHH” is used
for computing individual-level indicators (such as education) and
“WGTPOP” is used for population level indicators (such as income
indicators). There are no strata variables available in the data. We
will use “WGTPOP” for the anonymization of the household variables and
“WGTHH” for the anonymization of the individual-level variables.

\sphinxstylestrong{Step 3: Type of release}

In this case study, we assume that the file will be released as a PUF,
which will be freely available to anyone interested in the data (see
Section 3.1 for the conditions and more information on the release of
PUFs). The PUF release is intended for users with lower information
requirements (e.g., students) and researchers interested in the
structure of the data and willing to do preliminary research. The PUF
file can give an idea to the researcher whether it is worthwhile for
their research to apply for access to the SUF file. Researchers willing
to do more in-depth research will most likely apply for SUF access.
Generally, users of a PUF file are not restricted by an agreement that
prevents them from using the data to re-identify individuals and hence
the accepted risk level is much lower than in the case of the SUF and
the set of released variables is limited.

\sphinxstylestrong{Step 4: Intruder scenarios and choice of key variables}

Next, based on the release type, we reformulate the intruder scenarios
for the PUF release. This leads to the selection of a set of
quasi-identifiers. Since this case study is based on a demo dataset, we
do not have a real context and we cannot define exact disclosure
scenarios. Therefore, we make hypothetical assumptions on possible
disclosure scenarios. We consider two types of disclosure scenarios: 1)
matching with other publically available datasets and 2) spontaneous
recognition. Since the dataset will be distributed as PUF, there are de
facto no restrictions on the use of the dataset by intruders.

For the sake of illustration, we assume that population registers are
available with the demographic variables gender, age, place of residence
(region, urban/rural), religion and other variables such as marital
status and variables relating to education and professional status that
are also present in our dataset. In addition, we assume that there is a
publically available cadastral register on land ownership. Based on this
analysis of available data sources, we have selected in case study 1 the
variables “REGION”, “URBRUR”, “HHSIZE”, “OWNAGLAND”, “RELIG”, “GENDER”,
“REL” (relationship to household head), “MARITAL” (marital status),
“AGEYRS”, “INDUSTRY1” and two variables relating to school attendance as
categorical quasi-identifiers, the expenditure and income variables as
well as LANDSIZEHA as continuous quasi-identifiers. According to our
assessment, these variables might enable an intruder to re-identify an
individual or household in the dataset by matching with other available
datasets. The key variables for PUF release generally coincide with the
key variables for the SUF release. Possibly, more variables could be
added, since the user has more possibilities to match the data
extensively and is not bound by any contract, as is in the case of the
SUF file. Equally, some key variables in the SUF file may not be
released in the PUF file and, as a consequence, these variables are
removed from the list of key variables.

Upon further consideration, this initial set of identifying variables is
too large for a PUF release, as the number of possible combinations
(keys) is very high and hence many respondents could be identified based
on these variables. Therefore, we decide to limit the set of key
variables, by excluding variables from the dataset for PUF release. The
choice of variables to be removed is led by the needs of the intended
PUF users. Assuming the typical users are mainly interested in aggregate
income and expenditure data, we can therefore remove from the initial
set of key variables “OWNAGLAND”, “RELIG” and “LANDSIZEHA” at the
household level and “EDYRSCURRAT” and “ATSCHOOL” at the individual
level. \sphinxstylestrong{NOTE: These variables will not be released in the PUF file.}
We also remove the income and expenditure components from the list of
key variables, since we reduce their information content by building
proportions (see Step 8a). The list of the remaining key variables is
presented in Table 9.19.

Table 9.19: Overview of selected key variables for PUF file


\begin{savenotes}\sphinxattablestart
\centering
\begin{tabulary}{\linewidth}[t]{|T|T|T|}
\hline
\sphinxstyletheadfamily 
\sphinxstylestrong{Variable name}
&\sphinxstyletheadfamily 
\sphinxstylestrong{Variable
description}
&\sphinxstyletheadfamily 
\sphinxstylestrong{Measurement level}
\\
\hline
REGION
&
region
&
Household,
categorical
\\
\hline
URBRUR
&
area of residence
&
Household,
categorical
\\
\hline
HHSIZE
&
household size
&
Household,
categorical
\\
\hline
TANHHEXP
&
total expenditure
&
Household, continuous
\\
\hline
INCTOTGROSSHH
&
total income
&
Household, continuous
\\
\hline
GENDER
&
gender
&
Individual,
categorical
\\
\hline
REL
&
relationship to
household head
&
Individual,
categorical
\\
\hline
MARITAL
&
marital status
&
Individual,
categorical
\\
\hline
AGEYRS
&
age in completed
years
&
Individual,
semi-continuous/categ
orical
\\
\hline
EDUCY
&
highest level of
education completed
&
Individual,
categorical
\\
\hline
INDUSTRY1
&
industry
classification
&
Individual,
categorical
\\
\hline
\end{tabulary}
\par
\sphinxattableend\end{savenotes}

The decision to release the dataset as a PUF means the level of
anonymization will be relatively high and consequently, the variables
are less detailed (e.g., after recoding) and a scenario of spontaneous
recognition is less likely. Nevertheless, we should check for rare
combinations or unusual patterns in the variables. Variables that may
lead to spontaneous recognition in our sample are amongst others
“HHSIZE” (household size) as well as “INCTOTGROSSHH” (aggregate income)
and “TANHHEXP” (total expenditure). Large households and households with
high income are easily identifiable, especially when combined with other
identifying variables such as a geographical identifier (“REGION”).
There might be only one or a few households/individuals in a certain
region with a high income, such as the local doctor. Variables that are
easily observable and known by neighbors such as “ROOF”, “TOILET”,
“WATER”, “ELECTCON”, “FUELCOOK”, “OWNMOTORCYCLE”, “CAR”, “TV” and
“LIVESTOCK” may also need protection depending on what stands out in the
community, since a user might be able to identify persons (s)he knows.
This is called the nosy-neighbor scenario.

\sphinxstylestrong{Step 5: Data key uses and selection of utility measures}

A PUF file contains less information and the file is generally used by
students as a teaching file, by researchers to get an idea about the
data structure, and for simple analyses. The users have generally lower
requirements than for a SUF file and the results of analysis may be less
precise. The researcher interested in a more detailed dataset, would
have to apply for access to the SUF file. Therefore, we select more
aggregate utility measures for the PUF file that reflect the intended
use of a PUF file. Data intensive measures, such as the Gini
coefficient, cannot be computed from the PUF file. Besides the standard
utility measures, such as tabulations, we evaluate the decile dispersion
ratio and a regression with the income deciles as regressand.

To measure the information loss, we should compare the initial data file
before any anonymization (including the anonymization for the SUF) with
the file after the anonymization for the PUF. Comparing the files
directly before and after the PUF anonymization would underestimate the
information loss, as this would omit the information loss due to SUF
anonymization. Therefore, in Step 2, we also loaded the raw dataset.

\sphinxstylestrong{Hierarchical (household) structure}

As noted in case study 1, the data has a household structure. For the
SUF release, we protected large households by removing these from the
dataset. Since some variables are measured on the household level and
thus have identical values for each household member, the values of the
household variables should be treated in the same way for each household
member (see Section 5.5). Therefore, we first anonymize only the
household variables. After this, we merge them with the individual-level
variables and then anonymize the individual-level and household-level
variables jointly.

Since the data has a hierarchical structure, Steps 6 through 10 are
repeated twice: Steps 6a through 10a are for the household-level
variables and Steps 6b through 10b for the combined dataset. In this
way, we ensure that household-level variable values remain consistent
across household members for each household and the household structure
cannot be used to re-identify individuals. This is further explained in
Sections 4.4 and 7.6.

Before continuing to Step 6a, we select the categorical key variables,
continuous key variables and any variables selected for use in PRAM
routines, as well as household-level sampling weights in \sphinxstyleemphasis{R}. We also
collect the variable names of the variables that will not be released.
The PRAM variables are variables select for the PRAM routine, which we
discuss further in Step 8a. We extract these selected household
variables from the SUF dataset and save them as “fileHH”. The choice of
PRAM variables is further explained in Step 8a. Example 9.29 illustrates
how these steps are done in \sphinxstyleemphasis{R} (see also Section 7.6).

Example 9.29: Selecting the variables for the household-level
anonymization

\begin{DUlineblock}{0em}
\item[] \sphinxstyleemphasis{\# Categorical key variables at household level}
\item[] selectedKeyVarsHH \textless{}- \sphinxstylestrong{c}(‘URBRUR’, ‘REGION’, ‘HHSIZE’)
\item[] \sphinxstyleemphasis{\# Continuous key variables}
\item[] numVarsHH \textless{}- \sphinxstylestrong{c}(‘TANHHEXP’, ‘INCTOTGROSSHH’)
\item[] \sphinxstyleemphasis{\# PRAM variables}
\item[] pramVarsHH \textless{}- \sphinxstylestrong{c}(‘ROOF’, ‘TOILET’, ‘WATER’, ‘ELECTCON’,
‘FUELCOOK’, ‘OWNMOTORCYCLE’, ‘CAR’, ‘TV’, ‘LIVESTOCK’)
\item[] \sphinxstyleemphasis{\# Household weight}
\item[] weightVarHH \textless{}- \sphinxstylestrong{c}(‘WGTPOP’)
\item[] \sphinxstyleemphasis{\# Variables not suitable for release in PUF (HH level)}
\item[] varsNotToBeReleasedHH \textless{}- \sphinxstylestrong{c}(“OWNAGLAND”, “RELIG”, “LANDSIZEHA”)
\item[] \sphinxstyleemphasis{\# Vector with names of all HH level variables}
\item[] HHVars \textless{}- \sphinxstylestrong{c}(‘IDH’, selectedKeyVarsHH, pramVarsHH, numVarsHH,
weightVarHH)
\end{DUlineblock}

\begin{DUlineblock}{0em}
\item[] \sphinxstyleemphasis{\# Create subset of file with only HH level variables}
\item[] fileHH \textless{}- file{[},HHVars{]}
\end{DUlineblock}

Every household has the same number of entries as it has members (e.g.,
a household of three will be repeated three times in “fileHH”). Before
analyzing the household-level variables, we select only one entry per
household, as illustrated in Example 9.30. This is further explained in
Section 7.6. In the same way we extract “fileOrigHH” from “fileOrig”.
“fileOrigHH” contains all variables from the raw data, but contains
every household only once. We need “fileOrigHH” in Steps 8a and 10a for
undoing some perturbative methods used in the SUF file and computing
utility measures from the raw data respectively.

Example 9.30: Taking a subset with only households

\begin{DUlineblock}{0em}
\item[] \sphinxstyleemphasis{\# Remove duplicated rows based on IDH, one row per household in
fileHH}
\item[] fileHH \textless{}- fileHH{[}\sphinxstylestrong{which}(!**duplicated**(fileHH\$IDH)),{]} \sphinxstyleemphasis{\# SUF
file}
\item[] fileOrigHH \textless{}-
fileOrig{[}\sphinxstylestrong{which}(!**duplicated**(fileOrig\$IDH)),{]} \sphinxstyleemphasis{\# original
dataset}
\item[] \sphinxstyleemphasis{\# Dimensions of fileHH}
\item[] \sphinxstylestrong{dim}(fileHH)
\end{DUlineblock}

\sphinxcode{\sphinxupquote{\#\# {[}1{]} 1970   16}}

\sphinxstylestrong{dim}(fileOrigHH)

\sphinxcode{\sphinxupquote{\#\# {[}1{]} 2000   68}}

The file “fileHH” contains 1,970 households and 16 variables. We are now
ready to create our \sphinxstyleemphasis{sdcMicro} object with the corresponding variables
we selected in Example 9.28. For our case study, we will create an
\sphinxstyleemphasis{sdcMicro} object called “sdcHH” based on the data in “fileHH”, which we
will use for steps 6a \textendash{} 10a (see Example 9.31). \sphinxstylestrong{NOTE: When the
sdcMicro object is created, the sdcMicro package automatically
calculates and stores the risk measures for the data.} This leads us to
Step 6a.

Example 9.31: Creating a \sphinxstyleemphasis{sdcMicro} object for the household variables

\begin{DUlineblock}{0em}
\item[] \sphinxstyleemphasis{\# Create initial sdcMicro object for household level variables}
\item[] sdcHH \textless{}- \sphinxstylestrong{createSdcObj}(dat = fileHH, keyVars = selectedKeyVarsHH,
\item[] pramVars = pramVarsHH, weightVar = weightVarHH, numVars = numVarsHH)
\item[] numHH \textless{}- \sphinxstylestrong{length}(fileHH{[},1{]}) \sphinxstyleemphasis{\# number of households}
\end{DUlineblock}

\sphinxstylestrong{Step 6a: Assessing disclosure risk (household level)}

Based on the key variables selected in the disclosure scenarios, we can
evaluate the risk at the household level. The PUF risk measures show a
lower risk level than in the SUF file after anonymization in case study
1. The reason is that the set of key variables is smaller, since some
variables will not be released in the PUF file. Removing (key) variables
reduces the risk, and it is one of the most straightforward SDC methods.

As a first measure, we evaluate the number of households violating
\(k\)-anonymity at the levels 2, 3 and 5. Table 9.20 shows the
number of violating households as well as the percentage of the total
number of households. Example 9.32 illustrates how to find these values
with \sphinxstyleemphasis{sdcMicro}. The print() function in \sphinxstyleemphasis{sdcMicro} shows only the
values for thresholds 2 and 3. Values for other thresholds can be
calculated manually by summing up the frequencies smaller than the
\(k\)-anonymity threshold, as shown in Example 9.32. The number of
violators is already at a low level, due to the prior anonymization of
the SUF file and the reduced set of key variables.

Table 9.20: Number and proportion of households violating
\(\mathbf{k}\)-anonymity


\begin{savenotes}\sphinxattablestart
\centering
\begin{tabulary}{\linewidth}[t]{|T|T|T|}
\hline
\sphinxstyletheadfamily 
\sphinxstylestrong{k-anonymity level}
&\sphinxstyletheadfamily 
\sphinxstylestrong{Number of HH
violating}
&\sphinxstyletheadfamily 
\sphinxstylestrong{Percentage of total
number of HH}
\\
\hline
2
&
0
&
0.0\%
\\
\hline
3
&
18
&
0.9\%
\\
\hline
5
&
92
&
4.7\%
\\
\hline
\end{tabulary}
\par
\sphinxattableend\end{savenotes}

Example 9.32: Showing number of households violating
\(\mathbf{k}\)-anonymity for levels 2, 3 and 5

\begin{DUlineblock}{0em}
\item[] \sphinxstyleemphasis{\# Number of observations violating k-anonymity (thresholds 2 and 3)}
\item[] \sphinxstylestrong{print}(sdcHH)
\end{DUlineblock}

\begin{DUlineblock}{0em}
\item[] \sphinxcode{\sphinxupquote{\#\# Infos on 2/3-Anonymity:}}
\item[] \sphinxcode{\sphinxupquote{\#\#}}
\item[] \sphinxcode{\sphinxupquote{\#\# Number of observations violating}}
\item[] \sphinxcode{\sphinxupquote{\#\#  - 2-anonymity: 0}}
\item[] \sphinxcode{\sphinxupquote{\#\#  - 3-anonymity: 18}}
\item[] \sphinxcode{\sphinxupquote{\#\#}}
\item[] \sphinxcode{\sphinxupquote{\#\# Percentage of observations violating}}
\item[] \sphinxcode{\sphinxupquote{\#\#  - 2-anonymity: 0.000 \%}}
\item[] \sphinxcode{\sphinxupquote{\#\#  - 3-anonymity: 0.914 \%}}
\item[] \sphinxcode{\sphinxupquote{-{-}-{-}-{-}-{-}-{-}-{-}-{-}-{-}-{-}-{-}-{-}-{-}-{-}-{-}-{-}-{-}-{-}-{-}-{-}-{-}-{-}-{-}-{-}-{-}-{-}-{-}-{-}-{-}-{-}-{-}-{-}-{-}-{-}-{-}-{-}-{-}-{-}}}
\end{DUlineblock}

\begin{DUlineblock}{0em}
\item[] \sphinxstyleemphasis{\# Calculate sample frequencies and count number of obs. violating k
(5) - anonymity}
\item[] kAnon5 \textless{}- \sphinxstylestrong{sum}(sdcHH@risk\$individual{[},2{]} \textless{} 5)
\end{DUlineblock}

kAnon5

\#\# {[}1{]} 92

\begin{DUlineblock}{0em}
\item[] \sphinxstyleemphasis{\# As percentage of total}
\item[] kAnon5 / numHH
\end{DUlineblock}

\#\# {[}1{]} \sphinxcode{\sphinxupquote{0.04670051}}

It is often useful to view the records of the household(s) that violate
\(k\)-anonymity. This might help to find which variables cause the
uniqueness of these households; this can then be used later when
choosing appropriate SDC methods. Example 9.32 shows how to access the
values of the households violating 3 and 5-anonymity. Not surprisingly,
the variable “HHSIZE” is responsible for many of the unique combinations
and the origin of much of the risk. This is even the case after removing
large households for the SUF release.

Example 9.33: Showing records of households that violate
\(\mathbf{k}\)-anonymity

\begin{DUlineblock}{0em}
\item[] \sphinxstyleemphasis{\# Show values of key variable of records that violate k-anonymity}
\item[] fileHH{[}sdcHH@risk\$individual{[},2{]} \textless{} 3, selectedKeyVarsHH{]} \sphinxstyleemphasis{\# for
3-anonymity}
\end{DUlineblock}

fileHH{[}sdcHH@risk\$individual{[},2{]} \textless{} 5, selectedKeyVarsHH{]} \sphinxstyleemphasis{\# for
5-anonymity}

We also assess the disclosure risk of the categorical variables with the
individual and global risk measures as described in Sections 4.5 and
4.8. In “fileHH” every entry represents a household. Therefore, we use
the individual non-hierarchical risk here, where the individual refers
in this case to a household. “fileHH” is a subset of the complete
dataset and contains only households and has, contrary to the complete
dataset, no hierarchical structure. In Step 6b, we evaluate the
hierarchical risk in the dataset “file”, the dataset containing both
households and individuals. The individual and global risk measures
automatically take into consideration the household weights, which we
defined in Example 9.29. In our file, the global risk measure calculated
using the chosen key variables is lower than 0.01\% (the smallest
reported value is 0.01\%, in fact the global risk is 0.0000642 \%). This
percentage is extremely low and corresponds to 0.13 expected
re-identifications. The results are also shown in Example 9.34. This low
figure can be explained by the relatively small sample size of 0.25\% of
the total population (see case study 1). Furthermore, one should keep in
mind that this risk measure is based only on the categorical
quasi-identifiers at the household level.

Example 9.34: Printing global risk measures

\sphinxstylestrong{print}(sdcHH, “risk”)

\begin{DUlineblock}{0em}
\item[] \sphinxcode{\sphinxupquote{\#\# Risk measures:}}
\item[] \sphinxcode{\sphinxupquote{\#\#}}
\item[] \sphinxcode{\sphinxupquote{\#\# Number of observations with higher risk than the main part of the data: 0}}
\item[] \sphinxcode{\sphinxupquote{\#\# Expected number of re-identifications:}}\sphinxcode{\sphinxupquote{0}}\sphinxcode{\sphinxupquote{.}}\sphinxcode{\sphinxupquote{1}}\sphinxcode{\sphinxupquote{3 (0.0}}\sphinxcode{\sphinxupquote{1}}\sphinxcode{\sphinxupquote{\%)}}
\end{DUlineblock}

The global risk measure does not provide information about the spread of
the individual risk measures. There might be a few households with
relatively high risk, while the global (average) risk is low. Therefore
we check the highest individual risk as shown in Example 9.35. The
individual risk of the household with the highest risk is 0.1 \%, which
is still very low.

Example 9.35 Determining the highest individual risk

\begin{DUlineblock}{0em}
\item[] \sphinxstyleemphasis{\# Highest individual risk}
\item[] \sphinxstylestrong{max}(sdcHH@risk\$individual{[}, “risk”{]})
\end{DUlineblock}

\sphinxcode{\sphinxupquote{\#\# {[}1{]} 0.001011633}}

Since the selected key variables at the household level are both
categorical and numerical, the individual and global risk measures based
on frequency counts do not completely reflect the disclosure risk of the
entire dataset. When generating the SUF file, we concluded that recoding
of continuous variables to make them all categorical would likely not
satisfy the needs of the SUF users. For the PUF file it is acceptable to
recode continuous variables, such as income and expenditures since PUF
content is typically less detailed. In Step 8a we will recode these
variables into deciles and convert them into categorical variables.
Therefore, we exclude these variables from the risk calculations now We
take these variables into account while remeasuring risk after
anonymization.

\sphinxstylestrong{Step 7a: Assessing utility measures (household level)}

The utility of the data does not only depend on the household level
variables, but on the combination of household-level and
individual-level variables. Therefore, it is not useful to evaluate all
the utility measures selected in Step 5 at this stage, i.e., before
anonymizing the individual level variables. We restrict the initial
measurement of utility to those measures that are solely based on the
household variables. In our dataset, these are the measures related to
income and expenditure and their distributions. The results are
presented in Step 10a, together with the results after anonymization,
which allow direct comparison. If after the next anonymization step it
appears that the data utility has been significantly decreased by the
suppression of some household level variables, we can return to this
step. \sphinxstylestrong{NOTE: to analyze the utility loss, the utility measures before
anonymization have to be calculated from the raw data and not from the
anonymized SUF file.} Not all measures from case study 1 can be
computed from the PUF file, since the information content is lower. The
set of utility measures we use to evaluate the information loss in the
PUF file consists of measures that need less detailed variables. This
reflects the lower requirements a PUF user has on the dataset.

\sphinxstylestrong{Step 8a: Choice and application of SDC methods (household level)}

This step is divided into the anonymization of the categorical key
variables and the continuous key variables, since different methods are
used for both sets of variables. As already discussed in Step 4, we do
not release all variables in the PUF file. At the household level
“RELIG” (religion of household head), “OWNAGLAND” (land ownership) and
“LANDSIZEHA” (plot size in ha) are not released in addition to the
variables removed for the SUF release. For the sake of illustration, we
assume that the variable “RELIG” is too sensitive and the variables
“OWNAGLAND” and “LANDSIZEHA” are too identifying.

\sphinxstylestrong{Categorical variables}

We are now ready to move on to the choice of SDC methods for the
categorical variables on the household level in our dataset. In the SUF
file we already recoded some of the key variables and used local
suppression. We only have three categorical key variables at the
household level; “URBRUR”, “REGION” and “HHSIZE”. The selected
categorical key variables at the household level are not suitable for
recoding at this point, since the values cannot be grouped further.
“URBRUR” has only two distinct categories and “REGION” has only six
non-combinable regions. As noted before, the variable “HHSIZE” can be
reconstructed by a headcount per household. Therefore, recoding of this
variable alone does not lead to disclosure control.

Due to the relatively low risk of re-identification based on the three
selected categorical household level variables, it is possible in this
case to use an option like local suppression to achieve our desired
level of risk. Applying local suppression when initial risk is
relatively low will likely only lead to suppression of few observations
and thus limit the loss of utility. If, however, the data had been
measured to have a relatively high risk, then applying local suppression
without previous recoding would likely result in a large number of
suppressions and greater information loss. Efforts such a recoding
should be taken first before suppressing values in cases where risk is
initially measured as high. Recoding will reduce risk with little
information loss and thus the number of suppressions, if local
suppression is applied as an additional step.

We apply local suppression to reach 5-anonymity. The chosen level of
five is higher than in the SUF release and is based on the release type
as PUF. This leads to a total of 39 suppressions, all in the variable
“HHSIZE”. As explained earlier, suppression of the value of the variable
“HHSIZE” does not lead to actual suppression of this information.
Therefore, we redo the local suppression, but this time we tell
\sphinxstyleemphasis{sdcMicro} to, if possible, not suppress “HHSIZE” but one of the other
variables. Alternatively, we could remove households with suppressed
values of the variable “HHSIZE”, remove large households or split
households.

In \sphinxstyleemphasis{sdcMicro} it is possible to tell the algorithm which variables are
important and less important for making small changes (see also Section
5.2.2). To prevent values of the variable “HHSIZE” being suppressed, we
set the importance of “HHSIZE” in the importance vectors to the highest
(i.e., 1). We try two different importance vectors: the first where
“REGION” is more important than “URBRUR” and the second with the
importance of “REGION” and “URBRUR” swapped. Example 9.36 shows how to
apply local suppression and put importance on the variable “HHSIZE”.
\sphinxstylestrong{NOTE: In} \sphinxstylestrong{Example 9.36 we use the undolast() function in sdcMicro
to go one step back after we had first applied local suppression with no
importance vector.} The undolast() function restores the \sphinxstyleemphasis{sdcMicro}
object back to the previous state (i.e., before we applied local
suppression), which allows us to rerun the same command, but this time
with an importance vector set. The undolast() function can only be used
to go one step back.

The suppression patterns of the three different options are shown in
Table 9.21. The importance is clearly reflected in the number of
suppressions per variable. The total number of suppressions is with an
importance vector higher than without an importance vector (44/73 vs.
39), but 5-anonymity is achieved in the dataset with no suppressions in
the variable “HHSIZE”. This means that we do not have to remove or split
households. The variable “REGION” is the type of variable that should
not have any suppressions either. From that perspective we chose the
third option. This leads to more suppressions, but no suppressions in
“HHSIZE” and as few as possible in “REGION”.

Table 9.21: Number of suppressions by variable after local suppression
with and without importance vector


\begin{savenotes}\sphinxattablestart
\centering
\begin{tabulary}{\linewidth}[t]{|T|T|T|T|}
\hline
\sphinxstyletheadfamily 
Key variable
&\sphinxstartmulticolumn{3}%
\begin{varwidth}[t]{\sphinxcolwidth{3}{4}}
\sphinxstyletheadfamily Number of suppressions and proportion of total
\par
\vskip-\baselineskip\vbox{\hbox{\strut}}\end{varwidth}%
\sphinxstopmulticolumn
\\
\hline&
\sphinxstyleemphasis{No importance
vector}
&
\sphinxstyleemphasis{Importance
HHSIZE, URBRUR,
REGION}
&
\sphinxstyleemphasis{Importance
HHSIZE, REGION,
URBRUR}
\\
\hline
\sphinxstyleemphasis{URBRUR}
&
0 (0.0 \%)
&
2 (0.1 \%)
&
61 (3.1 \%)
\\
\hline
\sphinxstyleemphasis{REGION}
&
0 (0.0 \%)
&
42 (2.1 \%)
&
12 (0.6 \%)
\\
\hline
\sphinxstyleemphasis{HHSIZE}
&
39 (2.0 \%)
&
0 (0.0 \%)
&
0 (0.0 \%)
\\
\hline
\end{tabulary}
\par
\sphinxattableend\end{savenotes}

Example 9.36: Local suppression with and without importance vector

\begin{DUlineblock}{0em}
\item[] \sphinxstyleemphasis{\# Local suppression to achieve 5-anonimity}
\item[] sdcHH \textless{}- \sphinxstylestrong{localSuppression}(sdcHH, k = 5, importance = NULL) \sphinxstyleemphasis{\# no
importance vector}
\item[] \sphinxstylestrong{print}(sdcHH, “ls”)
\end{DUlineblock}

\begin{DUlineblock}{0em}
\item[] \sphinxcode{\sphinxupquote{\#\# Local Suppression:}}
\item[] \sphinxcode{\sphinxupquote{\#\#  KeyVar \textbar{} Suppressions (\#) \textbar{} Suppressions (\%)}}
\item[] \sphinxcode{\sphinxupquote{\#\#  URBRUR \textbar{}                0 \textbar{}            0.000}}
\item[] \sphinxcode{\sphinxupquote{\#\#  REGION \textbar{}                0 \textbar{}            0.000}}
\item[] \sphinxcode{\sphinxupquote{\#\#  HHSIZE \textbar{}               39 \textbar{}            1.980}}
\item[] \sphinxcode{\sphinxupquote{\#\# -{-}-{-}-{-}-{-}-{-}-{-}-{-}-{-}-{-}-{-}-{-}-{-}-{-}-{-}-{-}-{-}-{-}-{-}-{-}-{-}-{-}-{-}-{-}-{-}-{-}-{-}-{-}-{-}-{-}-{-}-{-}-{-}-{-}-{-}-{-}-{-}-{-}-}}
\end{DUlineblock}

\begin{DUlineblock}{0em}
\item[] sdcHH \textless{}- \sphinxstylestrong{undolast}(sdcHH) \sphinxstyleemphasis{\# undo suppressions to see the effect
of an importance vector}
\item[] \sphinxstyleemphasis{\# Redo local suppression minimizing the number of suppressions in
HHSIZE}
\item[] sdcHH \textless{}- \sphinxstylestrong{localSuppression}(sdcHH, k = 5, importance = \sphinxstylestrong{c}(2,
3, 1))
\item[] \sphinxstylestrong{print}(sdcHH, “ls”)
\end{DUlineblock}

\begin{DUlineblock}{0em}
\item[] \sphinxcode{\sphinxupquote{\#\# Local Suppression:}}
\item[] \sphinxcode{\sphinxupquote{\#\#  KeyVar \textbar{} Suppressions (\#) \textbar{} Suppressions (\%)}}
\item[] \sphinxcode{\sphinxupquote{\#\#  URBRUR \textbar{}                2 \textbar{}            0.102}}
\item[] \sphinxcode{\sphinxupquote{\#\#  REGION \textbar{}               42 \textbar{}            2.132}}
\item[] \sphinxcode{\sphinxupquote{\#\#  HHSIZE \textbar{}                0 \textbar{}            0.000}}
\item[] \sphinxcode{\sphinxupquote{\#\# -{-}-{-}-{-}-{-}-{-}-{-}-{-}-{-}-{-}-{-}-{-}-{-}-{-}-{-}-{-}-{-}-{-}-{-}-{-}-{-}-{-}-{-}-{-}-{-}-{-}-{-}-{-}-{-}-{-}-{-}-{-}-{-}-{-}-{-}-{-}-{-}-{-}-}}
\end{DUlineblock}

\begin{DUlineblock}{0em}
\item[] sdcHH \textless{}- \sphinxstylestrong{undolast}(sdcHH) \sphinxstyleemphasis{\# undo suppressions to see the effect
of a different importance vector}
\item[] \sphinxstyleemphasis{\# Redo local suppression minimizing the number of suppressions in
HHSIZE}
\item[] sdcHH \textless{}- \sphinxstylestrong{localSuppression}(sdcHH, k = 5, importance = \sphinxstylestrong{c}(3,
2, 1))
\item[] \sphinxstylestrong{print}(sdcHH, “ls”)
\end{DUlineblock}

\begin{DUlineblock}{0em}
\item[] \sphinxcode{\sphinxupquote{\#\# Local Suppression:}}
\item[] \sphinxcode{\sphinxupquote{\#\#  KeyVar \textbar{} Suppressions (\#) \textbar{} Suppressions (\%)}}
\item[] \sphinxcode{\sphinxupquote{\#\#  URBRUR \textbar{}}}\sphinxcode{\sphinxupquote{61}}\sphinxcode{\sphinxupquote{\textbar{}}}\sphinxcode{\sphinxupquote{3}}\sphinxcode{\sphinxupquote{.}}\sphinxcode{\sphinxupquote{096}}
\item[] \sphinxcode{\sphinxupquote{\#\#  REGION \textbar{}}}\sphinxcode{\sphinxupquote{1}}\sphinxcode{\sphinxupquote{2 \textbar{}}}\sphinxcode{\sphinxupquote{0}}\sphinxcode{\sphinxupquote{.}}\sphinxcode{\sphinxupquote{609}}
\item[] \sphinxcode{\sphinxupquote{\#\#  HHSIZE \textbar{}                0 \textbar{}            0.000}}
\item[] \sphinxcode{\sphinxupquote{\#\# -{-}-{-}-{-}-{-}-{-}-{-}-{-}-{-}-{-}-{-}-{-}-{-}-{-}-{-}-{-}-{-}-{-}-{-}-{-}-{-}-{-}-{-}-{-}-{-}-{-}-{-}-{-}-{-}-{-}-{-}-{-}-{-}-{-}-{-}-{-}-{-}-{-}-}}
\end{DUlineblock}

In case study 1 we applied invariant PRAM to the variables “ROOF”,
“TOILET”, “WATER”, “ELECTCON”, “FUELCOOK”, “OWNMOTORCYCLE”, “CAR”, “TV”
and “LIVESTOCK”, since these variables are not sensitive and were not
selected as quasi-identifiers because we assumed that there are no
external data sources containing this information that could be used for
matching. Values can be easily observed or be known to neighbors,
however, and therefore are important, together with other variables, for
the nosy neighbor scenario. For the PUF release we would like to level
of uncertainty by increasing the number of changes. Therefore, we redo
PRAM with a different transition matrix. As discussed in Section 5.3.1,
the invariant PRAM method has the property that the univariate
distributions do not change. To maintain this property, we reapply PRAM
to the raw data, rather than to the already PRAMmed variables in the SUF
file.

Example 9.37 illustrates how to apply PRAM. We use the original values
to apply PRAM and replace the values in the \sphinxstyleemphasis{sdcMicro} object with these
values. We choose the parameter ‘pd’, the lower bound for the
probability that a value is not changed, to be relatively low at 0.6.
This is a lower value than the 0.8 used in the SUF file and will lead to
a higher number of changes (cf. Example 9.17 on page 135). This is
acceptable for a PUF file and introduces more uncertainty as required
for a PUF release. Example 9.37 also shows the number of changed records
per variables. Because PRAM is a probabilistic method, we set a seed for
the random number generator before applying PRAM to ensure
reproducibility of the results. \sphinxstylestrong{Note: In some cases the choice of the
seed matters. The choice of seed changes the results.} The seed should
not be released, since it allows for reconstructing the original values
if combined with the transition matrix. The transition matrix can be
released: this allows for consistent statistical inference by correcting
the statistical methods used if the researcher has knowledge about the
PRAM method (at this point \sphinxstyleemphasis{sdcMicro} does not allow to retrieve the
transition matrix).

Example 9.37: Applying PRAM

\begin{DUlineblock}{0em}
\item[] \sphinxstyleemphasis{\# PRAM}
\item[] \sphinxstylestrong{set.seed}(10987)
\item[] \sphinxstyleemphasis{\# Replace PRAM variables in sdcMicro object sdcHH with the original
raw values}
\item[] sdcHH@origData{[},pramVarsHH{]} \textless{}- fileHH{[}\sphinxstylestrong{match}(fileHH\$IDH,
fileOrigHH\$IDH), pramVarsHH{]}
\item[] \sphinxhref{mailto:sdcHH@manipPramVars}{sdcHH@manipPramVars} \textless{}- fileHH{[}\sphinxstylestrong{match}(fileHH\$IDH,
fileOrigHH\$IDH), pramVarsHH{]}
\item[] sdcHH \textless{}- \sphinxstylestrong{pram}(obj = sdcHH, pd = 0.6)
\end{DUlineblock}

\begin{DUlineblock}{0em}
\item[] \sphinxcode{\sphinxupquote{\#\# Number of changed observations:}}
\item[] \sphinxcode{\sphinxupquote{\#\# - - - - - - - - - - -}}
\item[] \sphinxcode{\sphinxupquote{\#\# ROOF != ROOF\_pram : 305 (15.48\%)}}
\item[] \sphinxcode{\sphinxupquote{\#\# TOILET != TOILET\_pram : 260 (13.2\%)}}
\item[] \sphinxcode{\sphinxupquote{\#\# WATER != WATER\_pram : 293 (14.87\%)}}
\item[] \sphinxcode{\sphinxupquote{\#\# ELECTCON != ELECTCON\_pram : 210 (10.66\%)}}
\item[] \sphinxcode{\sphinxupquote{\#\# FUELCOOK != FUELCOOK\_pram : 315 (15.99\%)}}
\item[] \sphinxcode{\sphinxupquote{\#\# OWNMOTORCYCLE != OWNMOTORCYCLE\_pram : 95 (4.82\%)}}
\item[] \sphinxcode{\sphinxupquote{\#\# CAR != CAR\_pram : 255 (12.94\%)}}
\item[] \sphinxcode{\sphinxupquote{\#\# TV != TV\_pram : 275 (13.96\%)}}
\item[] \sphinxcode{\sphinxupquote{\#\# LIVESTOCK != LIVESTOCK\_pram : 109 (5.53\%)}}
\end{DUlineblock}

PRAM has changed values within the variables according to the invariant
transition matrices. Since we used the invariant PRAM method (see
Section 5.3.1), the absolute univariate frequencies remain approximately
unchanged. This is not the case for the multivariate frequencies. In
Step 10a we compare the multivariate frequencies before and after
anonymization for the PRAMmed variables.

\sphinxstylestrong{Continuous variables}

We have selected the variables “INCTOTGROSSHH” (total income) and
“TANHHEXP” (total expenditure) as numerical quasi-identifiers, as
discussed in Step 4. In Step 5 we identified variables having high
interest for the users of our data: many users use the data for
measuring inequality and expenditure patterns. The noise addition in the
SUF file does not protect these variables sufficiently, especially,
because outliers are not protected. Therefore, we decide to recode total
income and total expenditure into deciles.

As with PRAM, we want to compute the deciles from the raw data rather
than from the perturbed values in the SUF file. We compute the deciles
directly from the raw data and overwrite these values in the \sphinxstyleemphasis{sdcMicro}
object. Subsequently, we compute the mean of each decile from the raw
data and replace the values for total income and total expenditures with
the mean of the respective decile. In this way the mean of both
variables does not change. This approach can be interpreted as
univariate microaggregation with very large groups (group size n/10)
with the mean as replacement value (see Section 5.3.2).

The information in the income and expenditure variables by component is
too sensitive to release as PUF, and, summing those variables would
allow an intruder to reconstruct the totals. PUF users might however be
interested in the shares. Therefore, we decide to keep the income and
expenditure components as proportions of the raw totals, rounded to two
digits. The anonymization of the income and expenditure variables is
shown in Example 9.38.

Example 9.38: Anonymization of income and expenditure variables

\begin{DUlineblock}{0em}
\item[] \sphinxstyleemphasis{\# Create bands (deciles) for income and expenditure variables
(aggregates) based on the original data}
\item[] decExp \textless{}-
\sphinxstylestrong{as.numeric}(\sphinxstylestrong{cut}(fileOrigHH{[}\sphinxstylestrong{match}(fileHH\$IDH,
fileOrigHH\$IDH), “TANHHEXP”{]},
\sphinxstylestrong{quantile}(fileOrigHH{[}\sphinxstylestrong{match}(fileHH\$IDH, fileOrigHH\$IDH),
“TANHHEXP”{]}, (0:10)/10, na.rm = T), include.lowest = TRUE, labels =
\sphinxstylestrong{c}(1, 2, 3, 4, 5, 6, 7, 8, 9, 10)))
\item[] decInc \textless{}-
\sphinxstylestrong{as.numeric}(\sphinxstylestrong{cut}(fileOrigHH{[}\sphinxstylestrong{match}(fileHH\$IDH,
fileOrigHH\$IDH), “INCTOTGROSSHH”{]},
\sphinxstylestrong{quantile}(fileOrigHH{[}\sphinxstylestrong{match}(fileHH\$IDH, fileOrigHH\$IDH),
“INCTOTGROSSHH”{]}, (0:10)/10, na.rm = T), include.lowest = TRUE, labels
= \sphinxstylestrong{c}(1, 2, 3, 4, 5, 6, 7, 8, 9, 10)))
\item[] \sphinxstyleemphasis{\# Mean values of deciles}
\item[] decExpMean \textless{}-
\sphinxstylestrong{round}(\sphinxstylestrong{sapply}(\sphinxstylestrong{split}(fileOrigHH{[}\sphinxstylestrong{match}(fileHH\$IDH,
fileOrigHH\$IDH), “TANHHEXP”{]}, decExp), mean))
\item[] decIncMean \textless{}-
\sphinxstylestrong{round}(\sphinxstylestrong{sapply}(\sphinxstylestrong{split}(fileOrigHH{[}\sphinxstylestrong{match}(fileHH\$IDH,
fileOrigHH\$IDH), “INCTOTGROSSHH”{]}, decInc), mean))
\item[] \sphinxstyleemphasis{\# Replace with mean value of decile}
\item[] \sphinxhref{mailto:sdcHH@manipNumVars\$TANHHEXP}{sdcHH@manipNumVars\$TANHHEXP} \textless{}- decExpMean{[}\sphinxstylestrong{match}(decExp,
\sphinxstylestrong{names}(decExpMean)){]}
\item[] \sphinxhref{mailto:sdcHH@manipNumVars\$INCTOTGROSSHH}{sdcHH@manipNumVars\$INCTOTGROSSHH} \textless{}- decIncMean{[}\sphinxstylestrong{match}(decInc,
\sphinxstylestrong{names}(decIncMean)){]}
\item[] \sphinxstyleemphasis{\# Recalculate risks after manually changing values in sdcMicro
object}
\item[] \sphinxstylestrong{calcRisks}(sdcHH)
\end{DUlineblock}

\begin{DUlineblock}{0em}
\item[] \sphinxstyleemphasis{\# Extract data from sdcHH}
\item[] HHmanip \textless{}- \sphinxstylestrong{extractManipData}(sdcHH) \sphinxstyleemphasis{\# manipulated variables HH}
\item[] \sphinxstyleemphasis{\# Keep components of expenditure and income as share of total, use
original data since previous data was perturbed}
\item[] compExp \textless{}- \sphinxstylestrong{c}(‘TFOODEXP’, ‘TALCHEXP’, ‘TCLTHEXP’, ‘THOUSEXP’,
‘TFURNEXP’, ‘THLTHEXP’, ‘TTRANSEXP’, ‘TCOMMEXP’, ‘TRECEXP’, ‘TEDUEXP’,
‘TRESTHOTEXP’, ‘TMISCEXP’)
\item[] compInc \textless{}- \sphinxstylestrong{c}(‘INCRMT’, ‘INCWAGE’, ‘INCFARMBSN’, ‘INCNFARMBSN’,
‘INCRENT’, ‘INCFIN’, ‘INCPENSN’, ‘INCOTHER’)
\item[] HHmanip \textless{}- \sphinxstylestrong{cbind}(HHmanip,
\sphinxstylestrong{round}(fileOrigHH{[}\sphinxstylestrong{match}(fileHH\$IDH, fileOrigHH\$IDH),
compExp{]} / fileOrigHH{[}\sphinxstylestrong{match}(fileHH\$IDH, fileOrigHH\$IDH),
“TANHHEXP”{]}, 2))
\item[] HHmanip \textless{}- \sphinxstylestrong{cbind}(HHmanip,
\sphinxstylestrong{round}(fileOrigHH{[}\sphinxstylestrong{match}(fileHH\$IDH, fileOrigHH\$IDH),
compInc{]} / fileOrigHH{[}\sphinxstylestrong{match}(fileHH\$IDH, fileOrigHH\$IDH),
“INCTOTGROSSHH”{]}, 2))
\end{DUlineblock}

\sphinxstylestrong{Step 9a: Re-measure risk (household level)}

For the categorical variables, we conclude that we have achieved
5-anonymity in the data with local suppression. 5-anonymity also implies
2- and 3-anonymity. The global risk stayed close to zero (as the
expected number of re-identifications), which is very low. Therefore, we
conclude that based on the categorical variables, the data has been
sufficiently anonymized. One should keep in mind that the anonymization
methods applied are complementing the ones used for the SUF. \sphinxstylestrong{NOTE: The
methods selected methods in this case study alone would not be
sufficient to protect the data set for a PUF release.}

We have reduced the risk of spontaneous recognition of households, by
removing the variable “LANDSIZEHA” and PRAMming the variables identified
to be important in the nosy neighbor scenario. An intruder cannot know
with certainty whether a household that (s)he recognizes in the data is
the correct household, due to the noise in these variables.

These measures refer only to the categorical variables. To evaluate the
risk of the continuous variables we could use an interval measure or
closest neighbor algorithm. These risk measures are discussed in Section
4.7. We chose to use an interval measure, since exact value matching is
not our largest concern based on the assumed scenarios and external data
sources. Instead, datasets with similar values but not the exact same
values could be used for matching. Here the main concern is that the
values are sufficiently far from the original values, which is measured
with an interval measure.

Example 9.39 shows how to evaluate the interval measure for the
variables “INCTOTGROSSHH” and “TANHHEXP” (total income and expenditure).
The different values of the parameter \(k\) in the function dRisk()
define the size of the interval around the original value as a function
of the standard deviation, as explained in Section 4.7.2\sphinxstyleemphasis{.} The larger
\(k\), the larger the intervals, the higher the probability that a
perturbed value is in the interval around the original value and the
higher the risk measure. The results are satisfactory, especially when
keeping in mind that there are only 10 distinct values in the dataset
(the means of each of the deciles). All outliers have been recoded.
Looking at the proportions of the components, we do not detect any
outliers (households with an unusual high or low spending pattern in one
component).

Example 9.39: Measuring risk of re-identification of continuous
variables

\begin{DUlineblock}{0em}
\item[] \sphinxstyleemphasis{\# Risk evaluation continuous variables}
\item[] \sphinxstylestrong{dRisk}(sdcHH@origData{[},**c**(“TANHHEXP”, “INCTOTGROSSHH”){]}, xm
= sdcHH@manipNumVars{[},\sphinxstylestrong{c}(“TANHHEXP”, “INCTOTGROSSHH”){]}, k =
0.01)
\end{DUlineblock}

\sphinxcode{\sphinxupquote{\#\# {[}1{]} 0.4619289}}

\sphinxstylestrong{dRisk}(sdcHH@origData{[},**c**(“TANHHEXP”, “INCTOTGROSSHH”){]}, xm =
sdcHH@manipNumVars{[},\sphinxstylestrong{c}(“TANHHEXP”, “INCTOTGROSSHH”){]}, k = 0.02)

\sphinxcode{\sphinxupquote{\#\# {[}1{]} 0.642132}}

\sphinxstylestrong{dRisk}(sdcHH@origData{[},**c**(“TANHHEXP”, “INCTOTGROSSHH”){]}, xm =
sdcHH@manipNumVars{[},\sphinxstylestrong{c}(“TANHHEXP”, “INCTOTGROSSHH”){]}, k = 0.05)

\sphinxcode{\sphinxupquote{\#\# {[}1{]} 0.8258883}}

\sphinxstylestrong{Step 10a Re-measure utility (household level)}

The utility in the data has decreased compared to the raw data, mainly
because variables were completely removed. Many of the utility measures
used in case study 1 are not applicable to the PUF file. However, by
replacing the deciles with their means, we can still use the income and
expenditure variables for arithmetic operations. Also the shares of the
income and expenditure components can still be used, since they are
based on the raw data.

We select two additional utility measures: the decile dispersion ratio
and the share of total consumption by the poorest decile. The decile
dispersion ratio is the ratio of the average income of the top decile
and the average income of the bottom decile. Example 9.40 shows how to
compute these from the raw data and the household variables after
anonymization. Table 9.22 presents the estimated values. The differences
are small and mainly due to the removed households.

Table 9.22: Comparison of utility measures


\begin{savenotes}\sphinxattablestart
\centering
\begin{tabulary}{\linewidth}[t]{|T|T|T|}
\hline
\sphinxstyletheadfamily &\sphinxstyletheadfamily 
\sphinxstylestrong{Raw data}
&\sphinxstyletheadfamily 
\sphinxstylestrong{PUF file}
\\
\hline
\sphinxstylestrong{Decile dispersion ratio}
&
24.12
&
23.54
\\
\hline
\sphinxstylestrong{Share of consumption by the poorest decile}
&
0.0034
&
0.0035
\\
\hline
\end{tabulary}
\par
\sphinxattableend\end{savenotes}

Example 9.40: Computation of decile dispersion ratio and share of total
consumption by the poorest decile

\begin{DUlineblock}{0em}
\item[] \sphinxstyleemphasis{\# Decile dispersion ratio}
\item[] \sphinxstyleemphasis{\# raw data}
\item[] \sphinxstylestrong{mean}(\sphinxstylestrong{tail}(\sphinxstylestrong{sort}(fileOrigHH\$INCTOTGROSSHH), n = 200))
/ \sphinxstylestrong{mean}(\sphinxstylestrong{head}(\sphinxstylestrong{sort}(fileOrigHH\$INCTOTGROSSHH), n =
200))
\end{DUlineblock}

\sphinxcode{\sphinxupquote{\#\# {[}1{]} 24.12152}}

\sphinxstylestrong{mean}(\sphinxstylestrong{tail}(\sphinxstylestrong{sort}(HHmanip\$INCTOTGROSSHH), n = 197)) /
\sphinxstylestrong{mean}(\sphinxstylestrong{head}(\sphinxstylestrong{sort}(HHmanip\$INCTOTGROSSHH), n = 197))

\sphinxcode{\sphinxupquote{\#\# {[}1{]} 23.54179}}

\begin{DUlineblock}{0em}
\item[] \sphinxstyleemphasis{\# Share of total consumption by the poorest decile households}
\item[] \sphinxstylestrong{sum}(\sphinxstylestrong{head}(\sphinxstylestrong{sort}(fileOrigHH\$TANHHEXP), n = 200)) /
\sphinxstylestrong{sum}(fileOrigHH\$TANHHEXP)
\end{DUlineblock}

\sphinxcode{\sphinxupquote{\#\# {[}1{]} 0.003411664}}

\sphinxstylestrong{sum}(\sphinxstylestrong{head}(\sphinxstylestrong{sort}(HHmanip\$TANHHEXP), n = 197)) /
\sphinxstylestrong{sum}(HHmanip\$TANHHEXP)

\sphinxcode{\sphinxupquote{\#\# {[}1{]} 0.003530457}}

\sphinxstylestrong{Merging the household- and individual-level variables}

The next step is to merge the treated household variables with the
untreated individual variables for the anonymization of the individual
level variables. Example 9.41 shows the steps to merge these files. This
also includes the selection of variables used in the anonymization of
the individual-level variables. We create the \sphinxstyleemphasis{sdcMicro} object for the
anonymization of the individual variables in the same way as for the
household variable in Example 9.31. Generally, at this stage, the
household level and individual level variables should be combined and
quasi-identifiers at both levels be used (see Section 4.4).
Unfortunately, in our dataset, this leads to long computation times.
Therefore, we create two \sphinxstyleemphasis{sdcMicro} objects, one with all key variables
(“sdcCombinedAll”) and one with only the individual level key variables
(“sdcCombined”). In Step 6b we compare the risk measures for both cases
and in Step 8b we discuss alternative approaches to keeping the complete
set of variables. We now repeat Steps 6-10 for the individual-level
variables.

Example 9.41: Merging the files with household and individual-level
variables and creating an \sphinxstyleemphasis{sdcMicro} object for the anonymization of the
individual-level variables

\begin{DUlineblock}{0em}
\item[] \#\#\# Select variables (individual level)
\item[] selectedKeyVarsIND = \sphinxstylestrong{c}(‘GENDER’, ‘REL’, ‘MARITAL’, ‘AGEYRS’,
‘EDUCY’, ‘INDUSTRY1’) \sphinxstyleemphasis{\# list of selected key variables}
\item[] \sphinxstyleemphasis{\# sample weight (WGTHH, individual weight)}
\item[] selectedWeightVarIND = \sphinxstylestrong{c}(‘WGTHH’)
\item[] \sphinxstyleemphasis{\# Household ID}
\item[] selectedHouseholdID = \sphinxstylestrong{c}(‘IDH’)
\item[] \sphinxstyleemphasis{\# Variables not suitable for release in PUF (IND level)}
\item[] varsNotToBeReleasedIND \textless{}- \sphinxstylestrong{c}(“ATSCHOOL”, “EDYRSCURRAT”)
\item[] \sphinxstyleemphasis{\# All individual level variables}
\item[] INDVars \textless{}- \sphinxstylestrong{c}(selectedKeyVarsIND)
\item[] \sphinxstyleemphasis{\# Recombining anonymized HH data sets and individual level variables}
\item[] indVars \textless{}- \sphinxstylestrong{c}(“IDH”, “IDP”, selectedKeyVarsIND, “WGTHH”) \sphinxstyleemphasis{\# HID
and all non HH vars}
\item[] fileInd \textless{}- file{[}indVars{]} \sphinxstyleemphasis{\# subset of file without HHVars}
\item[] fileCombined \textless{}- \sphinxstylestrong{merge}(HHmanip, fileInd, by.x = \sphinxstylestrong{c}(‘IDH’))
\item[] fileCombined \textless{}- fileCombined{[}\sphinxstylestrong{order}(fileCombined{[},’IDH’{]},
fileCombined{[},’IDP’{]}),{]}
\item[] \sphinxstylestrong{dim}(fileCombined)
\end{DUlineblock}

\sphinxcode{\sphinxupquote{\#\# {[}1{]} 10068    44}}

\begin{DUlineblock}{0em}
\item[] \sphinxstyleemphasis{\# SDC objects with only IND level variables}
\item[] sdcCombined \textless{}- \sphinxstylestrong{createSdcObj}(dat = fileCombined, keyVars =
\sphinxstylestrong{c}(selectedKeyVarsIND), weightVar = selectedWeightVarIND, hhId =
selectedHouseholdID)
\end{DUlineblock}

\begin{DUlineblock}{0em}
\item[] \sphinxstyleemphasis{\# SDC objects with both HH and IND level variables}
\item[] sdcCombinedAll \textless{}- \sphinxstylestrong{createSdcObj}(dat = fileCombined, keyVars =
\sphinxstylestrong{c}(selectedKeyVarsIND, selectedKeyVarsHH ), weightVar =
selectedWeightVarIND, hhId = selectedHouseholdID)
\end{DUlineblock}

\sphinxstylestrong{Step 6b: Assessing disclosure risk (individual level)}

As first measure, we evaluate the number of records violating
\(k\)-anonymity at the levels 2, 3 and 5. Table 9.23 shows the
number of violating individuals as well as the percentage of the total
number of households. The second and third column refer to “sdcCombined”
and the fourth and fifth column to “sdcCombinedAll”. We see that
combining the individual level and household level variables leads to a
large increase in the number of \(k\)-anonymity violators. The
choice not to include the household level variables is pragmatically
driven by the computation time and can be justified by the different
type of variables on the household level and individual level. One could
assume that these variables are not available in the same dataset and
can therefore not simultaneously be used by an intruder to re-identify
individuals.

Table 9.23: Number of records violating k-anonimity


\begin{savenotes}\sphinxattablestart
\centering
\begin{tabulary}{\linewidth}[t]{|T|T|T|T|T|}
\hline
\sphinxstyletheadfamily &\sphinxstartmulticolumn{2}%
\begin{varwidth}[t]{\sphinxcolwidth{2}{5}}
\sphinxstyletheadfamily sdcCombined
\par
\vskip-\baselineskip\vbox{\hbox{\strut}}\end{varwidth}%
\sphinxstopmulticolumn
&\sphinxstartmulticolumn{2}%
\begin{varwidth}[t]{\sphinxcolwidth{2}{5}}
\sphinxstyletheadfamily sdcCombinedAll
\par
\vskip-\baselineskip\vbox{\hbox{\strut}}\end{varwidth}%
\sphinxstopmulticolumn
\\
\hline
k-anonymity
&
Number of
records
violating
&
Percentage

of total
records
&
Number of
records
violating
&
Percentage

of total
records
\\
\hline
2
&
0
&
0.0 \%
&
4,048
&
40.2 \%
\\
\hline
3
&
167
&
1.7 \%
&
6,107
&
60.7 \%
\\
\hline
5
&
463
&
4.6 \%
&
8,292
&
82.4 \%
\\
\hline
\end{tabulary}
\par
\sphinxattableend\end{savenotes}

The global hierarchical risk measure is 0.095\%, which corresponds to
approximately 10 expected re-identifications. We use here the
hierarchical risk measure, since the re-identification of a single
household member would lead to the re-identification of the other
members of the same household too. This number is low compared to the
number of \(k\)-anonymity violations, due to the high sample
weights, which protect the data already to a large extent. Only 24
observations have an individual hierarchical risk higher than 1\%, with a
maximum of 1.17\%. This is mainly because of the lower sample weights of
these records. Example 9.42 shows how to retrieve these measures in \sphinxstyleemphasis{R}.

Example 9.42: Risk measures before anonymization

numIND \textless{}- \sphinxstylestrong{length}(fileCombined{[},1{]}) \sphinxstyleemphasis{\# number of households}

\begin{DUlineblock}{0em}
\item[] \sphinxstyleemphasis{\# Number of observations violating k-anonymity}
\item[] \sphinxstylestrong{print}(sdcCombined)
\end{DUlineblock}

\begin{DUlineblock}{0em}
\item[] \sphinxcode{\sphinxupquote{\#\# Infos on 2/3-Anonymity:}}
\item[] \sphinxcode{\sphinxupquote{\#\#}}
\item[] \sphinxcode{\sphinxupquote{\#\# Number of observations violating}}
\item[] \sphinxcode{\sphinxupquote{\#\#  - 2-anonymity: 0}}
\item[] \sphinxcode{\sphinxupquote{\#\#  - 3-anonymity: 167}}
\item[] \sphinxcode{\sphinxupquote{\#\#}}
\item[] \sphinxcode{\sphinxupquote{\#\# Percentage of observations violating}}
\item[] \sphinxcode{\sphinxupquote{\#\#  - 2-anonymity: 0.000 \%}}
\item[] \sphinxcode{\sphinxupquote{\#\#  - 3-anonymity: 1.659 \%}}
\item[] \sphinxcode{\sphinxupquote{\#\# -{-}-{-}-{-}-{-}-{-}-{-}-{-}-{-}-{-}-{-}-{-}-{-}-{-}-{-}-{-}-{-}-{-}-{-}-{-}-{-}-{-}-{-}-{-}-{-}-{-}-{-}-{-}-{-}-{-}-{-}-{-}-{-}-{-}-{-}-{-}-{-}-{-}-}}
\end{DUlineblock}

\begin{DUlineblock}{0em}
\item[] \sphinxstyleemphasis{\# Calculate sample frequencies and count number of obs. violating k
(3,5) - anonymity}
\item[] kAnon5 \textless{}- \sphinxstylestrong{sum}(sdcCombined@risk\$individual{[},2{]} \textless{}5)
\item[] kAnon5
\end{DUlineblock}

\sphinxcode{\sphinxupquote{\#\# {[}1{]} 463}}

\begin{DUlineblock}{0em}
\item[] \sphinxstyleemphasis{\# As percentage of total}
\item[] kAnon5 / numIND
\end{DUlineblock}

\sphinxcode{\sphinxupquote{\#\# {[}1{]} 0.04598729}}

\begin{DUlineblock}{0em}
\item[] \sphinxstyleemphasis{\# Global risk on individual level}
\item[] \sphinxstylestrong{print}(sdcCombined, ‘risk’)
\end{DUlineblock}

\begin{DUlineblock}{0em}
\item[] \sphinxcode{\sphinxupquote{\#\# Risk measures:}}
\item[] \sphinxcode{\sphinxupquote{\#\#}}
\item[] \sphinxcode{\sphinxupquote{\#\# Number of observations with higher risk than the main part of the data: 0}}
\item[] \sphinxcode{\sphinxupquote{\#\# Expected number of re-identifications: 1.69 (0.02 \%)}}
\item[] \sphinxcode{\sphinxupquote{\#\#}}
\item[] \sphinxcode{\sphinxupquote{\#\# Information on hierarchical risk:}}
\item[] \sphinxcode{\sphinxupquote{\#\# Expected number of re-identifications: 9.57 (0.10 \%)}}
\item[] \sphinxcode{\sphinxupquote{\#\# -{-}-{-}-{-}-{-}-{-}-{-}-{-}-{-}-{-}-{-}-{-}-{-}-{-}-{-}-{-}-{-}-{-}-{-}-{-}-{-}-{-}-{-}-{-}-{-}-{-}-{-}-{-}-{-}-{-}-{-}-{-}-{-}-{-}-{-}-{-}-{-}-{-}-}}
\end{DUlineblock}

\begin{DUlineblock}{0em}
\item[] \sphinxstyleemphasis{\# Number of observation with relatively high risk}
\item[] \sphinxstylestrong{dim}(fileCombined{[}sdcCombined@risk\$individual{[}, “hier\_risk”{]} \textgreater{}
0.01,{]})
\end{DUlineblock}

\sphinxcode{\sphinxupquote{\#\# {[}1{]} 24 44}}

\begin{DUlineblock}{0em}
\item[] \sphinxstyleemphasis{\# Highest individual risk}
\item[] \sphinxstylestrong{max}(sdcCombined@risk\$individual{[}, “hier\_risk”{]})
\end{DUlineblock}

\sphinxcode{\sphinxupquote{\#\# {[}1{]} 0.01169091}}

\sphinxstylestrong{Step 7b: Assessing utility measures (individual level)}

We evaluate the utility measures as discussed in Step 5 based on the raw
data (before applying any anonymization measures). The results are
presented in Step 10b together with the values after anonymization to
allow for direct comparison.

\sphinxstylestrong{Step 8b: Choice and application of SDC methods (individual level)}

In this step, we discuss four different techniques used for
anonymization: 1) removing variables from the dataset to be released, 2)
recoding of categorical variables to reduce the level of detail, 3)
local suppression to achieve the required level of \(k\)-anonymity,
4) randomization of the order of the records in the file. Finally, we
discuss some alternative options for treating the household structure in
the dataset.

\sphinxstylestrong{Removing variables}

Additional to the variables removed from the dataset for the SUF release
(see case study 1), we further reduce the number of variables in the
dataset to be released. This is normal practice for PUF releases.
Sensitive or identifying variables are removed, which allows to release
other variables at a more detailed level. In a PUF release, the set of
key variables should be limited.

In our case, we decide to remove at the individual level the variables
“EDYRSCURRAT”, as this variable is too identifying (identifies whether
there are school-going children in the household). We keep the variable
“EDUCY” (highest level of education attended) for information on
education. \sphinxstylestrong{NOTE: As an alternative to removing the variables from the
dataset, one could also set all values to missing. This would allow the
user to see the structure and variables contained in the SUF file.}

\sphinxstylestrong{Recoding}

As noted before, PUF users require a lower level of information and
therefore we can recode the key variables even further to reduce the
disclosure risk. The recoding of variables in case study 1 is not
sufficient for a PUF release. Therefore, we recode most of the
categorical key variables from Table 9.19 to reduce the risk and number
of necessary suppressions by local suppression. Table 9.24 gives an
overview of the recodes made. All new categories are formed with the
needs of the data user in mind. Example 9.43 shows how to do this in \sphinxstyleemphasis{R}
and also shows value labels and the univariate tabulations of these
variables before and after recoding.

Table 9.24: Overview of recodes of categorical variables at individual
level


\begin{savenotes}\sphinxattablestart
\centering
\begin{tabulary}{\linewidth}[t]{|T|T|}
\hline
\sphinxstyletheadfamily 
\sphinxstylestrong{Variable}
&\sphinxstyletheadfamily 
\sphinxstylestrong{Recoding}
\\
\hline
REL (relation to household head)
&
recode ‘Father/Mother’, ‘
Grandchild’, ‘Son/Daughter in
law’, ‘Other relative’ to ‘Other
relative’ and recode ‘Domestic
help’ and ‘Non-relative’ to
‘Other’
\\
\hline
MARITAL (marital status)
&
recode ‘Married monogamous’,
‘Married polygamous’, ’Common
law, union coutumiere, union
libre, living together’ to
‘Married/living together’ and
‘Divorced/Separated’ and
‘Widowed’ to
‘Divorced/Separated/Widowed’
\\
\hline
AGEYRS (age in completed years)
&
recode values under 15 to 7
(other values have been recoded
for SUF)
\\
\hline
EDUCY (highest level of education
completed)
&
recode ‘Completed lower secondary
(or post-primary vocational
education) but less than
completed upper secondary’,
‘Completed upper secondary (or
extended vocational/technical
education)’, ‘Post secondary
technical’ and ‘University and
higher’ to ‘Completed lower
secondary or higher’
\\
\hline
INDUSTRY1
&
recode to ‘primary’, ‘secondary’
and ‘tertiary’
\\
\hline
\end{tabulary}
\par
\sphinxattableend\end{savenotes}

Example 9.43: Recoding the categorical and continuous variables

\begin{DUlineblock}{0em}
\item[] \sphinxstyleemphasis{\# Recode REL (relation to household head)}
\item[] \sphinxstylestrong{table}(\sphinxhref{mailto:sdcCombined@manipKeyVars\$REL}{sdcCombined@manipKeyVars\$REL}, useNA = “ifany”)
\end{DUlineblock}

\begin{DUlineblock}{0em}
\item[] \sphinxcode{\sphinxupquote{\#\#}}
\item[] \sphinxcode{\sphinxupquote{\#\#    1    2    3    4    5    6    7    8    9 \textless{}NA\textgreater{}}}
\item[] \sphinxcode{\sphinxupquote{\#\# 1698 1319 4933   52  765   54  817   40   63  327}}
\end{DUlineblock}

\begin{DUlineblock}{0em}
\item[] \sphinxstyleemphasis{\# 1 - Head, 2 - Spouse, 3 - Child, 4 - Father/Mother, 5 - Grandchild,
6 - Son/Daughter in law}
\item[] \sphinxstyleemphasis{\# 7 - Other relative, 8 - Domestic help, 9 - Non-relative}
\item[] sdcCombined \textless{}- \sphinxstylestrong{groupVars}(sdcCombined, var = “REL”, before =
\sphinxstylestrong{c}(“4”, “5”, “6”, “7”), after = \sphinxstylestrong{c}(“7”, “7”, “7”, “7”)) \sphinxstyleemphasis{\#
other relative}
\item[] sdcCombined \textless{}- \sphinxstylestrong{groupVars}(sdcCombined, var = “REL”, before =
\sphinxstylestrong{c}(“8”, “9”), after = \sphinxstylestrong{c}(“9”, “9”)) \sphinxstyleemphasis{\# other}
\item[] \sphinxstylestrong{table}(\sphinxhref{mailto:sdcCombined@manipKeyVars\$REL}{sdcCombined@manipKeyVars\$REL}, useNA = “ifany”)
\end{DUlineblock}

\begin{DUlineblock}{0em}
\item[] \sphinxcode{\sphinxupquote{\#\#}}
\item[] \sphinxcode{\sphinxupquote{\#\#    1    2    3    7    9 \textless{}NA\textgreater{}}}
\item[] \sphinxcode{\sphinxupquote{\#\# 1698 1319 4933 1688  103  327}}
\end{DUlineblock}

\begin{DUlineblock}{0em}
\item[] \sphinxstyleemphasis{\# Recode MARITAL (marital status)}
\item[] \sphinxstylestrong{table}(\sphinxhref{mailto:sdcCombined@manipKeyVars\$MARITAL}{sdcCombined@manipKeyVars\$MARITAL}, useNA = “ifany”)
\end{DUlineblock}

\begin{DUlineblock}{0em}
\item[] \sphinxcode{\sphinxupquote{\#\#}}
\item[] \sphinxcode{\sphinxupquote{\#\#    1    2    3    4    5    6 \textless{}NA\textgreater{}}}
\item[] \sphinxcode{\sphinxupquote{\#\# 3542 2141  415  295  330  329 3016}}
\end{DUlineblock}

\begin{DUlineblock}{0em}
\item[] \sphinxstyleemphasis{\# 1 - Never married, 2 - Married monogamous, 3 - Married polygamous,}
\item[] \sphinxstyleemphasis{\# 4 - Common law, union coutumiere, union libre, living together, 5 -
Divorced/Separated, 6 - Widowed}
\item[] sdcCombined \textless{}- \sphinxstylestrong{groupVars}(sdcCombined, var = “MARITAL”, before =
\sphinxstylestrong{c}(“2”, “3”, “4”), after = \sphinxstylestrong{c}(“2”, “2”, “2”)) \sphinxstyleemphasis{\#
married/living together}
\item[] sdcCombined \textless{}- \sphinxstylestrong{groupVars}(sdcCombined, var = “MARITAL”, before =
\sphinxstylestrong{c}(“5”, “6”), after = \sphinxstylestrong{c}(“9”, “9”)) \sphinxstyleemphasis{\#
divorced/seperated/widowed}
\item[] \sphinxstylestrong{table}(\sphinxhref{mailto:sdcCombined@manipKeyVars\$MARITAL}{sdcCombined@manipKeyVars\$MARITAL}, useNA = “ifany”)
\end{DUlineblock}

\begin{DUlineblock}{0em}
\item[] \sphinxcode{\sphinxupquote{\#\#}}
\item[] \sphinxcode{\sphinxupquote{\#\#    1    2    9 \textless{}NA\textgreater{}}}
\item[] \sphinxcode{\sphinxupquote{\#\# 3542 2851  659 3016}}
\end{DUlineblock}

\begin{DUlineblock}{0em}
\item[] \sphinxstyleemphasis{\# Recode AGEYRS (0-15 years)}
\item[] \sphinxstylestrong{table}(\sphinxhref{mailto:sdcCombined@manipKeyVars\$AGEYRS}{sdcCombined@manipKeyVars\$AGEYRS}, useNA = “ifany”)
\end{DUlineblock}

\begin{DUlineblock}{0em}
\item[] \sphinxcode{\sphinxupquote{\#\#}}
\item[] \sphinxcode{\sphinxupquote{\#\#    0    1    2    3    4    5    6    7    8    9   10   11   12   13   14}}
\item[] \sphinxcode{\sphinxupquote{\#\#  311  367  340  332  260  334  344  297  344  281  336  297  326  299  263}}
\item[] \sphinxcode{\sphinxupquote{\#\#   20   30   40   50   60   65 \textless{}NA\textgreater{}}}
\item[] \sphinxcode{\sphinxupquote{\#\# 1847 1220  889  554  314  325  188}}
\end{DUlineblock}

\begin{DUlineblock}{0em}
\item[] sdcCombined \textless{}- \sphinxstylestrong{groupVars}(sdcCombined, var = “AGEYRS”, before =
\sphinxstylestrong{c}(“0”, “1”, “2”, “3”, “4”, “5”, “6”, “7”, “8”, “9”, “10”, “11”,
“12”, “13”, “14”)
\item[] , after = \sphinxstylestrong{rep}(“7”, 15))
\item[] \sphinxstylestrong{table}(\sphinxhref{mailto:sdcCombined@manipKeyVars\$AGEYRS}{sdcCombined@manipKeyVars\$AGEYRS}, useNA = “ifany”)
\end{DUlineblock}

\begin{DUlineblock}{0em}
\item[] \sphinxcode{\sphinxupquote{\#\#}}
\item[] \sphinxcode{\sphinxupquote{\#\#    7   20   30   40   50   60   65 \textless{}NA\textgreater{}}}
\item[] \sphinxcode{\sphinxupquote{\#\# 4731 1847 1220  889  554  314  325  188}}
\end{DUlineblock}

\begin{DUlineblock}{0em}
\item[] sdcCombined \textless{}- \sphinxstylestrong{calcRisks}(sdcCombined)
\item[] \sphinxstyleemphasis{\# Recode EDUCY (highest level of educ compl)}
\item[] \sphinxstylestrong{table}(\sphinxhref{mailto:sdcCombined@manipKeyVars\$EDUCY}{sdcCombined@manipKeyVars\$EDUCY}, useNA = “ifany”)
\end{DUlineblock}

\begin{DUlineblock}{0em}
\item[] \sphinxcode{\sphinxupquote{\#\#}}
\item[] \sphinxcode{\sphinxupquote{\#\#    0    1    2    3    4    5    6 \textless{}NA\textgreater{}}}
\item[] \sphinxcode{\sphinxupquote{\#\# 1582 4755 1062  330  139   46  104 2050}}
\end{DUlineblock}

\begin{DUlineblock}{0em}
\item[] \sphinxstyleemphasis{\# 0 - No education, 1 - Pre-school/ Primary not completed, 2 -
Completed primary, but less than completed lower secondary}
\item[] \sphinxstyleemphasis{\# 3 - Completed lower secondary (or post-primary vocational
education) but less than completed upper secondary}
\item[] \sphinxstyleemphasis{\# 4 - Completed upper secondary (or extended vocational/technical
education), 5 - Post secondary technical}
\item[] \sphinxstyleemphasis{\# 6 - University and higher}
\item[] sdcCombined \textless{}- \sphinxstylestrong{groupVars}(sdcCombined, var = “EDUCY”, before =
\sphinxstylestrong{c}(“3”, “4”, “5”, “6”), after = \sphinxstylestrong{c}(“3”, “3”, “3”, “3”)) \sphinxstyleemphasis{\#
completed lower secondary or higher}
\item[] \sphinxstylestrong{table}(\sphinxhref{mailto:sdcCombined@manipKeyVars\$EDUCY}{sdcCombined@manipKeyVars\$EDUCY}, useNA = “ifany”)
\end{DUlineblock}

\begin{DUlineblock}{0em}
\item[] \sphinxcode{\sphinxupquote{\#\#}}
\item[] \sphinxcode{\sphinxupquote{\#\#    0    1    2    3 \textless{}NA\textgreater{}}}
\item[] \sphinxcode{\sphinxupquote{\#\# 1582 4755 1062  619 2050}}
\end{DUlineblock}

\begin{DUlineblock}{0em}
\item[] \sphinxstyleemphasis{\# Recode INDUSTRY1 ()}
\item[] \sphinxstylestrong{table}(\sphinxhref{mailto:sdcCombined@manipKeyVars\$INDUSTRY1}{sdcCombined@manipKeyVars\$INDUSTRY1}, useNA = “ifany”)
\end{DUlineblock}

\begin{DUlineblock}{0em}
\item[] \sphinxcode{\sphinxupquote{\#\#}}
\item[] \sphinxcode{\sphinxupquote{\#\#    1    2    3    4    5    6    7    8    9   10 \textless{}NA\textgreater{}}}
\item[] \sphinxcode{\sphinxupquote{\#\# 5300   16  153    2   93  484   95   17   70  292 3546}}
\end{DUlineblock}

\begin{DUlineblock}{0em}
\item[] \sphinxstyleemphasis{\# 1 - Agriculture and Fishing, 2 - Mining, 3 - Manufacturing, 4 -
Electricity and Utilities}
\item[] \sphinxstyleemphasis{\# 5 - Construction, 6 - Commerce, 7 - Transportation, Storage and
Communication, 8 - Financial, Insurance and Real Estate}
\item[] \sphinxstyleemphasis{\# 9 - Services: Public Administration, 10 - Other Services, 11 -
Unspecified}
\item[] sdcCombined \textless{}- \sphinxstylestrong{groupVars}(sdcCombined, var = “INDUSTRY1”, before
= \sphinxstylestrong{c}(“1”, “2”), after = \sphinxstylestrong{c}(“1”, “1”)) \sphinxstyleemphasis{\# primary}
\item[] sdcCombined \textless{}- \sphinxstylestrong{groupVars}(sdcCombined, var = “INDUSTRY1”, before
= \sphinxstylestrong{c}(“3”, “4”, “5”), after = \sphinxstylestrong{c}(“2”, “2”, “2”)) \sphinxstyleemphasis{\#
secondary}
\item[] sdcCombined \textless{}- \sphinxstylestrong{groupVars}(sdcCombined, var = “INDUSTRY1”, before
= \sphinxstylestrong{c}(“6”, “7”, “8”, “9”, “10”), after = \sphinxstylestrong{c}(“3”, “3”, “3”,
“3”, “3”)) \sphinxstyleemphasis{\# tertiary}
\item[] \sphinxstylestrong{table}(\sphinxhref{mailto:sdcCombined@manipKeyVars\$INDUSTRY1}{sdcCombined@manipKeyVars\$INDUSTRY1}, useNA = “ifany”)
\end{DUlineblock}

\begin{DUlineblock}{0em}
\item[] \sphinxcode{\sphinxupquote{\#\#}}
\item[] \sphinxcode{\sphinxupquote{\#\#    1    2    3 \textless{}NA\textgreater{}}}
\item[] \sphinxcode{\sphinxupquote{\#\# 5316  248  958 3546}}
\end{DUlineblock}

\sphinxstylestrong{Local suppression}

The recoding has reduced the risk already considerably. We use local
suppression to achieve the required level of \(k\)-anonymity.
Generally, the required level of \(k\)-anonymity for PUF files is 3
or 5. In this case study, we require 5-anonimity. Example 9.44 shows the
suppression pattern without specifying an importance vector. All
suppressions are made in the variable “AGEYRS”. This is the variable
with the highest number of different values, and hence considered first
by the algorithm. We try different suppression patterns by specifying
importance vectors, but we decide that the pattern without importance
vector yields the best result. This is also the result with the lowest
total number of suppressions. Less than 1 percent suppression in the age
variable is acceptable. We could reduce this number by further recoding
the variable “AGEYRS”.

Example 9.44: Local suppression to reach 5-anonimity

\begin{DUlineblock}{0em}
\item[] \sphinxstyleemphasis{\# Local suppression without importance vector}
\item[] sdcCombined \textless{}- \sphinxstylestrong{localSuppression}(sdcCombined, k = 5, importance =
NULL)
\item[] \sphinxstyleemphasis{\# Number of suppressions per variable}
\item[] \sphinxstylestrong{print}(sdcCombined, “ls”)
\end{DUlineblock}

\begin{DUlineblock}{0em}
\item[] \sphinxcode{\sphinxupquote{\#\# Local Suppression:}}
\item[] \sphinxcode{\sphinxupquote{\#\#     KeyVar \textbar{} Suppressions (\#) \textbar{} Suppressions (\%)}}
\item[] \sphinxcode{\sphinxupquote{\#\#     GENDER \textbar{}                0 \textbar{}            0.000}}
\item[] \sphinxcode{\sphinxupquote{\#\#        REL \textbar{}                0 \textbar{}            0.000}}
\item[] \sphinxcode{\sphinxupquote{\#\#    MARITAL \textbar{}                0 \textbar{}            0.000}}
\item[] \sphinxcode{\sphinxupquote{\#\#     AGEYRS \textbar{}               91 \textbar{}            0.904}}
\item[] \sphinxcode{\sphinxupquote{\#\#      EDUCY \textbar{}                0 \textbar{}            0.000}}
\item[] \sphinxcode{\sphinxupquote{\#\#  INDUSTRY1 \textbar{}                0 \textbar{}            0.000}}
\end{DUlineblock}

\sphinxstylestrong{Randomization of order of records}

The records in the dataset are ordered by region and household ID. There
is a certain geographical order of the households within the regions,
due to the way the households IDs were assigned. Intruders could
reconstruct suppressed values by using this structure. To prevent this,
we randomly reorder the records within the regions. Example 9.45 shows
how to do this in \sphinxstyleemphasis{R}. We first count the number of records per region
(\sphinxstylestrong{NOTE: Some records have their region value suppressed, so we include
the count of NAs}). Subsequently, we draw randomly household IDs, in
such way that the regional division is respected. Finally, we sort the
file by the new, randomized, individual ID (“IDP”). Households with
suppressed values for “REGION” will be last in the reordered file.
Before randomizing the order, we extract the data from the \sphinxstyleemphasis{sdcMicro}
object “sdcCombined” as shown in Example 9.45.

Example 9.45: Randomizing the order of records within regions

\begin{DUlineblock}{0em}
\item[] \sphinxstyleemphasis{\# Randomize order of households dataAnon and recode IDH to random
number (sort file by region)}
\item[] \sphinxstylestrong{set.seed}(97254)
\item[] \sphinxstyleemphasis{\# Sort by region}
\item[] dataAnon \textless{}- dataAnon{[}\sphinxstylestrong{order}(dataAnon\$REGION),{]}
\end{DUlineblock}

\begin{DUlineblock}{0em}
\item[] \sphinxstyleemphasis{\# Number of households per region}
\item[] hhperregion \textless{}-
\sphinxstylestrong{table}(dataAnon{[}\sphinxstylestrong{match}(\sphinxstylestrong{unique}(dataAnon\$IDH),
dataAnon\$IDH), “REGION”{]}, useNA = “ifany”)
\item[] \sphinxstyleemphasis{\# Randomized IDH (household ID)}
\item[] randomHHid \textless{}- \sphinxstylestrong{c}(\sphinxstylestrong{sample}(1:hhperregion{[}1{]}, hhperregion{[}1{]}),
\sphinxstylestrong{unlist}(\sphinxstylestrong{lapply}(1:(\sphinxstylestrong{length}(hhperregion)-1),
function(i)\{\sphinxstylestrong{sample}((\sphinxstylestrong{sum}(hhperregion{[}1:i{]}) + 1):
\sphinxstylestrong{sum}(hhperregion{[}1:(i+1){]}), hhperregion{[}(i+1){]})\})))
\end{DUlineblock}

\begin{DUlineblock}{0em}
\item[] dataAnon\$IDH \textless{}- \sphinxstylestrong{rep}(randomHHid,
\sphinxstylestrong{table}(dataAnon\$IDH){[}\sphinxstylestrong{match}(\sphinxstylestrong{unique}(dataAnon\$IDH),
\sphinxstylestrong{as.numeric}(\sphinxstylestrong{names}(\sphinxstylestrong{table}(dataAnon\$IDH)))){]})
\item[] \sphinxstyleemphasis{\# Sort by IDH (and region)}
\item[] dataAnon \textless{}- dataAnon{[}\sphinxstylestrong{order}(dataAnon\$IDH),{]}
\end{DUlineblock}

\sphinxstylestrong{Alternative options for dealing with household structure}

In Step 6b we compared the disclosure risk for two cases: one with only
individual level key variables and another with individual level and
household level key variables combined. We decided to use only the
individual level key variables to reduce the computation time and
justified this choice by arguing that intruders cannot use household and
individual level variables simultaneously. This might not always be the
case. Therefore we explore other options to reduce the risk when taking
both individual level and household level variables into account. We
present two options: removing the household structure; and using options
in the local suppression algorithm.

\sphinxstyleemphasis{Removing household structure}

We consider the risk emanating from the household structure in the
dataset to be very high. We can remove the hierarchical household
structure completely and treat all variables at the individual level.
This entails, besides removing the household id (“IDH”), also treating
variables that could be used for reconstructing households. These are,
for instance, “REL” (relation to household head), “HHSIZE” (household
size), and any of the household level variables, such as income and
expenditure. However, not all household level variables need to be
treated. For example, “REGION” is a household level variable, but the
probability that this variable leads to the reconstruction of a
household is low. Also, we need to reorder the records in the file, as
they are sorted by households. Note that by removing the household
structure, we interpret all variables as individual level variables for
measuring disclosure risk. This leads to a lower need for recoding and
suppression, since the hierarchical risk disappears. The reason why we
did not opt for this approach is the loss of utility for the user. The
household structure is an important feature of the data, and should be
kept in the PUF file.

\sphinxstyleemphasis{Using different options for local suppression}

The long running time is mainly due to the local suppression algorithm.
In Section 5.2.2 we discuss options to reduce the running time of the
local suppression in case of many key variables. The all-\(m\)
approach reduces the running time by first considering subsets of the
complete set of key variables. This reduces the complexity of the
problem and leads to lower computation times. However, the total number
of suppressions made is likely to be higher. Also, if not explicitly
specified, it is not guaranteed that the required level for
\(k\)-anonymity is automatically achieved on the complete set of key
variables. It is therefore important to check the results.

\sphinxstylestrong{Step 9b: Re-measure risk}

We re-evaluate the risk measures selected in Step 6. Table 9.25 shows
that local suppression, not surprisingly, has reduced the number of
individuals violating 5-anonymity to 0. The global hierarchical risk was
reduced to 0.02\%, which corresponds to approximately 2 correct
re-identifications. The highest individual hierarchical
re-identification risk is 0.2\%. These risk levels are acceptable for a
PUF release. Furthermore, the recoding has removed any unusual
combinations in the data. \sphinxstylestrong{NOTE: The risk may be underestimated by
excluding the household level variables.}

Table 9.25: k-anonymity violations


\begin{savenotes}\sphinxattablestart
\centering
\begin{tabulary}{\linewidth}[t]{|T|T|T|}
\hline
\sphinxstyletheadfamily 
\sphinxstylestrong{k-anonymity}
&\sphinxstyletheadfamily 
\sphinxstylestrong{Number of records violating}
&\sphinxstyletheadfamily 
\sphinxstylestrong{Percentage}
\\
\hline
2
&
0
&
0.0 \%
\\
\hline
3
&
0
&
0.0 \%
\\
\hline
5
&
0
&
0.0 \%
\\
\hline
\end{tabulary}
\par
\sphinxattableend\end{savenotes}

\sphinxstylestrong{Step 10b: Re-measure utility}

We compare (cross-)tabulations before and after anonymization, which are
illustrated in the \sphinxstyleemphasis{R} code to this case study. We note that due to the
recoding in Step 8b, the detail in the variables is reduced. This
reduces the number of necessary suppressions and is acceptable for a
public use file.

\sphinxstylestrong{Step 11: Audit and reporting}

In the audit step, we check whether the data allow for reproduction of
published figures from the original dataset and relationships between
variables and other data characteristics are preserved in the
anonymization process. In short, we check whether the dataset is valid
for analytical purposes. There are no figures available that were
published from the dataset and need to be reproducible from the
anonymized data.

In Step 2, we explored the data characteristics and relationships
between variables. These data characteristics and relationships have
been mainly preserved, since we took them into account when choosing the
appropriate anonymization methods. Since values of the variable “AGEYRS”
were not perturbed, but only recoded and suppressed, we did not
introduce unlikely combinations, such as a 60-year-old individual
enrolled in primary education. Also, by separating the anonymization
process into two parts, one for household-level variables and one for
individual-level variables, the values of variables measured at the
household level agree for all members of each household.

Furthermore, we drafted two reports, internal and external, on the
anonymization of the case study dataset. The internal report includes
the methods used, the risk before and after anonymization as well as the
reasons for the selected methods and their parameters. The external
report focuses on the changes in the data and the loss in utility. Focus
here should be on the number of suppressions as well as the perturbative
methods (PRAM). This is described in the previous steps. \sphinxstylestrong{NOTE: When
creating a PUF, it is inevitable that there will be a loss of
information and it is very important for the users to be aware of these
changes and release them in a report that accompanies the data.}
Appendix C provides examples of an internal and external report of the
anonymization process of this dataset. Depending on the users and
readers of the reports, the content may differ. \sphinxstylestrong{NOTE: The report()
function in sdcMicro} \sphinxstylestrong{is at this point not useful, since this will
only report on the SDC measures in the second case study.} However, the
report should contain the entire process, including the measures applied
in case study 1.

\sphinxstylestrong{Step 12: Data release}

The final step is the release of the anonymized dataset together with
the external report. Example 9.46 shows how to export the anonymized
dataset as \sphinxstyleemphasis{STATA} file. Section 7.2 presents functions for exporting
files in other data formats.

Example 9.46: Exporting the anonymized PUF file

\begin{DUlineblock}{0em}
\item[] \sphinxstyleemphasis{\# Create STATA file}
\item[] \sphinxstylestrong{write.dta}(dataframe = dataAnon, file= ‘Case2DataAnon.dta’,
convert.dates=TRUE)
\end{DUlineblock}

\begin{sphinxthebibliography}{DoTo01a}
\bibitem[Bran02]{\detokenize{Bran02}}{\phantomsection\label{\detokenize{bibliography:bran02}} 
Brand, R. (2002).
\sphinxstylestrong{Microdata Protection through Noise Addition.}
In J. Domingo-Ferrer (Ed.), Inference Control in Statistical Databases - From Theory to Practice (Vol. Lecture Notes in Computer Science Series Volume 2316, pp. 97-116). Berlin Heidelberg, Germany: Springer.
}
\bibitem[BCRZ13]{\detokenize{BCRZ13}}{\phantomsection\label{\detokenize{bibliography:bcrz13}} 
Burgert, C. R., Colston, J., Roy, T., \& Zachary, B. (2013).
\sphinxstylestrong{Geographic Displacement Procedure and Georeferenced Data Release Policy for the Demographic and Health Surveys.}
DHS Spatial Analysis Report No. 7.
}
\bibitem[DaRe78]{\detokenize{DaRe78}}{\phantomsection\label{\detokenize{bibliography:dare78}} 
Dalenius, T., \& Reiss, S. (1978).
\sphinxstylestrong{Data-swapping: A Technique for Disclosure Control (extended abstract).}
American Statistical Association Proceedings of the Section on Survey Research Methods, (pp. 191-194).
}
\bibitem[WaWi99]{\detokenize{WaWi99}}{\phantomsection\label{\detokenize{bibliography:wawi99}} 
de Waal, A., \& Willenborg, L. (1999).
\sphinxstylestrong{Information Loss Through Global Recoding and Local Suppression.}
Netherlands Official Statistics , 14 (Special issue on SDC), 17-20.
}
\bibitem[WGKW98]{\detokenize{WGKW98}}{\phantomsection\label{\detokenize{bibliography:wgkw98}} 
de Wolf, P., Gouweleew, J. M., Kooiman, P., \& Willenborg, L. (1998).
\sphinxstylestrong{Reflections on PRAM.}
Proceedings of the conference “Statistical Data Protection”. Lisbon.
}
\bibitem[Wolf15]{\detokenize{Wolf15}}{\phantomsection\label{\detokenize{bibliography:wolf15}} 
de Wolf, P.-P. (2015).
\sphinxstylestrong{Public Use Files of EU-SILC and EU-LFS data.}
}
\bibitem[DoMa02]{\detokenize{DoMa02}}{\phantomsection\label{\detokenize{bibliography:doma02}} 
Domingo-Ferrer, J., \& Mateo-Sanz, J. M. (2002).
\sphinxstylestrong{Practical data-oriented microaggregation for statistical disclosure control.}
IEEE Transactions on Knowledge and Data Engineering , 14, 189-201.
}
\bibitem[DoTo01a]{\detokenize{DoTo01a}}{\phantomsection\label{\detokenize{bibliography:doto01a}} 
Domingo-Ferrer, J., \& Torra, V. (2001).
\sphinxstylestrong{A Quantitative Comparison of Disclosure Control Methods for Microdata.}
In P. Doyle, J. Lane, J. Theeuwes, \& L. Zayatz (Eds.), Confidentiality, Disclosure and Data Access: Theory and Practical Applications for Statistical Agencies (pp. 111-133). Amsterdam, North-Holland: Elsevier Science.
}
\bibitem[DoTo01b]{\detokenize{DoTo01b}}{\phantomsection\label{\detokenize{bibliography:doto01b}} 
Domingo-Ferrer, J., \& Torra, V. (2001).
\sphinxstylestrong{Disclosure Protection Methods and Information Loss for Microdata.}
In P. Doyle, J. Lane, J. Theeuwes, \& Z. L., Theory and Practical Applications for Statistical Agencies (pp. 91-110). Amsterdam.
}
\bibitem[DoMT01]{\detokenize{DoMT01}}{\phantomsection\label{\detokenize{bibliography:domt01}} 
Domingo-Ferrer, J., Mateo-Sanz, J., \& Torra, V. (2001).
\sphinxstylestrong{Comparing SDC Methods for Microdata on the basis of Information Loss and Disclosure Risk.}
Pre-proceedings of ETK-NTTS 2001, (pp. 807-826). Crete, Greece.
}
\bibitem[DuBo10]{\detokenize{DuBo10}}{\phantomsection\label{\detokenize{bibliography:dubo10}} 
Dupriez, O., \& Boyko, E. (2010).
\sphinxstylestrong{Dissemination of Microdata Files; Principles, Procedures and Practices.}
International Household Survey Network (IHSN).
}
\bibitem[ElMa03]{\detokenize{ElMa03}}{\phantomsection\label{\detokenize{bibliography:elma03}} 
Elliot , M., \& Manning, A. M. (2003).
\sphinxstylestrong{Using DIS to Modify the Classification of Special Uniques.}
Invited Paper. Joint ECE/Eurostat Work Session on Statistical Data Confidentiality. Luxemboug 2-9 April 2003.
}
\bibitem[ElMF02]{\detokenize{ElMF02}}{\phantomsection\label{\detokenize{bibliography:elmf02}} 
Elliot, M. J., Manning, A. M., \& Ford, R. W. (2002).
\sphinxstylestrong{A Computational Algorithm for Handling the Special Uniques Problem.}
International Journal of Uncertainty, Fuzziness and Knowledge Based System , 10 (5), 493-509.
}
\bibitem[EMMG05]{\detokenize{EMMG05}}{\phantomsection\label{\detokenize{bibliography:emmg05}} 
Elliot, M. J., Manning, A., Mayes, K., Gurd, J., \& Bane, M. (2005).
\sphinxstylestrong{SUDA: A Program for Detecting Special Uniques.}
Joint UNECE/Eurostat Work Session on Statistical Data Confidentiality. Geneva.
}
\bibitem[GKWW98a]{\detokenize{GKWW98a}}{\phantomsection\label{\detokenize{bibliography:gkww98a}} 
Gouweleeuw, J. M., Kooiman, P., Willenborg, L., \& de Wolf, P. (1998a).
\sphinxstylestrong{Post Randomization for Statistical Disclosure Control:Theory and Implementation.}
Journal of Official Statistics , 14 (4), 463-478.
}
\bibitem[GKWW98b]{\detokenize{GKWW98b}}{\phantomsection\label{\detokenize{bibliography:gkww98b}} 
Gouweleeuw, J. M., Kooiman, P., Willenborg, L., \& de Wolf, P. P. (1998b).
\sphinxstylestrong{The Post Randomization Method for Protecting Microdata.}
Qüestiió, Quaderns d’Estadística i Investigació Operativa , 22 (1), 145-156.
}
\bibitem[HaMu03]{\detokenize{HaMu03}}{\phantomsection\label{\detokenize{bibliography:hamu03}} 
Hansen, S. L., \& Mukherjee, S. (2003).
\sphinxstylestrong{A polynomial algorithm for univariate optimal.}
IEEE Transactions on Knowledge and Data Engineering , 15, 1043-1044.
}
\bibitem[HuDr15]{\detokenize{HuDr15}}{\phantomsection\label{\detokenize{bibliography:hudr15}} 
Hu, J., \& Drechsler, J. (2015).
\sphinxstylestrong{Generating synthetic geocoding infromation for public release.}
NTTS - Conferences on New Techniques and Technologies for Statistics. Brussels.
}
\bibitem[HDFG06]{\detokenize{HDFG06}}{\phantomsection\label{\detokenize{bibliography:hdfg06}} 
Hundepool, A., Domingo-Ferrer, J., Franconi, L., Giessing, S., Lenz, R., Naylor, J., et al. (2006).
\sphinxstylestrong{Handbook on Statistical Disclosure Control.}
ESSNet SDC.
}
\bibitem[HDFG12]{\detokenize{HDFG12}}{\phantomsection\label{\detokenize{bibliography:hdfg12}} 
Hundepool, A., Domingo-Ferrer, J., Franconi, L., Giessing, S., Nordholt, E. S., Spicer, K., et al. (2012).
\sphinxstylestrong{Statistical Disclosure Control.}
Chichester, UK: John Wiley \& Sons Ltd.
}
\bibitem[HWRF08]{\detokenize{HWRF08}}{\phantomsection\label{\detokenize{bibliography:hwrf08}} 
Hundepool, A., Van de Wetering, A., Ramaswamy, R., Franconi, L., Polettini, S., Capobianchi, A., et al. (2008).
\sphinxstylestrong{\(\mu\)-Argus User’s Manual.}
The Hague: Statistics Netherlands.
}
\bibitem[KiWi03]{\detokenize{KiWi03}}{\phantomsection\label{\detokenize{bibliography:kiwi03}} 
Kim, J. J., \& Winkler, W. W. (2003, April 17).
\sphinxstylestrong{Multiplicative Noise for Masking Continuous Data.}
Research Report Series.
}
\bibitem[KTMF13]{\detokenize{KTMF13}}{\phantomsection\label{\detokenize{bibliography:ktmf13}} 
Kowarik, A., Templ, M., Meindl, B., \& Fonteneau, F. (2013).
\sphinxstylestrong{Graphical User Interface for Package sdcMicro.}
Retrieved from \sphinxurl{http://cran.r-project.org/web/packages/sdcMicroGUI/index.html}
}
\bibitem[Lamb93]{\detokenize{Lamb93}}{\phantomsection\label{\detokenize{bibliography:lamb93}} 
Lambert, D. (1993).
\sphinxstylestrong{Measures of Disclosure Risk and Harm.}
Journal of Official Statistics , 9 (2), 313-331.
}
\bibitem[MKGV07]{\detokenize{MKGV07}}{\phantomsection\label{\detokenize{bibliography:mkgv07}} 
Machanavajjhala, A., Kifer, D., Gehrke, J., \& Venkitasubramaniam, M. (2007).
\sphinxstylestrong{L-diversity: Privacy Beyond K-anonymity.}
ACM Trans. Knowl. Discov. Data , 1 (Article 3) (1556-4681).
}
\bibitem[MaHK08]{\detokenize{MaHK08}}{\phantomsection\label{\detokenize{bibliography:mahk08}} 
Manning, A. M., Haglin, D. J., \& Keane, J. A. (2008).
\sphinxstylestrong{A Recursive Search Algorithm for Statistical Disclosure Assessment.}
Data Mining and Knowledge Discovery , 16 (2), 165-196.
}
\bibitem[MaTo10]{\detokenize{MaTo10}}{\phantomsection\label{\detokenize{bibliography:mato10}} 
Marés, J., \& Torra, V. (2010).
\sphinxstylestrong{PRAM Optimization Using an Evolutionary Algorithm.}
In J. Domingo-Ferrer, \& E. Magkos, Privacy in Statistical Databases (pp. 97-106). Corfú, Greece: Springer.
}
\bibitem[Mivu12]{\detokenize{Mivu12}}{\phantomsection\label{\detokenize{bibliography:mivu12}} 
Mivule, K. (July 16-19, 2012).
\sphinxstylestrong{Utilizing Noise Addition for Data Privacy, An Overview.}
International Conference on Information and Knowledge Engineering (IKE 2012), (pp. 65-71). Las Vegas, Nevada.
}
\bibitem[MuSa06]{\detokenize{MuSa06}}{\phantomsection\label{\detokenize{bibliography:musa06}} 
Muralidhar, K., \& Sarathy, R. (2006).
\sphinxstylestrong{Data Shuffling- A New Masking Approach for Numerical Data.}
Management Science , 658-670.
}
\bibitem[TeMK14]{\detokenize{TeMK14}}{\phantomsection\label{\detokenize{bibliography:temk14}} 
Templ, M., Meindl, B., \& Kowarik, A. (2014, August).
\sphinxstylestrong{Tutorial for SDCMicroGUI.}
Retrieved from International Household Survey Network (IHSN): \sphinxurl{http://www.ihsn.org/home/software/disclosure-control-toolbox}
}
\bibitem[TMKC14]{\detokenize{TMKC14}}{\phantomsection\label{\detokenize{bibliography:tmkc14}} 
Templ, M., Meindl, B., Kowarik, A., \& Chen, S. (2014, August 1).
\sphinxstylestrong{Introduction to Statistical Disclosure Control (SDC).}
Retrieved November 13, 2014, from \sphinxurl{http://www.ihsn.org/home/software/disclosure-control-toolbox}.
}
\bibitem[ToCa11]{\detokenize{ToCa11}}{\phantomsection\label{\detokenize{bibliography:toca11}} 
Torra, V., \& Cano, I. (2011).
\sphinxstylestrong{Edit Constraints on Microaggregation and Additive Noise.}
In A. G.-D. Christos Dimitrakakis (Ed.), Privacy and Security Issues in Data Mining and Machine Learning (Vol. Lecture Notes in Computer Science Volume 6549, pp. 1-14). Berlin Heidelberg: Springer.
}
\bibitem[TuLS15]{\detokenize{TuLS15}}{\phantomsection\label{\detokenize{bibliography:tuls15}} 
Tudor, C., Lowthian, P., \& Spicer, K. (2015).
\sphinxstylestrong{Opening up governemnt data while maintaining data privacy.}
EDBT/ICDT 2015 Joint Conference.
}
\bibitem[UKIn]{\detokenize{UKIn}}{\phantomsection\label{\detokenize{bibliography:ukin}} 
UK Information Commissioner’s Office.
\sphinxstylestrong{Anonymisation: Managing Data Protection Risk Code of Practice.}
UK Information Commissioner’s Office.
}
\bibitem[Unit07]{\detokenize{Unit07}}{\phantomsection\label{\detokenize{bibliography:unit07}} 
United Nations. (2007).
\sphinxstylestrong{Managing Statistical Confidentiality \& Microdata Access: Principles and Guidelines of Good Practice.}
New York and Geneva: United Nations.
}
\bibitem[Warn65]{\detokenize{Warn65}}{\phantomsection\label{\detokenize{bibliography:warn65}} 
Warner, S. (1965).
\sphinxstylestrong{Randomized Response: A Survey Technique for Eliminating Evasive Answer Bias.}
Journal of American Statistical Association , 57, 622-627.
}
\bibitem[YaWC02]{\detokenize{YaWC02}}{\phantomsection\label{\detokenize{bibliography:yawc02}} 
Yancey, W. W., Winkler, W. E., \& Creecy, R. H. (2002).
\sphinxstylestrong{Disclosure Risk Assessment in Perturbative Microdata Protection.}
Research Report Series , Statistics 2002-01.
}
\bibitem[YaSB08]{\detokenize{YaSB08}}{\phantomsection\label{\detokenize{bibliography:yasb08}} 
Yang, A., Shan, Y., \& Bui, L. T. (2008).
\sphinxstylestrong{Success in Evolutionary Computation.}
Springer Science \& Business Media.
}
\end{sphinxthebibliography}



\renewcommand{\indexname}{Index}
\printindex
\end{document}